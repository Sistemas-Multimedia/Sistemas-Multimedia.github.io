%\newcommand{\SM}{\href{http://cms.ual.es/UAL/estudios/masteres/plandeestudios/asignaturas/asignatura/MASTER7114?idAss=71142105&idTit=7114}{Sistemas Multimedia}}
\newcommand{\SM}{\href{https://sistemas-multimedia.github.io/}{Sistemas Multimedia}}

\newcommand{\theproject}{\href{https://github.com/Sistemas-Multimedia/MRVC}{MRVC}}

\newcommand{\SW}{\href{https://github.com/Sistemas-Multimedia/MRVC}{MRVC}}

\title{\SM{} - Study Guide - Milestone 4: Understanding the Video Data}

\maketitle

\tableofcontents

\section{Description}

This milestone shows the structure of the
\href{https://en.wikipedia.org/wiki/Video}{videos} used in the
\theproject{} project. Basically, a video is a
\href{https://en.wikipedia.org/wiki/Sequence}{sequence} of
(\href{https://en.wikipedia.org/wiki/Digital_data}{digital})
\href{https://en.wikipedia.org/wiki/Image}{images}, that we prefer to
name ``\href{https://en.wikipedia.org/wiki/Film_frame}{frames}''. In
the context of \theproject{}, a frame can be described as a
\href{https://en.wikipedia.org/wiki/Matrix_(mathematics)}{matrix} of
\href{https://en.wikipedia.org/wiki/Pixel}{pixels}.

We are going to store a video as a sequence of
\href{https://en.wikipedia.org/wiki/Computer_file}{files} with names
<prefix>000.png, <prefix>001.png, ..., where <prefix> specifies the
video name and ``.png'' indicates that the frames are coded using the
\href{https://en.wikipedia.org/wiki/Lossless_compression}{lossless}
\href{https://en.wikipedia.org/wiki/Portable_Network_Graphics}{PNG}
image format~\cite{roelofs1999png}. PNG uses
\href{https://en.wikipedia.org/wiki/Differential_pulse-code_modulation}{DPCM}
and the \href{https://en.wikipedia.org/wiki/DEFLATE}{DEFLATE} codec
(\href{https://en.wikipedia.org/wiki/Huffman_coding}{Huffman coding} +
\href{https://en.wikipedia.org/wiki/Lempel-Ziv-Storer-Szymanski}{LZSS})~\cite{nelson96datacompression}
to remove the
\href{https://en.wikipedia.org/wiki/Image_compression}{spatial} and
\href{https://en.wikipedia.org/wiki/Data_compression}{statistical
  redudancy}. Notice that no
\href{https://en.wikipedia.org/wiki/YUV}{color transform} is used.

Finally, notice that all the frames of a video:
\begin{enumerate}
\item Have the same
  \href{https://en.wikipedia.org/wiki/Image_resolution}{spatial
    resolution}.
\item Are represented in the
  \href{https://en.wikipedia.org/wiki/RGB_color_model}{RGB domain}.
\item Each color component (channel) has
  \href{https://en.wikipedia.org/wiki/Glossary_of_computer_graphics#bit_depth}{bit
    depth} of 16 bits, representing an unsigned
  \href{https://en.wikipedia.org/wiki/Integer_(computer_science)}{integer}.
\item The pixel
  \href{https://en.wikipedia.org/wiki/Luminous_intensity}{intensities}
  are displaced to the center of the
  \href{https://en.wikipedia.org/wiki/Range_(computer_programming)}{range
    of values} ([0, 65535]) that the components can take (for example,
  in \href{https://en.wikipedia.org/wiki/Disk_storage}{disk}, the
  component intensity 0 is represented by 32768).
\end{enumerate}
  
\section{What you have to do?}

\begin{enumerate}
\item Download \theproject{}, preferably cloning your fork.
\item Run this \href{https://jupyter.org/}{jupyter}
  \href{https://github.com/Sistemas-Multimedia/Sistemas-Multimedia.github.io/blob/master/milestone/04-the_data/display_video.ipynb}{notebook}
  that shows how to read and write frames.
%\item Understand the insights of the notebook.
\end{enumerate}

\section{Timming}

Please, finish this milestone before the next class session.

\section{Deliverables}

None.

\section{Resources}

\renewcommand{\addcontentsline}[3]{}% Remove functionality of \addcontentsline
\bibliography{image-compression,text-compression}
