%\newcommand{\SM}{\href{http://cms.ual.es/UAL/estudios/masteres/plandeestudios/asignaturas/asignatura/MASTER7114?idAss=71142105&idTit=7114}{Sistemas Multimedia}}
\newcommand{\SM}{\href{https://sistemas-multimedia.github.io/}{Sistemas Multimedia}}

\newcommand{\theproject}{\href{https://github.com/Sistemas-Multimedia/MRVC}{MRVC}}

\newcommand{\SW}{\href{https://github.com/Sistemas-Multimedia/MRVC}{MRVC}}

\title{\SM{} - Study Guide - Milestone 10: IPP... Coding in the Image Domain}

\maketitle

\tableofcontents

\section{Description}

It's time to test the performance of the ME/MC process previously
described. We will encode a sequence of frames $\{W_k\}$ using the
pattern IPP..., which means that the first frame will be intra-coded
(I-type frame) and the rest of frames of the GOF (Group Of
\href{https://en.wikipedia.org/wiki/Group_of_pictures}{Frames}) will
be predicted-coded (P-type frame), respect to the previous one.

\begin{figure}
  \centering
    \svg{graphics/codec}{1100}
  \caption{A simple IPP... image codec.}
\label{fig:IPP_codec}
\end{figure}

IPP... coding can be done by the codec shown in the
Fig.~\ref{fig:IPP_codec}, where (for the case of the YCoCg transform):

\begin{equation}
  V_k \leftarrow \text{C}(W_k) =
  \begin{bmatrix}
    \frac{1}{4} &  \frac{1}{2}  &  \frac{1}{4} \\ 
    \frac{1}{2} &            0  & -\frac{1}{2} \\
    -\frac{1}{4} &  \frac{1}{2}  & -\frac{1}{4}
  \end{bmatrix}
  \begin{bmatrix}
    W_k.\text{R} \\
    W_k.\text{G} \\
    W_k.\text{B}
  \end{bmatrix}
  , \tag{a}
\end{equation}

\begin{equation}
  Z^{-1}(V_k) = V_{k-1},
  \tag{b}
\end{equation}
and by definition, $Z^{-1}(V_{-1}) = 0$,

\begin{equation}
  \overset{k\rightarrow k-1}{V} \leftarrow \text{M}(V_k, V_{k-1}),
  \tag{c}
\end{equation}
where M stands for Motion Estimation, and by definition,
$\overset{0\rightarrow (-1)}{V}=0$,

\begin{equation}
  \overset{\sim}{\overset{k\rightarrow k-1}{V}} \leftarrow \text{E}_{\overset{\rightarrow}{V}}(\overset{k\rightarrow k-1}{V}),
  \tag{d}
\end{equation}
where E$_{\overset{\rightarrow}{V}}(\cdot)$ represents the lossy
  compression of the motion data,

\begin{equation}
  \overset{\sim}{\overset{k\rightarrow k-1}{V}} \leftarrow \text{D}_{\overset{\rightarrow}{V}}(\overset{\sim}{\overset{k\rightarrow k-1}{V}}),
  \tag{e}
\end{equation}
where D$_{\overset{\rightarrow}{V}}(\cdot)$ represents the 
decompression of the motion data,

\begin{equation}
  E_k \leftarrow V_k - \overset{\wedge}{{V}}_k,
  \tag{f}
\end{equation}
where the symbol $-$ represents to the pixel-wise substraction,

\begin{equation}
  \overset{\sim}{E_k} \leftarrow \text{E}_{E}(E_k),
  \tag{g}
\end{equation}
where E$_{E}(\cdot)$ represents the lossy compression of the
prediction error texture data,

\begin{equation}
  \overset{\sim}{E}_k \leftarrow \text{D}_{E}(\overset{\sim}{E}_k),
  \tag{h}
\end{equation}
where D$_{E}(\cdot)$ represents the decompression of the prediction
error texture data,

\begin{equation}
  \overset{\sim}{V}_k \leftarrow \overset{\sim}{E}_k + \overset{\wedge}{V}_k,
  \tag{i}
\end{equation}
and notice that if $\overset{\wedge}{V}_k=0$, then
$\overset{\sim}{E}_k = \overset{\sim}{V}_k$,

\begin{equation}
  \overset{\wedge}{V}_k \leftarrow \text{P}(\overset{\sim}{\overset{k\rightarrow k-1}{V}}, \overset{\sim}{V}_{k-1}),
  \tag{j}
\end{equation}
where P$(\cdot,\cdot)$ is a motion compensated predictor.

Notice that if $\overset{\wedge}{{V}}_k$ is similar to $V_k$, then
$E_k$ will be approximately zero, and therefore, easely
compressed. Another interesting aspect to highlight is that the
encoder replicates de decoder in order to use the reconstructed images
as reference and avoid the drift error.

\section{What do I have to do?}

Run the notebook \href{https://github.com/Sistemas-Multimedia/MRVC/blob/master/src/image_IPP.ipynb}{image\_IPP.ipynb}. Use a different video (see
this
\href{https://github.com/Sistemas-Multimedia/MRVC/tree/master/sequences}{directory}). Check
the improvement of IPP... coding over III... coding. Visualize the
reconstructed sequence using \href{https://ffmpeg.org/ffplay.html}{\texttt{ffplay}}.

\section{Timming}

\section{Deliverables}

\section{Resources}

\renewcommand{\addcontentsline}[3]{}% Remove functionality of \addcontentsline
\bibliography{image-pyramids,DWT,motion-estimation,HEVC}
