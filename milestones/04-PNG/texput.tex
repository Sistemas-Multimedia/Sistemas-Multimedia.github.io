%\newcommand{\SM}{\href{http://cms.ual.es/UAL/estudios/masteres/plandeestudios/asignaturas/asignatura/MASTER7114?idAss=71142105&idTit=7114}{Sistemas Multimedia}}
\newcommand{\SM}{\href{https://sistemas-multimedia.github.io/}{Sistemas Multimedia}}

\newcommand{\theproject}{\href{https://github.com/Sistemas-Multimedia/MRVC}{MRVC}}

\newcommand{\SW}{\href{https://github.com/Sistemas-Multimedia/MRVC}{MRVC}}

\title{\SM{} - Study Guide - Milestone 4: Entropy coding with PNG (Portable Network Graphics)}

\maketitle

\tableofcontents

\section{Description}

This milestone shows the structure of the
\href{https://en.wikipedia.org/wiki/Video}{videos} used in the rest of
this course (and in the \theproject{} project). Basically, a
(\href{https://en.wikipedia.org/wiki/Raster_graphics}{raster}) digital
video is a \href{https://en.wikipedia.org/wiki/Sequence}{sequence} of
(\href{https://en.wikipedia.org/wiki/Digital_data}{digital})
\href{https://en.wikipedia.org/wiki/Image}{images} (or
\href{https://en.wikipedia.org/wiki/Film_frame}{frames}). In the
context, an image can be described as a
\href{https://en.wikipedia.org/wiki/Matrix_(mathematics)}{matrix} of
\href{https://en.wikipedia.org/wiki/Pixel}{pixels} (PIcture X-ray
ELementS), arranged in a regular rectangular grid. Such sequences are
usully captured by a
\href{https://en.wikipedia.org/wiki/Charge-coupled_device}{CCD (Charge
  Coupled Device)}.

From a mathematical perspective, a video can be denoted by
\begin{equation*}
  \{{\mathbf V}_{t,y,x}\in\mathbb{P}^{k}, (t,y,x)\in\mathbb{N}^3\}
\end{equation*}
where $t$ is a temporal coordinate, $(y,x)$ is a spatial coordinate,
and $\mathbb{P}^{k}$ represents the (pixels) \emph{values domain}, with
dimension (number of pixel components)
$k$~\cite{burger2016digital}.

Depending on $k$, we can have grayscale pixels ($k=1$), color pixels
($k=3$), and multispectral pixels ($k>3$).\footnote{When $k>3$ we also
use the term ``super/hyper-spectral'' pixels/images/videos, depending
on the value of $k$.} Finally, if we take into consideration the
number of bits/component that usually is $8$ (although this value can be larger
specially for grayscale images), the pixel-depth will be $8$
bits/pixel for grayscale sequences or $3\times 8=24$ bits/pixel for
color ones.

We are going to store a video as a sequence of
\href{https://en.wikipedia.org/wiki/Computer_file}{files} with names
<prefix>000.png, <prefix>001.png, ..., where <prefix> specifies the
video name and ``.png'' indicates that the frames are coded using the
\href{https://en.wikipedia.org/wiki/Lossless_compression}{lossless}
\href{https://en.wikipedia.org/wiki/Portable_Network_Graphics}{PNG}
(pronounced ``ping'') image
format~\cite{roelofs1999png,world2003portable}. PNG uses
\href{https://en.wikipedia.org/wiki/Differential_pulse-code_modulation}{DPCM}
and the \href{https://en.wikipedia.org/wiki/DEFLATE}{DEFLATE} codec
(\href{https://en.wikipedia.org/wiki/Huffman_coding}{Huffman coding} +
\href{https://en.wikipedia.org/wiki/Lempel-Ziv-Storer-Szymanski}{LZSS})~\cite{nelson96datacompression}
to remove the
\href{https://en.wikipedia.org/wiki/Image_compression}{spatial} and
\href{https://en.wikipedia.org/wiki/Data_compression}{statistical
  redudancy}. Notice that no
\href{https://en.wikipedia.org/wiki/YUV}{color transform} is
used. Therefore, we will use
\href{https://vicente-gonzalez-ruiz.github.io/PNG/}{PNG} basically as
an \href{https://en.wikipedia.org/wiki/Entropy_encoding}{entropy
  codec}.

% We will use \href{https://vicente-gonzalez-ruiz.github.io/PNG/}{PNG}
% basically as an
% \href{https://en.wikipedia.org/wiki/Entropy_encoding}{entropy codec},
% which is a combination of
% \href{https://vicente-gonzalez-ruiz.github.io/Huffman_coding/}{Huffman
%   Coding} and
% \href{https://vicente-gonzalez-ruiz.github.io/LZ77/}{Lempel-Ziv-77
%   Coding}. However, notice that any other lossless image compressing
% file format (such as \href{https://en.wikipedia.org/wiki/TIFF}{TIFF})
% could be used.

Finally, notice that all the frames of a video usually:
\begin{enumerate}
\item Have the same
  \href{https://en.wikipedia.org/wiki/Image_resolution}{spatial
    resolution}.
\item Are represented in the
  \href{https://en.wikipedia.org/wiki/RGB_color_model}{RGB domain}.
\item Depending on the frame, in PNG, each color component (channel) has a
  \href{https://en.wikipedia.org/wiki/Glossary_of_computer_graphics#bit_depth}{bit
    depth} of $8$ or $16$ bits, representing an unsigned
  \href{https://en.wikipedia.org/wiki/Integer_(computer_science)}{integer}.
%\item The pixel
%  \href{https://en.wikipedia.org/wiki/Luminous_intensity}{intensities}
%  are displaced to the center of the
%  \href{https://en.wikipedia.org/wiki/Range_(computer_programming)}{range
%    of values} ([0, 65535]) that the components can take (for example,
%  in \href{https://en.wikipedia.org/wiki/Disk_storage}{disk}, the
%  component intensity 0 is represented by 32768).
\end{enumerate}
  
\section{What do I have to do?}

\begin{enumerate}
\item Download \theproject{}, preferably cloning your own fork.
\item Run this \href{https://jupyter.org/}{jupyter}
  \href{https://github.com/Sistemas-Multimedia/Sistemas-Multimedia.github.io/blob/master/milestones/04-PNG/display_video.ipynb}{notebook}
  that shows how to read and write frames.
%\item Please, try to dig a little bit into the insights of the notebook.
%\item Understand the insights of the notebook.
\end{enumerate}

\section{Timming}

Please, finish this milestone before the next class session.

\section{Deliverables}

None.

\section{Resources}

\renewcommand{\addcontentsline}[3]{}% Remove functionality of \addcontentsline
\bibliography{image-processing,image-compression,text-compression}
