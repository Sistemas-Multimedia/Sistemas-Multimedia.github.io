%\newcommand{\SM}{\href{http://cms.ual.es/UAL/estudios/masteres/plandeestudios/asignaturas/asignatura/MASTER7114?idAss=71142105&idTit=7114}{Sistemas Multimedia}}
\newcommand{\SM}{\href{https://sistemas-multimedia.github.io/}{Sistemas Multimedia}}

\newcommand{\theproject}{\href{https://github.com/Sistemas-Multimedia/MRVC}{MRVC}}

\newcommand{\SW}{\href{https://github.com/Sistemas-Multimedia/MRVC}{MRVC}}

\title{\SM{} - Study Guide - Milestone 12: IPP... coding in MRVC (Multi Resolution Video Codec)}

\maketitle

\tableofcontents

\section{Description}

\subsection{The IPP... decorrelation pattern}
It's time to put together all the ``tools'' that we have developed for
encoding a sequence of frames $\{V_k\}$. First, the sequence will be
splitted into GOFs
(\href{https://en.wikipedia.org/wiki/Group_of_pictures}{Group Of
  Frames}), and the structure of each GOF will be
IPP...~\cite{le1991mpeg}, which means that the first frame of each GOF
will be intra-coded (I-type), and the rest of frames of the GOF will
be predicted-coded (P-type), respect to the previous one\footnote{A
  P-type frame except for the second one, that always has a I-frame as
  reference.}. Notice that in an I-type frame all the coefficients
(\emph{coeffs} in short, remember that we are compensating the motion
in the DWT domain) will be I-type coeffs, and in a P-type frame, the
different coeffs will be I-type or P-type.

\subsection{A block diagram of the step codec}

\begin{figure}
  \centering
  \svg{graphics/codec3}{1000}
  \caption{The IPP... MRVC step codec. Notice that the input to the
    (step) encoder is a DWT transformed sequence of frames.}
  \label{fig:codec}
\end{figure}

The MRVC IPP... step (one resolution level) codec has been described in the
Fig.~\ref{fig:codec}. The equations that describe this system are:

\begin{equation}
  (V_k.L, V_k.H) \leftarrow \text{DWT}(V_k),
  \tag{a}
\end{equation}
where $\leftarrow$ denotes the
\href{https://en.wikipedia.org/wiki/Assignment_(computer_science)}{assignment}
operator, and $V_k$ is the $k$-th frame of the sequence,

\begin{equation}
  [V_k.L] \leftarrow \text{DWT}^{-1}(V_k.L, 0),
  \tag{E.a}
\end{equation}

\begin{equation}
  Z^{-1}([V_k.L]) = [V_{k-1}.L],
  \tag{E.b}
\end{equation}
and by definition, $Z^{-1}([V_{-1}.L]) = 0$,

\begin{equation}
  \overset{k\rightarrow k-1}{V} \leftarrow \text{M}([V_k.L], [V_{k-1}.L]),
  \tag{E.c}
\end{equation}
where M stands for Motion estimation, and by definition,
$\overset{0\rightarrow (-1)}{V}=0$,

\begin{equation}
  [\hat{V}_k.L] \leftarrow \text{P}(\overset{k\rightarrow k-1}{V}, [V_{k-1}.L]),
  \tag{E.d}
\end{equation}
where P stands for motion compensated Prediction,

\begin{equation}
  [E_k.L] \leftarrow [V_k.L] - [\hat{V}_k.L],
  \tag{E.e}
\end{equation}

\begin{equation}
  \{[M_k],[S_k]\} \leftarrow \text{EW-min}([V_k.L], [E_k.L])
  \tag{E.f}
\end{equation}
where
\begin{equation}
  [M_k]_{i,j}=\text{min}([V_k.L]_{i,j}, [E_k.L]_{i,j}),
\end{equation}
and $[S_k]$ is a binary matrix defined by
\begin{equation}
  [S_k]_{i,j} = \left\{
  \begin{array}{lll}
    0 & \text{if}~[V_k.L]_{i,j} < [E_k.L]_{i,j} & \text{(I-type coeff)} \\
    1 & \text{otherwise}                      & \text{(P-type coeff)},
  \end{array}
  \right.
  \label{eq:matrix}
\end{equation}
(notice that $[M_k]$, that contains the element-wise minimum of both
matrices, is discarded)

\begin{equation}
  [V_k.H] \leftarrow \text{DWT}^{-1}(0, V_k.H),
  \tag{b}
\end{equation}

\begin{equation}
  [E_k.H] \leftarrow [V_k.H] - [\hat{V}_k.H],
  \tag{c}
\end{equation}
where, notice that
\begin{equation}
  [E_k.H]_{i,j} = \left\{
  \begin{array}{ll}
    {[}V_k.H{]}_{i,j}                       & \text{if}~{[}\hat{V}'_k.H{]}_{i,j} = 0~\text{(I-type coeff)} \\
    {[}V_k.H{]}_{i,j} - [\hat{V}'_k.H]_{i,j} & \text{otherwise}~\text{(P-type coeff)},
  \end{array}
\right.
\end{equation}

\begin{equation}
  [\tilde{E}_k.H] \leftarrow \text{Q}([E_K.H]),
  \tag{d}
\end{equation}

\begin{equation}
  [\tilde{E}_k.H] \leftarrow  \text{Q}^{-1}([\tilde{E}_K.H]),
  \tag{E.g}
\end{equation}

\begin{equation}
  [\tilde{V}_k.H] \leftarrow [\tilde{E}_k.H] + [\hat{V}'_k.H],
  \tag{E.h}
\end{equation}
and notice that if $[\hat{V}_k.H]=0$, then $[\tilde{V}_k.H] =
[\tilde{E}_k.H]$,

\begin{equation}
  Z^{-1}([\tilde{V}_k.H]) = [V_{k-1}.H],
  \tag{E.i}
\end{equation}
and by definition, $Z^{-1}([V_{-1}.H]) = 0$,

\begin{equation}
  [\hat{V}_k.H] \leftarrow \text{MCP}(\overset{k+1\rightarrow k}{V}, [V_{k-1}.H]),
  \tag{E.j}
\end{equation}

\begin{equation}
  [\hat{V}'_k.H]_{i,j} \leftarrow \left\{
    \begin{array}{ll}
      {[}\hat{V}_k.H{]}_{i,j}  & \text{if}~{[}E_k.L{]}_{i,j} < {[}V_k.L{]}_{i,j} \text{(P-type coeff)} \\
      0                       & \text{otherwise (I-type coeff)},
    \end{array}
  \right.
  \tag{E.k}
\end{equation}
  
\begin{equation}
  (0, \tilde{E}_k.H) \leftarrow \text{DWT}([\tilde{E}_k.H]),
  \tag{f}
\end{equation}

\begin{equation}
  \{V_k.L, \tilde{E}_k.H\} \leftarrow \text{E}(V_k.L, \tilde{E}_k.H),
  \tag{g}
\end{equation}
where E represents the entropy coding of both data sources, in two
different code-streams,

\begin{equation}
  (V_k.L, \tilde{E}_k.H) \leftarrow \text{E}^{-1}(\{V_k.L, \tilde{E}_k.H\}),
  \tag{h}
\end{equation}

\begin{equation}
  [\tilde{E}_k.H] \leftarrow \text{DWT}^{-1}(0, \tilde{E}_k.H),
  \tag{i}
\end{equation}

\begin{equation}
  (0, \tilde{V}_k.H) \leftarrow \text{DWT}(0, [\tilde{V}_k.H]),
  \tag{j}
\end{equation}

and

\begin{equation}
  \tilde{V}_k \leftarrow \text{DWT}^{-1}(V_k.L, \tilde{V}_k.H).
  \tag{k}
\end{equation}

The IPP... codec is inspired in Differential Pulse Code Moldulation.
This
\href{https://github.com/Sistemas-Multimedia/Sistemas-Multimedia.github.io/blob/master/milestones/12-IPP_coding/DPCM.ipynb}{notebook}
shows how to implement a simple DPCM codec.

\subsection{Spatial multiresolution}
As it can be seen in the previous section, in each IPP... iteration of
the step encoder, only the high-frequency information of the sequence
of frames is decorrelated ($H$ subbands) considering the information
provided by the low-frequences ($L$ subband), which are losslessly
transmitted between the encoder and the decoder. Notice also that if
the $L$ data cannot be transmitted to the decoder, a drift error will
occur because the matrix $S_k$ will be different in the encoder and
the decoder for the same frame $k$.

Obviously (and unfortunately), the lossless transmission of the $L$'s
bounds the compression ratio that we will get. One solution is to
perform more (than 1) levels at the DWT stage (see Eq.~(a)) and to
apply the IPP... MRVC step encoder by spatial resolutions, starting at
the lowest, as the decoder will do at decompression time. If we
represent the Spatial Resolution Level (SRL) with an superindex, being
0 the original SRL, we can express the operation of the codec
described in the Fig.~\ref{fig:codec} by
\begin{equation}
  \left\{
    \begin{array}{l}
      \text{SE}(V^0_k) = \{V^0_k.L, \tilde{E}^0_k.H\} = \{V^1_k, \tilde{E}^0_k.H\} \\
      \text{SD}(\{V^1_k, \tilde{E}^0_k.H\}) = \tilde{V}^0_k,
    \end{array}
  \right.
  \label{eq:codec_1l}
\end{equation}
where $\text{SE}(\cdot)$ represents to the operation of the
IPP... step encoder and $\text{SD}(\cdot)$ to the operation of the
IPP... step decoder. As it can be seen, Eq.~\ref{eq:codec_1l} is only
valid when only one level of the DWT has been applied.

In general, for $s$ levels of the DWT, we have that
\begin{equation}
  \left\{
    \begin{array}{l}
      \text{SE}(V^{s-1}_k) = \{V^s_k, \tilde{E}^{s-1}_k.H\} \\
      \text{SD}(\{V^s_k, \tilde{E}^{s-1}_k.H\}) = \tilde{V}^{s-1}_k,
    \end{array}
  \right.
  \label{eq:codec_sl}
\end{equation}
where $\tilde{V}^{s-1}$ is the $(s-1)$-th SRL of the reconstructed
sequence $\tilde{V}$.

The next SRL ($s-2$), $\tilde{V}^{s-2}$, is determined by
\begin{equation}
  \left\{
    \begin{array}{l}
      \text{SE}(\tilde{V}^{s-2}_k) = \{\tilde{V}^{s-1}_k, \tilde{E}^{s-2}_k.H\} \\
      \text{SD}(\{\tilde{V}^{s-1}_k, \tilde{E}^{s-2}_k.H\}) = \tilde{V}^{s-2}_k,
    \end{array}
  \right.
  \label{eq:codec_s1l}
\end{equation}
and finally, for the highest SRL, we get $\tilde{V}^0$ defined by
Eq.~\ref{eq:codec_1l}.
%\begin{equation}
%  \left\{
%    \begin{array}{l}
%      \text{C}(\tilde{V}^0_k) = \{\tilde{V}^1_k, \tilde{E}^0_k.H\} \\
%      \text{D}(\{\tilde{V}^1_k, \tilde{E}^0_k.H\}) = \tilde{V}^0_k.
%    \end{array}
%  \right.
%  \label{eq:codec_s1l}
%\end{equation}

\begin{figure}
  \centering
  \svg{graphics/encoder}{600}
  \caption{The IPP... MRVC encoder.}
  \label{fig:encoder}
\end{figure}

So, only the lowest SRL of $\tilde{V}$, $\tilde{V}^s$ is an
III... pure sequence of small frames, losslessly encoded. This
multiresolution (multistage) scheme has been described in the
Fig.~\ref{fig:encoder}. The output of an IPP... encoder will be
refered as ``spatial-layers'', or simply as ``S-layers''.

The Fig.~\ref{fig:decoder} shows the decoder. It inputs the collection
of subbands generated by the encoder and for each one, a video with a
different spatial resolution is obtained.

\begin{figure}
  \centering
  \svg{graphics/decoder}{320}
  \caption{The IPP... MRVC decoder.}
  \label{fig:decoder}
\end{figure}

\subsection{SQ (layers) progression (next milestone?)}
The logical order of the S-layers in the code-stream is the one that
allows, when the code-stream is decoded sequentially, the progressive
increase of the spatial resolution of the video. For example, if
$s$ is the number of levels of the DWT, the generated
code-stream has $(s+1)$ S-layers
\begin{equation*}
  \{V^s,\tilde{E}^{s-1}.H,\tilde{E}^{s-2}.H,\cdots,\tilde{E}^0.H\},
\end{equation*}
which are able to generate $(s+1)$ progressive
reconstructions
\begin{equation*}
  \{V^s,\tilde{V}^{s-1},\tilde{V}^{s-2},\cdots,\tilde{V}^0\}.
\end{equation*}

Moreover, Quality (Q-progression) scalability in each SRL can be also
achieved in the high-frequency textures if a quality-scalable image
codec such as JPEG2000~\cite{taubman2002jpeg2000} replaces the PNG
compressor, generating a number $q$ of quality-layers (``Q-layers'')
by each motion compensated high-frequency subband. A SQ-progression is
defined considering both forms of scalability (spatial and quality),
with a higher number of layers. For example, if $s=3$ (2 IPP...-type
iterations) and $q=2$, the progression of layers would be
\begin{equation*}
  \{V^s[1],V^s[0],\tilde{E}^{s-1}.H[1],\tilde{E}^{s-1}.H[0],\tilde{E}^{s-2}.H[1],\tilde{E}^{s-2}.H[0],\cdots,\tilde{E}^0.H[1],\tilde{E}^0.H[0]\}.
\end{equation*}
%Considering both forms of scalability (spatial and quality),
%the SQ-progression, that can be Pythonically-described by
%\begin{verbatim}
%progression = []
%for S_layer in range(N_SRL-1, 0, -1):
%  for Q_layer in range(N_QL-1, 0, -1):
%    progression.append((S_layer, Q_layer))
%\end{verbatim}

%For example, if $N_{\text{SRL}}=3$ (2 IPP...-type iterations) and
%$N_{\text{QL}}=2$, the progression of layers would be:
%\begin{verbatim}
%progression = [
%  (S_layer=2, Q_layer=1),  <-- First one to be decoded
%  (S_layer=2, Q_layer=0),
%  (S_layer=1, Q_layer=1),
%  (S_layer=1, Q_layer=0),
%  (S_layer=0, Q_layer=1),
%  (S_layer=0, Q_layer=0) ] <-- Last one
%\end{verbatim}

The use of quality scalability boosts the possibilities in real-time
streaming scenarios where the transmission bit-rate can be variable
(sending more or less Q-layers of a given spatial resolution depending
on the bit-rate). Notice that the SQ-progression is free of
drift-error.\footnote{Other progressions such as the QS-progression
  should generate drift because the decoder at some truncation points
  of the code-stream would use a low-frequency information with a
  different quality than the encoderd used.}

%However, notice that the rendering of the frames at
%the decoder side with a smaller quality (and this will depends on the
%transmission bit-rate) will generate a drift-error in the
%reconstruction of the video because the predictions used at the
%encoder and the decoder are not identical.

%When only spatial scalability is used, this scheme is free of
%drift-error.

\section{What you have to do?}

Implement the codec described in the Fig.~\ref{fig:codec},
preferiblemente in a (or several) Jupyter notebook(s). Test it by
tracing a RD curve for different $\Delta$ values for the quantizer
(see Eqs. (d) and (g)). Compare it with the curve generated without
using MC (use always I-type frames).

\section{Timming}

In groups, you will present the results for this milestone during the
exam time.

\section{Deliverables}

The notebook(s) and the presentation.

\section{Resources}

\renewcommand{\addcontentsline}[3]{}% Remove functionality of \addcontentsline
\bibliography{video_compression,JPEG2000}
