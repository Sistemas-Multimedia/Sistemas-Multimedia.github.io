%\newcommand{\SM}{\href{http://cms.ual.es/UAL/estudios/masteres/plandeestudios/asignaturas/asignatura/MASTER7114?idAss=71142105&idTit=7114}{Sistemas Multimedia}}
\newcommand{\SM}{\href{https://sistemas-multimedia.github.io/}{Sistemas Multimedia}}

\newcommand{\theproject}{\href{https://github.com/Sistemas-Multimedia/MRVC}{MRVC}}

\newcommand{\SW}{\href{https://github.com/Sistemas-Multimedia/MRVC}{MRVC}}

\title{\SM{} - Study Guide - Milestone 7: Removing Spatial Redundancy with the Discrete Cosine Transform}

\maketitle

\tableofcontents

\section{Description}

Transforms concentrate (compact) information in a few output
coefficients by decorrelating the input
samples~\cite{sayood2017introduction}.

The DCT (discrete

In the 1D case, the forward DCT for a signal $g(n)$ of
length $N$ is defined as~\cite{burger2016digital}
\begin{equation}
  G(u) = \sqrt{\frac{2}{N}}\sum_{x=0}^{N-1}g(x)C_u\cos\Big(\pi\frac{u(2x+1)}{2N}\Big),
\end{equation}
for $0<u<N$, and the inverse transform is
\begin{equation}
  g(n) = \sqrt{\frac{2}{N}}\sum_{u=0}^{N-1}G(u)C_u\cos\Big(\pi\frac{u(2x+1)}{2N}\Big),
\end{equation}
for $0<n<N$, with
\begin{equation}
  c_u = \left\{
  \begin{array}{ll}
    \frac{1}{\sqrt{2}} & \quad for $u=0$, \\
    1 & \quad \text{otherwise}.
    \end{array}
\end{equation}

The 2D-DCT is separable, which means that it can be computed by
appliying the 1D-DCT to the two dimenssions of the signal. For the
inverse 2D-DCT, the procedure is similar, but appliying the inverse
1D-DWT in reverse order.

The DCT can be also expressed in matrix form as
\begin{equation}
  G = Ag

\section{Timming}

Please, finish this milestone before the next class session.

\section{Deliverables}

None.

\section{Resources}

\renewcommand{\addcontentsline}[3]{}% Remove functionality of \addcontentsline
\bibliography{maths,data-compression,signal-processing,DWT,image-compression,image-processing}
