%\newcommand{\SM}{\href{http://cms.ual.es/UAL/estudios/masteres/plandeestudios/asignaturas/asignatura/MASTER7114?idAss=71142105&idTit=7114}{Sistemas Multimedia}}
\newcommand{\SM}{\href{https://sistemas-multimedia.github.io/}{Sistemas Multimedia}}

\newcommand{\theproject}{\href{https://github.com/Sistemas-Multimedia/MRVC}{MRVC}}

\newcommand{\SW}{\href{https://github.com/Sistemas-Multimedia/MRVC}{MRVC}}

\title{\SM{} - Study Guide - Milestone 7: Removing Spatial Redundancy with the Discrete Cosine Transform}

\maketitle

\tableofcontents

\section{Description}

Transforms concentrate (compact) information (that can be estimated
through the variance of the entropy) in a few output coefficients by
decorrelating the input samples~\cite{sayood2017introduction}. Such
concentration normally implies an increment in the number of bits
necessary to represent the coefficients with a higher information.

The DCT (Discrete Cosine Transform) is used in some of the most
succesful image and video compressors, such as
\href{https://en.wikipedia.org/wiki/JPEG}{JPEG} and
\href{https://en.wikipedia.org/wiki/Advanced_Video_Coding}{H.264},
\href{https://en.wikipedia.org/wiki/Advanced_Video_Coding}{HEVC} and
\href{https://en.wikipedia.org/wiki/Versatile_Video_Coding}{VVC}.

In the 1D case, the forward DCT for a signal $g(n)$ of
length $N$ is defined as~\cite{burger2016digital}
\begin{equation}
  {\mathbf h}_u = \sqrt{\frac{2}{N}}\sum_{x=0}^{N-1}{\mathbf
    g}_xC_u\cos\Big(\pi\frac{u(2x+1)}{2N}\Big),
\end{equation}
for $0<u<N$, and the inverse transform is
\begin{equation}
  {\mathbf g}_x = \sqrt{\frac{2}{N}}\sum_{u=0}^{N-1}{\mathbf
    h}_uC_u\cos\Big(\pi\frac{u(2x+1)}{2N}\Big),
\end{equation}
for $0<n<N$, with
\begin{equation}
  c_u = \left\{
  \begin{array}{ll}
    \frac{1}{\sqrt{2}} & \quad \text{for} u=0, \\
    1 & \quad \text{otherwise}.
  \end{array}
  \right.
\end{equation}

The transformed signal is a sequence of coefficients ${\mathbf h}_u$
with the same length than the original signal ${\mathbf g}_x$, and the
position of the coefficients in the transform domain indicate the
contribution of such coefficient to the corresponding (in this case,
spatial) frequency. For example, the coefficient at the position 0
(that is commonly refered as DC (Direct Current)) is equal to the
average of the signal. The rest of coefficients high-frequency
coefficients are called AC (Alternating Current) coefficients.

The 2D-DCT is separable, which means that it can be computed by
appliying the 1D-DCT to the two dimensions of the signal. For the
inverse 2D-DCT, the procedure is similar, but appliying the inverse
1D-DWT in reverse order. The Fig.~\ref{fig:2D-DCT_basis} show the
first 64 2D-DCT basis (see this \href{}{notebook}).

\begin{figure}
  \centering
  \png{2D-DCT_basis}{600}
  \caption{First 64 2D-DCT basis functions.}
  \label{fig:2D-DCT_basis}
\end{figure}

The DCT can be also expressed in matrix~\cite{sayood2017introduction}
form as
\begin{equation}
  {\mathbf h} = {\mathbf A}{\mathbf g},
\end{equation}
where ${\mathbf A}$ is the forward transform matrix. The rows of the
transform matrix are often referred to as the basis vectors for the
transform because they form an orthonormal basis set (see these
\href{https://cseweb.ucsd.edu/classes/fa17/cse166-a/lec13.pdf}{slides}),
and the elements of the transformed sequence are often called the
transform coefficients. Obviously, the inverse transform is computed
by
\begin{equation}
  {\mathbf g} = {\mathbf A}^{-1}{\mathbf h},
\end{equation}
where ${\mathbf A}^{-1}$ denotes the inverse matrix of ${\mathbf
  A}$. In the case of the DCT, ${\mathbf A}$ and therefore, ${\mathbf
  A}^{-1}={\mathbf A}^{\text T}$, where $\cdot^{\text T}$ denotes the
transpose of ${\mathbf A}$.

The DCT is usually applied by 2D blocks which, in most of the previous
image and video compression standards, are of 8x8 pixels. This a
direct consequence of that, usually the DCT losses compactation
efficiency when the block size is increased (although this depends on
the signal characteristics). When we group the coefficients of several
blocks are considered together, they form a subband, and the
collection of subbands, a
decomposition~\cite{vetterli2014foundations}.

The DCT is orthonormal (orthogonal + unitary). Orthonormal transforms
are energy preserving; that is, the sum of the squares of the
transformed sequence is the same as the sum of the squares of the
original sequence.

\section{Timming}

Please, finish this milestone before the next class session.

\section{Deliverables}

None.

\section{Resources}

\renewcommand{\addcontentsline}[3]{}% Remove functionality of \addcontentsline
\bibliography{maths,data-compression,signal-processing,DWT,image-compression,image-processing}
