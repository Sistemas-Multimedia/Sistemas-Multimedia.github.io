%\newcommand{\SM}{\href{http://cms.ual.es/UAL/estudios/masteres/plandeestudios/asignaturas/asignatura/MASTER7114?idAss=71142105&idTit=7114}{Sistemas Multimedia}}
\newcommand{\SM}{\href{https://sistemas-multimedia.github.io/}{Sistemas Multimedia}}

\newcommand{\theproject}{\href{https://github.com/Sistemas-Multimedia/MRVC}{MRVC}}

\newcommand{\SW}{\href{https://github.com/Sistemas-Multimedia/MRVC}{MRVC}}

\title{\SM{} - Study Guide - Milestone 7: Removing Spatial Redundancy with the 2D-Discrete Cosine Transform}

\maketitle

\tableofcontents

\section{Description}

\subsection{Decorrelating with the \href{https://en.wikipedia.org/wiki/Discrete_cosine_transform}{DCT
(Discrete Cosine Transform)}}

Transforms concentrate (compact) information (that can be estimated
through the \href{https://en.wikipedia.org/wiki/Variance}{variance} or
the \href{https://en.wikipedia.org/wiki/Entropy}{entropy}) in a few
output coefficients by decorrelating the input
samples~\cite{sayood2017introduction}. Such energy concentration can
be used to find good compression schemes. In this milestone, we will
use DCT to remove the spatial redundancy in the components of an
image. Notice that the DCT is used in some of the most succesful image and
video compressors, such as
\href{https://en.wikipedia.org/wiki/JPEG}{JPEG} and
\href{https://en.wikipedia.org/wiki/Advanced_Video_Coding}{H.264},
\href{https://en.wikipedia.org/wiki/Advanced_Video_Coding}{HEVC} and
\href{https://en.wikipedia.org/wiki/Versatile_Video_Coding}{VVC}.

In the 1D case, the forward DCT for a signal $\mathbf{g}[n]$ of
length $N$ is defined as~\cite{burger2016digital}
\begin{equation}
  {\mathbf h}[u] = \sqrt{\frac{2}{N}}\sum_{n=0}^{N-1}{\mathbf
    g}[x]{\mathbf c}[u]\cos\Big(\pi\frac{u(2x+1)}{2N}\Big),
\end{equation}
for $0<u<N$, and the inverse transform is
\begin{equation}
  {\mathbf g}[n] = \sqrt{\frac{2}{N}}\sum_{u=0}^{N-1}{\mathbf
    h}[u]{\mathbf c}[u]\cos\Big(\pi\frac{u(2x+1)}{2N}\Big),
\end{equation}
for $0<n<N$, with
\begin{equation}
  {\mathbf c}[u] = \left\{
  \begin{array}{ll}
    \frac{1}{\sqrt{2}} & \quad \text{for}~u=0, \\
    1 & \quad \text{otherwise}.
  \end{array}
  \right.
\end{equation}

The transformed signal is a sequence of coefficients ${\mathbf h}_u$
with the same length than the original signal ${\mathbf g}_x$, and the
position of the coefficients in the transform domain indicate the
contribution of such coefficient to the corresponding (in this case,
spatial) frequency. For example, the coefficient at the position 0
(that is commonly refered as DC (Direct Current)) is equal to the
average of the signal. The rest of high-frequency coefficients are
called AC (Alternating Current) coefficients.

The DCT can be also expressed in matrix~\cite{sayood2017introduction}
form as
\begin{equation}
  {\mathbf h} = {\mathbf K}{\mathbf g},
\end{equation}
where ${\mathbf K}$ is the forward transform matrix. The rows of the
transform matrix are often referred to as the basis vectors for the
transform because they form an orthonormal basis set (see these
\href{https://cseweb.ucsd.edu/classes/fa17/cse166-a/lec13.pdf}{slides}),
and the elements of the transformed sequence are often called the
transform coefficients. Obviously, the inverse transform is computed
by
\begin{equation}
  {\mathbf g} = {\mathbf K}^{-1}{\mathbf h},
\end{equation}
where ${\mathbf K}^{-1}$ denotes the inverse matrix of ${\mathbf
  K}$. In the case of the DCT, ${\mathbf K}$ and therefore, ${\mathbf
  K}^{-1}={\mathbf K}^{\text T}$, where $\cdot^{\text T}$ denotes the
transpose of ${\mathbf K}$.

\subsection{The 2D-DCT}

The 2D-DCT is separable, which means that it can be computed by
appliying the 1D-DCT to the two dimensions of the signal. For the
inverse 2D-DCT, the procedure is similar, but appliying the inverse
1D-DWT in reverse order. The Fig.~\ref{fig:2D-DCT_basis} show the
first 64 2D-DCT basis (see this
\href{https://github.com/Sistemas-Multimedia/Sistemas-Multimedia.github.io/blob/master/milestones/07-DCT/DCT_basis.ipynb}{notebook}).

\begin{figure}
  \centering
  \png{2D-DCT_basis}{600}
  \caption{First 64 2D-DCT basis functions.}
  \label{fig:2D-DCT_basis}
\end{figure}

\subsection{Block-based transform coding}

The DCT is usually applied by 2D blocks which, in most of the previous
image and video compression standards are of 8x8 pixels. This a direct
consequence of that, usually the DCT losses compaction efficiency when
the block size is increased (although this depends on the signal
characteristics). When the coefficients of several blocks are
considered together, they form a
\href{https://en.wikipedia.org/wiki/Sub-band_coding}{subband}, and the
collection of subbands, a
\href{https://en.wikipedia.org/wiki/Discrete_wavelet_transform}{decomposition}~\cite{vetterli2014foundations},
and the
\href{https://en.wikipedia.org/wiki/Array_data_structure#Element_identifier_and_addressing_formulas}{index}
of the subband is related to the
\href{https://en.wikipedia.org/wiki/Frequency}{frequency} of the
signal. For example, in the case of the
\href{https://en.wikipedia.org/wiki/Digital_image}{images}, the
position of the
\href{https://en.wikipedia.org/wiki/Coefficient}{coefficients} in the
subbands is related to
\href{https://github.com/Sistemas-Multimedia/Sistemas-Multimedia.github.io/blob/master/milestones/07-DCT/block_DCT_compression.ipynb}{the
  spatial area where the corresponding pixels are found}.

The DCT is orthonormal (orthogonal + unitary). Orthonormal transforms
are energy preserving; that is, the sum of the squares of the
transformed sequence is the same as the sum of the squares of the
original sequence.

\subsection{Quantization in the (spatial) DCT domain}

The DCT is orthonormal, which implies that the gain of each
block-based 2D-DWT subband is equal to 1, and also that the
quantization error is \emph{amplificated} with the same gain in all
the subbands. However, this does not mean that all the subbands should
be quantized using the same $\Delta$, because some subbands can be
compressed more efficienly than others and their SNR (after
quantization) can be
different. This \href{https://github.com/Sistemas-Multimedia/Sistemas-Multimedia.github.io/blob/master/milestones/07-DCT/block_DCT_compression.ipynb}{notebook}
shows how to use RD optimization to determine, given a target
bit-rate, the best combination of quantization steps per subbands.

\section{What do I have to do?}
With the aim of checking that distortion can be measured in both, the
transform and the image (RGB) domain, extend the
previous \href{https://github.com/Sistemas-Multimedia/Sistemas-Multimedia.github.io/blob/master/milestones/07-DCT/block_DCT_compression.ipynb}{notebook}
to compute the distortion in the RGB domain. Check that the result is
the same.

\section{Timming}

Please, finish this milestone before the next class session.

\section{Deliverables}

None.

\section{Resources}

\renewcommand{\addcontentsline}[3]{}% Remove functionality of \addcontentsline
\bibliography{maths,data-compression,signal-processing,DWT,image-compression,image-processing}
