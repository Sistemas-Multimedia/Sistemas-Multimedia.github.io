\newcommand{\SM}{\href{http://cms.ual.es/UAL/estudios/masteres/plandeestudios/asignaturas/asignatura/MASTER7114?idAss=71142105&idTit=7114}{Sistemas Multimedia}}

\newcommand{\theproject}{\href{}}{MCDWT}
\title{\SM{} - Study Guide - Milestone 6: Removing Intercomponent Redundancy}

\maketitle

\section{Description}

Redundancy in signals is usually expressed as a correlation or
statistical/spatio/temporal dependency between samples. In the case of
a video, a sample is a color component and one source of redundancy is
the correlation between color components.

One way of estimating this redudancy is to compute the 0-order entropy
of the signal: the higher the entropy, the lower the redudancy. In
fact, if we suppose that the samples of the signal are uncorrelated,
the 0-order entropy is an exact mesarure of the expected bit-rate
achieved by an arithmetic codec.

An alternative way of estimating the redundancy is to use a
compressor: the higher the length of the compressed file, the higher
the redundancy. This estimation is more accurate than the entropy, in
general, but notice that depends on the encoding algorithm (different
algoritms can provide different extimations).

In this milestone we are going to estimate the intercomponent
redundancy of a video frame, i.e., the redundancy generated by the
color space used. The two color spaced that we are going to research
are: (1) \href{https://en.wikipedia.org/wiki/RGB_color_model}{the RGB
  color space}, and (2) the
\href{https://en.wikipedia.org/wiki/YCbCr}{YCC color space}. To travel
from RGB to , we will use
\href{https://docs.opencv.org/3.4/de/d25/imgproc_color_conversions.html}{the
  RGB to YCC transform} defined as
\begin{equation}
  \begin{array}{lcl}
    \text{Y}  & = & 0.299\text{R} + 0.587\text{G} + 0.114\text{B} \\
    \text{Cr} & = & 0.713(\text{R} - \text{Y}) + \delta  \\
    \text{Cb} & = & 0.564(\text{B} - \text{Y}) + \delta,
  \end{array}
\end{equation}
where
\begin{equation}
  \delta = \left\{
  \begin{array}{ll}
    128 & \text{for 8 bits images},\\
    32768 & \text{for 16 bits images},\\
    0.5 & \text{for floating point images}.
  \end{array}
  \right.
\end{equation}

The inverse transform is defined by
\begin{equation}
  \begin{array}{lcl}
    \text{R} & = & \text{Y} + 1.403(\text{Cr} - \delta) \\
    \text{G} & = & \text{Y} - 0.714(\text{Cr} - \delta) - 0.344(\text{Cb} - \delta)\\
    \text{B} & = & \text{Y} + 1.773(\text{Cb} - \delta).
  \end{array}
\end{equation}

\section{What you have to do?}
  
Please, modify the
\href{https://github.com/Sistemas-Multimedia/Sistemas-Multimedia.github.io/blob/master/study_guide/05-quantization/quantize_a_frame.ipynb}{notebook}
in order to use the
\href{https://docs.opencv.org/master/d4/da8/group__imgcodecs.html}{TIFF
  and JPEG 2000 image formats} instead of PNG. Compare the RD curves.

\section{Timming}

Please, finish this notebook before the next class session.

\section{Deliverables}

None.

\section{Resources}

\bibliography{data-compression,signal-processing}
