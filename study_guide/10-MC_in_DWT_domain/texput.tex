%\newcommand{\SM}{\href{http://cms.ual.es/UAL/estudios/masteres/plandeestudios/asignaturas/asignatura/MASTER7114?idAss=71142105&idTit=7114}{Sistemas Multimedia}}
\newcommand{\SM}{\href{https://sistemas-multimedia.github.io/}{Sistemas Multimedia}}

\newcommand{\theproject}{\href{https://github.com/Sistemas-Multimedia/MRVC}{MRVC}}

\newcommand{\SW}{\href{https://github.com/Sistemas-Multimedia/MRVC}{MRVC}}

\title{\SM{} - Study Guide - Milestone 10: Motion Compensation in the DWT Domain}

\maketitle

\section{Description}

\subsection{Removing the temporal redundancy through Motion Compensation (MC)}
The next natural step in the process of decorrelating the sequence of
frames is to remove the temporal redundancy by means of Motion
Compensation (MC). Basically, MC consists in substracting to the video
data a prediction performed with the information that is avaliable to
the decoder. If this prediction is accurate, the result of this
operation is a residual video with a lower temporal redundancy, that
can be compressed with a higher compression ratio because there is
less information to encode in the residue sequence than in the
original one.

\subsection{Integer pixel accuracy In-Band Motion Estimation and Compensation}
At this stage of the encoding process, the video data is represented
in the DWT domain, and therefore, we need to perform a In-Band Motion
Estimation and Compensation~\cite{andreopoulos2005complete}). Let's
suppose that the number of levels of the DWT is 1, and therefore, each
frame has been decomposed into two 2D subbands $L$ and $H$ (remember
that using the notation introduces in the previous milestone, $H$ has
inside the three high-frequency subbands: $LH$, $HL$ and $HH$, and
that $L=LL$). This discussion will be also constrained to the case in
which the movement of the objects in the scene is a integer number of
pixels.

\subsection{The lack of shift-invariance in the DWT domain}
Unfortunately, DWT decompositions are shift-variant as a consequence
of the downsampling performed during the DWT to achieve a critical
representation. This can be seen in the
Fig.~\ref{fig:dwt_shift_variance} were some DWT coefficients of a test
video with three test frames has been shown. As it can be seen, when
the circle moves to the left only one pixel (between frames 0 and 1),
the value of the coefficients that correspond to the circunference of
the circle are different between the reference frame (0) and the
predicted frame (1). This makes quite difficult to estimate and
compensate the motion between frames. Notice also that the effects of
shift-variance is also visible after using the inverse transform when
the coefficients are filtered or quantized, because the aliasing
between the filters is not completely cancelled in this
case~\cite{bradley2003shift}.

\begin{figure}
  \centering
  \begin{tabular}{ccc}
    \vbox{\pngfig{moving_circle_000}{4cm}{400}} &
    \vbox{\myfig{movement}{4cm}{400}} &
    \vbox{\pngfig{difference}{4cm}{400}} \\
    \vbox{\myfig{haar_LL}{4cm}{400}} &
    \vbox{\myfig{db5_LL}{4cm}{400}} &
    \vbox{\myfig{bior35_LL}{4cm}{400}} \\
    \vbox{\myfig{haar_LH}{4cm}{400}} &
    \vbox{\myfig{db5_LH}{4cm}{400}} &
    \vbox{\myfig{bior35_LH}{4cm}{400}} \\ 
    \vbox{\myfig{haar_HL}{4cm}{400}} &
    \vbox{\myfig{db5_HL}{4cm}{400}} &
    \vbox{\myfig{bior35_HL}{4cm}{400}} \\
    \vbox{\myfig{haar_HH}{4cm}{400}} &
    \vbox{\myfig{db5_HH}{4cm}{400}} &
    \vbox{\myfig{bior35_HH}{4cm}{400}} 
  \end{tabular}
  \caption{A demonstration of the shift-variance of the DWT.}
\label{fig:dwt_shift_variance}
\end{figure}

However, suprisingly, at it can be also seen in the
Fig.~\ref{fig:dwt_shift_variance}, when the circle has traveled two
pixels (frames 0 and 2), a perfect match is achieved! The reason why
the 1-pixel movement is generating different coefficients in the
reference and the predicted frames is because a 1-pixel motion cannot
be represented selecting always the same phase at the downsamplers
(remember that with the downsampler we are basically selecting only
one the two possible phases of the output of the analysis filters: the
even samples or the odd samples, see this notebook).

Lets suppose that
the downsampler discards the odd coefficients (let's refer them as
odd-phase coefficients).

In this case, the even-phase cofficients of
the reference frame are the same than the odd-phase coefficients of
the predicted frame (this can be seen in this notebook). Therefore, in
the 1D case, when the motion is ``even''-type (that is, a displacement
of a even number of samples) we should compensate the even-phase
coefficients of the reference and the predicted frame, while when the
motion is ``odd''-type we should compensate the odd-phase coefficients
of the predicted frame with a prediction generated with the even-phase
coefficients of the reference frame, or viceversa.

There are different alternatives for recovering the ``lost'' phase
during the DWT (in the 1D case):
\begin{enumerate}
\item MC-then-downsample: Perform first the MC stage directly over the
  output of the analysis filters, and then, selectively downsample the
  result. Notice that the downsampler should select the right phase,
  depending on the type of motion detected (``odd'' or ``even''). This
  information (the selected phase), should be available at the
  decoder, along with the motion fields.
\item Delay-then-DWT: Perform two identical DWTs, one to the original
  signal, and the other to a one-sample delayed signal (remember than
  a movement of one pixel will change the phase at the output of the
  DWT). Thus, the the DWT applied to the original signal will generate
  one of the phases and the DWT applied to the delayed signal will
  generate the other one.
\item CODWT: Use the current (single phase) L and H coefficients to
  compute the missing phase, using the CODWT (Complete-to-Overcomplete
  DWT)~\cite{andreopoulos2005complete} (a new type of DWT applied to
  the DWT coefficients).
\end{enumerate}
Each alternative has pros and cons. If the DWT has been implemented
using convolution, MC-then-downsample should be a fast
alternative. However, if the DWT uses Lifting, Multiple-DWT-then-MC
should be fast also, because only one phase is computed by the
DWT. These two options can be used with any DWT filters. On the other
hand, CODWT needs specific designs form each DWT filters. Notice that, in any case, the solution is reached after using the ODWT domain.

In the 2D case, and always working with only one level of the DWT, we
have up to four different phases: (even, even)-, (even, odd)-, (odd,
even)-, and (odd, odd)-phase coefficients. Thus, depending on the type
of motion detected, the corresponding phase should be selected.

\subsection{Near shift-invariance in the IDWT (Interpolated DWT) domain}
As it was commented before, the causant of the shift-variance in the
critically sampled DWT domain is the use of the downsamplers. At this
point we have basically two different alternatives:
\begin{enumerate}
\item Use the Algorithme \`a Trous (AaT)~\cite{mallat1999wavelet},
  which removes the downsamplers from the DWT, generating the so
  called Overcomplete DWT (ODWT). Notice that, because the
  downsamplers are removed, the aliasing artifacts produced by the
  downsamplers is also avoided.
\item Approximate the AaT coefficients by interpolating the DWT
  coefficients using the DWT synthesis filters. In this case, the
  aliasing is not avoided, but the shift-variance problem is
  reduced.
\end{enumerate}

\subsection{Subpixel accuracy}
Objects in real scenes usually move a rational number of pixels, and
therefore, even when the input frames seems to be the same,
numerically they aren't. To deal with this drawback, interpolation can
be used to increase the resolution of the frames (MC in the frame
domain) or the subbands (MC in the subband domain), performing thus a
MC with increased accuracy.

Interpolation and DWT are both linear operators, and therefore, are
interchangeable. This means that we can interpolate the input frames and work as if the motion where integer-pixel, or we can interpolate the DWT coefficients. In both options, the number of 

\section{What you have to do?}
  
Please, using this
\href{https://github.com/Sistemas-Multimedia/Sistemas-Multimedia.github.io/blob/master/study_guide/10-MC_in_DWT_domain/DWT_shift_variance.ipynb}{notebook},
research the posibilities for performing MC of other DWTs available at
\href{https://pywavelets.readthedocs.io/en/latest/}{PyWavelets}.

\section{Timming}

Please, finish this notebook before the next class session.

\section{Deliverables}

None.

\section{Resources}

\bibliography{python}
