%\newcommand{\SM}{\href{http://cms.ual.es/UAL/estudios/masteres/plandeestudios/asignaturas/asignatura/MASTER7114?idAss=71142105&idTit=7114}{Sistemas Multimedia}}
\newcommand{\SM}{\href{https://sistemas-multimedia.github.io/}{Sistemas Multimedia}}

\newcommand{\theproject}{\href{https://github.com/Sistemas-Multimedia/MRVC}{MRVC}}

\newcommand{\SW}{\href{https://github.com/Sistemas-Multimedia/MRVC}{MRVC}}

\title{\SM{} - Study Guide - Milestone 8: Motion Compensation in the DWT Domain?}

\maketitle

\section{Description}

\subsection{Removing the temporal redundancy through Motion Compensation (MC)}
The next level in the process of decorrelating the sequence of frames
is to remove the temporal redundancy by means of Motion Compensation
(MC). Basically, MC consist in substracting to the video data a
prediction performed with the information that is avaliable to the
decoder. If this prediction is accurate, the result of this operation
is a residual video with a lower temporal redundancy, that can be
compressed with a higher compression ratio.

\subsection{The lack of shift-invariance in the DWT domain}
In our case, the video data is represented in the MDWT domain. Let's
suppose that the number of levels of the DWT is 1, and therefore, each
frame has been decomposed into two 2D subbands $L$ and $H$ (remember
that using the notation introduces in the previous milestone, $H$ has
inside the three high-frequency subbands: $LH$, $HL$ and $HH$, and
that $L=LL$). So, after using the MDET, MC must be performed using the
DWT coefficients.

Unfortunately, as a consequence of the downsamplers used during the
DWT to achieve critical sampling and the aliasing between the
subbands, all DWT decompositions are shift variant. This can be seen
in the Fig.~\ref{fig:dwt_shift_variance} were some DWT coefficients of a
test video with two frames (that are ``empty'' images where a circle
has been moved one pixel to the left) has been shown.

\begin{figure}
  \centering
  \myfig{bior35_LL}{3cm}{300}
  %\begin{tabular}{ccc}
    % & \myfig{movement}{3cm}{300} &\\
  %  &\vbox{\myfig{movement}{3cm}{300}}&
                                         %    \end{tabular}
  \caption{hola}
\label{fig:dwt_shift_variance}
\end{figure}

Shift variance is also generated after the inverse transform when the
coefficients are filtered or quantized, because the aliasing between
the filters is not completely cancelled in this
case~\cite{bradley2003shift}.

\section{What you have to do?}
  
Please, complete this
\href{https://github.com/Sistemas-Multimedia/Sistemas-Multimedia.github.io/blob/master/study_guide/MDWT/MDWT.ipynb}{notebook}.

\section{Timming}

Please, finish this notebook before the next class session.

\section{Deliverables}

None.

\section{Resources}

\bibliography{python}
