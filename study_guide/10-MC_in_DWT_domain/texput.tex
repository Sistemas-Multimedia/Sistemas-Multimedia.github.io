%\newcommand{\SM}{\href{http://cms.ual.es/UAL/estudios/masteres/plandeestudios/asignaturas/asignatura/MASTER7114?idAss=71142105&idTit=7114}{Sistemas Multimedia}}
\newcommand{\SM}{\href{https://sistemas-multimedia.github.io/}{Sistemas Multimedia}}

\newcommand{\theproject}{\href{https://github.com/Sistemas-Multimedia/MRVC}{MRVC}}

\newcommand{\SW}{\href{https://github.com/Sistemas-Multimedia/MRVC}{MRVC}}

\title{\SM{} - Study Guide - Milestone 8: Motion Compensation in the DWT Domain}

\maketitle

\section{Description}

\subsection{Removing the temporal redundancy through Motion Compensation (MC)}
The next natural step in the process of decorrelating the sequence of
frames is to remove the temporal redundancy by means of Motion
Compensation (MC). Basically, MC consists in substracting to the video
data a prediction performed with the information that is avaliable to
the decoder. If this prediction is accurate, the result of this
operation is a residual video with a lower temporal redundancy, that
can be compressed with a higher compression ratio.

\subsection{The lack of shift-invariance in the DWT domain}
In our case, the video data is represented in the DWT domain, and
therefore, we need to perform the so called In-Band Motion Estimation
and Compensation~\cite{andreopoulos2005complete}). Let's suppose that
the number of levels of the DWT is 1, and therefore, each frame has
been decomposed into two 2D subbands $L$ and $H$ (remember that using
the notation introduces in the previous milestone, $H$ has inside the
three high-frequency subbands: $LH$, $HL$ and $HH$, and that
$L=LL$). So, after using the MDWT, MC must be performed using the DWT
coefficients.

Unfortunately, as a consequence of the downsamplers used during the
DWT to achieve critical sampling and the aliasing between the
subbands, DWT decompositions are shift-variant. This can be seen in
the Fig.~\ref{fig:dwt_shift_variance} were some DWT coefficients of a
test video with three frames (with a 1-pixel constant speed moving (to
the left) cicle``empty'') has been shown. As it can be seen, when the
circle has been moved only one pixel, the value of the coefficients
that correspond to the circunference of the circle are different
between the reference frame and the predicted frame. This makes quite
difficult to estimate the motion, and therefore, compensate
it. However, when the circle has traveled two pixels, a perfect match
is performed.

\begin{figure}
  \centering
  \begin{tabular}{ccc}
    \vbox{\pngfig{moving_circle_000}{4cm}{400}} &
    \vbox{\myfig{movement}{4cm}{400}} &
    \vbox{\pngfig{difference}{4cm}{400}} \\
    \vbox{\myfig{haar_LL}{4cm}{400}} &
    \vbox{\myfig{db5_LL}{4cm}{400}} &
    \vbox{\myfig{bior35_LL}{4cm}{400}} \\
    \vbox{\myfig{haar_LH}{4cm}{400}} &
    \vbox{\myfig{db5_LH}{4cm}{400}} &
    \vbox{\myfig{bior35_LH}{4cm}{400}} \\ 
    \vbox{\myfig{haar_HL}{4cm}{400}} &
    \vbox{\myfig{db5_HL}{4cm}{400}} &
    \vbox{\myfig{bior35_HL}{4cm}{400}} \\
    \vbox{\myfig{haar_HH}{4cm}{400}} &
    \vbox{\myfig{db5_HH}{4cm}{400}} &
    \vbox{\myfig{bior35_HH}{4cm}{400}} 
  \end{tabular}
  \caption{A demonstration of the shift-variance of the DWT.}
\label{fig:dwt_shift_variance}
\end{figure}

The reason why the 1-pixel movement is generating different
coefficients in the reference and the predicted frames is because a
1-pixel motion cannot be represented using always the same phase
(remember that with the downsampler we are basically selecting only
one the two possible phases of the output of the analysis
filters). Lets suppose that the downsampler ignores the even
coefficients (let's refer them as 0-phase coefficients). In this case,
the 0-phase cofficients of the reference frame are the same than the
1-phase coefficients of the predicted frame. This can be seen in this
notebook. Therefore, in the 1D case, when the motion is ``even'' we
should compensate the 0-phase coefficients of the reference and the
predicted frame, while when the motion is ``odd'' we should compensate
the 1-phase coefficients of the predicted frame with a prediction
generated with the 0-phase coefficients of the reference frame.

There are different alternatives for recovering the ``lost'' phase
during the DWT (in the 1D case):
\begin{enumerate}
\item Perform first the MC stage directly over the output of the
  analysis filters, and then, downsample the result. Notice that the
  downsampler should select the right phase, depending on the type of
  motion detected (``odd'' or ``even''). This information (the
  selected phase), should be available at the decoder, along with the
  motion fields.
\item Perform two identical DWTs, one to the original signal, and the
  other to a one-sample delayed signal. The output of the DWT applied
  to the original signal will generate one of the phases and the
  output of the DWT applied to the delayed signal will generate the
  other one.
\item Use the current (single phase) coefficients to compute the
  missing phase.
\end{enumerate}

In the 2D case, and always working with only one level of the DWT, we
have up to four different phase: (0,0)-phase, (0,1)-phase,
(1,0)-phase, and (1,1)-phase coefficients. When all the phases are considered, we say that we are using the Overcomplete DWt (ODWT).

As it was indicated before, one

Shift-variance is also generated after the inverse transform when the
coefficients are filtered or quantized, because the aliasing between
the filters is not completely cancelled in this
case~\cite{bradley2003shift}.

\subsection{Near shift-invariance in the IDWT (Interpolated DWT) domain}
As it was commented before, the causant of the shift-variance in the
critically sampled DWT domain is the use of the downsamplers. At this
point we have basically two different alternatives:
\begin{enumerate}
\item Use the Algorithme \`a Trous (AaT)~\cite{mallat1999wavelet}, which
  removes the downsamplers from the DWT. Notice that, because the
  downsamplers are removed, the aliasing artifacts produced by the
  downsamplers is also avoided.
\item Approximate the AaT coefficients by interpolating the DWT
  coefficients using the DWT synthesis filters. In this case, the
  aliasing is not avoided, but the shift-variance problem is
  reduced.
\end{enumerate}


\section{What you have to do?}
  
Please, using this
\href{https://github.com/Sistemas-Multimedia/Sistemas-Multimedia.github.io/blob/master/study_guide/10-MC_in_DWT_domain/DWT_shift_variance.ipynb}{notebook},
research the posibilities for performing MC of other DWTs available at
\href{https://pywavelets.readthedocs.io/en/latest/}{PyWavelets}.

\section{Timming}

Please, finish this notebook before the next class session.

\section{Deliverables}

None.

\section{Resources}

\bibliography{python}
