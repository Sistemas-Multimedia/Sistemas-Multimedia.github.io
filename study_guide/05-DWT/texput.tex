%\newcommand{\SM}{\href{http://cms.ual.es/UAL/estudios/masteres/plandeestudios/asignaturas/asignatura/MASTER7114?idAss=71142105&idTit=7114}{Sistemas Multimedia}}
\newcommand{\SM}{\href{https://sistemas-multimedia.github.io/}{Sistemas Multimedia}}

\newcommand{\theproject}{\href{https://github.com/Sistemas-Multimedia/MRVC}{MRVC}}

\newcommand{\SW}{\href{https://github.com/Sistemas-Multimedia/MRVC}{MRVC}}

\title{\SM{} - Study Guide - Milestone 5: Energy Concentration and Multiresolution with the Discrete Wavelet Transform}

\maketitle

\section{Description}

Technically, the
\href{https://en.wikipedia.org/wiki/Discrete_wavelet_transform}{DWT
  (Discrete Wavelet Transform)} is a
\href{https://en.wikipedia.org/wiki/Linearity}{linear}
\href{https://en.wikipedia.org/wiki/Change_of_basis}{basis}
\href{https://www.youtube.com/watch?v=P2LTAUO1TdA}{expansion} which
compute a
\href{https://www.dsprelated.com/freebooks/sasp/Critically_Sampled_Perfect_Reconstruction.html}{critically}-\href{https://en.wikipedia.org/wiki/Nyquist-Shannon_sampling_theorem}{sampled}
\href{https://en.wikipedia.org/wiki/Octave_(electronics)}{octave}-\href{https://en.wikipedia.org/wiki/Frequency_band}{band}
\href{https://www.sciencedirect.com/topics/engineering/wavelet-decomposition}{decomposition}~\cite{vetterli2014foundations,kovacevic2013fourier}.

\subsection{Transform coding}

\href{https://web.stanford.edu/class/ee398a/handouts/lectures/07-TransformCoding.pdf}{Transform
  coding} can exploit
\href{https://en.wikipedia.org/wiki/Correlation_and_dependence}{correlation}
in \href{https://en.wikipedia.org/wiki/Signal}{signals} to concentrate
the
\href{https://en.wikipedia.org/wiki/Energy_(signal_processing)}{energy}. Specifically,
the DWT is a family of transforms whose output describes the (input)
signal in a set of
\href{https://en.wikipedia.org/wiki/Sub-band_coding}{subbands}. This
set is also called a decomposition and the
\href{https://en.wikipedia.org/wiki/Array_data_structure#Element_identifier_and_addressing_formulas}{index}
of the subband is related to the
\href{https://en.wikipedia.org/wiki/Frequency}{frequency} of the
signal. For example, in the case of the
\href{https://en.wikipedia.org/wiki/Digital_image}{images}, the
position of the
\href{https://en.wikipedia.org/wiki/Coefficient}{coefficients} in the
subbands is related to
\href{https://en.wikipedia.org/wiki/Discrete_wavelet_transform#/media/File:Jpeg2000_2-level_wavelet_transform-lichtenstein.png}{the
  spatial area where the related pixels are found}.

All linear\footnote{Non-linear transform are also possible, but their
  mathematical operation can not be described using linear algebra.}
transforms can be described by
\href{https://en.wikipedia.org/wiki/Matrix_multiplication}{matrix-vector
  product}~\cite{strang4linear} as
\begin{equation}
  y = Kx,
  \label{eq:forward_transform_matrix_form}
\end{equation}
where $x$ is the input signal, $K$ is the analysis transform matrix,
and $y$ is the output decomposition. Notice that
\begin{equation}
  y_i = \langle K_i, x_i\rangle,
\end{equation}
where $K_i$ is the $i$-th row of $K$, and $\langle\cdot,\cdot\rangle$
denotes the
\href{https://mathworld.wolfram.com/InnerProduct.html}{inner
  product}. This basically means that $y_i$ is proportional to the
\href{https://en.wikipedia.org/wiki/Similarity_(geometry)}{similarity}
between the input signal $x$ and the
\href{https://en.wikipedia.org/wiki/Finite_impulse_response}{taps} of
the \href{https://en.wikipedia.org/wiki/Digital_filter}{filter}
$K_i$. These
\href{https://cseweb.ucsd.edu/classes/fa17/cse166-a/lec13.pdf}{slides}
can help you with this key idea. The inverse (synthesis) transform is
computed by
\begin{equation}
  x = K^{-1}y,
  \label{eq:backward_transform_matrix_form}
\end{equation}
where $K^{-1}$ denotes to the inverse matrix of $K$.

For example, the
2x2-\href{https://en.wikipedia.org/wiki/Karhunen-Loeve_theorem}{KLT}~\cite{sayood2017introduction}
is defined by
\begin{equation}
  \begin{bmatrix}
    y_0 \\
    y_1
  \end{bmatrix}
  = 
  \begin{bmatrix} \frac{1}{\sqrt{2}} & \frac{1}{\sqrt{2}} \\ \frac{1}{\sqrt{2}} & \frac{-1}{\sqrt{2}} \end{bmatrix}
  \begin{bmatrix}
    x_0 \\
    x_1
  \end{bmatrix},
  \label{eq:KLT_transform}
\end{equation}
and it holds that
\begin{equation}
  K=K^{-1}=K^{\text T},
  \label{eq:orthogonal_matrix}
\end{equation}
where $K^{\text T}$ represents the transpose matrix of $K$. The
Eq.~\ref{eq:orthogonal_matrix} is true for all
\href{https://en.wikipedia.org/wiki/Orthogonality}{orthogonal}
transforms, and therefore
\begin{equation}
  \langle K_i, K_j\rangle = 0, \forall i\neq j.
\end{equation}

\subsection{An example of DWT: the Haar DWT}
As the rest of transforms, the DWT is also described by the filters of
$K$. Haar defined the analysis
\href{https://en.wikipedia.org/wiki/Downsampling_(signal_processing)}{downsampling}
filters
\begin{equation}
  \begin{bmatrix}
    l^{s+1}_n \\
    h^{s+1}_n
  \end{bmatrix}
  = 
  \begin{bmatrix} \frac{1}{\sqrt{2}} & \frac{1}{\sqrt{2}} \\ \frac{1}{\sqrt{2}} & \frac{-1}{\sqrt{2}} \end{bmatrix}
  \begin{bmatrix}
    l^s_{2n} \\
    l^s_{2n+1}
  \end{bmatrix},
  \label{eq:Haar_transform}
\end{equation}
where the superindex $s$ denotes the subband, and $n$ refers to the
$n$-th element of the signal. By definition (and notice that this
holds for all DWTs),
\begin{equation}
  l^0=x.
\end{equation}

\begin{figure}
  \centering
  \myfig{haar_modulus}{6cm}{600}
  \caption{Haar filters's response in the frequency domain (see this
    \href{https://github.com/Sistemas-Multimedia/Sistemas-Multimedia.github.io/blob/master/study_guide/05-DWT/dwt_filters_analysis.ipynb}{notebook}).
    $|K_i(e^{j\omega})|$ denotes the
    \href{https://en.wikipedia.org/wiki/Absolute_value}{modulus} of
    the \href{https://en.wikipedia.org/wiki/Fourier_transform}{Fourier
      transform} of $K_i$.}
  \label{fig:Haar_modulus}
\end{figure}

\begin{figure}
  \centering
  \myfig{db5_modulus}{6cm}{600}
  \caption{Daubechies 5 filters's response in the frequency domain
    (see this
    \href{https://github.com/Sistemas-Multimedia/Sistemas-Multimedia.github.io/blob/master/study_guide/05-DWT/dwt_filters_analysis.ipynb}{notebook}).}
  \label{fig:db5_modulus}
\end{figure}

\begin{figure}
  \centering
  \myfig{bior3.5_modulus}{6cm}{600}
  \caption{Biorthiogonal 3.5 filters's response in the frequency
    domain (see this
    \href{https://github.com/Sistemas-Multimedia/Sistemas-Multimedia.github.io/blob/master/study_guide/05-DWT/dwt_filters_analysis.ipynb}{notebook}).}
  \label{fig:bior3.5_modulus}
\end{figure}

\begin{figure}
  \centering
  \myfig{haar_phase}{6cm}{600}
  \caption{Phase response of the Haar filters (see this
    \href{https://github.com/Sistemas-Multimedia/Sistemas-Multimedia.github.io/blob/master/study_guide/05-DWT/dwt_filters_analysis.ipynb}{notebook}).}
  \label{fig:haar_phase}
\end{figure}

\begin{figure}
  \centering \myfig{db5_phase}{6cm}{600}
  \caption{Phase response of the Daubechies 5 filters (see this
    \href{https://github.com/Sistemas-Multimedia/Sistemas-Multimedia.github.io/blob/master/study_guide/05-DWT/dwt_filters_analysis.ipynb}{notebook}).}
  \label{fig:db5_phase}
\end{figure}

As it can be seen in the Fig.~\ref{fig:Haar_modulus}, $K_0$ is a
\href{https://en.wikipedia.org/wiki/Low-pass_filter}{low-pass filter}
and $K_1$ is a
\href{https://en.wikipedia.org/wiki/High-pass_filter}{high-pass
  filter} (this holds for all DWTs). Considering the Haar filter, we
can conclude that:
\begin{enumerate}
\item There exists
  \href{https://en.wikipedia.org/wiki/Aliasing}{aliasing} between the
  filters (this holds for all DWTs) and this is a drawback because:
  \begin{enumerate}
  \item The
    \href{https://en.wikipedia.org/wiki/Information}{information} with
    the same frequency can be found in both subbands ($l$ and
    $h$). Therefore, the concentration of the energy in one of the
    subbands is smaller.
  \item At it can be seen also in Eq.~\ref{eq:Haar_transform}, the
    subbands are downsampled by 2 and both subbands should have, at
    most, a
    \href{https://en.wikipedia.org/wiki/Bandwidth_(signal_processing)}{bandwidth}
    of $1/2$
    \href{https://en.wikipedia.org/wiki/Radian}{radians}/\href{https://en.wikipedia.org/wiki/Sampling_(signal_processing)}{sample}
    in order to avoid the aliasing during the
    \href{https://en.wikipedia.org/wiki/Downsampling_(signal_processing)}{subsampling},
    and increasing thus the
    \href{https://en.wikipedia.org/wiki/Perception}{perceptible}
    \href{https://en.wikipedia.org/wiki/Signal-to-noise_ratio}{quality}
    of the low-pass subband.
  \end{enumerate}
  These problems can be only solved using filters that have
  \href{https://en.wikipedia.org/wiki/Transfer_function}{transfer
    functions} that overlaps a smaller area (see
  Figs.~\ref{fig:db5_modulus} and
  \ref{fig:bior3.5_modulus}).
%\item The response of the filter bank is flat, which means that the gain of the different frequencie
%  $|y(e^{j\omega})|=a|x(e^{j\omega})|, a\in\mathbb{R}$ (the
%  reconstructed signal $y$ has not been filtered). % OJO
\item There is not
  \href{https://en.wikipedia.org/wiki/Linear_phase}{phase distortion}
  (the phase of the filters is a linear function of the frequency, see
  the Fig.~\ref{fig:haar_phase}). This means that coefficients
  $l^{s+1}_n$ and $h^{s+1}_n$ refers to the section of signal
  $\{l^s_{2n}, l^s_{2n+1}\}$, allowing the design of entropy codecs
  for the decomposition which exploit the correlation between
  subbands. Notice that this does not hold for all filters (see
  Fig.~\ref{fig:db5_phase}).
\end{enumerate}

% Polyphase implementation of the multilevel DWT
\begin{figure}
  \centering
  \myfig{graphics/DWT}{10cm}{1000}
  \caption{Computation of the $S$-levels DWT and the generated
    subbands.}
  \label{fig:DWT}
\end{figure}

Eq.~\ref{eq:Haar_transform} computes the 1-levels Haar DWT. A more general expresion for this equation is (and this holds for all DWTs)
\begin{equation}
  \{l^{s+1}, h^{s+1}\} = \text{DWT}(l^s),
  \label{eq:DWT}
\end{equation}
where $\{\cdot,\cdot\}$ denotes the concatenation of subbands. As can
be seen in the Fig.~\ref{fig:DWT}, it's possible to compute the
$S$-levels DWT, using Eq.~\ref{eq:DWT} iteratively over the low-frequency subband, generating the decomposition
\begin{equation}
  \{l^S_0, h^S_0, h^{S-1}_0 h^{S-1}_1, h^{S-2}_0 h^{S-2}_1 h^{S-2}_2
  h^{S-2}_3, \cdots, h^1_0 h^1_1 \cdots
  h^1_{2^{n-1}-1}\}=\text{DWT}^S(l^0),
  \label{eq:S_levels_DWT}
\end{equation}
where
\begin{equation}
  n = \log_2(N)
\end{equation}
where $N$ is the number of samples.

The $S$-levels inverse DWT is defined by (this holds for all DWTs) by the iteration of the $1$ levels inverse DWT found by
\begin{equation}
  l^s = \text{DWT}^{-1}(\{l^{s+1}, h^{s+1}\}),
\end{equation}
where, it can be seen in the case of the Haar DWT, $\text{DWT}^{-1}$
is the result of solving $l^{s+1}$ and $h^{s+1}$ in the
Eq.~\ref{eq:Haar_transform}.

Notice that in the decomposition described in
Eq.~\ref{eq:S_levels_DWT} there are $S+1$ subbands, and therefore,
$S+1$ resolution levels are posible:
\begin{description}
\item 1. $l^S$, with only one sample.
\item 2. $l^{S-1}=\text{DWT}^{-1}(\{l^S, h^S\})$, with two samples.
\item 3. $l^{S-2}=\text{DWT}^{-1}(\{l^{S-1}, h^{S-1}\})$, with four
  samples.
\item $\vdots$.
\item $S+1$. $l^0=\text{DWT}^{-1}(\{l^1, h^1\})$.
\end{description}  

\subsection{Orthogonal and biorthogonal DWT}
In the DWT context, orthogonality provides intersubband decorrelation,
which basically means that the contribution of each subbands to the
reconstruction (this holds for all DWTs) of the signal are
independent. Using a similar notation that the followed by PyWavelets,
all transforms are defined by 4 filters:
\begin{enumerate}
\item Analysis (downsampling) low-pass filter $\tilde\phi$ (the decomposition
  scaling function), which computes $l^{s+1}$ from $l^s$.
\item Analysis (downsampling) high-pass filter $\tilde\psi$ (the decomposition
  wavelet function), which computes $h^{s+1}$ frin $l^s$.
\item Synthesis (upsampling) high-pass filter $\phi$ (the
  reconstruction scaling function), which computes $\{l^s_{2n}\}$ from
  $l^{s+1}$ and $h^{s+1}$.
\item Synthesis (upsampling) high-pass filter $\psi$ (the
  reconstruection wavelet function), which computes $\{l^s_{2n+1}\}$
  from $l^{s+1}$ and $h^{s+1}$.
\end{enumerate}

Orthogonal filters can be recognized because:
\begin{enumerate}
\item $\tilde\phi\bot\tilde\psi$ and $\phi\bot\psi$.
\item Their taps are all different (asymmetric).
\item The synthesis filters can be determined simply by
  \emph{reflecting} analysis filters.
\item In general, except for the Haar filters, orthogonal filters
  generate phase distortion.
\end{enumerate}

On the other hand, in a biorthogonal DWT:
\begin{enumerate}
\item $\tilde\phi\bot\tilde\psi$, $\phi\bot\psi$, $\psi\bot\tilde\phi$
  and $\tilde\psi\bot\phi$.
\item All the filters are symmetric.
\item $\psi=(-1)^n\tilde\phi; n\in\mathbb{N}$, and
  $\phi=(-1)^n; n\in\mathbb{N}$ (the odd taps are multiplied by $-1$).
\item No phase distortion.
\end{enumerate}

\subsection{Multiresolution}

\section{What you have to do?}
  
Please, using \href{}{this notebook}, add a new curve to the Fig.~\ref{}.

\section{Timming}

Please, finish this notebook before the next class session.

\section{Deliverables}

None.

\section{Resources}

\bibliography{signal-processing,maths,data-compression}
