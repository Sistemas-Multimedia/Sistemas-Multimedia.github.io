%\newcommand{\SM}{\href{http://cms.ual.es/UAL/estudios/masteres/plandeestudios/asignaturas/asignatura/MASTER7114?idAss=71142105&idTit=7114}{Sistemas Multimedia}}
\newcommand{\SM}{\href{https://sistemas-multimedia.github.io/}{Sistemas Multimedia}}

\newcommand{\theproject}{\href{https://github.com/Sistemas-Multimedia/MRVC}{MRVC}}

\newcommand{\SW}{\href{https://github.com/Sistemas-Multimedia/MRVC}{MRVC}}

\title{\SM - Study Guide}

\maketitle

\section{Methodology}
SM (Sistemas Multimedia) follows the PBL (Project-Based Learning)
methodology. The students, helped by the teacher, develop a project
during the course. This project is
\href{https://github.com/Sistemas-Multimedia/MRVC}{MRVC}, a video
compressor. In the
\href{http://portafirma.ual.es/pfirma/downloadReport/file?idDocument=Y3iGKKA5Lb&idRequest=vo8X8SLOXQ}{Teaching
  Guide} you can find the details about the evaluation of the subject.

The project is described as a sequence of milestones (or topics), each
one with the following structure:
\begin{enumerate}
\item A description of the milestone. This part is worked in class.
\item A \emph{What you have to do?} section, which describes the
  virtualized job of the students.
\item How much time you should spend working in the previous
  virtualized section.
\item The deliverables that you should generate and send to the
  teacher.
\item A list of bibliographic references.
\end{enumerate}

\section{Milestones}
A list of the current milestones is available in the
\href{https://sistemas-multimedia.github.io/syllabus}{syllabus} of the
subject.

\begin{comment}
\begin{enumerate}
\item \href{https://sistemas-multimedia.github.io/study_guide/01-provisioning/}{OS (Operating System) Provisioning}.
\item \href{https://sistemas-multimedia.github.io/study_guide/02-python/}{Installation and basic programming with Python}.
\item \href{https://sistemas-multimedia.github.io/study_guide/03-git/}{Git, GitHub and the Fork-and-Branch Git Workflow}.
\item \href{https://sistemas-multimedia.github.io/study_guide/04-the_data/}{Understanding the Video Data}.
\item \href{https://sistemas-multimedia.github.io/study_guide/05-quantization/}{Quantizing in the RGB Domain}.
\item \href{https://sistemas-multimedia.github.io/study_guide/06-color_transform/}{Removing Redundancy with a Color Transform}.
\item \href{https://sistemas-multimedia.github.io/study_guide/07-DWT/}{Energy Concentration and Spatial Multiresolution with the Discrete Wavelet Transform}.
\item \href{https://sistemas-multimedia.github.io/study_guide/08-2D-DWT/}{Removing Redundancy with the 2D-DWT}.
\item \href{https://sistemas-multimedia.github.io/study_guide/09-MDWT/}{The Motion Discrete Wavelet Transform}.
\item \href{https://sistemas-multimedia.github.io/study_guide/10-ME/}{Motion Estimation}.
\item \href{https://sistemas-multimedia.github.io/study_guide/11-MC_in_DWT_domain/}{Motion Compensation in the DWT Domain}.
\item \href{https://sistemas-multimedia.github.io/study_guide/12-IPP_coding/}{IPP... Coding in MRVC}.
  % Using JPEG2000 instead of PNG.
  % Exploiting the color redundancy of the HVS.
%\item \href{https://sistemas-multimedia.github.io/study_guide/08-MC_in_DWT_domain/}{Motion Compensation in the DWT Domain}.
%\item \href{https://sistemas-multimedia.github.io/study_guide/09-LPT/}{Laplacian Pyramid Transform (LPT)}.
%\item \href{https://sistemas-multimedia.github.io/study_guide/10-CS-LPT/}{Critically Sampled LPT (CS-LPT)}.
%\item \href{https://sistemas-multimedia.github.io/study_guide/11-OF/}{Motion Estimation with Optical Flow}.
%\item \href{https://sistemas-multimedia.github.io/study_guide/12-MC-CS-LPT/}{Motion Compensated CS-LPT (MC-CS-LPT)}.
%\item \href{https://sistemas-multimedia.github.io/study_guide/14-coding/}{MC-CS-LPT Subband Coding}.
\end{enumerate}
\end{comment}

