%\newcommand{\SM}{\href{http://cms.ual.es/UAL/estudios/masteres/plandeestudios/asignaturas/asignatura/MASTER7114?idAss=71142105&idTit=7114}{Sistemas Multimedia}}
\newcommand{\SM}{\href{https://sistemas-multimedia.github.io/}{Sistemas Multimedia}}

\newcommand{\theproject}{\href{https://github.com/Sistemas-Multimedia/MRVC}{MRVC}}

\newcommand{\SW}{\href{https://github.com/Sistemas-Multimedia/MRVC}{MRVC}}

\title{\SM{} - Study Guide - Milestone 6: Computing the 2D-DWT}

\maketitle

\section{Description}

(Digital) Images are 2D (discrete in space and amplitude) signals. The
2D-DWT of an image can be computed using (1) separable 1D filters, and
(2) nonseparable 2D filters. Except in very special cases, all 2D-DWT
implementations use separable filters.

We define
\begin{equation}
  \{ll^{s+1}, lh^{s+1}, hl^{s+1}, hh^{s+1}\} = \text{2D-DWT}(ll^s),
  \label{eq:2D-DWT}
\end{equation}
where $ll^s$ is a 2D-signal, and
\begin{equation}
  \{\{\{l^{s+1}\}_i, \{h^{s+1}\}\}_i\} = \{\text{DWT}(ll^s_i)\},
  \label{eq:DWT}
\end{equation}
is the result of using the 1D-DWT to each row of $ll^s$, generating
a sequence of low-pass subbands $\{l^{s+1}\}$ and sequence of
high-pass subbands $\{l^{s+1}\}$, and
\begin{equation}
   \text{2D-DWT}(ll^s) = \{\text{DWT}(\{\{\{l^{s+1}\}_i, \{h^{s+1}\}\}_i\})\},
  \label{eq:DWT}
\end{equation}
is the result of using the 1D-DWT to each column of the
$\{\{\{l^{s+1}\}_i, \{h^{s+1}\}\}_i\}$ structure.

\section{What you have to do?}
  
Please, using \href{}{this notebook}, add a new curve to the Fig.~\ref{}.

\section{Timming}

Please, finish this notebook before the next class session.

\section{Deliverables}

None.

\section{Resources}

\bibliography{python}
