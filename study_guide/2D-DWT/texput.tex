\newcommand{\SM}{\href{http://cms.ual.es/UAL/estudios/masteres/plandeestudios/asignaturas/asignatura/MASTER7114?idAss=71142105&idTit=7114}{Sistemas Multimedia}}

\newcommand{\theproject}{\href{}}{MCDWT}
\title{\SM{} - Study Guide - Milestone 6: Computing the 2D-DWT}

\maketitle

\section{Description}

(Digital) Images are 2D (discrete in space and amplitude) signals. The
2D-DWT of an image can be computed using (1) separable 1D filters, and
(2) nonseparable 2D filters. Except in very special cases, all 2D-DWT
implementations use separable filters.

Separability in the DWT context means that we can compute the
$D$-dimensional DWT using the 1D filters, simply by applying them to
each dimension, using \emph{in-place} operations. This procedure has
been described in the Algorithm~\ref{alg:2D-DWT}, where $X_{n,*}$
refeers to the $n$-th row of the matrix $X$ and $X_{*,m}$ to the
$m$-th column.

\begin{pseudocode}{\text{2D-DWT}}{X}
  \label{alg:2D-DWT}
  \FOR n \GETS 0 \TO N-1 \DO
  X_{n,*} = \text{DWT}(X_{n,*})
  \\
  \FOR m \GETS 0 \TO M-1 \DO
  X_{*,m} = \text{DWT}(X_{*,m})
\end{pseudocode}

The obtained 2D decomposition is expressed in the Eq.~\ref{eq:2D-DWT}.

Notice that this algorithm computes the $1$-levels
2D-DWT. The obtained the 2D decomposition shown in the
Figure~\ref{fig:2D-DWT}.

\begin{equation}
  \begin{bmatrix}
    \text{LL} & \text{HL} \\
    \text{LH} & \text{HH}
  \end{bmatrix}
  =
  \text{2D-DWT}(X)
  \label{eq:2D-DWT}
\end{equation}

Eq.~\ref{eq:2D-DWT} describes the $1$-levels 2D-DWT. Replacing
$\text{LL}=L^{s+1}$, $X=L^s$, and
$\{\text{HL}, \text{LH}, \text{HH}\}=H^{s+1}$ in the previous
expression, the $S$-levels 2D-DWT can be computed applying recursively
the 2D-DWT to the low-frequency subband as
\begin{equation}
  \{L^{s+1}, H^{s+1}\} = \text{2D-DWT}(L^s).
\end{equation}

The Figure~\ref{fig:lena-DWT} shows the $3$-levels 2D-DWT of an image.

\section{What you have to do?}
  
Please, using \href{}{this notebook}, add a new curve to the Fig.~\ref{}.

\section{Timming}

Please, finish this notebook before the next class session.

\section{Deliverables}

None.

\section{Resources}

\bibliography{python}
