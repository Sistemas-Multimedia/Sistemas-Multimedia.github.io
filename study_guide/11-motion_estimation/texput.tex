\newcommand{\SM}{\href{http://cms.ual.es/UAL/estudios/masteres/plandeestudios/asignaturas/asignatura/MASTER7114?idAss=71142105&idTit=7114}{Sistemas Multimedia}}

\newcommand{\theproject}{\href{}}{MCDWT}
\title{\SM{} - Study Guide - Milestone 11: Motion Estimation}

\maketitle

\section{Description}

Temporal correlation between video frames can be removed by MC (Motion
Compensation). MC implies to substract to the frames those
information, in form of a prediction frame $\hat{P}$, that can be
infered from neighbor frames, as long as these frames are known by the
decoder or the motion information is transmited from the encoder to
the decoder.

A MC Predictor (MCP) inputs one or more reference frames $R$, a
predicted frame $P$ (the frame that we want to compensate), and a
motion vectors field $\overset{\rightarrow}{V}$. In the last
milestone we saw how to find $\hat{P}$ and how to use it. In this
milestone we show how to compute $\overset{\rightarrow}{V}$.

Let's see two basic techniques to estimate the motion between 2
frames, $R$ and $P$. In this discussion it will be supposed that the
motion of the objects that are in both frames is bounded, and that the
luminance, at most, only varies smoothly between adjacent frames.

\subsection{Disjoint block matching}
$P$ can be divided, for example, in blocks of 16x16 pixels, and we can
apply some distortion metric (such as the MSE) to compute the distance
between each block of $P$ and its surrounding blocks in $R$. For each
block a motion vector indicating the best match (smaller distance) is
computed.

\subsection{Overlaped block matching}
This technique is similar to the previous one, except in which the
blocks can be overlaped. The number of motion vectors is higher, which
in general can model better the motion. The overlaped pixels are
averaged.

\subsection{(Dense) Optical flow}
Basically, the optical flow try to establish connections between the
pixels of frames $P$ and $R$ supposing:
\begin{enumerate}
\item That the pixel will have the same luminance.
\item That adjacent pixels follow parallel trajectories.
\end{enumerate}


\section{What you have to do?}

Please, implement Eq.~\ref{eq:output}, preferiblely in a Jupyter
notebook. Verify also Eq.~\ref{eq:crs}.

\section{Timming}

In groups, you will present the results for this milestone during the
examination time.

\section{Deliverables}

None.

\section{Resources}

\bibliography{image-pyramids,DWT,motion-estimation,HEVC}


%%%%%%%%%%%%%%%%%%%%%%%%%%%%%%%%%%%%%%%%%%%%%%%%%%%%%%%%%%%%%%%%

\begin{comment}

\subsection{Using (also) the lost phase in IBMC}
There are different alternatives for recovering the ``lost'' phases
(remember that we have two subbands, and two downsamplers) during the
DWT (in the 1D case). The result of this procedure is know as the
Overcomplete DWT (ODWT). Notice, however, that we are not interested
in encoding the ODWT domain, but in encoding the DWT that is more
compact. To achieve this, at least the following techniques can be
used:

\subsection{Performing IBMC in the 1-levels MDWT domain}
Once that the missing phases have been recovered, the MC procedure
between two frames (the reference frame $R$ and the predicted frame
$P$) that we are going to implement is:
\begin{enumerate}
\item  Therefore, estimate the motion between the overcomplete
  low-frequency subband of the reference frame $[R.L]$ and the
  overcomplete low-frequency subband of the predicted frame
  $[P.L]$. The output of this step is a motion vectors field
  $\overrightarrow{V}$, that describes how to project the $[R.L]$ onto
  $[P.L]$. Notice that $\overrightarrow{V}$ should be also a good
  candidate for mapping $R$ onto $P$.
  
\item Use $\overrightarrow{V}$ and $[R.H]$ (notice that in 2D case,
  $[R.H]=\{[R.HL], [R.LH], [R.HH]\}$) to generate a prediction
  $[\hat{P}.L]$. We define the prediction error
  in the overcomplete low-frequency subband as
  \begin{equation}
    [E.L] = [P.L] - [\hat{P}.L],
    \label{eq:prediction_error_L}
  \end{equation}
  and the prediction error in the overcomplete high-frequency subband
  as
  \begin{equation}
    [E.H] = [P.H] - [\hat{P}.H].
    \label{eq:prediction_error}
  \end{equation}

\item Selectively subsample $[E.H]$, picking out the right phase.
\end{enumerate}  


  Perform first the MC stage directly over the
  output of the analysis filters, and then, selectively downsample the
  result. Notice that the downsampler should select the right phase,
  depending on the type of motion detected (``odd'' or ``even''). This
  information (the selected phase), should be available at the
  decoder, along with the motion fields.
\item Delay-then-DWT: Perform two identical DWTs, one to the original
  signal, and the other to a one-sample delayed signal (remember than
  a movement of one pixel will change the phase at the output of the
  DWT). Thus, the the DWT applied to the original signal will generate
  one of the phases and the DWT applied to the delayed signal will
  generate the other one.
\item CODWT: Use the current (single phase) L and H coefficients to
  compute the missing phase, using the CODWT (Complete-to-Overcomplete
  DWT)~\cite{andreopoulos2005complete} (a new type of DWT applied to
  the DWT coefficients).
\end{enumerate}
Each alternative has pros and cons. If the DWT has been implemented
using convolution, MC-then-downsample should be a fast
alternative. However, if the DWT uses Lifting, Multiple-DWT-then-MC
should be fast also, because only one phase is computed by the
DWT. These two options can be used with any DWT filters. On the other
hand, CODWT needs specific designs form each DWT filters. Notice that, in any case, the solution is reached after using the ODWT domain.

---

is because a 1-pixel motion cannot
be represented selecting always the same phase at the downsamplers
(remember that with the downsampler we are basically selecting only
one the two possible phases of the output of the analysis filters: the
even samples or the odd samples, see this notebook).

Lets suppose that
the downsampler discards the odd coefficients (let's refer them as
odd-phase coefficients).

In this case, the even-phase cofficients of
the reference frame are the same than the odd-phase coefficients of
the predicted frame (this can be seen in this notebook). Therefore, in
the 1D case, when the motion is ``even''-type (that is, a displacement
of a even number of samples) we should compensate the even-phase
coefficients of the reference and the predicted frame, while when the
motion is ``odd''-type we should compensate the odd-phase coefficients
of the predicted frame with a prediction generated with the even-phase
coefficients of the reference frame, or viceversa.


\subsection{Near shift-invariance in the IDWT (Interpolated DWT) domain}
As it was commented before, the causant of the shift-variance in the
critically sampled DWT domain is the use of the downsamplers. At this
point we have basically two different alternatives:
\begin{enumerate}
\item Use the Algorithme \`a Trous (AaT)~\cite{mallat1999wavelet},
  which removes the downsamplers from the DWT, generating the so
  called Overcomplete DWT (ODWT). Notice that, because the
  downsamplers are removed, the aliasing artifacts produced by the
  downsamplers is also avoided.
\item Approximate the AaT coefficients by interpolating the DWT
  coefficients using the DWT synthesis filters. In this case, the
  aliasing is not avoided, but the shift-variance problem is
  reduced.
\end{enumerate}

\subsection{Subpixel accuracy}
Objects in real scenes usually move a rational number of pixels, and
therefore, even when the input frames seems to be the same,
numerically they aren't. To deal with this drawback, interpolation can
be used to increase the resolution of the frames (MC in the frame
domain) or the subbands (MC in the subband domain), performing thus a
MC with increased accuracy.

Interpolation and DWT are both linear operators, and therefore, are
interchangeable. This means that we can interpolate the input frames and work as if the motion where integer-pixel, or we can interpolate the DWT coefficients. In both options, the number of 

\end{comment}
