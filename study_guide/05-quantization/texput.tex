%\newcommand{\SM}{\href{http://cms.ual.es/UAL/estudios/masteres/plandeestudios/asignaturas/asignatura/MASTER7114?idAss=71142105&idTit=7114}{Sistemas Multimedia}}
\newcommand{\SM}{\href{https://sistemas-multimedia.github.io/}{Sistemas Multimedia}}

\newcommand{\theproject}{\href{https://github.com/Sistemas-Multimedia/MRVC}{MRVC}}

\newcommand{\SW}{\href{https://github.com/Sistemas-Multimedia/MRVC}{MRVC}}

\title{\SM{} - Study Guide - Milestone 5: Quantizing the RGB frames}

\maketitle

\section{Description}
The \href{https://en.wikipedia.org/wiki/Visual_system}{humans} are
quite efficient recognizing the information stored in images (and
frames of a video), even when this information has been degraded or
(partially)
lost. \href{https://en.wikipedia.org/wiki/Quantization_(signal_processing)}{Quantization}~\cite{sayood2017introduction,vetterli2014foundations}
is a technique that can remove these part of the visual information
less interesting for us.

In this milestone we are going to quantize a frame in
the \href{https://en.wikipedia.org/wiki/RGB_color_model}{RGB color
domain}. A \href{https://en.wikipedia.org/wiki/Quantization_(signal_processing)}{quantizer}
(typically denoted by $Q$) is
a \href{https://en.wikipedia.org/wiki/Data_compression}{encoding
system} that inputs a sequence of (digital) samples $x$ and output a
sequence of quantization indexes $k$ (see the Fig.~\ref{fig:Q}). The
inverse system, called a dequantizer (and denoted by $Q^{-1}$),
recover an approximate version (depending on the quantization step
$\Delta$) of $x$ that we will denote as $\tilde{x}$. Notice that
although $\Delta$ is an input parameter to the quantizer, it has not
been considered in the figure to keep it simpler. We define the
quantization error
\begin{equation}
  e_i = x_i - \tilde{x}_i,
\end{equation}
that obviously, should be minimized.

\begin{figure}
  \centering
  \myfig{graphics/Q}{2cm}{200}
  \caption{Scalar quantization and dequantization of a signal.}
  \label{fig:Q}
\end{figure}

Quantizers can be classifies into:
\begin{enumerate}
\item \textbf{Scalar quantizers}: those that produce a quantization
  index $k_i$ by each input sample $x_i$.
\item \textbf{Vector quantizers}: that process the input by
  blocks of samples, called vectors, producing quantization index by
  vector. Vector Quantization (VQ) can remove spatial correlation and
  therefore, are more efficient that Scalar Quantization (SQ). On the
  other hand, the computational requirements of VQ are, by far, higher
  than the needed for SQ. If we also consider that there are other
  techniques (such as transform coding) that are able to decorrelate
  the samples with less computation than VQ. For this reason, SQ has
  been selected, for example, in most video codecs.
\end{enumerate}

Quantizers, also, can be classified as:
\begin{enumerate}
\item \textbf{Uniform quantizers}: those in which $\Delta$ is
  idependent on the amplitude of the samples.
\item \textbf{Non-unifom quantizers}, on the contrary, when $\Delta$
  depends (directly or indirectly) on the samples.
\end{enumerate}

Non-uniform quantizers can be:
\begin{enumerate}
\item \textbf{Static quantizers}: if $\Delta$ depends on the amplitude
  of the samples (as happens for example in compended quantizers), but
  this dependency is pre-computed.
\item \textbf{Adaptive quantizers}: when $\Delta$ is adapted to
  minimize the quantization error, depending on the ``on-the-fly''
  characteristics of the signal.
\end{enumerate}

All uantizers can be classified into:
\begin{enumerate}
\item \textbf{Mid-tread quantizers}, if $\tilde{x}_i$ can be $0$.
\item \textbf{Mid-rised quantizers}, if $\tilde{x}_i$ cannot be $0$.
\end{enumerate}

Finally, quantizers can be classified as:
\begin{enumerate}
\item \textbf{Dead-zone quantizers}, that are characterized by a
  quantization step of length $2\Delta$ for $x_i=0$. Deadzone
  quantizers tends to remove the electronic noise (that usually has an
  small amplitude compared to the input signal $x$). Notice that
  dead-zone quantizers should not be consdered uniform, and that all
  dead-zone quantizers, by definition, are mid-tread.
\item \textbf{No dead-zone quantizers}, when the dead-zone does not
  exist.
\end{enumerate}

\begin{figure}
  \centering
  \myfig{iomap_mr}{5cm}{500}
  \caption{Input/output map of a mid-riser quantizer with $\Delta=2$.}
  \label{fig:iomap_mr}
\end{figure}

\begin{figure}
  \centering
  \myfig{qe_mr}{5cm}{500}
  \caption{Quantization error of a mid-riser quantizer with $\Delta=2$.}
  \label{fig:qe_mr}
\end{figure}

\begin{figure}
  \centering
  \myfig{iomap_mt}{5cm}{500}
  \caption{Input/output map of a mid-tread quantizer with $\Delta=2$.}
  \label{fig:iomap_mt}
\end{figure}

\begin{figure}
  \centering
  \myfig{qe_mt}{5cm}{500}
  \caption{Quantization error of a mid-tread quantizer with $\Delta=2$.}
  \label{fig:qe_mt}
\end{figure}

\begin{figure}
  \centering
  \myfig{iomap_dz}{5cm}{500}
  \caption{Input/output map of a dead-zone quantizer with $\Delta=2$.}
  \label{fig:iomap_dz}
\end{figure}

\begin{figure}
  \centering
  \myfig{qe_dz}{5cm}{500}
  \caption{Quantization error of a dead-zone quantizer with $\Delta=2$.}
  \label{fig:qe_dz}
\end{figure}

Figs.~\ref{fig:iomap_mr}, \ref{fig:qe_mr}, \ref{fig:iomap_mt},
\ref{fig:qe_mt}, \ref{fig:iomap_dz}, and \ref{fig:qe_dz} describe the
behaviour of 3 different quantizers.  Use this
\href{https://github.com/Sistemas-Multimedia/Sistemas-Multimedia.github.io/blob/master/study_guide/05-quantization/digital_quantization.ipynb}{notebook}
to gain more insights about quantization.

\section{What you have to do?}
  
Please, modify the
\href{https://github.com/Sistemas-Multimedia/Sistemas-Multimedia.github.io/blob/master/study_guide/05-quantization/quantize_a_frame.ipynb}{notebook}
in order to use the
\href{https://docs.opencv.org/master/d4/da8/group__imgcodecs.html}{TIFF
  and JPEG 2000 image formats} instead of PNG. Compare the RD curves.

\section{Timming}

Please, finish this notebook before the next class session.

\section{Deliverables}

None.

\section{Resources}

\bibliography{data-compression,signal-processing}
