%\newcommand{\SM}{\href{http://cms.ual.es/UAL/estudios/masteres/plandeestudios/asignaturas/asignatura/MASTER7114?idAss=71142105&idTit=7114}{Sistemas Multimedia}}
\newcommand{\SM}{\href{https://sistemas-multimedia.github.io/}{Sistemas Multimedia}}

\newcommand{\theproject}{\href{https://github.com/Sistemas-Multimedia/MRVC}{MRVC}}

\newcommand{\SW}{\href{https://github.com/Sistemas-Multimedia/MRVC}{MRVC}}

\title{\SM{} - Study Guide - Milestone 5: Quantizing in the RGB Domain}

\maketitle

\section{Description}
\href{https://en.wikipedia.org/wiki/Visual_system}{Humans} are quite
efficient recognizing the information stored in images (and frames of
a video), even when this information has been degraded or (partially)
lost. \href{https://en.wikipedia.org/wiki/Quantization_(signal_processing)}{Quantization}~\cite{sayood2017introduction,vetterli2014foundations}
is a technique that can remove the part of the visual information that
is less relevant for us, and implies
\href{https://en.wikipedia.org/wiki/Lossy_compression}{lossy coding},
which provides
\href{https://en.wikipedia.org/wiki/Data_compression_ratio}{compression
  ratios} usually at least one order of magnitude higher than using
\href{https://en.wikipedia.org/wiki/Lossless_compression}{lossless
  coding}.

In this milestone we are quantize a frame in the
\href{https://en.wikipedia.org/wiki/RGB_color_model}{RGB color
  domain}. A
\href{https://en.wikipedia.org/wiki/Quantization_(signal_processing)}{\textbf{quantizer}}
(typically denoted by $Q$) is a
\href{https://en.wikipedia.org/wiki/Data_compression}{encoding system}
that inputs a sequence of (digital) samples $x$ and output a sequence
of quantization indexes $k$ (see the Fig.~\ref{fig:Q}). The inverse
system, called a \textbf{dequantizer} (and denoted by $Q^{-1}$),
recovers an approximate version of $x$ that we will denote as
$\tilde{x}$, whose similarity with $x$ inversely depends on a frame
quantization step (QS) $\Delta$. Notice that although $\Delta$ is an
input parameter to the quantizer, it has not been considered in the
figure to keep it simpler.

We define the quantization error\footnote{Also called quantization
noise.}
\begin{equation}
  e_i = x_i - \tilde{x}_i,
\end{equation}
that obviously, should be minimized.

\begin{figure}
  \centering
  \myfig{graphics/Q}{2cm}{200}
  \caption{Scalar quantization and dequantization of a signal.}
  \label{fig:Q}
\end{figure}

Quantizers can be classifies into:
\begin{enumerate}
\item \textbf{Scalar quantizers}: those that produce a quantization
  index $k_i$ by each input sample $x_i$.
\item \textbf{Vector quantizers}: that process the input by blocks of
  samples, called vectors, producing a quantization index by vector,
  and usually, the length of the quantization index y much shorter
  than the length of the vector. Vector Quantization (VQ) can remove
  correlation and therefore, are more efficient that Scalar
  Quantization (SQ). Unfortunately, the computational requirements of VQ
  are, by far, higher than the needed by SQ. If we also consider that
  there are other techniques (such as transform coding) that are able
  to decorrelate the samples with less computation than VQ, we can
  understand why SQ has been selected, for example, in
  most \href{https://en.wikipedia.org/wiki/Video_coding_format}{video
  codecs}.
\end{enumerate}

Quantizers, also, can be classified as:
\begin{enumerate}
\item \textbf{Uniform quantizers}: those in which $\Delta$ is
  idependent on the amplitude of the samples.
\item \textbf{Non-unifom quantizers}, on the contrary, when $\Delta$
  depends (directly or indirectly) on the amplitude of the samples.
\end{enumerate}

Non-uniform quantizers can be:
\begin{enumerate}
\item \textbf{Static quantizers}: if $\Delta$ depends on the amplitude
  of the samples (as happens, for example, in
  \href{https://en.wikipedia.org/wiki/Companding}{companded
    quantizers}), but this dependency is known \emph{a-priori}.
\item \textbf{Adaptive quantizers}: when $\Delta$ is adapted to
  minimize the quantization error, depending ``on-the-fly'' on the
  characteristics\footnote{Depending, for example, on the
    \href{https://en.wikipedia.org/wiki/Probability_density_function}{PDF}
    of the signal.} of the signal.
\end{enumerate}

All quantizers can be classified into:
\begin{enumerate}
\item \textbf{Mid-tread quantizers}, if $\tilde{x}_i$ can be $0$.
\item \textbf{Mid-rised quantizers}, if $\tilde{x}_i$ never is $0$,
  even if $x_i=0$.
\end{enumerate}

Finally, quantizers can be classified as:
\begin{enumerate}
\item \textbf{Dead-zone quantizers}, that are characterized by a QS of
  length $2\Delta$ for $x_i=0$. Deadzone quantizers tends to remove
  the
  \href{https://en.wikipedia.org/wiki/Noise_(electronics)}{electronic
    noise} (that usually has an small amplitude compared to the input
  signal $x$). Notice that dead-zone quantizers should not be
  consdered uniform, and that all dead-zone quantizers, by definition,
  are mid-tread.
\item \textbf{No dead-zone quantizers}, when the dead-zone does not
  exist.
\end{enumerate}

\begin{figure}
  \centering
  \myfig{iomap_mr}{5cm}{500}
  \caption{Input/output map of a mid-riser quantizer with $\Delta=2$.}
  \label{fig:iomap_mr}
\end{figure}

\begin{figure}
  \centering
  \myfig{qe_mr}{5cm}{500}
  \caption{Quantization error of a mid-riser quantizer with $\Delta=2$.}
  \label{fig:qe_mr}
\end{figure}

\begin{figure}
  \centering
  \myfig{iomap_mt}{5cm}{500}
  \caption{Input/output map of a mid-tread quantizer with $\Delta=2$.}
  \label{fig:iomap_mt}
\end{figure}

\begin{figure}
  \centering
  \myfig{qe_mt}{5cm}{500}
  \caption{Quantization error of a mid-tread quantizer with $\Delta=2$.}
  \label{fig:qe_mt}
\end{figure}

\begin{figure}
  \centering
  \myfig{iomap_dz}{5cm}{500}
  \caption{Input/output map of a dead-zone quantizer with $\Delta=2$.}
  \label{fig:iomap_dz}
\end{figure}

\begin{figure}
  \centering
  \myfig{qe_dz}{5cm}{500}
  \caption{Quantization error of a dead-zone quantizer with $\Delta=2$.}
  \label{fig:qe_dz}
\end{figure}

Figs.~\ref{fig:iomap_mr}, \ref{fig:qe_mr}, \ref{fig:iomap_mt},
\ref{fig:qe_mt}, \ref{fig:iomap_dz}, and \ref{fig:qe_dz} describe the
behaviour of 3 different quantizers.  Use this
\href{https://github.com/Sistemas-Multimedia/Sistemas-Multimedia.github.io/blob/master/study_guide/05-quantization/digital_quantization.ipynb}{notebook}
to gain more insights about quantization.

\subsection{Quantization in the RGB domain}
Supposing that we will use a static uniform dead-zone quantizer with
quantization steps, a color RGB image is quantized chanel by channel,
using QSs $\Delta_{\text{R}}$, $\Delta_{\text{G}}$, and
$\Delta_{\text{B}}$. A reasonable question here is, how we should be
chosen these parameters?  At this point we can consider two different
optimization perspectives. In the first one, we consider strictly
visual considerations, and obviously, any alternative different from
\begin{equation}
  \Delta_{\text{R}} = \Delta_{\text{G}} = \Delta_{\text{B}}
  \label{eq:simple_Q}
\end{equation}
will produce some alteration in the color (sometimes called
``chroma'') of the reconstructed frame.

In the second one, only the
(\href{https://en.wikipedia.org/wiki/Rate-distortion_theory}{RD
  performance} is considered. From a RD perspective, the best
combination of QSs is those that optimizes (makes it closer to the
origin of coordinates) the RD curve. Thus, the optimal QSs should
operate in the curves with the same RD slope,
\begin{equation}
  \lambda_{\text{R}} = \lambda_{\text{G}} = \lambda_{\text{B}},
  \label{eq:optimal_quantization}
\end{equation}
for a given total bit-rate $R$, which implies that the contribution of each channel (the ratio between
quality and bit-rate) to the quality of $\tilde{x}$ has been highest
possible~\cite{vetterli1995wavelets,sayood2017introduction}.

Unfortunately, the previous procedure implies the computation of the
RD curve for each channel, which is a time-consuming operation. For
this reason, and supposing that the statistics of each channel are
similar, we can suppose that Eq.~\ref{eq:simple_Q} satisfies
Eq.~\ref{eq:optimal_quantization}.

\section{What you have to do?}
Please, create a jupyter notebook to test if Eq.~\ref{eq:simple_Q} is
not optimal (at least there is a different configuration of QSs better
that this equation).
%\begin{enumerate}
%\item Please, modify this
%  \href{https://github.com/Sistemas-Multimedia/Sistemas-Multimedia.github.io/blob/master/study_guide/05-quantization/performance.ipynb}{notebook}
%  in order to use the
%  \href{https://docs.opencv.org/master/d4/da8/group__imgcodecs.html}{TIFF
%    and JPEG 2000 image formats} instead of PNG. Compare the RD
%  curves.
%\item In the previous
%  \href{https://github.com/Sistemas-Multimedia/Sistemas-Multimedia.github.io/blob/master/study_guide/05-quantization/performance.ipynb}{notebook}
%  the three color channels, R, G, and B has been quantized using the
%  same QS ($\Delta_{\text{R}} = \Delta_{\text{G}} =
%  \Delta_{\text{B}}$). Do you think that this strategy minimizes the
%  quantization error?
%\item Compare the estimation provided by the entropy with the
%  DEFLATE's bit-rates.
%\end{enumerate}

\section{Timming}

Please, finish this notebook before the next class session.

\section{Deliverables}

None.

\section{Resources}

\bibliography{data-compression,signal-processing,DWT}
