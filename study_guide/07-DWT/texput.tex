%\newcommand{\SM}{\href{http://cms.ual.es/UAL/estudios/masteres/plandeestudios/asignaturas/asignatura/MASTER7114?idAss=71142105&idTit=7114}{Sistemas Multimedia}}
\newcommand{\SM}{\href{https://sistemas-multimedia.github.io/}{Sistemas Multimedia}}

\newcommand{\theproject}{\href{https://github.com/Sistemas-Multimedia/MRVC}{MRVC}}

\newcommand{\SW}{\href{https://github.com/Sistemas-Multimedia/MRVC}{MRVC}}

\title{\SM{} - Study Guide - Milestone 7: Energy Concentration and Spatial Multiresolution with the Discrete Wavelet Transform}

\maketitle

\section{Description}

Technically, the
\href{https://en.wikipedia.org/wiki/Discrete_wavelet_transform}{DWT
  (Discrete Wavelet Transform)} is a
\href{https://en.wikipedia.org/wiki/Linearity}{linear}
\href{https://en.wikipedia.org/wiki/Change_of_basis}{basis}
\href{https://www.youtube.com/watch?v=P2LTAUO1TdA}{expansion} which
computes a
\href{https://www.dsprelated.com/freebooks/sasp/Critically_Sampled_Perfect_Reconstruction.html}{critically}-\href{https://en.wikipedia.org/wiki/Nyquist-Shannon_sampling_theorem}{sampled}
\href{https://en.wikipedia.org/wiki/Octave_(electronics)}{octave}-\href{https://en.wikipedia.org/wiki/Frequency_band}{band}
\href{https://www.sciencedirect.com/topics/engineering/wavelet-decomposition}{decomposition}~\cite{vetterli2014foundations,kovacevic2013fourier}. Specifically,
the DWT is a family of transforms whose output describes the (input)
signal in a set of
\href{https://en.wikipedia.org/wiki/Sub-band_coding}{subbands}. This
set is also called
a \href{https://en.wikipedia.org/wiki/Discrete_wavelet_transform}{decomposition}
and the
\href{https://en.wikipedia.org/wiki/Array_data_structure#Element_identifier_and_addressing_formulas}{index}
of the subband is related to the
\href{https://en.wikipedia.org/wiki/Frequency}{frequency} of the
signal. For example, in the case of the
\href{https://en.wikipedia.org/wiki/Digital_image}{images}, the
position of the
\href{https://en.wikipedia.org/wiki/Coefficient}{coefficients} in the
subbands is related to
\href{https://en.wikipedia.org/wiki/Discrete_wavelet_transform#/media/File:Jpeg2000_2-level_wavelet_transform-lichtenstein.png}{the
  spatial area where the corresponding pixels are found}.


\subsection{An example of DWT: the Haar DWT}
As the rest of transforms, the DWT is also described by the filters of
the analysis matrix transform (the matrix $K$ in the
\href{https://sistemas-multimedia.github.io/study_guide/06-color_transform/}{previous
  milestone}). Haar defined the analysis
\href{https://en.wikipedia.org/wiki/Downsampling_(signal_processing)}{downsampling}
filters
\begin{equation}
  \begin{bmatrix}
    l^{s+1}_n \\
    h^{s+1}_n
  \end{bmatrix}
  = 
  \begin{bmatrix} \frac{1}{\sqrt{2}} & \frac{1}{\sqrt{2}} \\ \frac{1}{\sqrt{2}} & \frac{-1}{\sqrt{2}} \end{bmatrix}
  \begin{bmatrix}
    l^s_{2n} \\
    l^s_{2n+1}
  \end{bmatrix},
  \label{eq:Haar_transform}
\end{equation}
where the superindex $s$ denotes the subband, and $n$ refers to the
$n$-th element of the signal. By definition (and notice that this
holds for all DWTs),
\begin{equation}
  l^0=x.
\end{equation}

\begin{figure}
  \centering
  \myfig{haar_modulus}{6cm}{600}
  \caption{Haar filters's response in the frequency domain (see this
    \href{https://github.com/Sistemas-Multimedia/Sistemas-Multimedia.github.io/blob/master/study_guide/05-DWT/dwt_filters_analysis.ipynb}{notebook}).
    $|K_i(e^{j\omega})|$ denotes the
    \href{https://en.wikipedia.org/wiki/Absolute_value}{modulus} of
    the \href{https://en.wikipedia.org/wiki/Fourier_transform}{Fourier
      transform} of $K_i$.}
  \label{fig:Haar_modulus}
\end{figure}

\begin{figure}
  \centering
  \myfig{db5_modulus}{6cm}{600}
  \caption{Daubechies-5 filters's response in the frequency domain
    (see this
    \href{https://github.com/Sistemas-Multimedia/Sistemas-Multimedia.github.io/blob/master/study_guide/05-DWT/dwt_filters_analysis.ipynb}{notebook}).}
  \label{fig:db5_modulus}
\end{figure}

\begin{figure}
  \centering
  \myfig{bior3.5_modulus}{6cm}{600}
  \caption{Biorthiogonal-3.5 filters's response in the frequency
    domain (see this
    \href{https://github.com/Sistemas-Multimedia/Sistemas-Multimedia.github.io/blob/master/study_guide/05-DWT/dwt_filters_analysis.ipynb}{notebook}).}
  \label{fig:bior3.5_modulus}
\end{figure}

\begin{figure}
  \centering
  \myfig{haar_phase}{6cm}{600}
  \caption{Phase response of the Haar filters (see this
    \href{https://github.com/Sistemas-Multimedia/Sistemas-Multimedia.github.io/blob/master/study_guide/05-DWT/dwt_filters_analysis.ipynb}{notebook}).}
  \label{fig:haar_phase}
\end{figure}

\begin{figure}
  \centering \myfig{db5_phase}{6cm}{600}
  \caption{Phase response of the Daubechies-5 filters (see this
    \href{https://github.com/Sistemas-Multimedia/Sistemas-Multimedia.github.io/blob/master/study_guide/05-DWT/dwt_filters_analysis.ipynb}{notebook}).}
  \label{fig:db5_phase}
\end{figure}

As it can be seen in the Fig.~\ref{fig:Haar_modulus}, $K_0$ is a
\href{https://en.wikipedia.org/wiki/Low-pass_filter}{low-pass filter}
and $K_1$ is a
\href{https://en.wikipedia.org/wiki/High-pass_filter}{high-pass
  filter} (this holds for all DWTs). Considering the Haar filter, we
can conclude that:
\begin{enumerate}
\item There exists
  \href{https://en.wikipedia.org/wiki/Aliasing}{aliasing} between the
  filters (this holds for all DWTs) and this is a drawback
  because: \begin{enumerate} \item The same
    \href{https://en.wikipedia.org/wiki/Information}{information} can
    be found in both subbands ($l$ and $h$). Therefore, the
    concentration of the energy in one of the subbands is
    smaller.  \item At it can be seen also in
    Eq.~\ref{eq:Haar_transform}, the subbands are downsampled by 2
    (action usually represented by $\downarrow 2$) and both subbands
    should have, at most, a
    \href{https://en.wikipedia.org/wiki/Bandwidth_(signal_processing)}{bandwidth}
    of $\frac{1}{2}$
    \href{https://en.wikipedia.org/wiki/Radian}{radians}/\href{https://en.wikipedia.org/wiki/Sampling_(signal_processing)}{sample}
    in order to avoid the aliasing during the
    \href{https://en.wikipedia.org/wiki/Downsampling_(signal_processing)}{subsampling},
    thus increasing the
    \href{https://en.wikipedia.org/wiki/Perception}{perceptible}
    \href{https://en.wikipedia.org/wiki/Signal-to-noise_ratio}{quality}
    of the low-pass subband $l$.  \end{enumerate} These problems can
  be only solved using filters that have
  \href{https://en.wikipedia.org/wiki/Transfer_function}{transfer
    functions} that overlaps a smaller area (see
  Figs.~\ref{fig:db5_modulus} and \ref{fig:bior3.5_modulus}).
%\item The response of the filter bank is flat, which means that the gain of the different frequencie
%  $|y(e^{j\omega})|=a|x(e^{j\omega})|, a\in\mathbb{R}$ (the
%  reconstructed signal $y$ has not been filtered). % OJO
\item There is not
  \href{https://en.wikipedia.org/wiki/Linear_phase}{phase distortion}
  (the phase of the filters is a linear function of the frequency, see
  the Fig.~\ref{fig:haar_phase}). This means that coefficients
  $l^{s+1}_n$ and $h^{s+1}_n$ refers to the section of signal
  $\{l^s_{2n}, l^s_{2n+1}\}$, allowing the design of entropy codecs
  for the decomposition which exploit the correlation between
  subbands. Notice that this does not hold for all filters (see
  Fig.~\ref{fig:db5_phase}).
\end{enumerate}

% Polyphase implementation of the multilevel DWT
\begin{figure}
  \centering
  \myfig{graphics/DWT}{10cm}{1000}
  \caption{Computation of the $S$-levels DWT and the generated
    subbands.}
  \label{fig:DWT}
\end{figure}

Eq.~\ref{eq:Haar_transform} computes the 1-levels Haar DWT. A more
general expresion for this equation (and this holds for all DWTs) is
\begin{equation}
  \{l^{s+1}, h^{s+1}\} = \text{DWT}(l^s),
  \label{eq:DWT}
\end{equation}
where $\{\cdot,\cdot\}$ denotes the concatenation of subbands. As can
be seen in the Fig.~\ref{fig:DWT}, it's possible to compute the
$S$-levels DWT, using Eq.~\ref{eq:DWT} iteratively over the
low-frequency subband, generating the decomposition
\begin{equation}
  \{l^S_0, h^S_0, h^{S-1}_0 h^{S-1}_1, h^{S-2}_0 h^{S-2}_1 h^{S-2}_2
  h^{S-2}_3, \cdots, h^1_0 h^1_1 \cdots
  h^1_{2^{n-1}-1}\}=\text{DWT}^S(l^0),
  \label{eq:S_levels_DWT}
\end{equation}
where
\begin{equation}
  n = \log_2(N)
\end{equation}
where $N$ is the number of samples.

In the case of the Haar filters, the synthesis (inverse) $1$-levels
DWT (that we will denote by $\text{DWT}^{-1}$) is the result of
solving the coefficients $l^{s+1}_n$ and $h^{s+1}_n$ in the
Eq.~\ref{eq:Haar_transform}. In general, for longer DWT filters we are
going to have more coefficients, but the same procedure can be used
for finding the synthesis transform. Therefore, it can be written that
\begin{equation}
  l^s = \text{DWT}^{-1}(\{l^{s+1}, h^{s+1}\}),
\end{equation}
and $\text{DWT}^{-S}$, the $S$-levels synthesis DWT, can be computed
as it is also described in the Fig.~\ref{fig:DWT}, by simply reversing
the analysis steps.

\subsection{\href{https://en.wikipedia.org/wiki/Multiresolution_analysis}{Multiresolution analysis}}
In the decomposition described in Eq.~\ref{eq:S_levels_DWT} there are
$S+1$ subbands, and therefore, it is possible to compute $S+1$
resolution levels of the analyzed signal $l^0$:
\begin{itemize}
\item  $l^S$: the resolution level $S$ of $l^0$ (no $\text{DWT}^{-1}$ has been
  applied), with have only one sample $l^S_0$ that uses to be the
  \href{https://en.wikipedia.org/wiki/Arithmetic_mean}{arithmetic
    mean} of $l^0$. The coefficient $l^S_0$
  is also called the \href{https://en.wikipedia.org/wiki/DC_bias}{DC
    (Direct Current) coefficient}. On the contrary, the rest of
  coefficents (high-frequency subbands) are called the AC (Alternating
  Current) coefficients.
\item $l^{S-1}=\text{DWT}^{-1}(\{l^S, h^S\})$, the resolution level $S-1$ of $l^0$,
  with two samples.
\item $l^{S-2}=\text{DWT}^{-1}(\{l^{S-1}, h^{S-1}\})$, the resolution
  level $S-1$ of $l^0$, with four samples.
\item $\vdots$
\item The resolution level $0$, $l^0=\text{DWT}^{-1}(\{l^1, h^1\})$.
\end{itemize}  

\subsection{Orthogonal and biorthogonal DWT}

\begin{figure}
  \centering
  \myfig{graphics/PRFB}{5cm}{500}
  \caption{A 2-channels PRFB (Perfect Reconstruction Filter Bank).}
  \label{fig:PRFB}
\end{figure}

All DWTs are defined by 4 filters (using a similar notation that the
followed by \href{https://pywavelets.readthedocs.io}{PyWavelets}) (see
Fig.~\ref{fig:PRFB}):
\begin{enumerate}
\item Analysis (downsampling) low-pass filter $\tilde\phi$ (the decomposition
  scaling function), which computes $l^{s+1}$ from $l^s$.
\item Analysis (downsampling) high-pass filter $\tilde\psi$ (the decomposition
  wavelet function), which computes $h^{s+1}$ from $l^s$.
\item Synthesis (upsampling) high-pass filter $\phi$ (the
  reconstruction scaling function), which computes $\{l^s_{2n}\}$ from
  $l^{s+1}$ and $h^{s+1}$.
\item Synthesis (upsampling) high-pass filter $\psi$ (the
  reconstruction wavelet function), which computes $\{l^s_{2n+1}\}$
  from $l^{s+1}$ and $h^{s+1}$.
\end{enumerate}

In the context of the DWT, orthogonality provides intersubband
\href{https://en.wikipedia.org/wiki/Decorrelation}{decorrelation},
which basically means that the contribution of each subbands to the
reconstruction (and this holds for all DWTs) of the signal are
independent. Orthogonal filters can be recognized because:
\begin{enumerate}
\item $\tilde\phi\bot\tilde\psi$ and $\phi\bot\psi$, where $\bot$
  denotes orthogonality.
\item Their taps are all different (the filters are asymmetric).
\item The taps of the synthesis filters can be determined simply by
  ``reflecting'' the taps of the analysis filters.
\item In general, except for the Haar DWT, orthogonal filters generate
  phase distortion.
\end{enumerate}

On the other hand, in a biorthogonal DWT:
\begin{enumerate}
\item $\tilde\phi\bot\tilde\psi$, $\phi\bot\psi$, $\psi\bot\tilde\phi$
  and $\tilde\psi\bot\phi$.
\item All the filters are symmetric.
\item The filters satisfy that
  \begin{equation}
    \psi=(-1)^n\tilde\phi;~n\in\mathbb{N},
  \end{equation}
  and
  \begin{equation}
    \phi=(-1)^n\tilde\psi;~n\in\mathbb{N}
  \end{equation}
    (the odd taps are multiplied by $-1$). Therefore, if we have the
  analysis filters we can deduce the synthesis filters and viceversa.
\item No phase distortion is generated.
\end{enumerate}

\section{What you have to do?}

Please, using
\href{https://github.com/Sistemas-Multimedia/Sistemas-Multimedia.github.io/blob/master/study_guide/07-DWT/dwt_filters_analysis.ipynb}{this
  notebook}, determine the frequency and phase responses of the
\href{http://wavelets.pybytes.com/wavelet/db20/}{Daubechies 20} and
\href{http://wavelets.pybytes.com/wavelet/bior6.8/}{Biorthogonal-6.8}
filters. Determine also their relative gains, using the energy of the
signal resulting of applying the synthesis transform to the impulse
signal. The energy of a signal $x$ is found by
\begin{equation}
  \langle x, x\rangle =  \sum_{i}{x_i^2}.
\end{equation}

\section{Timming}

Please, finish this notebook before the next class session.

\section{Deliverables}

None.

\section{Resources}

\bibliography{signal-processing,maths,data-compression}
