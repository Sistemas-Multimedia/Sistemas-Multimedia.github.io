%\newcommand{\SM}{\href{http://cms.ual.es/UAL/estudios/masteres/plandeestudios/asignaturas/asignatura/MASTER7114?idAss=71142105&idTit=7114}{Sistemas Multimedia}}
\newcommand{\SM}{\href{https://sistemas-multimedia.github.io/}{Sistemas Multimedia}}

\newcommand{\theproject}{\href{https://github.com/Sistemas-Multimedia/MRVC}{MRVC}}

\newcommand{\SW}{\href{https://github.com/Sistemas-Multimedia/MRVC}{MRVC}}

\title{\SM{} - Study Guide - Milestone 8: Quantization}

\maketitle

The HVS (Human Visual System) works quite well recognizing the
information stored in images (and frames of a video), even when this
information has been degraded or
lost. \href{https://en.wikipedia.org/wiki/Quantization_(signal_processing)}{Quantization}~\cite{sayood2017introduction,vetterli2014foundations}
is a technique that can remove these part of the visual information
less interesting for us. In this milestone we are going to quantize
the MDWT domain.

A quantizer (typically denoted by $Q$) is a system that inputs a
sequence of (digital) samples $x$ and output a sequence of
quantization indexes $k$ (see the Fig.~\ref{fig:Q}). The inverse
system, called a dequantizer (and denoted by $Q^{-1}$), recover an
approximate version (depending on the quantization step $\Delta$) of
$x$ that we will denote as $\tilde{x}$. Notice that although $\Delta$
is an input parameter to the quantizer, it has not been considered in
the figure. We define the quantization error
\begin{equation}
  e_i = x_i - \tilde{x}_i,
\end{equation}
that obviously, should be minimized.

\begin{figure}
  \centering
  \myfig{graphics/Q}{4cm}{400}
  \caption{Scalar quantization and dequantization of a signal.}
  \label{fig:Q}
\end{figure}

Quantizers can be classifies into:
\begin{enumerate}
\item \textbf{Scalar quantizers}: those that produce a quantization
  index $k_i$ by each input sample $x_i$.
\item \textbf{Vector quantizers}: those that process the input by
  blocks of samples, called vectors, producing quantization index by
  vector. Vector Quantization (VQ) can remove spatial correlation and
  therefore, are more efficient that Scalar Quantization (SQ).
\end{enumerate}

Quantizers, also, can be classified as:
\begin{enumerate}
\item \textbf{Uniform quantizers}: those in which the quantization
  step ($\Delta$) is idependent on the amplitude of the samples.
\item \textbf{Non-unifom quantizers}, on the contrary, when $\Delta$
  depends on (directly or indirectly) the samples.
\end{enumerate}

Non-uniform quantizers can be:
\begin{enumerate}
\item \textbf{Static quantizers}: those que use the same quantization
  step ($\Delta$) for each samples or vector of the sequence.
\item \textbf{Adaptive quantizers}: when $\Delta$ is adapted to
  minimize the quantization error.
\end{enumerate}

All uantizers can be classified into:
\begin{enunerate}
\item \textbf{Mid-tread quantizers}, if $\tilde{x}_i$ can be $0$.
\item \textbf{Mid-rised quantizers}, if $\tilde{x}_i$ cannot be $0$.
\end{enumerate}

Finally, quantizers can be classified as:
\begin{enunerate}
\item \textbf{Dead-zone quantizers}, that are characterized by a
  quantization step of length $2\Delta$ for $x_i=0$. Deadzone
  quantizers tends to remove the electronic noise (that usually has an
  small amplitude compared to the input signal $x$).
\item \textbf{No dead-zone quantizers}, when the dead-zone does not
  exist.
\end{enumerate}

Use this notebook to gain more insights about quantization.

\section{Description}

The 2D-DWT can a used on sequences of images, by simply iterating as
it is described in the Algorithm~\ref{alg:MDWT}. $I$ is the number of
images in the sequence $V$.

\begin{pseudocode}{\text{MDWT}}{V}
  \label{alg:MDWT}
  \FOR i \GETS 0 \TO I-1 \DO
  V_i = \text{2D-DWT}(V_i)
\end{pseudocode}

\section{What you have to do?}
  
Please, complete this
\href{https://github.com/Sistemas-Multimedia/Sistemas-Multimedia.github.io/blob/master/study_guide/MDWT/MDWT.ipynb}{notebook}.

\section{Timming}

Please, finish this notebook before the next class session.

\section{Deliverables}

None.

\section{Resources}

\bibliography{python}
