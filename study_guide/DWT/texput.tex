%\newcommand{\SM}{\href{http://cms.ual.es/UAL/estudios/masteres/plandeestudios/asignaturas/asignatura/MASTER7114?idAss=71142105&idTit=7114}{Sistemas Multimedia}}
\newcommand{\SM}{\href{https://sistemas-multimedia.github.io/}{Sistemas Multimedia}}

\newcommand{\theproject}{\href{https://github.com/Sistemas-Multimedia/MRVC}{MRVC}}

\newcommand{\SW}{\href{https://github.com/Sistemas-Multimedia/MRVC}{MRVC}}

\title{\SM{} - Study Guide - Milestone 5: Energy Concentration and Multiresolution  with the Discrete Wavelet Transform}

\maketitle

\section{Description}

\href{https://web.stanford.edu/class/ee398a/handouts/lectures/07-TransformCoding.pdf}{Transform
coding} can remove spatial
\href{https://en.wikipedia.org/wiki/Correlation_and_dependence}{correlation}
in signals. The DWT (Discrete Wavelet Transform) is a family of
transforms whose output describes the (input) signal in a set of
subbands. This set is also called a decomposition and the index of the
subband is related to the frequency of the signal. In the case of the
images, the coefficients of the subbands are related to the space. 

All linear\footnote{Non-linear transform are also possible, but their
  mathematical operation can not be described using linear algebra.}
transforms can be described a matrix-vector
product~\cite{strang4linear} as
\begin{equation}
  y = Kx,
  \label{eq:forward_transform_matrix_form}
\end{equation}
where $x$ is the input signal, $K$ is the analysis transform matrix,
and $y$ is the output decomposition. Notice that
\begin{equation}
  y_1 = <K_1, x_1>

The DWT is described by the filters that form the rows o

\section{What you have to do?}
  
Please, complete this
\href{https://github.com/Sistemas-Multimedia/Sistemas-Multimedia.github.io/blob/master/study_guide/MDWT/MDWT.ipynb}{notebook}.

\section{Timming}

Please, finish this notebook before the next class session.

\section{Deliverables}

None.

\section{Resources}

\bibliography{python}
