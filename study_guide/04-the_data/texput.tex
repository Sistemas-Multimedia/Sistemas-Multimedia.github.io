%\newcommand{\SM}{\href{http://cms.ual.es/UAL/estudios/masteres/plandeestudios/asignaturas/asignatura/MASTER7114?idAss=71142105&idTit=7114}{Sistemas Multimedia}}
\newcommand{\SM}{\href{https://sistemas-multimedia.github.io/}{Sistemas Multimedia}}

\newcommand{\theproject}{\href{https://github.com/Sistemas-Multimedia/MRVC}{MRVC}}

\newcommand{\SW}{\href{https://github.com/Sistemas-Multimedia/MRVC}{MRVC}}

\title{\SM{} - Study Guide - Milestone 3: Understanding the Video Data}

\maketitle

\section{Description}

\theproject{} inputs \href{https://en.wikipedia.org/wiki/Video} data,
which basically is a
\href{https://en.wikipedia.org/wiki/Sequence}{sequence} of
(\href{https://en.wikipedia.org/wiki/Digital_data}{digital})
\href{https://en.wikipedia.org/wiki/Image}{images}. A image is a
\href{https://en.wikipedia.org/wiki/Matrix_(mathematics)}{matrix} of
\href{https://en.wikipedia.org/wiki/Pixel}{pixels}. This milestone
shows the structure of the videos used in \theproject{}.

We are going to store a video as a sequence of
\href{https://en.wikipedia.org/wiki/Computer_file}{files} with names
<prefix>000.png, <prefix>001.png, ..., where <prefix> specifies the
video name and ``.png'' indicates that the images are coded using the
\href{https://en.wikipedia.org/wiki/Portable_Network_Graphics}{PNG}
image format. All the images of a video have the same
\href{https://en.wikipedia.org/wiki/Image_resolution}{spatial
  resolution}, are represented in the
\href{https://en.wikipedia.org/wiki/RGB_color_model}{RGB domain}, and
each color component has
\href{https://en.wikipedia.org/wiki/Glossary_of_computer_graphics#bit_depth}{bit
  depth} of 16 bits, representing an unsigned
\href{https://en.wikipedia.org/wiki/Integer_(computer_science)}{integer}. The
pixel
\href{https://en.wikipedia.org/wiki/Luminous_intensity}{intensities}
are displaced to the center of the
\href{https://en.wikipedia.org/wiki/Range_(computer_programming)}{range
  of values} ([0, 65535]) that the components can take (for example,
in \href{https://en.wikipedia.org/wiki/Disk_storage}{disk}, the
component intensity 0 is represented by 32768).

\section{What you have to do?}

\begin{enumerate}
\item Download \theproject, preferably cloning your fork.
\item Display the first image of the stockholm video (placed in the
  sequences/stockholm directory) using this
  \href{https://jupyter.org/}{jupyter} \href{}{notebook}.
\item Understand the insights of the notebook.
\end{enumerate}

\section{Timming}

Please, finish this milestone before the next class session.

\section{Deliverables}

None.

\section{Resources}

\bibliography{python}
