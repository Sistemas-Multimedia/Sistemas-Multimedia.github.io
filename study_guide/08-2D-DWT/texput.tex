%\newcommand{\SM}{\href{http://cms.ual.es/UAL/estudios/masteres/plandeestudios/asignaturas/asignatura/MASTER7114?idAss=71142105&idTit=7114}{Sistemas Multimedia}}
\newcommand{\SM}{\href{https://sistemas-multimedia.github.io/}{Sistemas Multimedia}}

\newcommand{\theproject}{\href{https://github.com/Sistemas-Multimedia/MRVC}{MRVC}}

\newcommand{\SW}{\href{https://github.com/Sistemas-Multimedia/MRVC}{MRVC}}

\title{\SM{} - Study Guide - Milestone 6: Computing the 2D-DWT}

\maketitle

\section{Description}

(Digital) Images are 2D (2-Dimensional,
\href{https://en.wikipedia.org/wiki/Discrete_time_and_continuous_time}{discrete}
in \href{https://en.wikipedia.org/wiki/Space}{space} and
\href{https://en.wikipedia.org/wiki/Amplitude}{amplitude})
signals. The 2D-DWT of an image can be computed using (1)
\href{https://en.wikipedia.org/wiki/Separable_filter}{separable} 1D
filters, and (2)
\href{https://en.wikipedia.org/wiki/Non-separable_wavelet}{nonseparable}
2D filters.~\cite{sayood2017introduction} Except in very special
cases, all 2D-DWT implementations use separable filters.

Separability in the DWT context means that we can compute the
$D$-dimensional DWT using the 1D filters, simply applying them to each
dimension, and
using \href{https://en.wikipedia.org/wiki/In-place_algorithm}{in-place}
operations. This procedure has been described in the
Algorithm~\ref{alg:2D-DWT}, where $X_{r,*}$ refers to the $r$-th row
of the matrix $X$ and $X_{*,c}$ to the $c$-th column, being $R$ and
$C$ the number of rows and columns of the image $X$.

\begin{pseudocode}{\text{2D-DWT}}{X}
  \label{alg:2D-DWT}
  \FOR r \GETS 0 \TO R-1 \DO
  X_{r,*} = \text{DWT}(X_{r,*})
  \\
  \FOR c \GETS 0 \TO C-1 \DO
  X_{*,c} = \text{DWT}(X_{*,c})
\end{pseudocode}

The obtained 2D decomposition is expressed by
\begin{equation}
  \begin{bmatrix}
    \text{LL} & \text{HL} \\
    \text{LH} & \text{HH}
  \end{bmatrix}
  =
  \text{2D-DWT}(X),
  \label{eq:2D-DWT}
\end{equation}
where $L$ stands for low-pass filtering and $H$ for high-pass
filtering.

Eq.~\ref{eq:2D-DWT} describes the $1$-levels (analysis)
2D-DWT. Replacing $\text{LL}=L^{s+1}$, $X=L^s$, and
$\{\text{HL}, \text{LH}, \text{HH}\}=H^{s+1}$ in the previous
expression, the $S$-levels 2D-DWT of $L^0=X$ ($\text{2D-DWT}^S(L^0)$)
can be computed applying $S$-times
\begin{equation}
  \{L^{s+1}, H^{s+1}\} = \text{2D-DWT}(L^s)
\end{equation}
to the low-frequency subband $L^s$, generating the 2D decomposition
described in the Fig.~\ref{fig:2D-DWT}.

\begin{figure}
  \centering
  \myfig{graphics/2D-DWT}{5cm}{500}
  \caption{Decomposition generated by $\text{2D-DWT}^2(X)$. On the
    left, the traditional notation. On the right, the new one. Both
    represents the same data.}
  \label{fig:2D-DWT}
\end{figure}

As an example, the Fig.~\ref{fig:lena-DWT} shows the $3$-levels
2D-DWT of the lena image (see this notebook).

\begin{figure}
  \centering
  \pngfig{dwt_lena}{5cm}{500}
  \caption{$\text{2D-DWT}^2(\text{"lena"})$'s decomposition. See
    \href{https://github.com/Sistemas-Multimedia/Sistemas-Multimedia.github.io/blob/master/study_guide/06-2D-DWT/dwt_lena.ipynb}{this
      notebook}.}
  \label{fig:2D-DWT}
\end{figure}

\subsection{Multiresolution}

\section{What you have to do?}
Using
\href{https://github.com/Sistemas-Multimedia/Sistemas-Multimedia.github.io/blob/master/study_guide/06-2D-DWT/dwt_lena.ipynb}{this
  notebook}, compute the
$\text{2D-DWT}^3(\text{"a different image"})$.

\section{Timming}

Please, finish this notebook before the next class session.

\section{Deliverables}

None.

\section{Resources}

\bibliography{data-compression}
