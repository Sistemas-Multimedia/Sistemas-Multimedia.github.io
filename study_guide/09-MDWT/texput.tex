%\newcommand{\SM}{\href{http://cms.ual.es/UAL/estudios/masteres/plandeestudios/asignaturas/asignatura/MASTER7114?idAss=71142105&idTit=7114}{Sistemas Multimedia}}
\newcommand{\SM}{\href{https://sistemas-multimedia.github.io/}{Sistemas Multimedia}}

\newcommand{\theproject}{\href{https://github.com/Sistemas-Multimedia/MRVC}{MRVC}}

\newcommand{\SW}{\href{https://github.com/Sistemas-Multimedia/MRVC}{MRVC}}

\title{\SM{} - Study Guide - Milestone 9: The Motion Discrete Wavelet Transform (MDWT)}

\maketitle

\section{Description}

The
\href{https://sistemas-multimedia.github.io/study_guide/06-2D-DWT/}{2D-DWT}
can a used on sequences of frames (images) by simply iterating as it
is described in the
Algorithm~\ref{alg:MDWT}~\cite{taubman2002jpeg2000}. $N$ is the number
of frames in the sequence $V$, and $S$ the number of levels. The
synthesis transform is computed using the 2D-DWT$^{-S}$.

\begin{pseudocode}{$\text{MDWT}$}{V, S}
  \label{alg:MDWT}
  \FOR i \GETS 0 \TO N-1 \DO
  V_i = \text{2D-DWT}^S(V_i)
\end{pseudocode}

In the Fig.~\ref{fig:MDWT} there is an example of the decomposition
generated for three frames $V_0$, $V_1$ and $V_2$.

\begin{figure}
  \centering
  \myfig{graphics/forward_MDWT}{6cm}{600}
  \caption{Decomposition generated by $\text{MDWT}(\{V_0, V_1, \cdots, V_{N-1}\}, $S=1$)$.}
  \label{fig:MDWT}
\end{figure}

\subsection{Multiresolution}
The $\text{MDWT}$ provides dyadic spatial multiresolution and full
temporal multiresolution.

\subsection{Quantization}
Supposing that the QS is a good estimator of the slope of the RD
curves of the frames of the sequence, we will select is
$\Delta_0=\Delta_1=\dots=\Delta_{N-1}$, where $\Delta_i$ is the frame
quantization step used for the $i$-th frame of the sequence.

\section{What you have to do?}

Please, create an experiment in which the frame quantization step is
different for each frame of a sequence, and another experiment that
uses the same quantization step for all the frames. Compute the
average RD curves. Which one is better? Do you think that
\begin{equation*}
  D = \sum_i D_i
\end{equation*}
where $D$ is the distortion generated for the complete sequence and
$D_i$ is the distortion only for the $i$-th frame, holds?

%Try to answer:
%\begin{enumerate}
%\item How many Spatial Resolution Levels (SRL) of
%  $V=\{V_0, V_1, V_2\}$ are accesible from the decomposition of the
%  Fig.~\ref{fig:MDWT} using the $\text{MDWT}^{-1}$?
%\item Please, run the
%  \href{https://sistemas-multimedia.github.io/MRVC/#x1-80004.1}{Example:
%    1-iteration MDWT ($\mathtt{MDWT}(N=5)$)}, and the
%  \href{https://sistemas-multimedia.github.io/MRVC/#x1-90004.2}{Example:
%    2-iterations MDWT ($2\times\mathtt{MDWT}(N=5)$)}.
%\end{enumerate}

\section{Timming}

Please, finish this notebook before the next class session.

\section{Deliverables}

None.

\section{Resources}

\bibliography{JPEG2000}
