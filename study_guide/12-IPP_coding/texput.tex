%\newcommand{\SM}{\href{http://cms.ual.es/UAL/estudios/masteres/plandeestudios/asignaturas/asignatura/MASTER7114?idAss=71142105&idTit=7114}{Sistemas Multimedia}}
\newcommand{\SM}{\href{https://sistemas-multimedia.github.io/}{Sistemas Multimedia}}

\newcommand{\theproject}{\href{https://github.com/Sistemas-Multimedia/MRVC}{MRVC}}

\newcommand{\SW}{\href{https://github.com/Sistemas-Multimedia/MRVC}{MRVC}}

\title{\SM{} - Study Guide - Milestone 12: IPP... coding in MRVC (Multi Resolution Video Codec)}

\maketitle

\section{Description}

\subsection{The IPP... decorrelation pattern}
It's time to put together all the tools that we have collected for
encoding a sequence of frames. We will call to this set of images, a
Group Of Frames (GOF), and the structure of each GOP will be (except
for the last one) IPP..., which means that the first frame will be
intra-coded (I-type) and the rest of frames of the GOF will be
predicted-coded (P-type). Notice that in a I-type frame all the
coefficients will be I-type, and in a P-type frame, the different
coefficients can be I-type or P-type.

\subsection{Multiresolution iterations}
In each IPP... stage, only a sequence of $[H]$ subbands is
decorrelated. Therefore, in order to increase the CR, a minimum number
of IPP... iterations should be carried out.

\subsection{Scalability}
IPP... MRVC sequences are scalable in resolution (R). Quality (Q)
scalability in each spatial resolution level can be achieved by simply
using an scalable image codec such as JPEG2000. However, notice that
the only allowed progression is RQ (first the quality is incremented
and then, the resolution).

To provide, for example, a QR progression (reconstruct first by
quality and then, by resolution)

\subsection{A block diagram of the codec}
\svg{graphics/codec}{1200}

\section{What you have to do?}
  
Please, using this
\href{https://github.com/Sistemas-Multimedia/Sistemas-Multimedia.github.io/blob/master/study_guide/10-MC_in_DWT_domain/DWT_shift_variance.ipynb}{notebook},
research the posibilities for performing MC of other DWTs available at
\href{https://pywavelets.readthedocs.io/en/latest/}{PyWavelets}.

\section{Timming}

Please, finish this notebook before the next class session.

\section{Deliverables}

None.

\section{Resources}

\bibliography{python}
