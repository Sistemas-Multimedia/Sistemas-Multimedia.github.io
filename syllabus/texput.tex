%\newcommand{\SM}{\href{http://cms.ual.es/UAL/estudios/masteres/plandeestudios/asignaturas/asignatura/MASTER7114?idAss=71142105&idTit=7114}{Sistemas Multimedia}}
\newcommand{\SM}{\href{https://sistemas-multimedia.github.io/}{Sistemas Multimedia}}

\newcommand{\theproject}{\href{https://github.com/Sistemas-Multimedia/MRVC}{MRVC}}

\newcommand{\SW}{\href{https://github.com/Sistemas-Multimedia/MRVC}{MRVC}}

\title{\SM - Syllabus}

\maketitle

\section{About this course}

This course analyses the basic principles used to encode digital images
and videos, and how to apply them in commercial and scientific
environments.

Images and sequences of images (videos) are everywhere, and one of the
main problems that arise when we use them is the huge amount of data
that they require. This course introduces the theoretical contents
based on the development of a video encoding project. Therefore, it is
an imminently practical course, organized into a collection of
milestones.

\section{Timetable}

\begin{description}
\item [Day 1: {\normalfont Preparing the working
    environment.}] \emph{We will use a Linux operating system with
  Git and Python to develop our project.}
  \begin{description}
  \item [Milestone 1: {\normalfont \href{https://sistemas-multimedia.github.io/milestones/01-provisioning/}{Operating system provisioning}.}]
  \item [Milestone 2: {\normalfont \href{https://sistemas-multimedia.github.io/milestones/02-python/}{Installation and basic programming with Python}.}]
  \item [Milestone 3: {\normalfont \href{https://sistemas-multimedia.github.io/milestones/03-git/}{Git, GitHub and the fork-and-branch Git workflow}.}]
  %\item [Milestone 4: {\normalfont \href{https://sistemas-multimedia.github.io/milestones/04-the_data/}{Understanding the video data}.}]
  \end{description}
  ~\newline

\item [Day 2: {\normalfont Removing color redundancy.}] \emph{Color
    components are correlated, and human beings are more sensitive to
    slow variations of the color than to high frequency changes.}
  \begin{description}
  \item [Milestone 4: {\normalfont \href{https://sistemas-multimedia.github.io/milestones/04-Motion_PNG/}{Entropy coding of images with PNG}.}]
  \item [Milestone 5: {\normalfont \href{https://sistemas-multimedia.github.io/milestones/05-RGB_quantization/}{Quantizing in the RGB domain}.}]
  \item [Milestone 6: {\normalfont \href{https://sistemas-multimedia.github.io/milestones/06-YUV_quantization/}{Quantizing in the YUV domain}.}] %{Decorrelating with Color Transforms}.
  \end{description}
  ~\newline

\item [Day 3: {\normalfont Removing spatial redundancy with transforms.}]
  \emph{Pixels are spatially correlated, and human beings prefer low
    spatial frequencies to high ones.}
  \begin{description}
  \item [Milestone 7: {\normalfont \href{https://sistemas-multimedia.github.io/milestones/07-DWT/}{The DCT (Discrete Cosine Transform)}.}]
  \item [Milestone 8: {\normalfont \href{https://sistemas-multimedia.github.io/milestones/08-DWT/}{The DWT (Discrete Wavelet Transform)}.}]
  %\item [Milestone 9: {\normalfont \href{}{LP (Laplacian Pyramid) (unfinished)}.}]
  \end{description}
  ~\newline

\item [Day 4: {\normalfont Removing temporal redundancy with motion estimation.}]
  \emph{Video sequences are correlated in time, too!}
  \begin{description}
  \item [Milestone 9: {\normalfont \href{https://sistemas-multimedia.github.io/milestones/10-ME/}{Motion estimation and compensation in the image domain}.}]
  \item [Milestone 10: {\normalfont \href{https://sistemas-multimedia.github.io/milestones/11-image_domain_IPP/}{IPP... coding in the image domain}.}]
  \end{description}
  ~\newline

\item [Days 5-8: {\normalfont MRVC (Muti-Resolution Video Coding).}]
  \emph{Increasing the spatial scalabilty.}
  \begin{description}
  \item [Milestone 11: {\normalfont \href{https://sistemas-multimedia.github.io/milestones/12-transform_domain_MC/}{Motion compensation in the transform domain}.}]
  \item [Milestone 12: {\normalfont \href{https://sistemas-multimedia.github.io/milestones/13-transform_domain_IPP/}{IPP... coding in the transform domain}.}]
  \end{description}

\end{description}

