% Emacs, this is -*-latex-*-

%\newcommand{\SM}{\href{http://cms.ual.es/UAL/estudios/masteres/plandeestudios/asignaturas/asignatura/MASTER7114?idAss=71142105&idTit=7114}{Sistemas Multimedia}}
\newcommand{\SM}{\href{https://sistemas-multimedia.github.io/}{Sistemas Multimedia}}

\newcommand{\theproject}{\href{https://github.com/Sistemas-Multimedia/MRVC}{MRVC}}

\newcommand{\SW}{\href{https://github.com/Sistemas-Multimedia/MRVC}{MRVC}}

\title{\SM{} \\ Syllabus}

\maketitle
\tableofcontents

\section{About}

Sistemas Multimedia (SM) is a compulsory subject of the Master in
Technologies and Applications in Computer Science at the UAL
(University of Almería).

\section{Course Meeting Times}

See the current \href{https://www.ual.es/estudios/masteres/presentacion/plandeestudios/asignatura/7114/71142105}{time-table}.

\section{Scope}

SM introduces and develops some of the most used techniques used in
image and video coding systems.

\section{Main goals}

To develop:
\begin{enumerate}
\item The ability to use and develop methodologies, methods,
  techniques, specific use programs,norms and standards of graphic
  computing (TI10).
\item The ability to conceptualize, design, develop and evaluate the
  human-computer interaction ofproducts, systems, applications and IT
  services (TI11).
\item The capacity for the creation and exploitation of virtual
  environments, and for the creation, management and distribution of
  multimedia contents (TI12).
\end{enumerate}

\section{Methodology and \href{https://sistemas-multimedia.github.io/contents/}{Contents}}

SM follows the PBL (Project-Based Learning)
\href{http://portafirma.ual.es/pfirma/downloadReport/file?idDocument=4u61Ie5es2&idRequest=ZeBY35LlFa}{methodology}. The
students, helped by the lecturer, develop a project during the
classes. This project is
\href{https://github.com/Sistemas-Multimedia/VCF}{VCF}, a framework
for image and video coding.

The contents of the subject can be found \href{https://sistemas-multimedia.github.io/contents/}{here}.

\section{Attendance}

Currently,
\href{https://www.ual.es/estudios/masteres/presentacion/plandeestudios/asignatura/7114/71142105}{the
  course is organized in 8 lectures of 2 hours/session, during 5
  weeks}, and it has been developed to be both, virtual and
blended training. This means that all the content is available
online, but students are expected to regularly attend lectures (at
least, remotely), which are mainly practical.

\section{Evaluation Criteria and Instruments}

The work done by students in the SM subject is evaluated in two
different ways:

\begin{enumerate}
\item By delivering the solutions to the problems raised throughout
  the course.

\item Through a final exam, optional.
\end{enumerate}

In the first case, the qualification will depend on the number of
problems solved, their complexity, the quality of the solution, and
the number of members of the working group, which may be up to 4
people. More details are given in the Teaching Guide
(\href{https://portafirma.ual.es/pfirma/downloadReport/file?idDocument=BfF99nW06C&idRequest=miB6Ni0U5S}{Español})/(\href{https://portafirma.ual.es/pfirma/downloadReport/file?idDocument=4u61Ie5es2&idRequest=ZeBY35LlFa}{English}).

\section{Chronogram}

\subsection{Spanish group}
\begin{tabular}{|l|l|l|}
  \hline
  Start & End & Task \\
  \hline
  2023/11/28 - 16:00 & 2023/12/04 - 15:59 & \href{https://sistemas-multimedia.github.io/contents/working_environment/}{Working Environment} \\
  2023/11/28 - 16:00 & 2023/12/04 - 15:59 & \href{https://sistemas-multimedia.github.io/contents/entropy_coding/}{Entropy Coding - Theory} \\
  2023/11/28 - 16:00 & 2024/01/30 - 18:00 & \href{https://sistemas-multimedia.github.io/contents/entropy_coding/#x1-110007}{Entropy Coding - To-Do} \\
  2023/12/05 - 16:00 & 2023/12/12 - 15:59 & \href{https://sistemas-multimedia.github.io/contents/quantization/}{Digital (Re-)Quantization - Theory} \\
  2023/12/05 - 16:00 & 2024/01/30 - 18:00 & \href{https://sistemas-multimedia.github.io/contents/quantization/#x1-150008}{Digital (Re-)Quantization - To-Do} \\
  2023/12/12 - 16:00 & 2023/12/14 - 15:59 & \href{https://sistemas-multimedia.github.io/contents/color_transforms/}{Color Transforms - Theory} \\
  2023/12/12 - 16:00 & 2024/01/30 - 18:00 & \href{https://sistemas-multimedia.github.io/contents/color_transforms/#x1-100006}{Color Transforms - To-Do} \\
  2023/12/14 - 16:00 & 2023/12/20 - 15:59 & \href{https://sistemas-multimedia.github.io/contents/spatial_transforms/}{Spatial Transforms - Theory} \\
  2023/12/14 - 16:00 & 2024/01/30 - 18:00 & \href{https://sistemas-multimedia.github.io/contents/spatial_transforms/#x1-80006}{Spatial Transforms - To-Do} \\
  2023/12/20 - 16:00 & 2024/01/17 - 15:59 & \href{https://sistemas-multimedia.github.io/contents/temporal_transforms/}{Temporal Transforms - Theory} \\
  2023/12/20 - 16:00 & 2024/01/30 - 18:00 & \href{https://sistemas-multimedia.github.io/contents/temporal_transforms/#x1-90008}{Temporal Transforms - To-Do} \\
  2024/01/10 - 16:00 & 2024/01/17 - 15:59 & \href{https://sistemas-multimedia.github.io/contents/perceptual_coding/}{Perceptual Coding - Theory} \\
  2024/01/10 - 16:00 & 2024/01/30 - 18:00 & \href{https://sistemas-multimedia.github.io/contents/perceptual_coding/#x1-100009}{Perceptual Coding - To-Do} \\
  2024/01/17 - 16:00 & 2024/01/24 - 15:59 & \href{https://sistemas-multimedia.github.io/contents/data_scalability/}{Code-stream Scalability - Theory} \\
  2024/01/17 - 16:00 & 2024/01/30 - 18:00 & \href{https://sistemas-multimedia.github.io/contents/data_scalability/#x1-150008}{Code-stream Scalability - To-Do} \\
  2024/01/24 - 16:00 & 2024/01/24 - 15:59 & \href{https://sistemas-multimedia.github.io/contents/standards/}{Image and Video Coding ``Standards'' - Theory} \\
  2024/01/17 - 16:00 & 2024/01/30 - 18:00 & \href{https://sistemas-multimedia.github.io/contents/standards/#x1-40003/}{Image and Video Coding ``Standards'' - To-Do} \\
  \hline
\end{tabular}

\subsection{English group}
\begin{tabular}{|l|l|l|}
  \hline
  Start & End & Task \\
  \hline
  2023/11/28 - 18:00 & 2023/12/04 - 17:59 & \href{https://sistemas-multimedia.github.io/contents/working_environment/}{Working Environment} \\
  2023/11/28 - 18:00 & 2023/12/04 - 17:59 & \href{https://sistemas-multimedia.github.io/contents/entropy_coding/}{Entropy Coding - Theory} \\
  2023/11/28 - 18:00 & 2024/01/30 - 18:00 & \href{https://sistemas-multimedia.github.io/contents/entropy_coding/#x1-110007}{Entropy Coding - To-Do} \\
  2023/12/05 - 18:00 & 2023/12/12 - 17:59 & \href{https://sistemas-multimedia.github.io/contents/quantization/}{Digital (Re-)Quantization - Theory} \\
  2023/12/05 - 18:00 & 2024/01/30 - 18:00 & \href{https://sistemas-multimedia.github.io/contents/quantization/#x1-150008}{Digital (Re-)Quantization - To-Do} \\
  2023/12/12 - 18:00 & 2023/12/14 - 17:59 & \href{https://sistemas-multimedia.github.io/contents/color_transforms/}{Color Transforms - Theory} \\
  2023/12/12 - 18:00 & 2024/01/30 - 18:00 & \href{https://sistemas-multimedia.github.io/contents/color_transforms/#x1-100006}{Color Transforms - To-Do} \\
  2023/12/14 - 18:00 & 2023/12/20 - 17:59 & \href{https://sistemas-multimedia.github.io/contents/spatial_transforms/}{Spatial Transforms - Theory} \\
  2023/12/14 - 18:00 & 2024/01/30 - 18:00 & \href{https://sistemas-multimedia.github.io/contents/spatial_transforms/#x1-80006}{Spatial Transforms - To-Do} \\
  2023/12/20 - 18:00 & 2024/01/17 - 17:59 & \href{https://sistemas-multimedia.github.io/contents/temporal_transforms/}{Temporal Transforms - Theory} \\
  2023/12/20 - 18:00 & 2024/01/30 - 18:00 & \href{https://sistemas-multimedia.github.io/contents/temporal_transforms/#x1-90008}{Temporal Transforms - To-Do} \\
  2024/01/10 - 18:00 & 2024/01/17 - 17:59 & \href{https://sistemas-multimedia.github.io/contents/perceptual_coding/}{Perceptual Coding - Theory} \\
  2024/01/10 - 18:00 & 2024/01/30 - 18:00 & \href{https://sistemas-multimedia.github.io/contents/perceptual_coding/#x1-100009}{Perceptual Coding - To-Do} \\
  2024/01/17 - 18:00 & 2024/01/24 - 17:59 & \href{https://sistemas-multimedia.github.io/contents/data_scalability/}{Code-stream Scalability - Theory} \\
  2024/01/17 - 18:00 & 2024/01/30 - 18:00 & \href{https://sistemas-multimedia.github.io/contents/data_scalability/#x1-150008}{Code-stream Scalability - To-Do} \\
  2024/01/24 - 18:00 & 2024/01/24 - 17:59 & \href{https://sistemas-multimedia.github.io/contents/standards/}{Image and Video Coding ``Standards'' - Theory} \\
  2024/01/17 - 18:00 & 2024/01/30 - 18:00 & \href{https://sistemas-multimedia.github.io/contents/standards/#x1-40003/}{Image and Video Coding ``Standards'' - To-Do} \\
  \hline
\end{tabular}
 