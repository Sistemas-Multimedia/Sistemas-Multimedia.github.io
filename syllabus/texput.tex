\newcommand{\SM}{\href{http://cms.ual.es/UAL/estudios/masteres/plandeestudios/asignaturas/asignatura/MASTER7114?idAss=71142105&idTit=7114}{Sistemas Multimedia}}

\newcommand{\theproject}{\href{}}{MCDWT}
\title{\SM - Syllabus}

\maketitle

\section{About this course}

This course shows the basic principles used to encode digital images
and videos, and how to apply them in commercial and scientific
environments.

Images and sequences of images (videos) are everywhere, and one of the
main problems that arise when we use them is the huge amount of data
that they require. This course introduces the theoretical contents
based on the development of a video encoding project. Therefore, it is
an imminently practical course, organized into a collection of
milestones.

\section{Timetable}

\begin{description}
\item [Week 1: {\normalfont Preparing the working
    environment.}] \emph{We will use a Linux operating system with
  Git and Python to develop our project.}
  \begin{description}
  \item [Milestone 1: {\normalfont \href{https://sistemas-multimedia.github.io/milestones/01-provisioning/}{Operating system provisioning}.}]
  \item [Milestone 2: {\normalfont \href{https://sistemas-multimedia.github.io/milestones/02-python/}{Installation and basic programming with Python}.}]
  \item [Milestone 3: {\normalfont \href{https://sistemas-multimedia.github.io/milestones/03-git/}{Git, GitHub and the fork-and-branch Git workflow}.}]
  %\item [Milestone 4: {\normalfont \href{https://sistemas-multimedia.github.io/milestones/04-the_data/}{Understanding the video data}.}]
  \end{description}
  ~\newline

\item [Week 2: {\normalfont Removing color redundancy.}] \emph{Color
    components are correlated, and human beings are more sensitive to
    slow variations of the color than to high frequency changes.}
  \begin{description}
  \item [Milestone 4: {\normalfont \href{https://sistemas-multimedia.github.io/milestones/04-Motion_PNG/}{Entropy coding of images with PNG}.}]
  \item [Milestone 5: {\normalfont \href{https://sistemas-multimedia.github.io/milestones/05-RGB_quantization/}{Quantizing in the RGB domain}.}]
  \item [Milestone 6: {\normalfont \href{https://sistemas-multimedia.github.io/milestones/06-YUV_quantization/}{Quantizing in the YUV domain}.}] %{Decorrelating with Color Transforms}.
  \end{description}
  ~\newline

\item [Week 3: {\normalfont Removing spatial redundancy I.}]
  \emph{Pixels are spatially correlated, and human beings prefer low
    spatial frequencies to high ones.}
  \begin{description}
  %\item [Milestone 4: {\normalfont \href{}{DPCM (Differential Pulse Code Modulation) (TO-DO)}.}]
  \item [Milestone 7: {\normalfont \href{}{DCT (unfinished)}.}] % Engloba hitos 7, 8 y 9. Aparte sería interesante hablar de la DCT.
  \end{description}
  ~\newline

\item [Week 4: {\normalfont Removing spatial redundancy I.}]
  \emph{More spatial decorrelating techniques.}
  \begin{description}
  \item [Milestone 8: {\normalfont \href{https://sistemas-multimedia.github.io/milestones/07-DWT/}{DWT (Discrete Wavelet Transform)}.}] %{Energy Concentration and Spatial Multiresolution with the Discrete Wavelet Transform (DWT)}.
  \item [Milestone 8: {\normalfont \href{https://sistemas-multimedia.github.io/milestones/08-2D-DWT/}{Removing Redundancy with the 2D-DWT}.}] % Juntar
  \item [Milestone 8: {\normalfont \href{https://sistemas-multimedia.github.io/milestones/09-MDWT/}{The Motion Discrete Wavelet Transform}.}] % Juntar
  \end{description}
  ~\newline

\item [Week 5: {\normalfont Removing spatial redundancy I.}]
  \emph{Even more spatial decorrelating techniques.}
  \begin{description}
 \item [Milestone 9: {\normalfont \href{}{LP (Laplacian Pyramid) (unfinished)}.}]
  \end{description}
  ~\newline

\item [Week 6: {\normalfont Removing temporal redundancy in the image domain.}]
  \emph{Video images are correlated in time, too!}
  \begin{description}
  \item [Milestone 10: {\normalfont \href{https://sistemas-multimedia.github.io/milestones/10-ME/}{Motion estimation and compensation in the image domain}.}]
  \item [Milestone 11: {\normalfont \href{https://sistemas-multimedia.github.io/milestones/11-image_domain_IPP/}{IPP... coding in the image domain}.}]
  \end{description}
  ~\newline

\item [Week 7: {\normalfont Removing temporal redundancy in the transform domain.}]
  \emph{Increasing data scalability.}
  \begin{description}
  \item [Milestone 12: {\normalfont \href{https://sistemas-multimedia.github.io/milestones/12-transform_domain_MC/}{Motion compensation in the transform domain}.}]
  \item [Milestone 13: {\normalfont \href{https://sistemas-multimedia.github.io/milestones/13-transform_domain_IPP/}{IPP... coding in the transform domain}.}]
  \end{description}
  ~\newline

%\item [Week 6: {\normalfont MCTF (Motion Compensated Temporal Filtering) in the imagen domain (TO-DO).}]
%  ~\newline

%\item [Week 7: {\normalfont IPP... coding in the DWT/LP domain.}]
%  \begin{description}
%  \item [Milestone 11: {\normalfont \href{https://sistemas-multimedia.github.io/milestones/11-MC_in_DWT_domain/}{Motion Compensation in the DWT Domain}.}]
%  \item [Milestone 12: {\normalfont \href{https://sistemas-multimedia.github.io/milestones/12-IPP_coding/}{IPP... coding in MRVC (Multi Resolution Video Codec)}.}]
%  \end{description}
%  ~\newline
%\item [Week 8: {\normalfont IBP... coding in the DWT/LP domain (TO-DO).}]
%  ~\newline
%\item [Week 9: {\normalfont MCTF in the DWT/LP domain (TO-DO).}]
\end{description}

