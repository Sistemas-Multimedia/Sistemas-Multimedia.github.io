%\newcommand{\SM}{\href{http://cms.ual.es/UAL/estudios/masteres/plandeestudios/asignaturas/asignatura/MASTER7114?idAss=71142105&idTit=7114}{Sistemas Multimedia}}
\newcommand{\SM}{\href{https://sistemas-multimedia.github.io/}{Sistemas Multimedia}}

\newcommand{\theproject}{\href{https://github.com/Sistemas-Multimedia/MRVC}{MRVC}}

\newcommand{\SW}{\href{https://github.com/Sistemas-Multimedia/MRVC}{MRVC}}

\title{\SM - Syllabus}

\maketitle

\section{About this course}

This course shows the basic principles used to encode digital images
and videos, and how to apply them in commercial and scientific
environments.

Images and sequences of images (videos) are everywhere, and one of the
main problems that arise when we use them is the huge amount of data
that they require. This course introduces the theoretical contents
based on the development of a video encoding project. Therefore, it is
an imminently practical course, organized into a collection of
milestones.

\section{Timetable}

\begin{description}
\item [Week 1: {\normalfont Preparing the working environment.}]
  \begin{description}
  \item [Milestone 1: {\normalfont \href{https://sistemas-multimedia.github.io/milestones/01-provisioning/}{Operating system Provisioning}.}]
  \item [Milestone 2: {\normalfont \href{https://sistemas-multimedia.github.io/milestones/02-python/}{Installation and basic programming with Python}.}]
  \item [Milestone 3: {\normalfont \href{https://sistemas-multimedia.github.io/milestones/03-git/}{Git, GitHub and the Fork-and-Branch Git Workflow}.}]
  \end{description}
  ~\newline
\item [Week 2: {\normalfont Video color domains.}]
  \begin{description}
  \item [Milestone 4: {\normalfont \href{https://sistemas-multimedia.github.io/milestones/04-the_data/}{Understanding the Video Data}.}]
  \item [Milestone 5: {\normalfont \href{https://sistemas-multimedia.github.io/milestones/05-quantization/}{Quantizing in the RGB Domain}.}]
  \item [Milestone 6: {\normalfont \href{https://sistemas-multimedia.github.io/milestones/06-color_transform/}{Removing Color Redundancy}.}] %{Decorrelating with Color Transforms}.
  \end{description}
  ~\newline
\item [Week 3: {\normalfont 2D transforms (spatial domains).}]
  \begin{description}
  \item [Milestone 7: {\normalfont \href{}{DCT (Discrete Cosine Transform) (TO-DO)}.}] % Engloba hitos 7, 8 y 9. Aparte sería interesante hablar de la DCT.
  \item [Milestone 8: {\normalfont \href{https://sistemas-multimedia.github.io/milestones/07-DWT/}{DWT (Discrete Wavelet Transform)}.}] %{Energy Concentration and Spatial Multiresolution with the Discrete Wavelet Transform (DWT)}.
  \item [Milestone 8: {\normalfont \href{https://sistemas-multimedia.github.io/milestones/08-2D-DWT/}{Removing Redundancy with the 2D-DWT}.}] % Juntar
  \item [Milestone 8: {\normalfont \href{https://sistemas-multimedia.github.io/milestones/09-MDWT/}{The Motion Discrete Wavelet Transform}.}] % Juntar
  \item [Milestone 9: {\normalfont \href{}{LP (Laplacian Pyramid)}.}]
  \end{description}
  ~\newline
\item [Week 4: {\normalfont IPP... motion (estimation and) compensation in the image domain.}]
  \begin{description}
  \item [Milestone 10: {\normalfont \href{https://sistemas-multimedia.github.io/milestones/10-ME/}{Motion Estimation}.}] % En el dominio de la imagen
  \end{description}
  ~\newline
\item [Week 5: {\normalfont IBP... motion compensation in the image domain (TO-DO).}]
  \begin{description}
  \item [Milestone 10: {\normalfont \href{https://sistemas-multimedia.github.io/milestones/10-ME/}{Motion Estimation}.}] % En el dominio de la imagen
  \end{description}
  ~\newline
\item [Week 6: {\normalfont MCTF (Motion Compensated Temporal Filtering) in the imagen domain (TO-DO).}]
  ~\newline
\item [Week 7: {\normalfont IPP... motion compensation in the DWT/LP domain.}]
  \begin{description}
  \item [Milestone 11: {\normalfont \href{https://sistemas-multimedia.github.io/milestones/11-MC_in_DWT_domain/}{Motion Compensation in the DWT Domain}.}]
  \item [Milestone 12: {\normalfont \href{https://sistemas-multimedia.github.io/milestones/12-IPP_coding/}{IPP... coding in MRVC (Multi Resolution Video Codec)}.}]
  \end{description}
  ~\newline
\item [Week 8: {\normalfont IBP... motion compensation in the DWT/LP domain (TO-DO).}]
  ~\newline
\item [Week 9: {\normalfont MCTF in the DWT/LP domain (TO-DO).}]
\end{description}

