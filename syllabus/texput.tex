\newcommand{\SM}{\href{http://cms.ual.es/UAL/estudios/masteres/plandeestudios/asignaturas/asignatura/MASTER7114?idAss=71142105&idTit=7114}{Sistemas Multimedia}}

\newcommand{\theproject}{\href{}}{MCDWT}
\title{\SM - Syllabus}

\maketitle

\section{About this course}

This course analyses the basic principles used to encode digital
images and videos, and how to apply such techniques in commercial and
scientific environments.

Images and sequences of images (videos) are everywhere, and one of the
main problems that arise when we use them is the huge amount of data
that they require. This course introduces theoretical and practical
contents based on the development of a video encoding project,
organized into a collection of milestones.

\section{Timetable}

\begin{description}
\item [Milestone 1: {\normalfont Preparing the working
    environment.}] {\emph{Usually, open source frameworks increase the portability and eases the deployment.}
  \begin{description}
  \item [Step 1: {\normalfont
      \href{https://sistemas-multimedia.github.io/content/provisioning/}{Operating
        system provisioning}.}]
  \item [Step 2: {\normalfont
      \href{https://sistemas-multimedia.github.io/content/python/}{Installation
        and basic programming with Python}.}]
  \item [Step 3: {\normalfont
      \href{https://sistemas-multimedia.github.io/content/git/}{Git,
        GitHub and the fork-and-branch Git workflow}.}]
  %\item [Milestone 4: {\normalfont \href{https://sistemas-multimedia.github.io/milestones/04-the_data/}{Understanding the video data}.}]
  \end{description}
  ~\newline

\item [Milestone 2: {\normalfont Exploiting the color redundancy:}]
  \emph{In color images, the RGB channels tend to be highly
  correlated. We can take advantage of this channel correlation to
  decrease the entropy and therefore, increase the compression ratio}.
  \begin{description}
  \item [Task 4: {\normalfont
      \href{htpps://sistemas-multimedia.github.io/content/color_DCT/}{The
        1D-DCT (1-Dimensional Discrete Cosine Transform) transform}.}]
  \item [Task 5: {\normalfont
      \href{htpps://sistemas-multimedia.github.io/content/YCrCb/}{The
        YCrCb transform}.}]
  \item [Task 6: {\normalfont
      \href{htpps://sistemas-multimedia.github.io/content/YCoCg/}{The
        YCoCg transform}.}]
  \end{description}
  ~\newline

\item [Milestone 3: {\normalfont Removing the spatial redundancy:}]
  \emph{Visual data is highly correlated in the spatial domain. Let's
  see how we can decrease the entropy by exploiting the spatial
  correlation in images}.
  \begin{description}
  \item [Task 7: {\normalfont
      \href{https://sistemas-multimedia.github.io/content/DPCM/}{2D-DPCM
        (2-Dimensional Differential Pulse Code Modulation)}.}]
  \item [Task 8: {\normalfont
      \href{https://sistemas-multimedia.github.io/content/spatialDCT/}{The
        2D-DCT (2-Dimensional Discrete Cosine Transform)}.}]
  \item [Task 9: {\normalfont
      \href{https://sistemas-multimedia.github.io/content/DWT/}{The
        2D-DWT (2-Dimensional Discrete Wavelet Transform)}.}]
  \end{description}
  ~\newline

\item [Milestone 4: {\normalfont Entropy coding:}] \emph{So far, the
minimization of the entropy has been our main objective. Let's see how
can we use an entropy codec to build a
\href{https://github.com/Sistemas-Multimedia/RIC}{lossless (or near
  lossless, depending on the implementation of the inter-channel and
  intra-channels transforms) image codec}.
  \begin{description}
  \item [Task 10: {\normalfont
      \href{https://sistemas-multimedia.github.io/content/PNG/}{PNG (Portable Network Grachics)}.}]
  \item [Task 11: {\normalfont
      \href{https://sistemas-multimedia.github.io/content/J2K/}{JPEG (Joint Photographic Expert Group) 2000}.}]
  \item [Task 12: {\normalfont
      \href{https://sistemas-multimedia.github.io/content/CBAC/}{CBAC (Context-Based Arithmetic Coding)}.}]
  \end{description}
  ~\newline

\item [Milestone 5: {\normalfont Rate-control through quantization:}]
  \emph{In most images, a large percentage of the code-stream is
  dedicated to encode inappreciable (by humans) visual
  information. Quantization can be used to remove such information and
  providing higher compression ratios}.
  \begin{description}
  \item [Task 13: {\normalfont
      \href{https://sistemas-multimedia.github.io/content/quantization/}{Quantization in the transform domain}.}]
  \item [Task 14: {\normalfont
      \href{https://sistemas-multimedia.github.io/content/RDO/}{RDO (Rate/Distortion Optimization).}]
  \end{description}
  ~\newline

\item [Milestone 6: {\normalfont Perceptual coding through chroma
    ``subsampling'':}] \emph{Humans hardly perceive high frequency
content in the chroma components and for this reason,
\href{https://github.com/Sistemas-Multimedia/IIC}{chroma information
  can be low-pass filtered without a significative perceived
  distortion}}.
  \begin{description}
  \item [Task 15: {\normalfont
      \href{https://sistemas-multimedia.github.io/content/chroma_filtering/}{Chroma filtering}.}]
  \end{description}
  ~\newline

\item [Milestone 7: {\normalfont Motion estimation:}] \emph{Most image
sequences are highly correlated in the temporal domain and motion
estimation can exploit it}.
  \begin{description}
  \item [Task 16: {\normalfont
    \href{https://sistemas-multimedia.github.io/content/motion_estimation/}{Motion estimation}.}]
  \end{description}
  ~\newline

\item [Milestone 8: {\normalfont Motion compensation:}] \emph{With the
motion information it is possible to make predictions that help to
decrease the entropy}.
  \begin{description}
  \item [Task 17: {\normalfont
      \href{https://sistemas-multimedia.github.io/content/motion_compensation/}{Motion compensation}.}]
  \end{description}
  ~\newline

%%%%%%%
  
\item [Milestone 2: {\normalfont Reversible Image Compression (RIC):}]
  \begin{description}
  \item [Task 4: {\normalfont \href{htpps://sistemas-multimedia.github.io/content/YCrCb/}{The YCoCg color transform}.}]
  \item [Task 5: {\normalfont \href{https://sistemas-multimedia.github.io/content/DWT/}{The Discrete Wavelet Transform (DWT)}.}]
  \item [Task 6: {\normalfont \href{https://sistemas-multimedia.github.io/content/PNG/}{The Portable Network Graphics format (PNG)}.}]
  \item [Task 7: {\normalfont \href{https://sistemas-multimedia.github.io/content/RIC_R_opt/}{Rate optimization}.}]
  \end{description}
  ~\newline

\item [Milestone 3: {\normalfont Irreversible Image Compression (IIC):}]
  \begin{description}
  \item [Task 8: {\normalfont \href{htpps://sistemas-multimedia.github.io/content/IIC_quantization/}{Quantization in the DWT domain}.}]
  \item [Task 9: {\normalfont \href{https://sistemas-multimedia.github.io/content/IIC_RD_opt/}{Rate/Distortion optimization in IIC}.}]
  \end{description}
  ~\newline

\item [Milestone 4: {\normalfont Progressive Image Compression (PIC):}]
  \begin{description}
  \item [Task 10: {\normalfont \href{htpps://sistemas-multimedia.github.io/content/PIC_prog_enc/}{Progressive encoding in the DWT domain}.}]
  \item [Task 11: {\normalfont \href{https://sistemas-multimedia.github.io/content/PIC_RD_opt/}{Rate/Distortion optimization in PIC}.}]
  \end{description}
  ~\newline

\item [Milestone 5: {\normalfont Progressive Image Compression (PIC):}]
  \begin{description}
  \item [Task 10: {\normalfont \href{htpps://sistemas-multimedia.github.io/content/PIC_prog_enc/}{Progressive encoding in the DWT domain}.}]
  \item [Task 11: {\normalfont \href{https://sistemas-multimedia.github.io/content/PIC_RD_opt/}{Rate/Distortion optimization in PIC}.}]
  \end{description}
  ~\newline


  
\item [Day 2: {\normalfont Compressing color images.}] \emph{Quantization and entropy coding, two key signal compression techniques.}
  \begin{description}
  \item [Milestone 4: {\normalfont \href{https://sistemas-multimedia.github.io/milestones/04-PNG/}{Entropy coding of images with PNG}.}]
  \item [Milestone 5: {\normalfont \href{https://sistemas-multimedia.github.io/milestones/05-RGB_compression/}{Compresion of RGB images}.}]
  \end{description}
  ~\newline

\item [Day 3: {\normalfont Removing color redundancy.}] \emph{Color
    components are correlated, and human beings are more sensitive to
    slow variations of the color than to high frequency changes.}
  \begin{description}
   \item [Milestone 6: {\normalfont \href{https://sistemas-multimedia.github.io/milestones/06-YUV_compression/}{Compression in a luma/chroma domain}.}]
  \end{description}
  ~\newline

\item [Day 4: {\normalfont Removing spatial redundancy with transforms I.}]
  \emph{Quantization in the transform domain is more efficient than in the image domain.}
  \begin{description}
  \item [Milestone 7: {\normalfont \href{https://sistemas-multimedia.github.io/milestones/07-DCT/}{The 2D-DCT (Discrete Cosine Transform)}.}]
  \end{description}
  ~\newline

\item [Day 5: {\normalfont Removing spatial redundancy with transforms II.}]
  \emph{Providing spatial multiresolution to image coding.}
  \begin{description}
  \item [Milestone 8: {\normalfont \href{https://sistemas-multimedia.github.io/milestones/08-DWT/}{The 2D-DWT (Discrete Wavelet Transform)}.}]
  %\item [Milestone 9: {\normalfont \href{}{LP (Laplacian Pyramid) (unfinished)}.}]
  \end{description}
  ~\newline

\begin{comment}
\item [Day 6: {\normalfont RDO in the III... domain.}]
  \emph{Rate/Distortion Optimization in a sequence of YCoCg+DCT
  (Intracoded) frames.}
  \begin{description}
  \item [Milestone 9: {\normalfont \href{https://sistemas-multimedia.github.io/milestones/09-III_coding/}{III\_coding}.}]
  \end{description}
  ~\newline
\end{comment}

\item [Day 6: {\normalfont Motion estimation.}]
  \emph{Video sequences are correlated in time, and determining how the objects move can help to increase the compression ratio.}
  \begin{description}
  \item [Milestone 9: {\normalfont \href{https://sistemas-multimedia.github.io/milestones/09-ME/}{Motion estimation}.}]
  \end{description}
  ~\newline

\item [Day 7: {\normalfont IPP... coding.}]
  \emph{Building a video compressor based on spatial and IPP... temporal decorrelation and entropy coding.}
  \begin{description}
  \item [Milestone 10: {\normalfont \href{https://sistemas-multimedia.github.io/milestones/10-image_domain_IPP/}{IPP... coding}.}]
  \end{description}
  ~\newline

\item [Day 8: {\normalfont MRVC (Muti-Resolution Video Coding).}]
  \emph{Providing spatial multiresolution to video coding.}
  \begin{description}
  \item [Milestone 11: {\normalfont \href{https://sistemas-multimedia.github.io/milestones/11-transform_domain_MC/}{Motion compensation in the transform domain}.}]
  \item [Milestone 12: {\normalfont \href{https://sistemas-multimedia.github.io/milestones/12-transform_domain_IPP/}{IPP... coding in the transform domain}.}]
  \end{description}
  
\begin{comment}
\item [Day 8: {\normalfont The final project.}]
  \emph{Incorporating chroma subsampling to the YCoCg+DCT+ME video compressor.}

\item [Days 7: {\normalfont MRVC (Muti-Resolution Video Coding).}]
  \emph{Providing spatial multiresolution to video coding.}
  \begin{description}
  \item [Milestone 11: {\normalfont \href{https://sistemas-multimedia.github.io/milestones/12-transform_domain_MC/}{Motion compensation in the transform domain}.}]
  \item [Milestone 12: {\normalfont \href{https://sistemas-multimedia.github.io/milestones/13-transform_domain_IPP/}{IPP... coding in the transform domain}.}]
  \end{description}
\end{comment}

\end{description}

