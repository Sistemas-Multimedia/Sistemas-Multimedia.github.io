% Emacs, this is -*-latex-*-

\newcommand{\SM}{\href{http://cms.ual.es/UAL/estudios/masteres/plandeestudios/asignaturas/asignatura/MASTER7114?idAss=71142105&idTit=7114}{Sistemas Multimedia}}

\newcommand{\theproject}{\href{}}{MCDWT}

\title{\SM{} - Entropy coding with
  \href{https://vicente-gonzalez-ruiz.github.io/PNG/}{PNG} (Portable
  Network Graphics)}

\maketitle

\tableofcontents

\section{Description}

This task introduces
\href{https://en.wikipedia.org/wiki/Portable_Network_Graphics}{PNG}~\cite{vruiz__PNG}
(pronounced ``ping'') a
\href{https://en.wikipedia.org/wiki/Lossless_compression}{lossless
  image compression format} used in this course for representing
\href{https://en.wikipedia.org/wiki/Digital_data}{digital}
\href{https://en.wikipedia.org/wiki/Digital_image}{images} and
\href{https://en.wikipedia.org/wiki/Video}{videos}~\cite{vruiz__image_video}.

In our experiments, we are going to store a video as a sequence of
\href{https://en.wikipedia.org/wiki/Computer_file}{files} with names
     {\tt <prefix>000.png}, {\tt <prefix>001.png}, ..., where {\tt
       <prefix>} specifies the sequence name and ``{\tt .png}''
     indicates that the images~\cite{vruiz__image_IO} are coded using
     the PNG image compression
     format~\cite{roelofs1999png,world2003portable}.

% We will use \href{https://vicente-gonzalez-ruiz.github.io/PNG/}{PNG}
% basically as an
% \href{https://en.wikipedia.org/wiki/Entropy_encoding}{entropy codec},
% which is a combination of
% \href{https://vicente-gonzalez-ruiz.github.io/Huffman_coding/}{Huffman
%   Coding} and
% \href{https://vicente-gonzalez-ruiz.github.io/LZ77/}{Lempel-Ziv-77
%   Coding}. However, notice that any other lossless image compressing
% file format (such as \href{https://en.wikipedia.org/wiki/TIFF}{TIFF})
% could be used.

%\item The pixel
%  \href{https://en.wikipedia.org/wiki/Luminous_intensity}{intensities}
%  are displaced to the center of the
%  \href{https://en.wikipedia.org/wiki/Range_(computer_programming)}{range
%    of values} ([0, 65535]) that the components can take (for example,
%  in \href{https://en.wikipedia.org/wiki/Disk_storage}{disk}, the
%  component intensity 0 is represented by 32768).
  
\section{What do I have to do?}

\begin{enumerate}
%\item Download \theproject{}, preferably cloning your own fork.
\item Run this \href{https://jupyter.org/}{jupyter}
  \href{https://github.com/Sistemas-Multimedia/Sistemas-Multimedia.github.io/blob/master/contents/PNG/display_video.ipynb}{notebook}
  that shows how to read and write PNG frames in Python.
%\item Please, try to dig a little bit into the insights of the notebook.
%\item Understand the insights of the notebook.
\end{enumerate}

\section{Timming}

Please, finish this task before the next class session.

\section{Deliverables}

None.

\section{Resources}

\renewcommand{\addcontentsline}[3]{}% Remove functionality of \addcontentsline
\bibliography{python,image-processing,image-compression,text-compression,image-video-theory,image-formats}
