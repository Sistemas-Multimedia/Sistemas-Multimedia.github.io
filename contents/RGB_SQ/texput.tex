% Emacs, this is -*-latex-*-

%\newcommand{\SM}{\href{http://cms.ual.es/UAL/estudios/masteres/plandeestudios/asignaturas/asignatura/MASTER7114?idAss=71142105&idTit=7114}{Sistemas Multimedia}}
\newcommand{\SM}{\href{https://sistemas-multimedia.github.io/}{Sistemas Multimedia}}

\newcommand{\theproject}{\href{https://github.com/Sistemas-Multimedia/MRVC}{MRVC}}

\newcommand{\SW}{\href{https://github.com/Sistemas-Multimedia/MRVC}{MRVC}}

\title{\SM{} - Study Guide - Milestone 3 - Task 4 - Scalar Quantization in the $\text{RGB}$ Color Domain}

\maketitle

\tableofcontents

\section{Some info about the $\text{RGB}$ color domain}
%{{{

In $\text{RGB}$ (Red, Green, and Blue) images, the color of a pixel
depends on the
\href{https://en.wikipedia.org/wiki/Visible_spectrum}{frequency of the
  light that the pixel represents}. Such information can be
represented in a number of different encoding systems known as
\href{https://en.wikipedia.org/wiki/Color_space}{color spaces}. Among
all those systems, the $\text{RGB}$ color space is the most used
because $\text{RGB}$ images can be obtained directly from the light
signal using color filters.\footnote{Specifically, a red (R) filter, a
green (G) filter and a blue (B) filter.}

%}}}

\section{Scalar quantization in the $\text{RGB}$ domain}
%{{{ 

The $\text{RGB}$ color domain is additive, which implies that the
quantization error is the same in the three color
components,\footnote{At least, if the quantization error is
insignificant compared to the amplitude of the components.}
$\text{R}$, $\text{G}$, and $\text{B}$. Therefore, in order to select
the same RD slope~\cite{vruiz__information_theory} in the three
components, the quantization pattern should be
\begin{equation}
  \Delta_\text{R} = \Delta_\text{G} = \Delta_\text{B}.
  \label{eq:same_delta}
\end{equation}

Unfortunately, Eq.~\eqref{eq:same_delta} not always will provide
optimality in a RD sense. The reason is that the quantization step
size does not defines accurately the neither the rate (R) nor the
distortion (D). This means that, if we want to be sure that the
quantization pattern is optimal, the RD slopes of each channel must be
taken intro consideration (see this
\href{https://github.com/Sistemas-Multimedia/Sistemas-Multimedia.github.io/blob/master/contents/RGB_quantization/RD_performance.ipynb}{notebook}).

%}}}

\section{What do I have to do?}

\begin{enumerate}
\item Please, run the
  \href{https://github.com/Sistemas-Multimedia/Sistemas-Multimedia.github.io/blob/master/contents/RGB_quantization/RD_performance.ipynb}{notebook}
  to learn some insights about the compression of RGB images.
\item Try to compress different images.
\end{enumerate}

\section{Timming}

Please, finish this milestone before the next class session.

\section{Deliverables}

None.

\section{Resources}

\renewcommand{\addcontentsline}[3]{}% Remove functionality of \addcontentsline
\bibliography{information_theory}
