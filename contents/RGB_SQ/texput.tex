% Emacs, this is -*-latex-*-

\newcommand{\SM}{\href{http://cms.ual.es/UAL/estudios/masteres/plandeestudios/asignaturas/asignatura/MASTER7114?idAss=71142105&idTit=7114}{Sistemas Multimedia}}

\newcommand{\theproject}{\href{}}{MCDWT}
\title{\SM{} - Study Guide - Scalar Quantization in the $\text{RGB}$ Color Domain}

\maketitle

\tableofcontents

\section{Objective}
In this task we are going to develop a rudimentery
\href{https://en.wikipedia.org/wiki/Lossy_compression}{lossy
  ($\text{RGB}$) image compressor} based on
SQ~\cite{vruiz__scalar_quantization} and PNG~\cite{vruiz__PNG}. We
will decide what quantization step will be applied to each
channel\footnote{Or component, both terms are
interchangeable. However, we will use channel to refer to the matrix
form by the same component of all the pixels of the image.}  of the
image, using the same step for all the pixel-components.

\section{Some info about the $\text{RGB}$ color domain}
%{{{

$\text{RGB}$ (Red, Green, and Blue) is an additive color system, which
means that all colors ``start'' with black and are created by adding
some intensity of the primary colors red, green and
blue~\cite{burger2016digital}. In $\text{RGB}$ images, the color of a
pixel depends on the
\href{https://en.wikipedia.org/wiki/Visible_spectrum}{frequency of the
  light that the pixel represents}. Such information can be
represented in a number of different encoding systems known as
\href{https://en.wikipedia.org/wiki/Color_space}{color spaces}. Among
all those systems, the $\text{RGB}$ color space is the most used
because $\text{RGB}$ images can be obtained directly from the light
signal using color filters.\footnote{Specifically, a red (R) filter, a
green (G) filter and a blue (B) filter.}

%}}}

\section{Scalar quantization in the $\text{RGB}$ domain}
%{{{ 

Supposing that we are using a dead-zone quantizer, a $\text{RGB}$
image can be quantized, channel by channel, using quantization steps
$\mathbf{\Delta}_{\text{R}}$, $\mathbf{\Delta}_{\text{G}}$, and
$\mathbf{\Delta}_{\text{B}}$ (being $\mathbf{\Delta}$ the
``quantization pattern''). A reasonable question that arises here is:
given a target bit-rate $R$ for the compressed frame, how the
quantization steps should be chosen to minimize the distortion?

At this point we can consider two different quantization
perspectives. In the first one, if we consider a strictly visual
effect, any alternative different from
\begin{equation}
  \mathbf{\Delta}_{\text{R}} = \mathbf{\Delta}_{\text{G}} = \mathbf{\Delta}_{\text{B}}
  \label{eq:simple_Q}
\end{equation}
will produce some alteration in the color (also called the
\href{https://en.wikipedia.org/wiki/Chrominance}{``chroma''}) of the
reconstructed image.

A second perspective considers only a pure
\href{https://en.wikipedia.org/wiki/Rate-distortion_theory}{Rate/Distortion
  (RD) performance}~\cite{vruiz__information_theory}. The $\text{RGB}$
color domain is additive, which means that the distortion generated in
the quantized image is the sum of the distortions generated in each of
the channels, when they are quantized independently. Therefore, the
optimal RD curve for the reconstructed image (using the $3$ channels)
can be obtained through the steps:
\begin{enumerate}
\item Compute the RD curve for each channel. Notice that here we must
  define a set of quantization steps for each channel. For simplicity,
  use the same set for the three channels.
\item Put all the RD points in a list and sort the list by the slopes
  of the points. Since the points have been ordered by their slope,
  the sequence of quantization patterns should generate the optimal
  curve.
\item Use the corresponding quantization patterns (defined by the points
  of the list) to compute the RD curve of the reconstructed image.
\end{enumerate}

This algorithm is independent of how the set of quantization steps
used in each channel has been selected. If the quantization steps used
to describe the RD curves of each channel are different, the points of
the definitive curve will be different, but not its shape or its
localization in the RD space.

If we are used the same quantization steps for each channel, the
points generated when
\begin{equation}
  \mathbf{\Delta}_{\text{R}} = \mathbf{\Delta}_{\text{G}} = \mathbf{\Delta}_{\text{B}}
\end{equation}
must occur in the definitive RD curve. Notice that the rest of points
of the curve should appear between these points, but do not
necessarily belong to the convex hull of the curve (altough they could
be used to perform a finer rate-control).

This algorithm has been implemented in this
\href{https://github.com/Sistemas-Multimedia/Sistemas-Multimedia.github.io/blob/master/contents/RGB_SQ/RGB_SQ.ipynb}{notebook}.

%}}}

\section{What do I have to do?}

\begin{enumerate}
\item Please, run the
  \href{https://github.com/Sistemas-Multimedia/Sistemas-Multimedia.github.io/blob/master/contents/RGB_quantization/RGB_SQ.ipynb}{notebook}
  to check if the previous theory really works.
\item Try to compress different images.
\end{enumerate}

\section{Timming}

Please, finish this milestone before the next class session.

\section{Deliverables}

None.

\section{Resources}

\renewcommand{\addcontentsline}[3]{}% Remove functionality of \addcontentsline
\bibliography{information_theory,image-formats}
