%\newcommand{\SM}{\href{http://cms.ual.es/UAL/estudios/masteres/plandeestudios/asignaturas/asignatura/MASTER7114?idAss=71142105&idTit=7114}{Sistemas Multimedia}}
\newcommand{\SM}{\href{https://sistemas-multimedia.github.io/}{Sistemas Multimedia}}

\newcommand{\theproject}{\href{https://github.com/Sistemas-Multimedia/MRVC}{MRVC}}

\newcommand{\SW}{\href{https://github.com/Sistemas-Multimedia/MRVC}{MRVC}}

\title{\SM{} - Image and Video Coding ``Standards''}

\maketitle

\tableofcontents

\section{What the standards define?}
Most of the encoding standards indicate the format of the code-stream
and how it should be decoded. About coding, they just provide
suggestions, and in general, it is an undefined process.


\section{A comparison}

\begin{tabular}{lrrrrrrrrrrr}
     &           &                          &          &           &              &         &              & Code-stream & & & \\
     &           &                          &          &   Maximum &              &         &              & scalability & & & \\
     &           &                          &          &     number&      Maximum &         &      Maximum &  T=temporal & & & \\
     &           &                          &          &   of bits &      spatial & Maximum &          bit &   S=Spatial & & & \\
     &   Name of &                          &          &       per &   resolution & picture &         rate &   Q=Quality & & Lossless & Patents \\
Year & the codec & Common encoding pipeline & Main use & component &(rows x cols) &    rate & (only video) &       R=\href{https://en.wikipedia.org/wiki/Region_of_interest}{ROI} & \href{https://en.wikipedia.org/wiki/Stereoscopy}{Stereo} & option & free\\
\hline

% Analog Video
1951 & % Year
\href{https://en.wikipedia.org/wiki/Video#Analog_video}{Analog Video} & % Codec
\href{https://en.wikipedia.org/wiki/Analog_signal}{Analog} \href{https://en.wikipedia.org/wiki/YUV}{YUV} & % Encoding
\href{https://en.wikipedia.org/wiki/Television}{TV} & % Main use
\href{https://en.wikipedia.org/wiki/N/A}{NA} & % bits
\href{https://en.wikipedia.org/wiki/Interlaced_video}{576}$\times$\href{https://en.wikipedia.org/wiki/N/A}{NA} (\href{https://en.wikipedia.org/wiki/PAL}{PAL}) & % Max resolution
30 (\href{https://en.wikipedia.org/wiki/NTSC}{NTSC}) & % Max pic-rate
\href{https://en.wikipedia.org/wiki/N/A}{NA} & % Max BR
No & % Scalability
No & % Stereo
NA & % Lossless
\href{https://en.wikipedia.org/wiki/N/A}{NA} \\ % free

% H.120
1984 & % Year
\href{https://en.wikipedia.org/wiki/H.120}{H.120} & % Codec
(\href{https://en.wikipedia.org/wiki/Chroma_subsampling#4:2:2}{$\text{E}'_{\text{Y}}$, $\text{E}'_{\text{R}} – \text{E}'_{\text{Y}}$, $\text{E}'_{\text{B}} – \text{E}'_{\text{Y}}$})+\href{https://en.wikipedia.org/wiki/Motion_compensation}{MC}+\href{https://en.wikipedia.org/wiki/Quantization_(signal_processing)}{SQ}+\href{https://en.wikipedia.org/wiki/Huffman_coding}{Huffman} & % Encoding
\href{https://en.wikipedia.org/wiki/Videotelephony}{Videotelephony} & % Main use
8 & % bits
$720\times 576$ (\href{https://en.wikipedia.org/wiki/PAL}{PAL}) & % Max res
30 (\href{https://en.wikipedia.org/wiki/NTSC}{NTSC}) & % Max pic-rate
2048 \href{https://en.wikipedia.org/wiki/Data-rate_units}{kbps} & % Max BR
No & % Scalability
No & % Stereo
No & % Lossless
\href{https://en.wikipedia.org/wiki/Term_of_patent}{Expired} \\ % free

% Digital Video
1986 & % Year
\href{https://en.wikipedia.org/wiki/Digital_video}{Digital Video} & % Codec
\href{https://en.wikipedia.org/wiki/Chroma_subsampling#4:2:2}{4:2:2}/\href{https://en.wikipedia.org/wiki/JPEG#Color_space_transformation}{YCbCr}+\href{https://en.wikipedia.org/wiki/Pulse-code_modulation}{PCM} & % Encoding
\href{https://en.wikipedia.org/wiki/Digital_television}{Digital TV} & % Main use
8 & % bits
$720\times 576$ (\href{https://en.wikipedia.org/wiki/PAL}{PAL}) & % Max res
30 (\href{https://en.wikipedia.org/wiki/NTSC}{NTSC}) & % Max pic-rate
140 \href{https://en.wikipedia.org/wiki/Data-rate_units}{Mbps} & % Max BR
No & % Scalability
No & % Stereo
NA & % Lossless
\href{https://en.wikipedia.org/wiki/Term_of_patent}{Expired} \\ % free

% GIF
1987 & % Year
\href{https://en.wikipedia.org/wiki/GIF}{GIF} & % Codec
\href{https://en.wikipedia.org/wiki/Color_quantization}{Color VQ}+\href{https://en.wikipedia.org/wiki/Lempel%E2%80%93Ziv%E2%80%93Welch}{LZW} & % Encoding
\href{https://en.wikipedia.org/wiki/GIF#Usage}{Web} & % Main use
8 & % bits
\href{https://www.w3.org/Graphics/GIF/spec-gif89a.txt}{$2^{16}-1\times 2^{16}-1$} & % Max res
\href{https://www.w3.org/Graphics/GIF/spec-gif89a.txt}{100} & % Max pic-rate
Undefined & % Max BR
\href{https://www.pcmag.com/encyclopedia/term/interlaced-gif}{S} & % Scalability
No & % Stereo
No & % Lossless
\href{https://en.wikipedia.org/wiki/Term_of_patent}{Expired} \\ % free

% JPEG
1992 & % Year
\href{https://en.wikipedia.org/wiki/JPEG}{JPEG} & % Codec
\href{https://en.wikipedia.org/wiki/Chroma_subsampling#4:2:0}{4:2:0}/\href{https://en.wikipedia.org/wiki/JPEG#Color_space_transformation}{YCbCr}+\href{https://en.wikipedia.org/wiki/Discrete_cosine_transform}{DCT}+\href{https://en.wikipedia.org/wiki/Quantization_(signal_processing)}{SQ}+\href{https://en.wikipedia.org/wiki/Run-length_encoding}{RLE}+\href{https://en.wikipedia.org/wiki/Huffman_coding}{Huffman} & % Encoding
\href{https://en.wikipedia.org/wiki/Image_compression}{Web} & % Main use
12 & % bits
$(2^{16}-1)\times (2^{16}-1$) & % Max res
\href{https://en.wikipedia.org/wiki/N/A}{NA} & % Max pic-rate
\href{https://en.wikipedia.org/wiki/N/A}{NA} & % Max BR
\href{https://www.liquidweb.com/kb/what-is-a-progressive-jpeg/}{S} & % Scalability
No & % Stereo
No & % Lossless
\href{https://en.wikipedia.org/wiki/Term_of_patent}{Expired} \\ % free

% Motion JPEG
1992 & % Year
\href{https://en.wikipedia.org/wiki/Motion_JPEG}{Motion JPEG} & % Codec
\href{https://en.wikipedia.org/wiki/Chroma_subsampling#4:2:0}{4:2:0}/\href{https://en.wikipedia.org/wiki/JPEG#Color_space_transformation}{YCbCr}+\href{https://en.wikipedia.org/wiki/Discrete_cosine_transform}{DCT}+\href{https://en.wikipedia.org/wiki/Quantization_(signal_processing)}{SQ}+\href{https://en.wikipedia.org/wiki/Run-length_encoding}{RLE}+\href{https://en.wikipedia.org/wiki/Huffman_coding}{Huffman} & % Encoding
\href{https://en.wikipedia.org/wiki/Image_compression}{Video edition} & % Main use
8 & % bits
$(2^{16}-1)\times (2^{16}-1)$ & % Max res
undefined & % Max pic-rate
undefined & % Max BR
TS & % Scalability
No & % Stereo
No & % Lossless
\href{https://en.wikipedia.org/wiki/Term_of_patent}{Expired} \\ % free

% MPEG-1 Part 2
1993 & % Year
\href{https://en.wikipedia.org/wiki/MPEG-1#Part_2:_Video}{MPEG-1 Part 2} & % Codec
\href{https://en.wikipedia.org/wiki/Chroma_subsampling#4:2:0}{4:2:0}/\href{https://en.wikipedia.org/wiki/JPEG#Color_space_transformation}{YCbCr}+\href{https://en.wikipedia.org/wiki/Motion_compensation}{MC}+\href{https://en.wikipedia.org/wiki/Discrete_cosine_transform}{DCT}+\href{https://en.wikipedia.org/wiki/Quantization_(signal_processing)}{SQ}+\href{https://en.wikipedia.org/wiki/Run-length_encoding}{RLE}+\href{https://en.wikipedia.org/wiki/Huffman_coding}{Huffman} & % Encoding
\href{https://en.wikipedia.org/wiki/Video_CD}{Video CD} & % Main use
8 & % bits
$352\times 288$ & % Max res
30 (\href{https://en.wikipedia.org/wiki/NTSC}{NTSC}) & % Max pic-rate
1150 \href{https://en.wikipedia.org/wiki/Data-rate_units}{kbps} & % Max BR
\href{https://www.etsist.upm.es/uploaded/392/02_Video_Compression_Basics_MPEG1_MPEG2_MPEG4.pdf}{T} & % Scalability
No & % Stereo
No & % Lossless
\href{https://en.wikipedia.org/wiki/Term_of_patent}{Expired} \\ % free

% MPEG-2 Video (H.262)
1995 & % Year
\href{https://en.wikipedia.org/wiki/H.262/MPEG-2_Part_2}{MPEG-2 Video (H.262)} & % Codec
\href{https://en.wikipedia.org/wiki/Chroma_subsampling#4:2:0}{4:2:0}/\href{https://en.wikipedia.org/wiki/JPEG#Color_space_transformation}{YCbCr}+\href{https://en.wikipedia.org/wiki/Motion_compensation}{MC}+\href{https://en.wikipedia.org/wiki/Discrete_cosine_transform}{DCT}+\href{https://en.wikipedia.org/wiki/Quantization_(signal_processing)}{SQ}+\href{https://en.wikipedia.org/wiki/Run-length_encoding}{RLE}+\href{https://en.wikipedia.org/wiki/Huffman_coding}{Huffman} & % Encoding
\href{https://en.wikipedia.org/wiki/DVD}{DVD} \& \href{https://en.wikipedia.org/wiki/DVB}{DVB} & % Main use
8 & % bits
$1920\times 1080$ & % Max res
60 (\href{https://en.wikipedia.org/wiki/NTSC}{NTSC}) & % Max pic-rate
300 \href{https://en.wikipedia.org/wiki/Data-rate_units}{Mbps} & % Max BR
\href{https://www.etsist.upm.es/uploaded/392/02_Video_Compression_Basics_MPEG1_MPEG2_MPEG4.pdf}{T}\href{https://mpeg.chiariglione.org/standards/mpeg-2/video}{SQ} & % Scalability
No & % Stereo
No & % Lossless
\href{https://en.wikipedia.org/wiki/Term_of_patent}{Expired} \\ % free

% DV
1995 & % Year
\href{https://en.wikipedia.org/wiki/DV}{DV} & % Codec
\href{https://en.wikipedia.org/wiki/Chroma_subsampling#4:2:0}{4:2:0}/\href{https://en.wikipedia.org/wiki/JPEG#Color_space_transformation}{YCbCr}+\href{https://en.wikipedia.org/wiki/Discrete_cosine_transform}{DCT}+\href{https://en.wikipedia.org/wiki/Quantization_(signal_processing)}{SQ}+\href{https://en.wikipedia.org/wiki/Run-length_encoding}{RLE}+\href{https://en.wikipedia.org/wiki/Huffman_coding}{Huffman} & % Encoding
\href{https://en.wikipedia.org/wiki/Digital_television}{Digital TV} & % Main use
8 & % bits
$720\times 576$ (\href{https://en.wikipedia.org/wiki/PAL}{PAL}) & % Max res
60 (\href{https://en.wikipedia.org/wiki/NTSC}{NTSC}) & % Max pic-rate
50 \href{https://en.wikipedia.org/wiki/Data-rate_units}{Mbps} & % Max BR
No & % Scalability
No & % Stereo
No & % Lossless
\href{https://en.wikipedia.org/wiki/Term_of_patent}{Expired} \\ % free

% H.263
1996 & % Year
\href{https://en.wikipedia.org/wiki/H.263}{H.263} & % Codec
\href{https://en.wikipedia.org/wiki/Chroma_subsampling#4:2:0}{4:2:0}/\href{https://en.wikipedia.org/wiki/JPEG#Color_space_transformation}{YCbCr}+\href{https://en.wikipedia.org/wiki/Motion_compensation}{MC}+\href{https://en.wikipedia.org/wiki/Discrete_cosine_transform}{DCT}+\href{https://en.wikipedia.org/wiki/Quantization_(signal_processing)}{SQ}+\href{https://en.wikipedia.org/wiki/Run-length_encoding}{RLE}+\href{https://en.wikipedia.org/wiki/Huffman_coding}{Huffman}+\href{https://en.wikipedia.org/wiki/Deblocking_filter}{DF} & % Encoding
\href{https://en.wikipedia.org/wiki/Integrated_Services_Digital_Network}{ISDN} \href{https://en.wikipedia.org/wiki/Videotelephony}{videotelephony} & % Main use
8 & % bits
$1408\times 1152$ & % Max res
30 & % Max pic-rate
$n\times 64$ \href{https://en.wikipedia.org/wiki/Data-rate_units}{kbps} & % Max BR
\href{https://en.wikipedia.org/wiki/H.263#H.263v2_(H.263+)}{TRS} & % Scalability
No & % Stereo
No & % Lossless
\href{https://en.wikipedia.org/wiki/Term_of_patent}{Expired} \\ % free

% PNG
1987 & % Year
\href{https://en.wikipedia.org/wiki/Portable_Network_Graphics}{PNG} & % Codec
\href{https://en.wikipedia.org/wiki/Portable_Network_Graphics#Compression}{DPCM}+\href{https://en.wikipedia.org/wiki/Deflate}{DEFLATE} & % Encoding
\href{https://www.w3.org/TR/png/}{Web} & % Main use
\href{https://en.wikipedia.org/wiki/Portable_Network_Graphics#Critical_chunks}{16} & % bits
\href{https://en.wikipedia.org/wiki/Portable_Network_Graphics#Critical_chunks}{$2^{32}-1\times 2^{32}-1$} & % Max res
\href{https://www.w3.org/Graphics/GIF/spec-gif89a.txt}{100} & % Max pic-rate
Undefined & % Max BR
\href{https://www.pcmag.com/encyclopedia/term/interlaced-gif}{S} & % Scalability
No & % Stereo
Yes & % Lossless
No \\ % free

% MPEG-4 Visual
1999 & % Year
\href{https://en.wikipedia.org/wiki/MPEG-4_Part_2}{MPEG-4 Visual} & % Codec
\href{https://en.wikipedia.org/wiki/Chroma_subsampling#4:2:0}{4:2:0}/\href{https://en.wikipedia.org/wiki/JPEG#Color_space_transformation}{YCbCr}+\href{https://en.wikipedia.org/wiki/Motion_compensation}{MC}+\href{https://en.wikipedia.org/wiki/Discrete_cosine_transform}{DCT}+\href{https://en.wikipedia.org/wiki/Quantization_(signal_processing)}{SQ}+\href{https://en.wikipedia.org/wiki/Run-length_encoding}{RLE}+\href{https://en.wikipedia.org/wiki/Huffman_coding}{Huffman} & % Encoding
\href{https://en.wikipedia.org/wiki/DivX}{``Internet movies''} & % Main use
8 & % bits
$1408\times 1152$ & % Max res
30 & % Max pic-rate
$n\times 64$ \href{https://en.wikipedia.org/wiki/Data-rate_units}{kbps} & % Max BR
No & % Scalability
No & % Stereo
No & % Lossless
\href{https://en.wikipedia.org/wiki/Term_of_patent}{Expired} \\ % free

% JPEG 2000
2000 & % Year
\href{https://en.wikipedia.org/wiki/JPEG_2000}{JPEG 2000} & % Codec
(\href{https://en.wikipedia.org/wiki/Chroma_subsampling#4:2:0}{4:2:0}/\href{https://en.wikipedia.org/wiki/JPEG#Color_space_transformation}{YCbCr})|\href{https://en.wikipedia.org/wiki/JPEG_2000#Color_components_transformation}{RCT}+\href{https://en.wikipedia.org/wiki/Discrete_wavelet_transform}{DWT}+\href{https://en.wikipedia.org/wiki/Quantization_(signal_processing)}{SQ}+\href{https://ijcsi.org/papers/IJCSI-8-4-1-531-536.pdf}{EBCOT} & % Encoding
\href{https://en.wikipedia.org/wiki/Medical_imaging}{Medical imaging} & % Main use
32 & % bits
$(2^{32}-1)\times (2^{32}-1)$ & % Max res
\href{https://en.wikipedia.org/wiki/N/A}{NA} & % Max pic-rate
\href{https://en.wikipedia.org/wiki/N/A}{NA} & % Max BR
SQR & % Scalability
No & % Stereo
Yes & % Lossless
No \\ % free

% Motion JPEG 2000
2001 & % Year
\href{https://en.wikipedia.org/wiki/Motion_JPEG_2000}{Motion JPEG 2000} & % Codec
(\href{https://en.wikipedia.org/wiki/Chroma_subsampling#4:2:0}{4:2:0}/\href{https://en.wikipedia.org/wiki/JPEG#Color_space_transformation}{YCbCr})|\href{https://en.wikipedia.org/wiki/YCoCg}{YCoCg}+\href{https://en.wikipedia.org/wiki/Discrete_wavelet_transform}{DWT}+\href{https://en.wikipedia.org/wiki/Quantization_(signal_processing)}{SQ}+\href{https://ijcsi.org/papers/IJCSI-8-4-1-531-536.pdf}{EBCOT} & % Encoding
\href{https://en.wikipedia.org/wiki/Digital_cinema}{Digital cinema} & % Main use
32 & % bits
\href{https://en.wikipedia.org/wiki/Digital_cinema}{$7680\times 4320$} (\href{https://en.wikipedia.org/wiki/8K_resolution}{8K UHD}) & % Max res
\href{https://en.wikipedia.org/wiki/Digital_cinema}{48} & % Max pic-rate
\href{https://en.wikipedia.org/wiki/Digital_Cinema_Package}{250} \href{https://en.wikipedia.org/wiki/Data-rate_units}{Mbps} & % Max BR
TSQR & % Scalability
No & % Stereo
Yes & % Lossless
No \\ % free

% MPEG-4 AVC (H.264)
2003 & % Year
\href{https://en.wikipedia.org/wiki/Advanced_Video_Coding}{MPEG-4 AVC (H.264)} & % Codec
(\href{https://en.wikipedia.org/wiki/Chroma_subsampling#4:2:0}{4:2:0}/\href{https://en.wikipedia.org/wiki/JPEG#Color_space_transformation}{YCbCr})+\href{https://en.wikipedia.org/wiki/Motion_compensation}{MC}+\href{https://en.wikipedia.org/wiki/Discrete_cosine_transform}{DCT}+\href{https://en.wikipedia.org/wiki/Quantization_(signal_processing)}{SQ}+\href{https://en.wikipedia.org/wiki/Context-adaptive_variable-length_coding}{CABAC} & % Encoding
\href{https://en.wikipedia.org/wiki/Blu-ray}{Blue-ray} \and \href{https://en.wikipedia.org/wiki/DVB}{DVB} & % Main use
10 & % bits
\href{https://en.wikipedia.org/wiki/Advanced_Video_Coding}{$7680\times 4320$} (\href{https://en.wikipedia.org/wiki/8K_resolution}{8K UHD}) & % Max res
\href{https://en.wikipedia.org/wiki/Advanced_Video_Coding#Levels}{130} & % Max pic-rate
\href{https://en.wikipedia.org/wiki/Advanced_Video_Coding#Levels}{800} \href{https://en.wikipedia.org/wiki/Data-rate_units}{Mbps} & % Max BR
TSQ & % Scalability
Yes & % Stereo
Yes & % Lossless
No \\ % free

% Theora
2004 & % Year
\href{https://en.wikipedia.org/wiki/Theora}{Theora} & % Codec
(\href{https://en.wikipedia.org/wiki/Chroma_subsampling#4:2:0}{4:2:0}/\href{https://en.wikipedia.org/wiki/JPEG#Color_space_transformation}{YCbCr})+\href{https://en.wikipedia.org/wiki/Motion_compensation}{MC}+\href{https://en.wikipedia.org/wiki/Discrete_cosine_transform}{DCT}+\href{https://en.wikipedia.org/wiki/Quantization_(signal_processing)}{SQ}+\href{https://en.wikipedia.org/wiki/Run-length_encoding}{RLE}+\href{https://en.wikipedia.org/wiki/Huffman_coding}{Huffman} & % Encoding
\href{https://en.wikipedia.org/wiki/Use_of_Ogg_formats_in_HTML5}{Web} & % Main use
8 & % bits
\href{https://theora.org/doc/Theora.pdf}{$1048560 × 1048560$} & % Max res
Undefined & % Max pic-rate
Undefined & % Max BR
T & % Scalability
No & % Stereo
No & % Lossless
Yes \\ % free

% VC-1
2006 & % Year
\href{https://en.wikipedia.org/wiki/VC-1}{VC-1} & % Codec
(\href{https://en.wikipedia.org/wiki/Chroma_subsampling#4:2:0}{4:2:0}/\href{https://en.wikipedia.org/wiki/JPEG#Color_space_transformation}{YCbCr})+\href{https://en.wikipedia.org/wiki/Motion_compensation}{MC}+\href{https://en.wikipedia.org/wiki/Discrete_cosine_transform}{DCT}+\href{https://en.wikipedia.org/wiki/Quantization_(signal_processing)}{SQ}+\href{https://en.wikipedia.org/wiki/Run-length_encoding}{RLE}+\href{https://en.wikipedia.org/wiki/Huffman_coding}{Huffman} & % Encoding
\href{https://en.wikipedia.org/wiki/Microsoft_Silverlight}{Microsoft Silverlight} & % Main use
8 & % bits
\href{https://en.wikipedia.org/wiki/VC-1}{$2048\times 1536$} & % Max res
\href{https://en.wikipedia.org/wiki/VC-1}{60} & % Max pic-rate
\href{https://en.wikipedia.org/wiki/VC-1}{135} \href{https://en.wikipedia.org/wiki/Data-rate_units}{Mbps} & % Max BR
T & % Scalability
No & % Stereo
No & % Lossless
No \\ % free

% Apple ProRes
2007 & % Year
\href{https://en.wikipedia.org/wiki/Apple_ProRes}{Apple ProRes} & % Codec
(\href{https://en.wikipedia.org/wiki/Chroma_subsampling#4:2:2}{4:2:2}/\href{https://en.wikipedia.org/wiki/JPEG#Color_space_transformation}{YCbCr})+\href{https://en.wikipedia.org/wiki/Discrete_cosine_transform}{DCT}+\href{https://en.wikipedia.org/wiki/Quantization_(signal_processing)}{SQ}+\href{https://en.wikipedia.org/wiki/Run-length_encoding}{RLE}+\href{https://en.wikipedia.org/wiki/Huffman_coding}{Huffman} & % Encoding
\href{https://en.wikipedia.org/wiki/Image_compression}{Video edition} & % Main use
12 & % bits
\href{https://en.wikipedia.org/wiki/Apple_ProRes}{$7680\times 4320$} (\href{https://en.wikipedia.org/wiki/8K_resolution}{8K UHD}) & % Max res
\href{https://en.wikipedia.org/wiki/Apple_ProRes#Data_rates}{50} & % Max pic-rate
\href{https://en.wikipedia.org/wiki/Apple_ProRes#Data_rates}{4424} \href{https://en.wikipedia.org/wiki/Data-rate_units}{Mbps} & % Max BR
T & % Scalability
No & % Stereo
No & % Lossless
No \\ % free

% JPEG XR (AVC intra)
2014 & % Year
\href{https://en.wikipedia.org/wiki/JPEG_XR}{JPEG XR} & % Codec
\href{https://en.wikipedia.org/wiki/Quadtree}{Quandtree}+\href{https://en.wikipedia.org/wiki/Chroma_subsampling#4:2:0}{4:2:0}/\href{https://en.wikipedia.org/wiki/JPEG#Color_space_transformation}{YCbCr}+\href{https://en.wikipedia.org/wiki/Quantization_(signal_processing)}{SQ}+\href{https://en.wikipedia.org/wiki/Context-adaptive_variable-length_coding}{CABAC} & % Encoding
\href{https://en.wikipedia.org/wiki/JPEG_XR}{Web} & % Main use
\href{https://en.wikipedia.org/wiki/JPEG_XR#Compression_algorithm}{32} & % bits
\href{bellard.org/bpg/}{$2^{32}-1\times 2^{32}-1$} & % Max res
\href{https://en.wikipedia.org/wiki/N/A}{NA} & % Max pic-rate
\href{https://en.wikipedia.org/wiki/N/A}{NA} & % Max BR
None & % Scalability
No & % Stereo
\href{https://en.wikipedia.org/wiki/JPEG_XR#Capabilities}{Yes} & % Lossless
No \\ % free

% WebP
2010 & % Year
\href{https://en.wikipedia.org/wiki/WebP}{WebP} & % Codec
\href{https://en.wikipedia.org/wiki/Chroma_subsampling#4:2:0}{4:2:0}/\href{https://en.wikipedia.org/wiki/JPEG#Color_space_transformation}{YCbCr} & % Encoding
\href{https://en.wikipedia.org/wiki/Image_file_format#WebP}{Web} & % Main use
\href{https://en.wikipedia.org/wiki/WebP#Restrictions}{8} & % bits
\href{https://en.wikipedia.org/wiki/WebP#Technology}{$2^{14}-1\times 2^{14}-1$} & % Max res
\href{https://en.wikipedia.org/wiki/N/A}{NA} & % Max pic-rate
Undefined & % Max BR
None & % Scalability
No & % Stereo
Yes & % Lossless
Yes \\ % free

% HEVC (H.265)
2013 & % Year
\href{https://en.wikipedia.org/wiki/High_Efficiency_Video_Coding}{HEVC (H.265)} & % Codec
(\href{https://en.wikipedia.org/wiki/Chroma_subsampling#4:2:0}{4:2:0}/\href{https://en.wikipedia.org/wiki/JPEG#Color_space_transformation}{YCbCr})+\href{https://en.wikipedia.org/wiki/Motion_compensation}{MC}+\href{https://en.wikipedia.org/wiki/Discrete_cosine_transform}{DCT}+\href{https://en.wikipedia.org/wiki/Quantization_(signal_processing)}{SQ}+\href{https://en.wikipedia.org/wiki/Context-adaptive_variable-length_coding}{CABAC} & % Encoding
\href{https://en.wikipedia.org/wiki/High_Efficiency_Video_Coding}{Video compression} & % Main use
16 & % bits
\href{https://en.wikipedia.org/wiki/High_Efficiency_Video_Coding}{$8192\times 4320$} & % Max res
\href{https://en.wikipedia.org/wiki/High_Efficiency_Video_Coding#Tiers_and_levels}{128} & % Max pic-rate
\href{https://en.wikipedia.org/wiki/High_Efficiency_Video_Coding#Tiers_and_levels}{800} \href{https://en.wikipedia.org/wiki/Data-rate_units}{Mbps} & % Max BR
TSQ & % Scalability
Yes & % Stereo
Yes & % Lossless
No \\ % free

% BPG
2014 & % Year
\href{https://en.wikipedia.org/wiki/Better_Portable_Graphics}{BPG} & % Codec
\href{https://en.wikipedia.org/wiki/Quadtree}{Quandtree}+\href{https://en.wikipedia.org/wiki/Chroma_subsampling#4:2:0}{4:2:0}/\href{https://en.wikipedia.org/wiki/JPEG#Color_space_transformation}{YCbCr}+\href{https://en.wikipedia.org/wiki/Quantization_(signal_processing)}{SQ}+\href{https://en.wikipedia.org/wiki/Context-adaptive_variable-length_coding}{CABAC} & % Encoding
\href{https://en.wikipedia.org/wiki/Better_Portable_Graphics}{Web} & % Main use
\href{https://en.wikipedia.org/wiki/Better_Portable_Graphics}{14} & % bits
\href{bellard.org/bpg/}{$2^{32}-1\times 2^{32}-1$} & % Max res
\href{https://en.wikipedia.org/wiki/N/A}{NA} & % Max pic-rate
\href{https://en.wikipedia.org/wiki/N/A}{NA} & % Max BR
None & % Scalability
No & % Stereo
Yes & % Lossless
Yes \\ % free

% AV1
2018 & % Year
\href{https://en.wikipedia.org/wiki/AV1}{AV1} & % Codec
(\href{https://en.wikipedia.org/wiki/Chroma_subsampling#4:2:0}{4:2:0}/\href{https://en.wikipedia.org/wiki/JPEG#Color_space_transformation}{YCbCr})+\href{https://en.wikipedia.org/wiki/Motion_compensation}{MC}+\href{https://en.wikipedia.org/wiki/Discrete_cosine_transform}{DCT}+\href{https://en.wikipedia.org/wiki/Quantization_(signal_processing)}{SQ}+\href{https://en.wikipedia.org/wiki/Range_coding}{Range coding} & % Encoding
\href{https://en.wikipedia.org/wiki/AV1}{Web} & % Main use
\href{https://en.wikipedia.org/wiki/AV1}{12} & % bits
\href{}{$8192\times 4320$} & % Max res
\href{https://en.wikipedia.org/wiki/AV1#Levels}{300} & % Max pic-rate
\href{https://en.wikipedia.org/wiki/AV1#Levels}{800} \href{https://en.wikipedia.org/wiki/Data-rate_units}{Mbps} & % Max BR
\href{https://en.wikipedia.org/wiki/AV1#Scalable_video_coding}{TS} & % Scalability
No & % Stereo
Yes & % Lossless
Yes \\ % free

% VVC (H.266)
2020 & % Year
\href{https://en.wikipedia.org/wiki/Versatile_Video_Coding}{VVC (H.266)} & % Codec
(\href{https://en.wikipedia.org/wiki/Chroma_subsampling#4:2:0}{4:2:0}/\href{https://en.wikipedia.org/wiki/JPEG#Color_space_transformation}{YCbCr})+\href{https://en.wikipedia.org/wiki/Motion_compensation}{MC}+\href{https://en.wikipedia.org/wiki/Discrete_cosine_transform}{DCT}+\href{https://en.wikipedia.org/wiki/Quantization_(signal_processing)}{SQ}+\href{https://en.wikipedia.org/wiki/Context-adaptive_variable-length_coding}{CABAC} & % Encoding
\href{https://en.wikipedia.org/wiki/Versatile_Video_Coding#Broadcast}{Broadcast} & % Main use
\href{https://en.wikipedia.org/wiki/Versatile_Video_Coding#Concept}{16} & % bits
\href{https://en.wikipedia.org/wiki/16K_resolution}{16K} & % Max res
\href{https://en.wikipedia.org/wiki/Versatile_Video_Coding#Concept}{120} & % Max pic-rate
Undefined & % Max BR
\href{https://en.wikipedia.org/wiki/Versatile_Video_Coding#Concept}{TSQ} & % Scalability
\href{https://en.wikipedia.org/wiki/Versatile_Video_Coding#Concept}{Yes} & % Stereo
\href{https://en.wikipedia.org/wiki/Versatile_Video_Coding#Concept}{Yes} & % Lossless
No \\ % free

\end{tabular}

\section{References}

\renewcommand{\addcontentsline}[3]{}% Remove functionality of \addcontentsline
\bibliography{projects}

\begin{comment}

$B$ = Block structure name
$S$ = Possible block sizes
$T$ = Tiling scheme name
$P$ = Configurable Perceptual Quantization
$D$ = Deblocking filtering
$A$ = Adaptive deblocking filtering
QP = Relation between Quantization Parameter and $\Delta$ (quantization step size)
MBD = Maximum Bit Depth

In chronologial order:

\section{JPEG}

\section{BPG}

\section{JPEG2000}

\section{ \href{https://en.wikipedia.org/wiki/JPEG_XL}{JPEG XL}}

\section{MPEG-1}

\section{MPEG-2}

\section{MPEG-4}

\section{H.264/AVC}

$B$=Macroblock
$S=\{16x16\}$
$T$=slice
$P$=yes
$D$=yes
$A$=yes
Modes = All intra, random access, low delay

\section{HEVC}

$B$=Tree Coding Unit
$S=\{8x8, 64x64\}$
$P$=yes
$D$=yes
$A$=yes
$\text{QP}=2^{\frac{\text{QP}-4}{6}}$
MBD = 10
Modes = All intra, random access, low delay

RS 
TSQ = Temporal, Spatial and Quality scalability

\end{comment}
