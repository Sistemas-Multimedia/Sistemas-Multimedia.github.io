% Emacs, this is -*-latex-*-

%\newcommand{\SM}{\href{http://cms.ual.es/UAL/estudios/masteres/plandeestudios/asignaturas/asignatura/MASTER7114?idAss=71142105&idTit=7114}{Sistemas Multimedia}}
\newcommand{\SM}{\href{https://sistemas-multimedia.github.io/}{Sistemas Multimedia}}

\newcommand{\theproject}{\href{https://github.com/Sistemas-Multimedia/MRVC}{MRVC}}

\newcommand{\SW}{\href{https://github.com/Sistemas-Multimedia/MRVC}{MRVC}}

\title{\SM{} - Scalar Quantization of Gray-scale Images}

\maketitle
\tableofcontents

\section{Objective}
This task shows how to implement a very simple
\href{https://en.wikipedia.org/wiki/Lossy_compression}{lossy
  (gray-scale)~\cite{vruiz__image_IO} image compressor} based on
Scalar
Quantization~\cite{vruiz__scalar_quantization,sayood2017introduction,vetterli2014foundations}
and PNG~\cite{vruiz__PNG}. This compressor basically removes the
number of different gray tones in the image
(\href{https://en.wikipedia.org/wiki/Visual_system}{humans} hardly
distinguish more than 64 different gray colors).

\section{Scalar quantization applied to the ``intensity'' domain}
The \href{}{compressor} uses a deadzone quantizer, and the quantization steps
are $\Delta\in\{128, 64, 32, 16, 8, 4, 2, 1\}$ (powers of $2$),
simulating a \href{https://en.wikipedia.org/wiki/Bit_plane}{bit-plane}
encoding. Since the quantization step sizes are predetermined, no rate
or distortion~\cite{vruiz__information_theory} control has been
incorporated to the compressor.

\section{What do I have to do?}
\begin{enumerate}
\item Which redundancies are we exploiting with our image compressor? 
\item Create a new notebook named ``bit-rate\_control.ipynb'' that,
  given a maximum bit-rate, finds the quantization step $\Delta^*$
  that minimizes the distortion. Use the
  quantizer~\cite{vruiz__scalar_quantization} that your prefer (see
  the notebooks
  \href{https://github.com/Sistemas-Multimedia/Sistemas-Multimedia.github.io/blob/master/contents/gray_SQ/gray_SQ_LloydMax.ipynb}{gray\_SQ\_LloydMax.ipynb},
  \href{https://github.com/Sistemas-Multimedia/Sistemas-Multimedia.github.io/blob/master/contents/gray_SQ/gray_SQ_companded.ipynb}{gray\_SQ\_companded.ipynb},
  \href{https://github.com/Sistemas-Multimedia/Sistemas-Multimedia.github.io/blob/master/contents/gray_SQ/gray_SQ_midrise.ipynb}{gray\_SQ\_midrise.ipynb},
  and
  \href{https://github.com/Sistemas-Multimedia/Sistemas-Multimedia.github.io/blob/master/contents/gray_SQ/gray_SQ_midtread.ipynb}{gray\_SQ\_midtread.ipynb}).
\end{enumerate}

\section{Timming}
Please, finish this milestone before the next class session.

\section{Deliverables}
None.

\section{Resources}
\renewcommand{\addcontentsline}[3]{} % Remove functionality of \addcontentsline
\bibliography{data-compression,signal-processing,DWT,image-processing,quantization,information_theory}

%%%%%%%%%%%

\begin{comment}
\section{Distortion control}
We do not provide any accurate distortion control algorithm. Our
compressor is able to control the number of gray tones in the
reconstructed image, using a deadzone scalar quantizer. The
quantization steps are selected simulating a
\href{https://en.wikipedia.org/wiki/Bit_plane}{bit-plane}
encoding. Therefore, for 8-bit images, the quantization steps are
$\Delta\in\{128, 64, 32, 16, 8, 4, 2, 1\}$ (powers of $2$). Notice
that we cannot control the distortion in terms of the RMSE, for
example.

\section{Rate control}
The bit-rate (the number of bits/pixel) obtained after using
scalar quantization quantization and PNG depends on:
\begin{enumerate}
\item The number of output indexes generated by the quantizer.
\item The capability of PNG for removing spatial/statistical
  redudancy, aspect that we cannot control with accuracy (only some
  levels of compression are usually available).
\item In the case of SQ, the quantization step size $\Delta$
  used. Notice that, usually, the higher the $\Delta$, the higher the
  compression ratio, the lower the number of bits/pixel, and the lower
  the quality of the reconstructed image.
\end{enumerate}

Therefore, variying $\Delta$ we should be able to generate a Rate/Distortion (RD)
curve~\cite{vruiz__information_theory}, where the $x$-axis represents the bit-rate (in bit/pixel, for
example) and the $y$-axis represents the distortion (the
\href{https://en.wikipedia.org/wiki/Root-mean-square_deviation}{Root
  Mean Square Error (RMSE)}, for example). However, notice that only seven RD points will be available\footnote{This is a direct consequence of using bit-plane encoding.}, and the rate of each point is not controllable.\footnote{Because we cannot control the bit-rate with PNG.}
\end{comment}
