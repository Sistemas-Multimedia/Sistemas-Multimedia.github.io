% Emacs, this is -*-latex-*-

%\newcommand{\SM}{\href{http://cms.ual.es/UAL/estudios/masteres/plandeestudios/asignaturas/asignatura/MASTER7114?idAss=71142105&idTit=7114}{Sistemas Multimedia}}
\newcommand{\SM}{\href{https://sistemas-multimedia.github.io/}{Sistemas Multimedia}}

\newcommand{\theproject}{\href{https://github.com/Sistemas-Multimedia/MRVC}{MRVC}}

\newcommand{\SW}{\href{https://github.com/Sistemas-Multimedia/MRVC}{MRVC}}

\title{\SM{} - Study Guide - Milestone 3 - Task 1: Scalar Quantization of Gray-scale Images}

\maketitle
\tableofcontents

\section{Objective}

This task shows how to implement a
\href{https://en.wikipedia.org/wiki/Lossy_compression}{lossy
  (gray-scale) image compressor} based on Scalar
Quantization~\cite{vruiz__scalar_quantization} and
PNG~\cite{vruiz__PNG}.

\href{https://en.wikipedia.org/wiki/Visual_system}{Humans} are quite
efficient recognizing the information stored in images (and videos),
even when this information has been degraded or partially
lost. Quantization~\cite{vruiz__scalpar_quantization,sayood2017introduction,vetterli2014foundations}
is a technique that can remove the visual information that is less
relevant for us, and implies a
\href{https://en.wikipedia.org/wiki/Lossy_compression}{lossy coding},
which provides
\href{https://en.wikipedia.org/wiki/Data_compression_ratio}{compression
  ratios}~\cite{vruiz__information_theory} usually at least one order
of magnitude higher than using
\href{https://en.wikipedia.org/wiki/Lossless_compression}{lossless
  coding}.

\section{Description}
%{{{

Gray-Scale (GS) images have only one (usually
\href{https://en.wikipedia.org/wiki/Luminance}{luminance}, expressed
proportionally by the gray colors) channel\footnote{They are usually
represented in color displays using gray tones (colors where the same
amount of red, green and blue components have been mixed). However, it
is quite difficult to find a color display with more than 256 gray
colors.}. In most of the cases, the number of bits per pixel (the so
called ``depth'' of the image) is $8$, but it is not rare to use up to
$16$ bits/pixel in specific contexts such as in
\href{https://en.wikipedia.org/wiki/Medical_imaging}{Medical Imaging}
and \href{https://en.wikipedia.org/wiki/Remote_sensing}{Remote
  Sensing}~\cite{burger2016digital}.

In this milestone we are going to see how to
\href{https://vicente-gonzalez-ruiz.github.io/quantization/}{quantize}
a GS image, and then compressing the image with PNG, to create a
\href{https://en.wikipedia.org/wiki/Lossy_compression}{lossy image
  compressor}.

\subsection{Scalar and vector quantization of GS images}
%{{{

The compression ratio of our lossy image compressor can be controled
by means of quantization. At this point we have basically two
alternatives:
\begin{enumerate}
\item \textbf{Scalar Quantization (SQ)}: where each pixel is quantized
  without considering the rest of pixels in the
  image~\cite{vruiz__scalar_quantization}.
\item \textbf{Vector Quantization (VQ)}: when the pixels are quantized
  block-by-block (2D vectors)~\cite{vruiz__vector_quantization}.
\end{enumerate}
Notice that VQ exploits the spatial correlation, but SQ doesn't. For
this reason, by default, we will use PNG after SQ, because PNG can
remove the spatial redundancy. In the case of VQ, only the statistical
redundancy remains, that can be exploited by any entropy codec (and
obviously, any image codec).

%}}}

\subsection{Bit-rate control in SQ + PNG}
%{{{

The bit-rate (the number of bits/pixel) obtained after using
\emph{quantization+PNG} depends on:
\begin{enumerate}
\item The number of output indexes generated by the quantizer.
\item The capability of PNG for removing spatial/statistical
  redudancy, aspect that we cannot control with accuracy (only some
  levels of compression are usually available).
\item In the case of SQ, the quantization step size $\Delta$
  used. Notice that, usually, the higher the $\Delta$, the higher the
  compression ratio, the lower the number of bits/pixel, and the lower
  the quality of the reconstructed image.
\end{enumerate}
Variying $\Delta$ we should be able to generate a Rate/Distortion (RD)
curve, where the $x$-axis represents the bit-rate (in bit/pixel, for
example) and the $y$-axis represents the distortion (the
\href{https://en.wikipedia.org/wiki/Root-mean-square_deviation}{Root
  Mean Square Error (RMSE)}, for example).

Normaly, RD curves are convex~\cite{vruiz__information_theory} (this
can be seen in the notebooks
\href{https://github.com/Sistemas-Multimedia/Sistemas-Multimedia.github.io/blob/master/contents/scalar_quantization/midtread.ipynb}{Midtread Quantization},
\href{https://github.com/Sistemas-Multimedia/Sistemas-Multimedia.github.io/blob/master/contents/scalar_quantization/midrise.ipynb}{Midrise Quantization},
\href{https://github.com/Sistemas-Multimedia/Sistemas-Multimedia.github.io/blob/master/contents/scalar_quantization/deadzone.ipynb}{Deadzone Quantization},
\href{https://github.com/Sistemas-Multimedia/Sistemas-Multimedia.github.io/blob/master/contents/scalar_quantization/companded.ipynb}{Companded Quantization},
and
\href{https://github.com/Sistemas-Multimedia/Sistemas-Multimedia.github.io/blob/master/contents/scalar_quantization/LloydMax.ipynb}{LloydMax Quantization}). This means that:
\begin{enumerate}
\item At low bit-rates the distortion decreases faster than at high
  bit-rates.
\item If we have a \emph{scalable} code-stream (we can decide how the
  code-stream will be decompressed), we should be aware that some
  parts of the code-stream can minimize faster the RD curve than
  others.
\end{enumerate}

As it can be also seen in the notebook
\href{https://github.com/Sistemas-Multimedia/Sistemas-Multimedia.github.io/blob/master/contents/scalar_quantization/compare_quantizers.ipynb}{quantizers\_comparison},
that the performance of the quantizers is not the same: usually
midrise and midtread, performs better than deadzone at intermediate
bit-rates, but deadzone is the best a low bit-rates (excluding
Lloyd-Max). Deadzone has also another advantage over midread and
midtread: when $\Delta$ is a power of 2 (which corresponds to a
bit-plane encoding), the obtained RD point is near optimal in the RD
space. Finally, the Lloyd-Max Quantizer reaches the highest
performance because it is adaptive.

%}}}

%}}}

\section{What do I have to do?}
%{{{

\begin{enumerate}
\item Create a notebook named ``bit-rate\_control.ipynb'' that, given
  a maximum bit-rate, finds the quantization step $\Delta^*$ that
  minimizes the distortion. Use the quantizer that your prefer.
\item Modify the notebook
  \href{https://github.com/Sistemas-Multimedia/Sistemas-Multimedia.github.io/blob/master/contents/scalar_quantization/LloydMax.ipynb}{LloydMax
    Quantization} to include in the rate the amount of bits necessary
  to encode the
  \href{https://en.wikipedia.org/wiki/Probability_density_function}{PDF}
  of the input image.
\end{enumerate}

%}}}

\section{Timming}

Please, finish this milestone before the next class session.

\section{Deliverables}

None.

\section{Resources}

\renewcommand{\addcontentsline}[3]{} % Remove functionality of \addcontentsline
\bibliography{data-compression,signal-processing,DWT,image-processing,quantization,information_theory}
