% Emacs, this is -*-latex-*-

\newcommand{\SM}{\href{http://cms.ual.es/UAL/estudios/masteres/plandeestudios/asignaturas/asignatura/MASTER7114?idAss=71142105&idTit=7114}{Sistemas Multimedia}}

\newcommand{\theproject}{\href{}}{MCDWT}

\title{\SM{} - Data Scalabilty}

\maketitle

\tableofcontents

\section{What means data scalability?}

\section{Temporal scalability using MCTF}

Is a special case of random-access mode in which the B-type frames
only contain B-type blocks and the reference frames are predetermined
using a dyadic pattern.

\section{Spatial scalability in image coding}

\subsection{Using the LPT}

\subsection{Using the DWT}

\section{Spatial scalability in video coding}

\subsection{Using the LPT}

\subsection{Using the DWT}

\section{Quality scalability in image coding}

If we have used a layered-based encoding system and we know the RD
points generated by each layer of each subband, we can sort the points
by their slope and use the progression of layers to easely select hown
many quality layers should be used satisfiying some maximum distorion
of bit-rate. This is done, for example, in JPEG2000.

\section{Quality scalability in video coding}

\section{To-Do}
\begin{enumerate}
\item The $2^n\times 2^n$-DCT domain can be decoded by resolution
  levels using a inverse $2^m\times 2^m$-DCT where $m=0,1,\cdots,n$,
  to the lower frequency subbands (notice that the inverse
  $0\times 0$-DCT do not perform any computation). Implement in VCF
  such image decoder. Complexity 4.
\end{enumerate}

\section{References}

\renewcommand{\addcontentsline}[3]{}% Remove functionality of \addcontentsline
\bibliography{text_compression}
