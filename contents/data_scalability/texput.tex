% Emacs, this is -*-latex-*-

\newcommand{\SM}{\href{http://cms.ual.es/UAL/estudios/masteres/plandeestudios/asignaturas/asignatura/MASTER7114?idAss=71142105&idTit=7114}{Sistemas Multimedia}}

\newcommand{\theproject}{\href{}}{MCDWT}

\title{\SM{} -  \href{https://github.com/Sistemas-Multimedia/Sistemas-Multimedia.github.io/tree/master/contents/data_scalability}{Code-stream Scalabilty}}

\maketitle
\tableofcontents

\section{What means code-stream scalability?}

Image and video codecs represent multidimensional signals, and this
makes possible to decode such information in several ways. When the
encoding scheme allows this, we say that the code-stream generated by
such scheme is scalable.

Scalability is interesting in several contexts and it has been
developed my most of the image and video encoding standards.

\section{Temporal scalability}

The king of the MC schemes is III..., which provides total temporal
scalability. Notice that the concept of temporal scalability does not
make sense in the context of image coding.

Temporal scalabilty is used in video streaming and the in fast-forward
and fast-backward modes of the video players.

\section{Using MCTF}

Random-access modes can provide dyadic temporal scalability in each
GOF if only B-type frames are used and they are generated using
MCTF~\cite{vruiz__MCTF}.

\section{Spatial scalability in image coding}

Compressed images can be partially decoded, resulting in a
reconstruction with a smaller resolution or in a reconstruction of a
WOI (Window Of Interest). Such forms of scalability are used in
interactive streaming to minimize the latency and to avoid sending
resolutions that some devices cannot display.

\subsection{Using the LPT}

Laplacian pyramids are 2D multiresolution structures that be used to
provide spatial scalability in image coding. The main issue to solve
is the data redundancy overhead of the LPT domain.

\subsection{Using the DWT}

2D-DWT domain are also 2D multiresolution structures that enable
spatial scalability, and in this case, the data redundancy overhead is
avoided. An example is JPEG 2000.

\section{Spatial scalability in video coding}

Similarly to image scalability, videos can be partially decoded to
obtain a version with a smaller resolution. This is used in video
streaming to avoid interruptions during the playing of the videos by
switching between resolutions, in video databases to save memory, and
in the rendering of the videos in displays with different resolution.

On the other hand, when the motion information is infered at the
decoder, this information will be less accurate that if we use all the
visual information available at the encoder.\footnote{The decoder, in
general, has a limited access to the original information because the
encoding system is lossy.}

\subsection{Using the LPT}

\subsection{Using the DWT}

\section{Quality scalability in image coding}

If we have used a layered-based encoding system and we know the RD
points generated by each layer of each subband, we can sort the points
by their slope and use the progression of layers to easely select hown
many quality layers should be used satisfiying some maximum distorion
of bit-rate. This is done, for example, in JPEG2000.

\section{Quality scalability in video coding}

\section{To-Do}
\begin{enumerate}
\item The $L$-levels DWT provides $L+1$ spatial resolution levels of
  an image. Modify the VCF pipeline to include this
  functionality. Complexity 2.
\item The $2^n\times 2^n$-DCT domain can be decoded by resolution
  levels using a inverse $2^m\times 2^m$-DCT where $m=0,1,\cdots,n$,
  to the lower frequency subbands (notice that the inverse
  $0\times 0$-DCT does not perform any computation). Implement in VCF
  such image decoder. Complexity 4.
\item In video coding, we can obtain spatial scalability if we build a
  laplacian pyramid of the frames and compress each level of the
  sequence using a normal video encoder. Notice that we can use the
  reconstructed sequence at the spatial level $l$ to improve the
  predictions for the level $l-1$. Incorporate this functionality to
  VCF. Complexity 6.
\item The spatial resolution level $l$ of the reconstructed video can
  be used (after interpolation) to estimate the motion at the $l-1$
  level, making the transmission of motion fields unnecessary for
  resolution level $l-1$. Explore this in VCF. Complexity 7.
\end{enumerate}

\section{References}

\renewcommand{\addcontentsline}[3]{}% Remove functionality of \addcontentsline
\bibliography{text_compression}
