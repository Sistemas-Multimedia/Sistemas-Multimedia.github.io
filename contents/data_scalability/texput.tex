% Emacs, this is -*-latex-*-

% Hablar de MDC (Multiple Description Coding)

%\newcommand{\SM}{\href{http://cms.ual.es/UAL/estudios/masteres/plandeestudios/asignaturas/asignatura/MASTER7114?idAss=71142105&idTit=7114}{Sistemas Multimedia}}
\newcommand{\SM}{\href{https://sistemas-multimedia.github.io/}{Sistemas Multimedia}}

\newcommand{\theproject}{\href{https://github.com/Sistemas-Multimedia/MRVC}{MRVC}}

\newcommand{\SW}{\href{https://github.com/Sistemas-Multimedia/MRVC}{MRVC}}


\title{\href{https://sistemas-multimedia.github.io/contents/data_scalability/}{Code-stream Scalabilty}}

\maketitle
\tableofcontents

\section{What means code-stream scalability?}
%{{{

Image and video codecs represent multidimensional signals, and this
enables to decode such information in several ways. When the
code-stream allows for this, we say that the code-stream generated by such
scheme is \emph{scalable}.

Scalability is interesting in many contexts, but specially in
streaming\footnote{Specifically, in real-time streaming scenarios we
  cannot prefetch much data before start the rendering of the image or
  video (casically, because we cannot wait too much. In this case, we
  can adapt the quality of the rendering to the available bandwidth, a
  factor that we cannot control in most of the situations.}, and has
been developed on most of the image and video encoding standards.

As a general remark, data scalability in media coding implies some
loss of RD efficiency.

%}}}

\section{Temporal scalability in video coding~\cite{vruiz__video_scalability}}

%{{{

In video coding, temporal scalability provides flexibility with the
number of decoded frames.\footnote{Notice that the concept of temporal
scalability cannot be applied to image coding.}

\subsection{GOF-level scalabilty}
GOF-splitting (see
\href{https://sistemas-multimedia.github.io/contents/temporal_transforms/}{temporal
  transforms}) provides basic temporal scalability (GOFs can be
decoded independently), and this is used in video streaming services
(such as YouTube) and video players to move along time using fast
forward and fast backward modes.
  
Notice that in this context, the maximum temporal scalability is
achieved when we use the intra-coding mode (III...), which provides
the maximum degree of temporal scalability.

\subsection{Frame-level scalabilty using MCTF}

Random access modes can provide dyadic temporal scalability in each
GOF if only B-type frames are used and generated using Motion
Compensated Temporal Filtering (MCTF)~\cite{vruiz__MC,vruiz__MCTF}.

%}}}

\section{Spatial scalability in image coding~\cite{vruiz__JPEG2000}}

%{{{ 

Compressed images can be partially decoded, resulting in a
reconstruction with smaller resolution or a reconstruction of a WOI
(Window Of Interest)~\cite{vruiz__JPEG2000}. Such forms of scalability
are used in interactive streaming to minimize latency and to avoid
sending resolutions that some devices cannot display. An example of
spatial scalability using JPEG2000~\cite{vruiz__JPEG2000} can be found
in the \href{https://www.jhelioviewer.org/}{JHelioviewer service}.

\subsection{Using the LPT (Laplacian Pyramid Transform)}

\href{https://en.wikipedia.org/wiki/Pyramid_(image_processing)#Laplacian_pyramid}{Laplacian
  pyramids} are 2D multiresolution structures that can be used to
provide spatial scalability in image coding. The main issue to solve
here is the data redundancy overhead of the LPT domain (in the
transform domain we have more coefficients than pixels in the image
domain).

\subsection{Using the DWT (Discrete Wavelet Transform)~\cite{vruiz__JPEG2000}}

2D-DWT domains are 2D multiresolution data structures that enable
spatial scalability, and in this case, compared to LPT, the data
redundancy overhead is avoided. JPEG2000~\cite{vruiz__JPEG2000} is
based on DWT.

%}}}

\section{Spatial scalability in video coding~\cite{vruiz__video_scalability}}

%{{{

In the case of video, spatial scalability provides 2D multiresolution
rendering using only one (partially decoded) code-stream. This
possibility is usually generated using the LPT because the DWT domain
is not shift invariant (the DWT is not invariant to displacement of
the pixels in the image domain).\footnote{The DWT domain is not
  redundant, but the shift invariant feature is not satisfied. To
  solve this problem, the DWT subbands must be interpolated to restore
  the lost phases. In this overcomplete domain, the ME/MC algorithms
  work, but the phase used of the predicted images must be represented
  in the code-stream.} The idea here is to apply
MC~\cite{vruiz__MC} to each level of the Laplacian pyramid.

Spatial scalability can be used in video streaming to avoid
interruptions during the playing of the videos by switching between
resolutions, in video databases to save memory, and in the rendering
of the videos in displays with different resolutions.

%}}}

\section{Quality scalability in image coding~\cite{vruiz__JPEG2000}}

%{{{ 

Quality scalability allows the possibility of adding or substracting
more or less visual information, depending on the amount of rendered
code-stream. The spatial and temporal resolutions remain constant.

\subsection{Using the DCT}

Some DCT-based image coding standards, such as JPEG, allow for
progressive decoding, where a increasing number of coefficients or bit
planes of those coefficients are rendered. Note that if for example,
$11$ is the number of bit planes required to represent the
coefficients, a total of $11\times 64\times 64$ quality levels is
possible (considering a block size of $64\times 64$).

\subsection{Using the DWT~\cite{vruiz__transform_coding}}

The idea of bit-plane encoding in the DWT domain is used in
JPEG 2000 \cite{vruiz__JPEG2000}. Compared to JPEG, the number of
quality levels is much higher, because in this case we can have up to
$R\times C\times B$, where $R$ is the number of rows, $C$ the number
of columns, and $B$ the number of bit planes in the DWT domain. In
JPEG 2000, the code-stream that represents a quality level is called
quality layer. We find that all the code-streams that belong to the
quality layer $L$ generate points in the RD curve that are to the
left of the RD points generated by the layer $L+1$. In other
words, the quality layers are sorted by their contribution to the
quality of the reconstruction.

\section{Quality scalability in video coding~\cite{vruiz__video_scalability}}

In the case of video, most video standards provide quality scalability
by applying MC to successive refinements of the reconstructed
sequence, at the maximum spatial resolution. This idea can be easily
understood if we imagine a spatially scalable code-stream generated
with the LPT, but in this case all the levels of the pyramid have the
same spatial resolution\footnote{Notice that in this case, we should
  use the term ``cubic building'' instead of ``pyramid''.}.

%}}}

\section{Simulcast VS Adaptive bit-rate streaming VS data-scalability}

%{{{ 

Although both terms can be confused,
\href{https://en.wikipedia.org/wiki/Simulcast}{simulcast} is an
streaming technique, while data scalability is a coding technique.

Simulcast (used, for example, in the
\href{https://en.wikipedia.org/wiki/DVB}{DVB},
\href{https://en.wikipedia.org/wiki/ATSC_standards}{ATSC} and
\href{https://en.wikipedia.org/wiki/ISDB}{ISDB}) is the process of
parallel transmission of media with different resolutions and/or
qualities, and this is usually deployed using different code-streams
on the sender side, although it could also be done using only one
scalable code-stream in spatial resolution.

Adaptive bit-rate streaming allows us to adapt the transmission bit-rate
in a point-to-point communication (of digital media) to the available
capacity of the link (which is typically time-varying). This
technique is used, for example, in the
\href{https://en.wikipedia.org/wiki/Dynamic_Adaptive_Streaming_over_HTTP}{DASH
  standard}, which is used, for example, by YouTube.

%}}}

\section{To-Do}
\begin{enumerate}
\item The $L$-levels DWT provides $L+1$ spatial resolution levels of
  an image. Modify \texttt{2D-DWT.py} to include this functionality
  (the possibility of decoding a reduced resolution version of the
  original image), which basically consist of ignoring the
  high-frequency subbands. \textbf{5 points}.
\item \label{TD:BP} Modify \texttt{2D-DWT.py} to allows the
  possibility of decoding by bit-planes (given as a parameter the
  number of bit-planes to decode). In order to achieve this, we must
  use a bit-plane entropy encoder (we can encode bits using an
  arithmetic coding, or conform symbols (\emph{symbol-blocks}) of more
  bits by taking small blocks of bits (of size 4x4, for example), in
  whose case we could use PNG of TIFF, for example). Notice that this is not
  incompatible to decode by spatial resolution levels (see previous
  To-Do). \textbf{8 points}.
\item The $2^n\times 2^n$-DCT domain can be decoded by resolution
  levels using an inverse $2^m\times 2^m$-DCT to the lower frequency
  subbands, where $m=0,1,\cdots,n$ (notice that the inverse
  $2^0\times 2^0$-DCT does not require any computation because the DC
  coefficients represent the values of the pixels of the image at the
  resolution level $n$). Implement in \texttt{2D-DCT.py} such image
  decoder. See the notebook
  \href{https://github.com/vicente-gonzalez-ruiz/DCT2D/blob/master/src/DCT2D/YCoCg_2D_DCT_SQ.ipynb}{Image
    Compression with YCoCg + 2D-DCT}. \textbf{6 points}.
\item Modify \texttt{2D-DCT.py} to allows the decoding by bit-planes
  (see To-Do~\ref{TD:BP}). Complexity 10.
\item It is possible to add WOI-decoding (Window Of Interest) to
  \texttt{2D-DCT.py} and \texttt{2D-DWT.py} if the bit-planes are
  encoded by (using the same nomenclature than JPEG2000)
  \emph{code-blocks} (notice that the size of the code-blocks must be
  a multiple of the symbol-blocks used to create the symbols used
  during the entropy encoding). \textbf{5 points}
\item In intra (III...) video coding, we can obtain spatial
  scalability if we build a
  \href{https://en.wikipedia.org/wiki/Pyramid_(image_processing)#Laplacian_pyramid}{Laplacian
    pyramid} of the frames and compress each level of the sequence
  using a standard image encoder (such for example \texttt{2D-DCT.py}
  or \texttt{2D-DWT.py}). Create a new Python module named
  \texttt{SS-III.py} (Spatial Scalable III... video encoding) with
  such functionality. \textbf{6 points}.
\item The Laplacian pyramid can be also used in motion compensated
  (IPP..., IBP..., and MCTF) video coding to provide spatial
  scalability. The key here is to encode each resolution level using a
  standard (non-scalable) video codec (see the
  \href{https://sistemas-multimedia.github.io/contents/temporal_transforms/}{Temporal
    Transforms}), starting at the lowest level. Name the corresponding
  Python module \texttt{LPIPP.py}, \texttt{LPIBP.py} or
  \texttt{LPMCTF.py}, depending on the motion compensation
  scheme. \textbf{8 points}.
\item In the previous proposals there is redundancy between the motion
  vectors of the different spatial resolution levels because the
  motion vector $(x, y)\rightarrow (x', y')$ at resolution level $l$
  will be close to the motion vector $(2x, 2y)\rightarrow(2x',2y')$ at
  resolution level $l-1$. Exploit such redundancy by adding the
  suitable functionality to the corresponding module
  \texttt{LPIPP.py}, \texttt{LPIBP.py} or
  \texttt{LPMCTF.py}. \textbf{7 points}.
\item The number of coefficients in a (2D) Laplacian pyramid is bigger
  than the number of coefficients in a 2D-DWT (the Laplacian pyramid
  domain in more redundant than the discrete wavelet domain). However,
  it is possible to avoid such redundancy when the filters used to
  build the Laplacian pyramid are the same than the filters used to
  compute the 2D-DWT because, in this case, when the 1-levels 2D-DWT
  is applied to any high-frequency level of the Laplacian pyramid,
  then the corresponding $LL$ subband is $\mathbf{0}$. Explore such
  improvement in the corresponding \texttt{LPIPP.py},
  \texttt{LPIBP.py} or \texttt{LPMCTF.py} module. See
  \href{https://vicente-gonzalez-ruiz.github.io/pyramids-and-wavelets/}{this}.
  \textbf{10 points}.
\end{enumerate}

\section{References}

\renewcommand{\addcontentsline}[3]{}% Remove functionality of \addcontentsline
\bibliography{JPEG2000,video_compression,motion_estimation,signal_processing}
