% Emacs, this is -*-latex-*-

\newcommand{\SM}{\href{http://cms.ual.es/UAL/estudios/masteres/plandeestudios/asignaturas/asignatura/MASTER7114?idAss=71142105&idTit=7114}{Sistemas Multimedia}}

\newcommand{\theproject}{\href{}}{MCDWT}
\title{\SM{} - Study Guide - Milestone 3 - Task 2: Vector Quantization of Gray-scale Images}

\maketitle
\tableofcontents

\section{Description}
%{{{

If we can found 2D redundancy in an image, Vector Quantization
(VQ)~\cite{vruiz__vector_quantization} applied to the spatial domain
of images can provide better RD curves than Scalar Quantization
(SQ)~\cite{vruiz__scalar_quantization}. After using VQ on the image we
will obtain an matrix of quantization indexes and we can use PNG as an
entropy codec to remove the statistical redundancy.\footnote{Notice
that PNG could also remove the remaining spatial redundancy, but we
can expect that only statistical redundancy can be found in the
sequence of quantization indexes.} Let's denote such image codec by
VQ+PNG.

%}}}

\subsection{Bit-rate control in VQ+PNG}
%{{{

If VQ+PNG is used to compress an image, we must realize that:
\begin{enumerate}
\item The rate of the code-stream (for example, the number of
  bits/pixels obtained after compressing) depends on the size $L$ of
  the vectors (usually squared blocks of pixels) and number the $K$ of
  different vectors that we consider in the
  code-book~\cite{vruiz__vector_quantization}. Notice that, without
  considering PNG, we will generate $\lceil\log_2 K\rceil$ bits per
  quantization index, and one quantization index will be generated
  each $L$ input pixels.\footnote{Resulting in a quantizer's rate of
  $\frac{\lceil\log_2 K\rceil}{L}$bits per pixel.} Hopefully, after
  using PNG on the indexes, this number of bits/vector will be futher
  reduced.
\item The distortion generated by VQ depends also on $V$, $C$, but
  also depends on:
  \begin{enumerate}
  \item The hability of VQ to chose the best vectors that will be used
    in the code-stream. Different algorithms will provide different RD
    curves~\cite{vruiz__information_theory}.
  \item The content of the input image. For example, images with
    complex textures will require, in general, smaller vectors or
    larger code-books.
  \end{enumerate}
\end{enumerate}

%}}}

%}}}

\section{What do I have to do?}
%{{{

\begin{enumerate}
\item Create a notebook named ``bit-rate\_control.ipynb'' that, given
  a maximum bit-rate, finds the combination of $V$ and $C$ that
  minimizes the distortion.
\end{enumerate}

%}}}

\section{Timming}

Please, finish this milestone before the next class session.

\section{Deliverables}

The notebook.

\section{Resources}

\renewcommand{\addcontentsline}[3]{} % Remove functionality of \addcontentsline
\bibliography{data-compression,signal-processing,DWT,image-processing,information_theory,quantization}
