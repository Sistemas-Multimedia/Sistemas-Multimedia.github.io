% Emacs, this is -*-latex-*-

%\newcommand{\SM}{\href{http://cms.ual.es/UAL/estudios/masteres/plandeestudios/asignaturas/asignatura/MASTER7114?idAss=71142105&idTit=7114}{Sistemas Multimedia}}
\newcommand{\SM}{\href{https://sistemas-multimedia.github.io/}{Sistemas Multimedia}}

\newcommand{\theproject}{\href{https://github.com/Sistemas-Multimedia/MRVC}{MRVC}}

\newcommand{\SW}{\href{https://github.com/Sistemas-Multimedia/MRVC}{MRVC}}

\title{\SM{} - Removing $\text{RGB}$ Redundancy with the DCT (Discrete Cosine Transform)}

\maketitle
\tableofcontents

\section{Objective}
This task shows how to exploit the inter-channel
redundancy~\cite{vruiz__visual_redundancy,vruiz__information_theory}
in $\text{RGB}$ images with the DCT~\cite{vruiz__DCT} applied over the
color dimension~\cite{vruiz__image_IO}. The DCT is one of the most
used transforms in audio, image and video coding.

\section{Color decorrelation with the DCT}
The DCT is a orthonormal transform that convolves cosine functions
with the input sinal (in this case, the components of the pixels of
the input image). The output of the transform is a set of three
subbands $\{\text{DCT}^0, \text{DCT}^1, \text{DCT}^2$. Notice that the
number of coefficients in a subband $\text{DCT}^s$ is the same that
the number of pixels in the original image $\mathbf{I}$ (or the number
of components in a channel $\mathbf{I}^c$).

As a result of the use of the DCT, in most of the images there is a
concentration of the energy of the image in the low frequency
subband~\cite{vruiz__transform_coding} ($\text{DCT}^0$) which
basically represents the
\href{https://en.wikipedia.org/wiki/Luminance}{luminance} of the
image.

\section{Effects of the transform}
The concentration of the energy in the subband $\text{DCT}^0$
increases the dynamic range of the DCT-color domain compared to the
dynamic range of the original $\text{RGB}$ image. This has two
interesting consequences:
\begin{enumerate}
\item The entropy of the progressive encoding (by bit-planes) of the
  integer representation of the DCT coefficients is reduced (compared
  to the entropy of the bit-planes of the original $\text{RGB}$
  pixels). Therefore, the RD performance is improved.
\item The number of different quantization steps can be higher, and
  therefore it is possible to increases the number of points in the RD
  curve.
\end{enumerate}

\section{Scalar quantization applied to the DCT domain}
Since the DCT is orthonormal (orthogonal and unitary), we can use the
simple quantization pattern
\begin{equation}
  \mathbf{\Delta}_{\text{DCT}^0} = \mathbf{\Delta}_{\text{DCT}^1} = \mathbf{\Delta}_{\text{DCT}^2}.
  \label{eq:simple_Q}
\end{equation}
Therefore, there is not differente between quantizing in the
$\text{RGB}$ or the ($\text{DCT}^0,\text{DCT}^1,\text{DCT}^2$) color
domain, except that in the last case, we must consider one extra
bit-plane to encode.

\section{A better rate control (TO-DO)}
\label{sec:increasing}
As we did in the case of the
\href{https://sistemas-multimedia.github.io/contents/RGB_SQ/}{Scalar
  Quantization in the $\text{RGB}$ Color Domain}, we can also use the
intermediate RD points between each Eq.~\eqref{eq:simple_Q}-point to
increase the resolution of the decompressed image.

\section{Implementation}
The described algorithm has been implemented in this
\href{https://github.com/Sistemas-Multimedia/Sistemas-Multimedia.github.io/blob/master/contents/RGB_DCT_SQ/RGB_DCT_SQ.ipynb}{notebook}.

\section{What do I have to do?}

\begin{enumerate}
\item Please, modify the
  \href{https://github.com/Sistemas-Multimedia/Sistemas-Multimedia.github.io/blob/master/contents/RGB_DCT_SQ/RGB_DCT_SQ.ipynb}{notebook}
  to increse the number of RD points following the procedure described
  in Section~\ref{sec:increasing}.
\item Which redundancies are exploited in the new improved image
  compressor?
\end{enumerate}

\section{Timming}

Please, finish this milestone before the next class session.

\section{Deliverables}

None.

\section{Resources}

\renewcommand{\addcontentsline}[3]{}% Remove functionality of \addcontentsline
\bibliography{maths,data-compression,signal-processing,DWT,image-compression,image-processing}

%%%%%%%%%%%%%%%%%%%%%%%%%%%%%%%%%%

\begin{comment}

Implement a lossy image compressor that exploits the inter-channel
redundancy~\cite{vruiz__information_theory} in $\text{RGB}$ images. To
achive this, we use the DCT~\cite{vruiz__DCT} applied over the color
dimension~\cite{vruiz__image_IO}, resulting in an concentration of the
energy of the image in the first
subband~\cite{vruiz__transform_coding} ($\text{DCT}^0$) which
basically represents the
\href{https://en.wikipedia.org/wiki/Luminance}{luminance} of the
image. Notice that the visual
redundancy~\cite{vruiz__visual_redundancy} is not considered.



\section{Distortion control}
We do not provide any accurate distortion control algorithm.  After
using the DCT, the coefficients are quantized with a deadzone
quantizer, simulating a bit-plane encoding. However, the dynamic range
of the $\text{DCT}^0$ coefficients doubles the dynamic range of the
pixels, and therefore, the highest quantization step will be
$\Delta=256$. Therefore, we will have eight points in our RD curve
that satisfy that
\begin{equation}
  \mathbf{\Delta}_{\text{DCT}^0} = \mathbf{\Delta}_{\text{DCT}^1} = \mathbf{\Delta}_{\text{DCT}^2}.
  \label{eq:simple_Q}
\end{equation}

Notice that the DCT is orthogonal and therefore, the contributions of
the subbands (to the quality of the reconstructed image) are
independent and therefore, additive. Because the DCT is also
orthonormal, we can measure the distortion in both, the image and the
transform domain.

\section{Rate control}
Since PNG does not provide accurate rate control, we can only select
between eight bit-rates that safisfy the Eq.~\eqred{eq:simple_Q}.

\section{Scalar quantization in the ($\text{DCT}_0,\text{DCT}_1,\text{DCT}_2$) domain}
%{{{

The DCT is orthonormal, and thereore, the distortion can be measured
directly in the transform domain. This means that we can treat the DCT
coefficients as if would be pixels. Therefore, an optimal algorithm
for RDO~\cite{vruiz__information_theory} is:

\begin{enumerate}
\item Find the RD curve for each subband. We will use the same
  quantization steps for all the subbands. This action generates three
  curves.
\item Join all the RD points in a single list and sort it by the slope
  of the
  points~\cite{vruiz__information_theory}. Track\footnote{Storing with
  the point the combination subband index and quantization step used
  for each point.} the quantization steps used for each point.
\item Use the quantization patters described by the points of the
  sorted list to obtain the definitive RD curve. Notice that, since
  the DCT is orthonormal, we could estimate accurately the distortion
  in the definitive curve using directly the distortions of the
  subbands.
\end{enumerate}

Notice that the points that satisfy that
\begin{equation}
  \mathbf{\Delta}_0 = \mathbf{\Delta}_1 = \mathbf{\Delta}_2
  \label{eq:deltas}
\end{equation}
should belong to the definitive RD curve and should describe also the
convex hull of the curve. There is not guarantee that the rest of
points in which the Eq.~\eqref{eq:deltas} is not true will belong to
the convex hull (although they could be used to perform a finer
rate-control).

This algorithm has been implemented in this \href{https://github.com/Sistemas-Multimedia/Sistemas-Multimedia.github.io/blob/master/contents/RGB_DCT/RGB_DCT.ipynb}{notebook}.

\section{Ablation study}
The improvement resulting from the use of the DCT applied over the RGB
domain can be quantified if we compare the RD curves considering or
not its use.\footnote{This kind of comparison is commonly called
``\emph{Ablation Study}'' because we are ''removing'' some
functionality of the system to see how its ausence impacts on the
performance.} 

\end{comment}

\begin{comment}
   The steps
of this procedure are (see the
\href{https://github.com/Sistemas-Multimedia/Sistemas-Multimedia.github.io/blob/master/contents/color_DCT/RGB_DCT.ipynb}{notebook}):
\begin{enumerate}
\item  This fact allows us to use a ``fast'' (with
  linear
  \href{https://en.wikipedia.org/wiki/Computational_complexity}{complexity})
  RDO (Rate/Distortion Optimization) algorithm to find which
  combination of quantization steps $\mathbf{\Delta}$ that
  ``minimizes'' the RD curve of the reconstructed image, which can be
  computed as the sum\footnote{Remember, the constribution of the
  subbands are independent and additive.} of the RD curves generated
  by the $3$ subbands ($\text{DCT}^0$, $\text{DCT}^1$, and
  $\text{DCT}^2$). Therefore, we can:
  \begin{enumerate}
  \item Quantize and compress\footnote{We can consider that the
  coefficients are gray-scale pixels and use a image compressor.} each
    subband with a quantization step $\mathbf{\Delta}_s$. This results
    in $3$ RD curves, one per subband. We define the slope of the
    $n$-th point in a RD curve as the slope of the straight line
    passing through the points $n$ and $n+1$ (see the
    Fig.~\ref{fig:slope_computation}). The slope of the point for the
    highest rate is defined to $0$.  The results generated by this
    step are $3$ RD curves (one per subband).
  \item Put all the RD points in a list, tracking\footnote{Remembering
  the combination subband index and quantization step used for each
  point.} the quantization pattern used for each RD point. Compute the
    list of points of the RD curve of the reconstructed image as the
    (in descending order) sorted list, by slope, of RD points.
  \item (Optional) If the image compressor used in the first step
    generates less data overhead when a color image is compressed
    considering the three channels at the same time, recompute the
    rate of each RD point of the definitive curve. For example, in the
    case of PNG~\cite{vruiz__PNG}, even when the intercomponent
    redundancy is not exploited, the number of headers is decreased if
    we perform only one compression per image.
  \end{enumerate}

  The orthonormality of the DCT also allows us to compute the
  definitive distortion directly in the DCT domain.

  Notice also that this algorithm has been designed considering the
  fact that, ideally (where the RD curve is continuous), the rate
  selected for each subband should be the
  same~\cite{vruiz__information_theory,vetterli2014foundations}, and
  therefore, should generate a RD curve such that the sum of the
  distances of the points of the RD curve to the point $(0,0)$ is
  minimized.
  
\end{enumerate}

As a final remark, take into consideration that the quantization steps
used in each subband should be selected considering aspects such as
the ganularity of the rate-control\footnote{How many points has the RD
curve.} and the features of the decoding process\footnote{For example, if we
want to provide progressive bit-plane decoding, the quantization steps
should be powers of $2$.}

\section{An even faster (but coarser) rate-control algorithm}

The running time of the previous algorithm depends on the number of
subbands, that in our case is only $3$. However, we can develop a
faster rate-control procedure (avoiding to compute the RD curves of
the subbands that is the heaviest part) if we suppose that the RD
slopes are basically determined by the distortion (i.e., without
considering the rate)\footnote{Notice that this basically means that
all the subband's curves have the
same \footnote{https://en.wikipedia.org/wiki/Domain_of_a_function}{domain}
and that the RD points occurs at the same rates}.

Obviously, in the practice it is very unlikely to happen, but if the
shapes of the curves are close enough, we can suppose that the slopes
are the same for the same points in the different curves. In the case
of the DCT that is orthonormal (the gains of the forward and the
backward filters is $1$), this is the same that define the
quantization pattern
\begin{equation}
  \mathbf{\Delta}_0 = \mathbf{\Delta}_1 = \mathbf{\Delta}_2.
\end{equation}

Notice that one way to check if this quantization algorithm is optimal
(for the obtained RD points) is to check if the ``slower'' algorithm
outputs such quantization patterns. If this is true, the ``faster''
rate-control algorithm is optimal, at least for the encoded image.

\end{comment}
