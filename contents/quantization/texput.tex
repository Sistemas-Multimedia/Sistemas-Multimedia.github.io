% Emacs, this is -*-latex-*-

%\newcommand{\SM}{\href{http://cms.ual.es/UAL/estudios/masteres/plandeestudios/asignaturas/asignatura/MASTER7114?idAss=71142105&idTit=7114}{Sistemas Multimedia}}
\newcommand{\SM}{\href{https://sistemas-multimedia.github.io/}{Sistemas Multimedia}}

\newcommand{\theproject}{\href{https://github.com/Sistemas-Multimedia/MRVC}{MRVC}}

\newcommand{\SW}{\href{https://github.com/Sistemas-Multimedia/MRVC}{MRVC}}


\title{\SM{} - Digital (Re-)Quantization}

\maketitle

\tableofcontents

\section{What is Quantization?}

In Information Theory, Quantization is any mapping process between two
sets of elements $A$ and $B$, where all the elements of $A$ are
associated to one element of $B$, and happens that $|A|>|B|$, where
$|\cdot|$ represents the orde (the number of elements) of a group.

Notice that because $|A|>|B|$, quantization is an irreversible process
because two o more elements of $A$ will be mapped to the same element
of $B$.

\section{What is Digital Quantization?}

In Digital Quantization, the elements of $A$ and $B$ are digital
samples of a signal. Since, by definition, a digital sample has been
quantized to be converted from the analog world to the digital one, we
are appliying a re-quantization of the signal. Again, such digital
quantization implies a loss of information in the re-quantized signal.

\section{Basic terminology}

Several types of quantizers divide the range of possible values that
the samples can take into a collection of intervals. The values that
define such intervals are called decision levels and the value that in
the quantized domain represents to all the input posible values that
fall in a interval is called the representation level of the interval.

The size of each interval is called the quantization step size.

\section{Classification of Quantizers}

\subsection{Scalar VS Vector Quantizers}

When quantization maps single elments of $A$ to single elements of
$B$, the quantizer is said scalar. When we map tuples of elements, we
are using Vector Quantization.

\subsection{Uniform VS Non-Uniform Quantizers}

In a uniform quantizer, all the intervals have the same size. In a
non-uniform quantizer, at least one of the intervals is different to
the rest. For example, a deadzone quantizer has a interval size for
the representation level 0 that doubles the size of the rest of
intervals. For this reason, a deadzone quantizer is a non-uniform
quantizer.

\subsection{Static VS Adaptive Quantizers}

Static quantizers are defined without considering the specific
characteristics of the sequence of samples to quantize. Adaptive
quantizers adapt to the sequence of samples that are quantized. For
example, a Lloyd-Max quantizer divides the range of input values in a
set of intervals whose size is inversely proportional to the
probability of using such interval. On the contrary, a static
quantizer keeps fixed the size of the intervals.

\section{Deazone quantization}

Deadzone quantizers are static quasi-uniform scalar quantizers. These
are used frequently in lossy compression sistems because: (1) when the
quantization step size is a power of two, and (2) the input sample
values are integers (and negative integers are represented using two's
complement), then the representation levels can be found by simply
discarding low-significant bits of the input samples (in other words,
we only need to perform a bit-shift operation). Moreover, if the signal to be quantized has a average close to 0 (where the deadzone is centered), SNR.

There are basically two types of entropy encoders depending on how the
sequence of symbols that make up the data to be encoded are processed:

\begin{enumerate}
\item Those that process the sequence symbol by symbol. Examples of
  this type of algorithms are Huffman
  Coding~\cite{vruiz__huffman_coding} and Arithmetic
  Coding~\cite{vruiz__arithmetic_coding}. These encoders are also
  called ``probabilistic encoders'' because the number of bits of code
  assigned to a symbol $s$ depends on the probability of the symbol
  $p(s)$.
\item Those that process the sequence by blocks of symbols
  (strings). Examples are Run-Length Encoding (RLE)~\cite{vruiz__rle}
  and all the dictionary-based text compressors, such as
  LZW~\cite{vruiz__LZW}.
\end{enumerate}
  
Those of the first type generally achieve more compact
representations, but are more computationally expensive. The
algorithms of the second class are usually faster, but slighlt worse
in compression ratio.

\section{\href{https://en.wikipedia.org/wiki/Huffman_coding}{Huffman Coding}}

A Huffman code is a
\href{https://en.wikipedia.org/wiki/Prefix_code}{prefix code} that
allows to ``navigate'' through the so called Huffman (inverted) tree
from the trunk to one of its leaves without
uncertainty~\cite{vruiz__huffman_coding}. The Huffman tree satisfies that
\begin{equation}
  l(c(s)) = \lceil I(s)\rceil,
  \label{eq:huffman_performance}
\end{equation}
where $l(c(s))$ is the length of the (Huffman) code assigned to the
symbol $s$ and $I(s)$ is the amount of information represented by $s$,
measured bits of information.

Notice that the minimum number of bits that can be used for
representing a symbol (using Huffman) is 1, which can be a problem
when the length of the alphabet is small. Notice also that, if the
probabilistic model remains constant, the code-words assigned to the
symbols will be also remain unchanged.

\section{\href{https://en.wikipedia.org/wiki/Arithmetic_coding}{Arithmetic Coding}}

In an arithmetic code~\cite{vruiz__arithmetic_coding}, the number of
bits of data that can be used for representing a symbol can match
exactly the number of bits of information, i.e,
\begin{equation}
  l(c(s)) = I(s).
\end{equation}

This also means that even if the size of the alphabet is small, the
coding performance of an arithmetic code is optimal, although this
optimality is only satisfied if the number of symbols to encode is
infinite. Notice also that, if the size of the alphabet is high, the
encoding performance difference between a Huffman code (or any other
prefix code) and an arithmetic code, vanishes.

\section{Portable Network Graphics (PNG)}

\href{https://en.wikipedia.org/wiki/Portable_Network_Graphics}{PNG}~\cite{vruiz__PNG}
(pronounced ``ping'') a dictionary-based
\href{https://en.wikipedia.org/wiki/Lossless_compression}{lossless
  image compression format} used for representing
\href{https://en.wikipedia.org/wiki/Digital_data}{digital}
\href{https://en.wikipedia.org/wiki/Digital_image}{images} and
\href{https://en.wikipedia.org/wiki/Video}{videos}~\cite{vruiz__image_video}
in III... format.

We must bear in mind that as such an image compressor, we can only
interact with PNG at the image level, that is, it only accepts images
(in shades of gray or in color, with the possibility of alpha channel)
and returns images.

% PNG is the default EC in the \href{https://github.com/Sistemas-Multimedia/VCF}{VCF project} because: (1) is lossless, (2) is fast, and (3) the compression performance is reasonable.

\section{Resources}

\renewcommand{\addcontentsline}[3]{}% Remove functionality of \addcontentsline
\bibliography{text-compression,image-formats,image-video-theory}
