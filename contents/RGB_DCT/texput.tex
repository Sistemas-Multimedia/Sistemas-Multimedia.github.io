% Emacs, this is -*-latex-*-

%\newcommand{\SM}{\href{http://cms.ual.es/UAL/estudios/masteres/plandeestudios/asignaturas/asignatura/MASTER7114?idAss=71142105&idTit=7114}{Sistemas Multimedia}}
\newcommand{\SM}{\href{https://sistemas-multimedia.github.io/}{Sistemas Multimedia}}

\newcommand{\theproject}{\href{https://github.com/Sistemas-Multimedia/MRVC}{MRVC}}

\newcommand{\SW}{\href{https://github.com/Sistemas-Multimedia/MRVC}{MRVC}}

\title{\SM{} - Study Guide - Milestone 4 - Task 2 - Removing $\text{RGB}$ Redundancy with the DCT (Discrete Cosine Transform)}

\maketitle

\tableofcontents

\section{Description of the task}
%{{{

This task explains how the RD
tradeoff~\cite{vruiz__information_theory} can be
improved\footnote{Compared to the coding algorithm in which only Q+PNG
is applied to each color component, directly.} when the color
redundancy~\cite{vruiz__visual_redundancy} is exploited. The idea is
to concentrate, as much as possible, the visual information in a
smaller number of color channels\footnote{Also refered by color
components.}. In the case of the DCT applied to the color domain, most
of the energy is ``concentrated'' in the first subband($\text{DCT}_0$)
which basically represents the
\href{https://en.wikipedia.org/wiki/Luminance}{luminance} of the
image.

%}}}


\section{Color decorrelation with the DCT}
%{{{ 

One way of achieving the decorrelation in the $\text{RGB}$ domain is
to use
\href{https://vicente-gonzalez-ruiz.github.io/transform_coding/}{Transform
  Coding}~\cite{vruiz__transform_coding} applied between the color
components to concentrate information\footnote{Information is usually
estimated through the measurement of the
\href{https://en.wikipedia.org/wiki/Variance}{variance} or the
\href{https://en.wikipedia.org/wiki/Entropy}{entropy}~\cite{vruiz__information_theory}.}
of the image in a small set of coefficients, that after a quantization
can be compressed with a better RD
performance~\cite{vruiz__information_theory}. For this task we will
use the Color-DCT~\cite{vruiz__DCT}. Notice also that, in general,
transform domains require larger
\href{https://en.wikipedia.org/wiki/Dynamic_range}{dynamic ranges}
than the original ones.

%\footnote{For example,
%if the energy of a color subband is low, quantization could completely
%makes zero such subband, but the reconstruction of the image would be
%reasonable. The most part of entropy codecs reach higher compression
%ratios with sequences of zeros.}

%This
%\href{https://github.com/Sistemas-Multimedia/Sistemas-Multimedia.github.io/blob/master/contents/color_DCT/coding_gain.ipynb}{notebook}
%shows the near-lossless coding gain of an image after using the
%color-DCT.

%This
%\href{https://github.com/Sistemas-Multimedia/Sistemas-Multimedia.github.io/blob/master/milestones/06-YUV_compression/color_redundancy.ipynb}{notebook}
%quantifies the redundancy related to the color domain.

%}}}

\section{RD gain provided by the DCT}
%{{{

The improvement resulting from the use of the DCT applied over the RGB
domain can be quantified if we compare the RD curves considering or
not its use.\footnote{This kind of comparison is commonly called
``\emph{Ablation Study}'' because we are ''removing'' some
functionality of the system to see how its ausence impacts on the
performance.} To measure the amount of energy compaction we can
quantize the coefficients (the output of the forward transform) and to
entropy encode~\cite{vruiz__information_theory} the result. The steps
of this procedure are (see the
\href{https://github.com/Sistemas-Multimedia/Sistemas-Multimedia.github.io/blob/master/contents/color_DCT/RGB_DCT.ipynb}{notebook}):
\begin{enumerate}
\item Apply the DCT to each RGB pixel of the image. This outputs a new
  image in which the components are called
  subbands~\cite{vruiz__transform_coding} and where there are three
  coefficients $\text{DCT}^0_{i,j}$, $\text{DCT}^1_{i,j}$, and
  $\text{DCT}^2_{i,j}$, per pixel $\mathbf{I}_{i,j}$. Notice that the
  number of coefficients in a subband $\text{DCT}^s$ is the same that
  the number of pixels in $\mathbf{I}$ (or the number of components in
  a channel $\mathbf{I}^c$).
\item The DCT is orthogonal and therefore, the contributions of the
  subbands (to the quality of the reconstructed image) are independent
  and therefore, additive. This fact allows us to use a ``fast'' (with
  linear
  \href{https://en.wikipedia.org/wiki/Computational_complexity}{complexity})
  RDO (Rate/Distortion Optimization) algorithm to find which
  combination of quantization steps $\mathbf{\Delta}$ that
  ``minimizes'' the RD curve of the reconstructed image, which can be
  computed as the sum\footnote{Remember, the constribution of the
  subbands are independent and additive.} of the RD curves generated
  by the $3$ subbands ($\text{DCT}^0$, $\text{DCT}^1$, and
  $\text{DCT}^2$). Therefore, we can:
  \begin{enumerate}
  \item Quantize and compress\footnote{We can consider that the
  coefficients are gray-scale pixels and use a image compressor.} each
    subband with a quantization step $\mathbf{\Delta}_s$. This results
    in $3$ RD curves, one per subband. We define the slope of the
    $n$-th point in a RD curve as the slope of the straight line
    passing through the points $n$ and $n+1$ (see the
    Fig.~\ref{fig:slope_computation}). The slope of the point for the
    highest rate is defined to $0$.  The results generated by this
    step are $3$ RD curves (one per subband).
  \item Put all the RD points in a list, tracking\footnote{Remembering
  the combination subband index and quantization step used for each
  point.} the quantization pattern used for each RD point. Compute the
    list of points of the RD curve of the reconstructed image as the
    (in descending order) sorted list, by slope, of RD points.
  \item (Optional) If the image compressor used in the first step
    generates less data overhead when a color image is compressed
    considering the three channels at the same time, recompute the
    rate of each RD point of the definitive curve. For example, in the
    case of PNG~\cite{vruiz__PNG}, even when the intercomponent
    redundancy is not exploited, the number of headers is decreased if
    we perform only one compression per image.
  \end{enumerate}

  The orthonormality of the DCT also allows us to compute the
  definitive distortion directly in the DCT domain.

  Notice also that this algorithm has been designed considering the
  fact that, ideally (where the RD curve is continuous), the rate
  selected for each subband should be the
  same~\cite{vruiz__information_theory,vetterli2014foundations}, and
  therefore, should generate a RD curve such that the sum of the
  distances of the points of the RD curve to the point $(0,0)$ is
  minimized.
  
\end{enumerate}

As a final remark, take into consideration that the quantization steps
used in each subband should be selected considering aspects such as
the ganularity of the rate-control\footnote{How many points has the RD
curve.} and the features of the decoding process\footnote{For example, if we
want to provide progressive bit-plane decoding, the quantization steps
should be powers of $2$.}

\section{An even faster (but coarser) rate-control algorithm}

The running time of the previous algorithm depends on the number of
subbands, that in our case is only $3$. However, we can develop a
faster rate-control procedure (avoiding to compute the RD curves of
the subbands that is the heaviest part) if we suppose that the RD
slopes are basically determined by the distortion (i.e., without
considering the rate)\footnote{Notice that this basically means that
all the subband's curves have the
same \footnote{https://en.wikipedia.org/wiki/Domain_of_a_function}{domain}
and that the RD points occurs at the same rates}.

Obviously, in the practice it is very unlikely to happen, but if the
shapes of the curves are close enough, we can suppose that the slopes
are the same for the same points in the different curves. In the case
of the DCT that is orthonormal (the gains of the forward and the
backward filters is $1$), this is the same that define the
quantization pattern
\begin{equation}
  \mathbf{\Delta}_0 = \mathbf{\Delta}_1 = \mathbf{\Delta}_2.
\end{equation}

Notice that one way to check if this quantization algorithm is optimal
(for the obtained RD points) is to check if the ``slower'' algorithm
outputs such quantization patterns. If this is true, the ``faster''
rate-control algorithm is optimal, at least for the encoded image.

%}}}

\section{What do I have to do?}

\begin{enumerate}
\item Please, run the
  \href{https://github.com/Sistemas-Multimedia/Sistemas-Multimedia.github.io/blob/master/contents/color_DCT/RGB_DCT.ipynb}{notebook}
  to learn some insights about the compression of images using the
  DCT.
\item Try to compress different images. Is the Color-DCT always
  effective (are RD curves better than compressing the image in
  the RGB domain)?
\end{enumerate}

\section{Timming}

Please, finish this milestone before the next class session.

\section{Deliverables}

None.

\section{Resources}

\renewcommand{\addcontentsline}[3]{}% Remove functionality of \addcontentsline
\bibliography{maths,data-compression,signal-processing,DWT,image-compression,image-processing}
