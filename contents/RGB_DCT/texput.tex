% Emacs, this is -*-latex-*-

\newcommand{\SM}{\href{http://cms.ual.es/UAL/estudios/masteres/plandeestudios/asignaturas/asignatura/MASTER7114?idAss=71142105&idTit=7114}{Sistemas Multimedia}}

\newcommand{\theproject}{\href{}}{MCDWT}
\title{\SM{} - Study Guide - Milestone 4 - Task 2 - Removing $\text{RGB}$ Redundancy with the Color-DCT}

\maketitle

\tableofcontents

\section{Description of the task}
%{{{

This task explains how the RD
tradeoff~\cite{vruiz__information_theory} can be
improved\footnote{Compared to the coding algorithm in which only Q+PNG
is applied to each color component, directly.} when the color
redundancy~\cite{vruiz__visual_redundancy} is exploited. The idea is
to concentrate, as much as possible, the visual information in a small
number of color channels\footnote{Also refered by color components.},
in this case, it is only one and basically represents the
\href{https://en.wikipedia.org/wiki/Luminance}{luminance} of the
image.

%}}}



\section{Color-DCT decorrelation}
%{{{ 

One way of achieving the decorrelation in the $\text{RGB}$ domain is
to use
\href{https://vicente-gonzalez-ruiz.github.io/transform_coding/}{Transform
  Coding}~\cite{vruiz__transform_coding} applied between the color
components to ``concentrate'' information\footnote{Information is
usually estimated through the measurement of the
\href{https://en.wikipedia.org/wiki/Variance}{variance} or the
\href{https://en.wikipedia.org/wiki/Entropy}{entropy}.} of the image
in a small set of coefficients, that after quantization can be
compressed with a better RD
performancemore~\cite{vruiz__information_theory}. For this task we
will use the Color-DCT~\cite{vruiz__color_DCT}. Notice that, in
general, transform domains require larger
\href{https://en.wikipedia.org/wiki/Dynamic_range}{dynamic ranges}
than in the original signal.

%\footnote{For example,
%if the energy of a color subband is low, quantization could completely
%makes zero such subband, but the reconstruction of the image would be
%reasonable. The most part of entropy codecs reach higher compression
%ratios with sequences of zeros.}

%This
%\href{https://github.com/Sistemas-Multimedia/Sistemas-Multimedia.github.io/blob/master/contents/color_DCT/coding_gain.ipynb}{notebook}
%shows the near-lossless coding gain of an image after using the
%color-DCT.

%This
%\href{https://github.com/Sistemas-Multimedia/Sistemas-Multimedia.github.io/blob/master/milestones/06-YUV_compression/color_redundancy.ipynb}{notebook}
%quantifies the redundancy related to the color domain.

%}}}

\section{RD gain provided by the Color-DCT}
%{{{

The improvement resulting from the use of the Color-DCT can be
quantified if we compare the RD curves considering or not the
transform.\footnote{This kind of comparison is commonly called
``\emph{Ablation Study}'' because we are ''removing'' some
functionality of the system to see how its ausence impacts on the
performance.} To measure the amount of energy compaction we can
quantize the coefficients (the output of the forward transform) and to
entropy encode~\cite{vruiz__information_theory} the result. The steps
of this procedure are (see the
\href{https://github.com/Sistemas-Multimedia/Sistemas-Multimedia.github.io/blob/master/contents/color_DCT/RD_performance.ipynb}{notebook}):
\begin{enumerate}
\item Apply the Color-DCT to each pixel of the image. This outputs a
  new image in which the components are called
  subbands~\cite{vruiz__transform_coding} where there is a coefficient
  per pixel.
\item Quantize and compress each component with a quantization step
  size ${\mathbf\Delta}_i$ such that the overall RD is minimized. To
  optimize the steps sizes, the selected RD slope used in each
  component should be the
  same~\cite{vruiz__information_theory,vetterli2014foundations}. Notice
  that, because the DCT is orthonormal, the distortion (generated by
  the quantization in the transform domain) is not affected by the
  Color-DCT and therefore, the distortion in the reconstructed image
  can be computed directly in the Color-DCT domain.
\item Compress (again and definitively) the quantization indexes
  resulting of the quantization. For the sake of simplicity, we can
  consider the indexes as ``pixels'', and use a lossless image
  compressor, such as PNG (that does not removes the intercomponent
  redundancy)~\cite{vruiz__PNG}. In addition, the selected
  ${\mathbf\Delta}$ must be encoded in the
  code-stream.\footnote{Notice the ``bold font'' selected for
  representing the quantization step size: there are 3 different
  quantization steps because we have 3 subbands.}
\end{enumerate}

%}}}

\section{What do I have to do?}

\begin{enumerate}
\item Please, run the
  \href{https://github.com/Sistemas-Multimedia/Sistemas-Multimedia.github.io/blob/master/contents/color_DCT/RD_performance.ipynb}{notebook}
  to learn some insights about the compression of images using the
  Color-DCT.
\item Try to compress different images. Is the Color-DCT always
  effective (are RD curves better than compressing the image in
  the RGB domain)?
\end{enumerate}

\section{Timming}

Please, finish this milestone before the next class session.

\section{Deliverables}

None.

\section{Resources}

\renewcommand{\addcontentsline}[3]{}% Remove functionality of \addcontentsline
\bibliography{maths,data-compression,signal-processing,DWT,image-compression,image-processing}
