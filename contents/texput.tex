% Emacs, this is -*-latex-*-

%\newcommand{\SM}{\href{http://cms.ual.es/UAL/estudios/masteres/plandeestudios/asignaturas/asignatura/MASTER7114?idAss=71142105&idTit=7114}{Sistemas Multimedia}}
\newcommand{\SM}{\href{https://sistemas-multimedia.github.io/}{Sistemas Multimedia}}

\newcommand{\theproject}{\href{https://github.com/Sistemas-Multimedia/MRVC}{MRVC}}

\newcommand{\SW}{\href{https://github.com/Sistemas-Multimedia/MRVC}{MRVC}}

\title{\SM{}\\Contents}

\maketitle

%\section{Contents}

\begin{quote}
0. \href{https://sistemas-multimedia.github.io/contents/working_environment}{The Working Environment}.\\
1. \href{https://sistemas-multimedia.github.io/contents/entropy_coding}{Entropy Coding}. \\
2. \href{https://sistemas-multimedia.github.io/contents/quantization}{Quantization}. \\
3. \href{https://sistemas-multimedia.github.io/contents/color_transforms/}{Color Transforms}. \\
4. \href{https://sistemas-multimedia.github.io/contents/spatial_transforms}{Spatial Transforms}. \\
5. \href{https://sistemas-multimedia.github.io/contents/temporal_transforms}{Temporal Transforms}. \\
6. \href{https://sistemas-multimedia.github.io/contents/data_scalability}{Code-stream Scalability}. \\
7. \href{https://sistemas-multimedia.github.io/contents/perceptual_coding}{Perceptual Coding}. \\
8. \href{https://sistemas-multimedia.github.io/contents/standards}{Image and Video Coding Standards}.  % 8
% Learned Image Compression ?? (maybe, spreaded over all the previous topics)
\end{quote}

%\section{Resources}

%\bibliography{DWT,data_compression,image_compression,image_formats,image_processing,image_pyramids,image_video_theory,information_theory,JPEG,JPEG2000,maths,motion_estimation,neural_nets,perception,projects,quantization,text_compression,rate_control,signal_processing,video_compression,wikipedia}

\end{document}

%\item \href{https://sistemas-multimedia.github.io/milestones/01-provisioning/}{OS (Operating System) provisioning}.
%\item \href{https://sistemas-multimedia.github.io/milestones/02-python/}{Installation and basic programming with Python}.
%\item \href{https://sistemas-multimedia.github.io/milestones/03-git/}{Git, GitHub and the Fork-and-Branch Git workflow}.
%\item \href{https://sistemas-multimedia.github.io/milestones/04-PNG/}{Entropy coding with PNG}.
%\item \href{https://sistemas-multimedia.github.io/milestones/05-RGB_compression/}{Quantizing and compressing in the RGB domain}.
%\item \href{https://sistemas-multimedia.github.io/milestones/06-YUV_compression/}{Removing redundancy with a color transform}.
%\item \href{https://sistemas-multimedia.github.io/milestones/07-DCT/}{Image compression with the block-DCT}.
%\item \href{https://sistemas-multimedia.github.io/milestones/08-DWT/}{Energy Concentration and Spatial Multiresolution with the Discrete Wavelet Transform}.
%\item \href{https://sistemas-multimedia.github.io/milestones/09-III_coding/}{RDO in III... compression}.
%\item \href{https://sistemas-multimedia.github.io/milestones/10-ME/}{Motion estimation}.
%\item \href{https://sistemas-multimedia.github.io/milestones/11-image_domain_IPP/}{IPP... compression in the image domain}.
%\item \href{https://sistemas-multimedia.github.io/milestones/12-transform_domain_MC/}{Motion compensation in the transform domain}.
%\item \href{https://sistemas-multimedia.github.io/milestones/13-transform_domain_IPP/}{IPP... compression in the transform domain}.
  % Using JPEG2000 instead of PNG.
  % Exploiting the color redundancy of the HVS.
%\item \href{https://sistemas-multimedia.github.io/milestones/08-MC_in_DWT_domain/}{Motion Compensation in the DWT Domain}.
%\item \href{https://sistemas-multimedia.github.io/milestones/09-LPT/}{Laplacian Pyramid Transform (LPT)}.
%\item \href{https://sistemas-multimedia.github.io/milestones/10-CS-LPT/}{Critically Sampled LPT (CS-LPT)}.
%\item \href{https://sistemas-multimedia.github.io/milestones/11-OF/}{Motion Estimation with Optical Flow}.
%\item \href{https://sistemas-multimedia.github.io/milestones/12-MC-CS-LPT/}{Motion Compensated CS-LPT (MC-CS-LPT)}.
%\item \href{https://sistemas-multimedia.github.io/milestones/14-coding/}{MC-CS-LPT Subband Coding}.

\begin{comment}
  \section{Milestones}

Such projects are organized in a collection of milestones. Some of
them has been previously developed and our goal is to understand how
the implemented techniques work and how to use them. Others are
focused on the implementation of new features.

\begin{description}
\item [Milestone 0: {\normalfont Preparing the framework.}]  \emph{The
working environment used in Sistemas Multimedia relies on Linux,
Python and Git.}
  \begin{description}
  \item [Task 1: {\normalfont
      \href{https://sistemas-multimedia.github.io/contents/provisioning/}{Operating
        system provisioning}.}]
  \item [Task 2: {\normalfont
      \href{https://sistemas-multimedia.github.io/contents/python/}{Installation
        and basic programming with Python}.}]
  \item [Task 3: {\normalfont
      \href{https://sistemas-multimedia.github.io/contents/git/}{Git,
        GitHub and the fork-and-branch Git workflow}.}]
    %\item [Milestone 4: {\normalfont \href{https://sistemas-multimedia.github.io/milestones/04-the_data/}{Understanding the video data}.}]
  \end{description}
  ~\newline

\item [Milestone 1: {\normalfont Entropy coding of images:}] \emph{The
entropy (in the context of the Information Theory) is a measure of the
amount of information represented by a piece of data. Most of image
and video compressors achieve a reduction in the amount of output data
though entropy coding. In this milestone we analyze some of the most
used lossless image compressors, and the correlation of their rate
performance with the entropy of the encoded images.}
  \begin{description}
  \item [Task 1: {\normalfont
      \href{https://sistemas-multimedia.github.io/contents/PNG/}{PNG
        (Portable Network Graphics)}.}]
  \item [Task 2: {\normalfont
      \href{https://sistemas-multimedia.github.io/contents/JPEG2000/}{JPEG
        (Joint Photographic Expert Group) 2000}. TO-DO.}]
  \item [Task 3: {\normalfont
      \href{https://sistemas-multimedia.github.io/contents/JPEG_XL/}{JPEG
        XL}. TO-DO.}]
  \item [Task 4: {\normalfont
      \href{https://sistemas-multimedia.github.io/contents/BPG/}{BPG
        (Better Portable Graphics)}. TO-DO.}]
  \end{description}
  ~\newline

\item [Milestone 2: {\normalfont Exploiting the pixel-color
    redundancy:}] \emph{Raw images and videos contain large amounts of
information that can be removed without generating a meaningful visual
distortion (because humans can distinguish only a finite number of
``visual levels''). For this task, we will use Scalar Quantization, a
basic basic technique that allows to remove information (by decreasing
such number of levels) in images and videos (and in signals in
general), in the signal domain}.
  \begin{description}
  \item [Task 1: {\normalfont
      \href{https://sistemas-multimedia.github.io/contents/gray_SQ/}{Scalar
        quantization of gray-scale images}.}]
  \item [Task 2: {\normalfont
      \href{https://sistemas-multimedia.github.io/contents/RGB_SQ/}{Scalar
        quantization in the $\text{RGB}$ domain}.}]
  \item [Task 3: {\normalfont
      \href{https://sistemas-multimedia.github.io/contents/gray_TCQ/}{Trellis
        Coded Quantization (TCQ) of gray-scale images}. TO-DO.}]
  \end{description}
  ~\newline
  
 \item [Milestone 3: {\normalfont Exploiting the inter-component
     redundancy in images:}] \emph{In color images, the RGB (Red,
 Green, and Blue) channels tend to be highly correlated. We can take
 advantage of this channel correlation to increase the compression
 ratio in image (and of course, in video) coding, without generating
 noticeable visual artifacts}.
   \begin{description}
   \item [Task 1: {\normalfont
       \href{https://sistemas-multimedia.github.io/contents/RGB_DCT_SQ/}{$\text{RGB}$
         Decorrelation with the DCT (Discrete Cosine Transform)}.}]
   \item [Task 2: {\normalfont
       \href{https://sistemas-multimedia.github.io/contents/YCoCg_SQ/}{$\text{RGB}$
         Decorrelation with the YCoCg (luminance, orange chroma, green
         chroma) Transform}.}]
   \item [Task 3: {\normalfont
       \href{https://sistemas-multimedia.github.io/contents/YCrCb_SQ/}{$\text{RGB}$
         Decorrelation with the YCrCb (luminance, red chroma, blue
         chroma) Transform}.}]
   \item [Task 4: {\normalfont
       \href{https://sistemas-multimedia.github.io/contents/RCT_SQ/}{$\text{RGB}$
         Decorrelation with the
         \href{https://en.wikipedia.org/wiki/JPEG_2000\#Color_components_transformation}{RCT
           (Reversible Color Transform)}}. TO-DO.}]
   \item [Task 4: {\normalfont
       \href{https://sistemas-multimedia.github.io/contents/LMS_SQ/}{$\text{RGB}$
         Decorrelation with the \href{https://en.wikipedia.org/wiki/LMS}{LMS
           (Long, Medium, Short) Transform}}. TO-DO.}]
   \item [Task 5: {\normalfont
       \href{https://sistemas-multimedia.github.io/contents/RGB_VQ/}{Vector
         Quantization of the $\text{RGB}$ color domain}.}]
   \end{description}
   ~\newline
   
\item [Milestone 4: {\normalfont Exploiting the spatial redundancy in
    images:}] \emph{Visual data is highly correlated in the spatial
domain and such correlation can be considered to achieve higher
compression ratios}.
  \begin{description}
  \item [Task 15: {\normalfont
      \href{https://sistemas-multimedia.github.io/contents/chroma_filtering/}{Chroma
        filtering and subsampling}.}]
  %\item [Task 13: {\normalfont
  %    \href{https://sistemas-multimedia.github.io/contents/DPCM/}{2D-DPCM
  %      (2-Dimensional Differential Pulse Code Modulation)}.}]
  \item [Task 1: {\normalfont
      \href{https://sistemas-multimedia.github.io/contents/2D_DCT/}{Spatial
        Decorrelation with the DCT}.}]
  \item [Task 2: {\normalfont
      \href{https://sistemas-multimedia.github.io/contents/DWT/}{Spatial
        Decorrelation with the DWT (Discrete Wavelet Transform)}.}]
  \item [Task 3: {\normalfont
      \href{https://sistemas-multimedia.github.io/contents/spatial_color_VQ/}
           {Vector Quantization in the Spatial Domain}. TO-DO}] % Podemos
                                                          % aplicar
                                                          % spatial-VQ
                                                          % a la
                                                          % imagen
    % color-VQ-ed
  \item [Task 4: {\normalfont
      \href{https://sistemas-multimedia.github.io/contents/JPEG/}{JPEG (Joint Photographic Expert Group) (1992)}. TO-DO.}]
  \item [Task 5: {\normalfont
      \href{https://sistemas-multimedia.github.io/contents/JPEG2000_lossy/}{Lossy JPEG 2000}. TO-DO.}]
  \item [Task 6: {\normalfont
      \href{https://sistemas-multimedia.github.io/contents/JPEG_XL_lossy/}{Lossy JPEG XL}. TO-DO.}]
  \item [Task 7: {\normalfont
      \href{https://sistemas-multimedia.github.io/contents/BPG_lossy/}{Lossy BPG}. TO-DO.}]
  \end{description}
  ~\newline

\item [Milestone 5: {\normalfont Exploiting the temporal redundancy:}]
  \emph{In most of the videos, adjacent images are quite similar, and
  this source of redundancy can be removed to find more compact
  representations}.
  \begin{description}
  \item [Task 1: {\normalfont
      \href{https://sistemas-multimedia.github.io/contents/motion_estimation/}{Motion
        estimation in the image domain}.}]
  \item [Task 2: {\normalfont
      \href{https://sistemas-multimedia.github.io/contents/motion_compensation/}{Motion
        compensation in the image domain}.}]
  \item [Task 1: {\normalfont
      \href{https://sistemas-multimedia.github.io/contents/IPP_coding/}{``Only-Intra'' (III) Video Coding}.}]
  \item [Task 2: {\normalfont
      \href{https://sistemas-multimedia.github.io/contents/CR_coding/}{Conditional Replenishment Video Coding}.}]
  \item [Task 3: {\normalfont
      \href{https://sistemas-multimedia.github.io/contents/IPP_coding_no_MC/}{``Inter'' (IPP) Video Coding without Motion Compensation}.}]
  \item [Task 4: {\normalfont
      \href{https://sistemas-multimedia.github.io/contents/IPP_coding/}{``Inter'' (IPP) Video Coding (with Motion Compensation)}.}]
  \item [Task 5: {\normalfont
      \href{https://sistemas-multimedia.github.io/contents/IBP_coding_MC/}{``Bidirectional'' (IBP) Video Coding}.}]
  \item [Task 6: {\normalfont
      \href{https://sistemas-multimedia.github.io/contents/MCTF/}{Motion Compensated Temporal Filtering}.}]
  \end{description}
  ~\newline

\item [Milestone 6: {\normalfont Data scalability and
    multiresolution:}] \emph{Images and videos can be
decompressed of many ways, depending on the requirements of the
users}.
  \begin{description}
  \item [Task 1: {\normalfont
      \href{https://sistemas-multimedia.github.io/contents/spatial_scalability/}{Spatial scalability}.}]
  \item [Task 2: {\normalfont
      \href{https://sistemas-multimedia.github.io/contents/temporal_scalability/}{Temporal scalability}.}]
  \item [Task 3: {\normalfont
      \href{https://sistemas-multimedia.github.io/contents/quality_scalability/}{Quality scalability}.}]
  \item [Task 1: {\normalfont
      \href{https://sistemas-multimedia.github.io/contents/ME_MC_in_DWT_domain/}{Motion
        compensation in the DWT domain}.}]
  \item [Task 2: {\normalfont
      \href{https://sistemas-multimedia.github.io/contents/ME_MC_in_DCT_domain/}{Motion
        compensation in the DCT domain}.}]
  \item [Task 3: {\normalfont
      \href{https://sistemas-multimedia.github.io/contents/ME_MC_in_LP_domain/}{Motion
        estimation and compensation in the LP (Laplacian Pyramid) image domain}.}]
  \item [Task 1: {\normalfont
      \href{https://github.com/Sistemas-Multimedia/MRVC}{MRVC
        (Multi-Resolution Video Codec}.TO-DO}]
  \end{description}
  ~\newline

  % A técnicas básicas
\begin{comment}
\item [Milestone ??: {\normalfont Removing temporal redundancy in the
    transform domain:}] \emph{Some transforms facilitate the
multiresolution representation of images and this can be interesting to
estimate and compensate the motion in this domain}.
  \begin{description}
  \end{description}
  ~\newline
\end{comment}

\begin{comment}
\item [Milestone 8: {\normalfont MRVC (Muti-Resolution Video
    Coding):}] \emph{Cooking all the ingredients together in a video compressor}.
  \begin{description}
  \end{description}
\end{comment}
  
  %%%%%%%%%%%%
  
  \begin{comment}

 \item [Milestone 2: {\normalfont Removing the color redundancy:}]
   \emph{In color images, the RGB channels tend to be highly
   correlated. We can take advantage of this channel correlation, for
   example, to decrease the entropy and therefore, increase the
   compression ratio in image and video coding}.
  \begin{description}
  \item [Task 4: {\normalfont
      \href{https://sistemas-multimedia.github.io/contents/color_DCT/}{The
        1D-DCT (1-Dimensional Discrete Cosine Transform) transform}
      applied over the color domain.}]
  \item [Task 5: {\normalfont
      \href{https://sistemas-multimedia.github.io/contents/YCrCb/}{The
        YCrCb transform}.}]
  \item [Task 6: {\normalfont
      \href{https://sistemas-multimedia.github.io/contents/YCoCg/}{The
        YCoCg transform}.}]
  \end{description}
  ~\newline

\item [Milestone 3: {\normalfont Removing the spatial redundancy:}]
  \emph{Visual data is highly correlated in the spatial domain and
  such correlation can be exploited to decrease the entropy in
  images}.
  \begin{description}
  \item [Task 7: {\normalfont
      \href{https://sistemas-multimedia.github.io/contents/DPCM/}{2D-DPCM
        (2-Dimensional Differential Pulse Code Modulation)}.}]
  \item [Task 8: {\normalfont
      \href{https://sistemas-multimedia.github.io/contents/spatialDCT/}{The
        2D-DCT (2-Dimensional Discrete Cosine Transform)}.}]
  \item [Task 9: {\normalfont
      \href{https://sistemas-multimedia.github.io/contents/DWT/}{The
        2D-DWT (2-Dimensional Discrete Wavelet Transform)}.}]
  \end{description}
  ~\newline

\item [Milestone 4: {\normalfont Entropy coding:}] \emph{So far, the
minimization of the entropy has been our main objective. Let's see how
can we use an entropy codec to build a lossless (or near lossless,
depending the accuracy of the transforms) image compressor.}
%\href{https://github.com/Sistemas-Multimedia/RIC}{lossless (or near
%  lossless, depending on the implementation of the inter-channel and
%  intra-channels transforms) image codec}.
  \begin{description}
  \item [Task 10: {\normalfont
      \href{https://sistemas-multimedia.github.io/contents/PNG/}{PNG
        (Portable Network Grachics)}.}]
  \item [Task 11: {\normalfont
      \href{https://sistemas-multimedia.github.io/contents/J2K/}{JPEG
        (Joint Photographic Expert Group) 2000} as an entropy codec.}]
  \item [Task 12: {\normalfont
      \href{https://sistemas-multimedia.github.io/contents/CBAC/}{CBAC
        (Context-Based Arithmetic Coding)}.}]
  \end{description}
  ~\newline

\item [Milestone 5: {\normalfont Rate-Distortion control through quantization:}]
  \emph{In most images, a large percentage of the code-stream is
  dedicated to encode inappreciable (by humans) visual
  information. Quantization can be used to remove such information,
  providing higher compression ratios}.
  \begin{description}
  \item [Task 13: {\normalfont
      \href{https://sistemas-multimedia.github.io/contents/quantization/}{Quantization
        in the transform domain}.}]
  \item [Task 14: {\normalfont
      \href{https://sistemas-multimedia.github.io/contents/RDO/}{RDO
        (Rate/Distortion Optimization)}.}]
  \end{description}
  ~\newline

\item [Milestone 6: {\normalfont Perceptual coding through chroma
    ``subsampling'':}] \emph{Humans hardly perceive high frequency
contents in the chroma components.}
  %and for this reason,
%\href{https://github.com/Sistemas-Multimedia/IIC}{chroma information
%  can be low-pass filtered without a significative perceived
%  distortion}}.
  \begin{description}
  \item [Task 15: {\normalfont
      \href{https://sistemas-multimedia.github.io/contents/chroma_filtering/}{Chroma
        filtering and subsampling}.}]
  \end{description}
  ~\newline

\item [Milestone 7: {\normalfont Motion estimation and compensation in
    the image domain:}] \emph{Most image sequences are highly
correlated in the temporal domain and motion estimation can exploit
this redundancy to, again, increase the compression ratio in video
coding}.
  \begin{description}
  \item [Task 16: {\normalfont
      \href{https://sistemas-multimedia.github.io/contents/motion_estimation/}{Motion
        estimation in the image domain}.}]
  \item [Task 17: {\normalfont
      \href{https://sistemas-multimedia.github.io/contents/motion_compensation/}{Motion
        compensation in the image domain}.}]
  \end{description}
  ~\newline

\item [Milestone 8: {\normalfont MRVC (Muti-Resolution Video
    Coding):}] \emph{It's possible to provide multiresolution
(spatially-scalable) code-streams by estimating and compensating
motion in the transform domain}.
  \begin{description}
  \item [Task 16: {\normalfont
      \href{https://sistemas-multimedia.github.io/contents/motion_estimation/}{Motion
        estimation and compensation in the transform domain}.}]
  \item [Task 17: {\normalfont
      \href{https://sistemas-multimedia.github.io/contents/motion_compensation/}{Symmetric video coding}.}]
  \end{description}

%\item [Day 8: {\normalfont .}]
%  \emph{Providing spatial multiresolution to video coding.}
%  \begin{description}
%  \item [Milestone 11: {\normalfont \href{https://sistemas-multimedia.github.io/milestones/11-transform_domain_MC/}{Motion compensation in the transform domain}.}]
%  \item [Milestone 12: {\normalfont \href{https://sistemas-multimedia.github.io/milestones/12-transform_domain_IPP/}{IPP... coding in the transform domain}.}]
%  \end{description}
  

  
%\item [Milestone 8: {\normalfont Motion compensation:}] \emph{With the
%motion information it is possible to make predictions that help to
%decrease the entropy}.
%  \begin{description}
%  \item [Task 17: {\normalfont
%      \href{https://sistemas-multimedia.github.io/contents/motion_compensation/}{Motion compensation}.}]
%  \end{description}
%  ~\newline

%%%%%%%
  
%\item [Milestone 2: {\normalfont Reversible Image Compression (RIC):}]
%  \begin{description}
%  \item [Task 4: {\normalfont \href{htpps://sistemas-multimedia.github.io/contents/YCrCb/}{The YCoCg color transform}.}]
%  \item [Task 5: {\normalfont \href{https://sistemas-multimedia.github.io/contents/DWT/}{The Discrete Wavelet Transform (DWT)}.}]
%  \item [Task 6: {\normalfont \href{https://sistemas-multimedia.github.io/contents/PNG/}{The Portable Network Graphics format (PNG)}.}]
%  \item [Task 7: {\normalfont \href{https://sistemas-multimedia.github.io/contents/RIC_R_opt/}{Rate optimization}.}]
%  \end{description}
%  ~\newline

%\item [Milestone 3: {\normalfont Irreversible Image Compression (IIC):}]
%  \begin{description}
%  \item [Task 8: {\normalfont \href{htpps://sistemas-multimedia.github.io/contents/IIC_quantization/}{Quantization in the DWT domain}.}]
%  \item [Task 9: {\normalfont \href{https://sistemas-multimedia.github.io/contents/IIC_RD_opt/}{Rate/Distortion optimization in IIC}.}]
%  \end{description}
%  ~\newline

%\item [Milestone 4: {\normalfont Progressive Image Compression (PIC):}]
%  \begin{description}
%  \item [Task 10: {\normalfont \href{htpps://sistemas-multimedia.github.io/contents/PIC_prog_enc/}{Progressive encoding in the DWT domain}.}]
%  \item [Task 11: {\normalfont \href{https://sistemas-multimedia.github.io/contents/PIC_RD_opt/}{Rate/Distortion optimization in PIC}.}]
%  \end{description}
%  ~\newline

%\item [Milestone 5: {\normalfont Progressive Image Compression (PIC):}]
%  \begin{description}
%  \item [Task 10: {\normalfont \href{htpps://sistemas-multimedia.github.io/contents/PIC_prog_enc/}{Progressive encoding in the DWT domain}.}]
%  \item [Task 11: {\normalfont \href{https://sistemas-multimedia.github.io/contents/PIC_RD_opt/}{Rate/Distortion optimization in PIC}.}]
%  \end{description}
%  ~\newline
 
%\item [Day 2: {\normalfont Compressing color images.}] \emph{Quantization and entropy coding, two key signal compression techniques.}
%  \begin{description}
%  \item [Milestone 4: {\normalfont \href{https://sistemas-multimedia.github.io/milestones/04-PNG/}{Entropy coding of images with PNG}.}]
%  \item [Milestone 5: {\normalfont \href{https://sistemas-multimedia.github.io/milestones/05-RGB_compression/}{Compresion of RGB images}.}]
%  \end{description}
%  ~\newline

%\item [Day 3: {\normalfont Removing color redundancy.}] \emph{Color
%    components are correlated, and human beings are more sensitive to
%    slow variations of the color than to high frequency changes.}
%  \begin{description}
%   \item [Milestone 6: {\normalfont \href{https://sistemas-multimedia.github.io/milestones/06-YUV_compression/}{Compression in a luma/chroma domain}.}]
%  \end{description}
%  ~\newline

%\item [Day 4: {\normalfont Removing spatial redundancy with transforms I.}]
%  \emph{Quantization in the transform domain is more efficient than in the image domain.}
%  \begin{description}
%  \item [Milestone 7: {\normalfont \href{https://sistemas-multimedia.github.io/milestones/07-DCT/}{The 2D-DCT (Discrete Cosine Transform)}.}]
%  \end{description}
%  ~\newline

%\item [Day 5: {\normalfont Removing spatial redundancy with transforms II.}]
%  \emph{Providing spatial multiresolution to image coding.}
%  \begin{description}
%  \item [Milestone 8: {\normalfont \href{https://sistemas-multimedia.github.io/milestones/08-DWT/}{The 2D-DWT (Discrete Wavelet Transform)}.}]
%  %\item [Milestone 9: {\normalfont \href{}{LP (Laplacian Pyramid) (unfinished)}.}]
%  \end{description}
%  ~\newline

%\begin{comment}
%\item [Day 6: {\normalfont RDO in the III... domain.}]
%  \emph{Rate/Distortion Optimization in a sequence of YCoCg+DCT
%  (Intracoded) frames.}
%  \begin{description}
%  \item [Milestone 9: {\normalfont \href{https://sistemas-multimedia.github.io/milestones/09-III_coding/}{III\_coding}.}]
%  \end{description}
%  ~\newline
%\end{comment}

%\item [Day 6: {\normalfont Motion estimation.}]
%  \emph{Video sequences are correlated in time, and determining how the objects move can help to increase the compression ratio.}
%  \begin{description}
%  \item [Milestone 9: {\normalfont \href{https://sistemas-multimedia.github.io/milestones/09-ME/}{Motion estimation}.}]
%  \end{description}
%  ~\newline

%\item [Day 7: {\normalfont IPP... coding.}]
%  \emph{Building a video compressor based on spatial and IPP... temporal decorrelation and entropy coding.}
%  \begin{description}
%  \item [Milestone 10: {\normalfont \href{https://sistemas-multimedia.github.io/milestones/10-image_domain_IPP/}{IPP... coding}.}]
%  \end{description}
%  ~\newline

%\item [Day 8: {\normalfont MRVC (Muti-Resolution Video Coding).}]
%  \emph{Providing spatial multiresolution to video coding.}
%  \begin{description}
%  \item [Milestone 11: {\normalfont \href{https://sistemas-multimedia.github.io/milestones/11-transform_domain_MC/}{Motion compensation in the transform domain}.}]
%  \item [Milestone 12: {\normalfont \href{https://sistemas-multimedia.github.io/milestones/12-transform_domain_IPP/}{IPP... coding in the transform domain}.}]
%  \end{description}
  
%\begin{comment}
%\item [Day 8: {\normalfont The final project.}]
%  \emph{Incorporating chroma subsampling to the YCoCg+DCT+ME video compressor.}

%\item [Days 7: {\normalfont MRVC (Muti-Resolution Video Coding).}]
%  \emph{Providing spatial multiresolution to video coding.}
%  \begin{description}
%  \item [Milestone 11: {\normalfont \href{https://sistemas-multimedia.github.io/milestones/12-transform_domain_MC/}{Motion compensation in the transform domain}.}]
%  \item [Milestone 12: {\normalfont \href{https://sistemas-multimedia.github.io/milestones/13-transform_domain_IPP/}{IPP... coding in the transform domain}.}]
%  \end{description}
%\end{comment}
\end{comment}

\end{description}

