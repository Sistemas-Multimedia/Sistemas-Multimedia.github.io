% Emacs, this is -*-latex-*-

%\newcommand{\SM}{\href{http://cms.ual.es/UAL/estudios/masteres/plandeestudios/asignaturas/asignatura/MASTER7114?idAss=71142105&idTit=7114}{Sistemas Multimedia}}
\newcommand{\SM}{\href{https://sistemas-multimedia.github.io/}{Sistemas Multimedia}}

\newcommand{\theproject}{\href{https://github.com/Sistemas-Multimedia/MRVC}{MRVC}}

\newcommand{\SW}{\href{https://github.com/Sistemas-Multimedia/MRVC}{MRVC}}

\title{\SM{} - Color Transforms}

\maketitle
\tableofcontents

\section{What is spectral decorrelation?}

Most image and video compressors exploit the statistical (and also
perceptual\footnote{This will be addresed latter in this course.})
correlation between the $\text{RGB}$ color
\emph{components}\footnote{A component of a pixel in the $\text{RGB}$
domain refer to one of the values $\text{R}$ (red), $\text{G}$ (green)
and $\text{B}$ (blue) coordinates in the $\text{RGB}$ color 3D space.}
of the pixels, using a color transform. Notice that color transforms
are pixel-wise operators. As a resul, each pixel is represented in a
different domain where three new \emph{coefficients}\footnote{Most
part of the transforms, including the color ones, analyze the signal
information from a frequency perspective, generating the so called
coefficients whose index in the transform domain is related to the
frequency.} express the same\footnote{In general, the color transforms
can be considered lossless, although this is only true when
fixed-point arithmetic is used.} information but in a different color
domain.

\section{Lumas and chromas}

Most color transforms are designed to split the color information of a
pixel in \emph{luminance} (luma) and \emph{chrominance} (chroma). The
luma is basically the low frequency\footnote{It is worth to understand
that the frequency concept in the transform domain is not related to
the frequency concept in the original pixel domain. For example, the
$\text{R}$ component or a pixel represents the amount of red in the pixel, and in
visible spectrum we are refering to frequencies that are smaller than
$\text{G}$ and $\text{B}$ components. However, in a color transformed
domain, the luma measures brightness level of the pixel, and we cannot
found a subband in the visible spectrum that can represent such
information.} information of the pixel, and the chroma (logically)
the high frequency information.

For example, in JPEG and H.264/AVC the color information of each pixel
is transformed from the $\text{RGB}$ color space to the $\text{YCrCb}$
color space, and in JPEG XR and in H.264/AVC, the $\text{YCoCg}$ color
space is used. In these luma-based color spaces, $\text{Y}$ represents
the luma coefficient. The other two, are is the chroma.

\subsection*{Channels and subbands (a word about notation)}

Apart from using the terms of component and coefficient, we will use
the term \emph{channel} to refeer to the same index component of all
the pixels of an image (or video), and \emph{subband} to denote the
same index coefficient of all the pixels of an image (or video). For
example, the $\text{R}$ channel of a color image corresponds to the
image generated by the $\text{R}$ component of all the pixels of the
image (that can be considered as a
\emph{single-channel}/\emph{mono-component}/\emph{monochromatic}
image), and for example, the $\text{Y}$ subband of a transformed
($\text{RGB}$) image corresponds to the $\text{Y}$ coefficients of
all the pixels (again a ``\emph{scalar}'' image).

Most color transforms map 3 channels ($\text{RGB}$) into 2 subbands.

\section{Benefits of the color transforms}

Color transforms applied to natural visual information generally
provide two main profits:
\begin{enumerate}
\item \textbf{Energy concentration:} In general, transforms ``move''
  the energy (information provided by the signals) between subbands,
  accumulating most of the energy in a reduced number of them (aspect
  related to the so called coding gain of the transform). In our case,
  where the transformations are between color spaces, in the transform
  domain most of the energy is concentrated in the $\text{Y}$
  subband. As a consequence of this, usually, \textbf{the entropy is
    decreased} and the \textbf{the dynamic range is increased}. The
  first means that will compress more, and the second that we will be
  able to use a higher range of quenaization step sizes, increasing
  also the number of feasible points in the RD curve.
\item \textbf{A different luma/croma sensitivity:} Our visual system
  is more sensitive in terms of spatial resolution to the luma
  (``black-and-white'') information than to the chroma
  (``color'') information.\footnote{Notice again, that we will study
  this effect in a posterior session.}
\end{enumerate}

\section{Rate/distortion optimization considering the color space}

When we use scalar quantization.

\subsection{In the $\text{RGB}$ domain}

The $\text{RGB}$ components are additive~\cite{burger2016digital} in
both, the distortion and the rate domains. Therefore, ideally, given a
target bit-rate or distortion, we should select the quantization step
size in each component that generates the same slope in the RD curve
of the component.

\subsection{In the luma/croma domain}

In this case, the optimal quantization pattern should satisfy the same
conditions as quantizing in the $\text{RGB}$ domain. However, we have
to consider also that the transform must be orthogonal, or at least,
bi-orthogonal (the luma/chroma subbands must be
independent). Otherwise, we cannot say that if we ``improve'' the
quality of one subband in a given amount, such improvement will be
also achieved in the $\text{RGB}$ domain.

\subsection{A easier (and faster) solution}

In general, the RD optimization considering the color domain implies
to know the RD curves of the subbands. If the transform is orthogonal,
this is equivalent to know the RD curves in the $\text{RGB}$ domain
beecause the subbands are additive. If the transform is not
orthogonal, the this last property is false.

Even if the curves are known, in general a good approximation to the
optimal solution (using scalar quantization) is to use the same
quantization step size for all the subbands. For example, in the
$\text{RGB}$ domain, this implies to use
\begin{equation}
  \mathbf{\Delta}^{\text{R}}_i = \mathbf{\Delta}^{\text{G}}_i =
  \mathbf{\Delta}^{\text{B}}_i,
  \label{eq:simple_Q}
\end{equation}
where $i$ represents the $i$-th quantization step size. Notice that
the quantization pattern defined in the Eq.~\eqref{eq:simple_Q} will
be optimal only if for all the quantization steps, the condition
\begin{equation}
  \mathbf{\lambda}^{\text{R}}_i = \mathbf{\lambda}^{\text{G}}_i =
  \mathbf{\lambda}^{\text{B}}_i
  \label{eq:optimal_lambda}
\end{equation}
is satisfied. In the notebook \href{https://github.com/Sistemas-Multimedia/Sistemas-Multimedia.github.io/blob/master/contents/RGB_SQ/RGB_SQ.ipynb}{} we can explore (at least, visually) the grade of compilance of
Eq.~\eqref{eq:optimal_lambda}.

%%%%%%%%%%%%%%


\section{The Color DCT Domain}

\subsection{Scalar quantization applied to the DCT domain}
Since the DCT is orthonormal (orthogonal and unitary), we can use the
simple quantization pattern
\begin{equation}
  \mathbf{\Delta}^{\text{DCT}^0} = \mathbf{\Delta}^{\text{DCT}^1} = \mathbf{\Delta}^{\text{DCT}^2}.
  \label{eq:simple_Q}
\end{equation}
Therefore, there is not differente between quantizing in the
$\text{RGB}$ or the ($\text{DCT}^0,\text{DCT}^1,\text{DCT}^2$) color
domain, except that in the last case, we must consider one extra
bit-plane to encode.

\subsection{A better rate control (TO-DO)}
\label{sec:increasing}
As we did in the case of the
\href{https://sistemas-multimedia.github.io/contents/RGB_SQ/}{Scalar
  Quantization in the $\text{RGB}$ Color Domain}, we can also use the
intermediate RD points between each Eq.~\eqref{eq:simple_Q}-point to
increase the resolution of the decompressed image.

%%%%%%%%%%%%%%%%%

\section{Scalar versus vector quantization of RGB images}
%{{{

SQ (Scalar
Quantization)~\cite{vruiz__scalar_quantization,sayood2017introduction}
would be an optimal solution only if the image colors are uniformly
distributed within
\href{https://en.wikipedia.org/wiki/RGB_color_model}{the RGB
  cube}. However, the typical color distribution in natural images is
anything but uniform, with some regions of the color space being
densely populated and many potentially used colors entirely
missing. In this case, depending on the quantization step
size~\cite{vruiz__signal_quantization}, SQ is not optimal because the
used colors may not be sampled with sufficient density while at the same
time the encoding system is considering colors that do not appear in
the image at all~\cite{burger2016digital}.

On the other hand, VQ (Vector
Quantization)~\cite{vruiz__vector_quantization,sayood2017introduction}
applied to the color domain does not treat the individual RGB
components (also refered by
\href{https://en.wikipedia.org/wiki/Color_image}{channel}s) separately
as does scalar quantization, but each used color vector ${\mathbf C}_i
= ({\mathbf R}_i, {\mathbf G}_i, {\mathbf B}_i )$ in the image is
treated as a minimum structure. Starting from a set of original color
tuples ${\mathbf C} = \{{\mathbf C}_1, {\mathbf C}_2, \ldots ,{\mathbf
  C}_m\}$, where $m$ is the number of different colors found in the
image, the task of a vector quantizer in this context is to:
\begin{enumerate}
\item Find a set of $n$ representative color vectors (the so called
  \emph{code-book}) ${\mathbf C}' = \{{\mathbf C}'_1, {\mathbf C}'_2
  ,\ldots , {\mathbf C}'_n \}$, where $n < m$.
\item Replace each original color ${\mathbf C}_i$ by one of the new
  color vectors ${\mathbf C}'_j\in {\mathbf C}'$, where the resulting
  deviation from the original image shall be minimal.
\end{enumerate}

%}}}


\section{Resources}

\renewcommand{\addcontentsline}[3]{}% Remove functionality of \addcontentsline
\bibliography{maths,data-compression,signal-processing,DWT,image-compression,image-processing,information_theory}

%%%%%%%%%%%%%%%%%%%%%%%%%%%%%%%%%%

\begin{comment}

Implement a lossy image compressor that exploits the inter-channel
redundancy~\cite{vruiz__information_theory} in $\text{RGB}$ images. To
achive this, we use the DCT~\cite{vruiz__DCT} applied over the color
dimension~\cite{vruiz__image_IO}, resulting in an concentration of the
energy of the image in the first
subband~\cite{vruiz__transform_coding} ($\text{DCT}^0$) which
basically represents the
\href{https://en.wikipedia.org/wiki/Luminance}{luminance} of the
image. Notice that the visual
redundancy~\cite{vruiz__visual_redundancy} is not considered.



\section{Distortion control}
We do not provide any accurate distortion control algorithm.  After
using the DCT, the coefficients are quantized with a deadzone
quantizer, simulating a bit-plane encoding. However, the dynamic range
of the $\text{DCT}^0$ coefficients doubles the dynamic range of the
pixels, and therefore, the highest quantization step will be
$\Delta=256$. Therefore, we will have eight points in our RD curve
that satisfy that
\begin{equation}
  \mathbf{\Delta}_{\text{DCT}^0} = \mathbf{\Delta}_{\text{DCT}^1} = \mathbf{\Delta}_{\text{DCT}^2}.
  \label{eq:simple_Q}
\end{equation}

Notice that the DCT is orthogonal and therefore, the contributions of
the subbands (to the quality of the reconstructed image) are
independent and therefore, additive. Because the DCT is also
orthonormal, we can measure the distortion in both, the image and the
transform domain.

\section{Rate control}
Since PNG does not provide accurate rate control, we can only select
between eight bit-rates that safisfy the Eq.~\eqred{eq:simple_Q}.

\section{Scalar quantization in the ($\text{DCT}_0,\text{DCT}_1,\text{DCT}_2$) domain}
%{{{

The DCT is orthonormal, and thereore, the distortion can be measured
directly in the transform domain. This means that we can treat the DCT
coefficients as if would be pixels. Therefore, an optimal algorithm
for RDO~\cite{vruiz__information_theory} is:

\begin{enumerate}
\item Find the RD curve for each subband. We will use the same
  quantization steps for all the subbands. This action generates three
  curves.
\item Join all the RD points in a single list and sort it by the slope
  of the
  points~\cite{vruiz__information_theory}. Track\footnote{Storing with
  the point the combination subband index and quantization step used
  for each point.} the quantization steps used for each point.
\item Use the quantization patters described by the points of the
  sorted list to obtain the definitive RD curve. Notice that, since
  the DCT is orthonormal, we could estimate accurately the distortion
  in the definitive curve using directly the distortions of the
  subbands.
\end{enumerate}

Notice that the points that satisfy that
\begin{equation}
  \mathbf{\Delta}_0 = \mathbf{\Delta}_1 = \mathbf{\Delta}_2
  \label{eq:deltas}
\end{equation}
should belong to the definitive RD curve and should describe also the
convex hull of the curve. There is not guarantee that the rest of
points in which the Eq.~\eqref{eq:deltas} is not true will belong to
the convex hull (although they could be used to perform a finer
rate-control).

This algorithm has been implemented in this \href{https://github.com/Sistemas-Multimedia/Sistemas-Multimedia.github.io/blob/master/contents/RGB_DCT/RGB_DCT.ipynb}{notebook}.

\section{Ablation study}
The improvement resulting from the use of the DCT applied over the RGB
domain can be quantified if we compare the RD curves considering or
not its use.\footnote{This kind of comparison is commonly called
``\emph{Ablation Study}'' because we are ''removing'' some
functionality of the system to see how its ausence impacts on the
performance.} 

\end{comment}

\begin{comment}
   The steps
of this procedure are (see the
\href{https://github.com/Sistemas-Multimedia/Sistemas-Multimedia.github.io/blob/master/contents/color_DCT/RGB_DCT.ipynb}{notebook}):
\begin{enumerate}
\item  This fact allows us to use a ``fast'' (with
  linear
  \href{https://en.wikipedia.org/wiki/Computational_complexity}{complexity})
  RDO (Rate/Distortion Optimization) algorithm to find which
  combination of quantization steps $\mathbf{\Delta}$ that
  ``minimizes'' the RD curve of the reconstructed image, which can be
  computed as the sum\footnote{Remember, the constribution of the
  subbands are independent and additive.} of the RD curves generated
  by the $3$ subbands ($\text{DCT}^0$, $\text{DCT}^1$, and
  $\text{DCT}^2$). Therefore, we can:
  \begin{enumerate}
  \item Quantize and compress\footnote{We can consider that the
  coefficients are gray-scale pixels and use a image compressor.} each
    subband with a quantization step $\mathbf{\Delta}_s$. This results
    in $3$ RD curves, one per subband. We define the slope of the
    $n$-th point in a RD curve as the slope of the straight line
    passing through the points $n$ and $n+1$ (see the
    Fig.~\ref{fig:slope_computation}). The slope of the point for the
    highest rate is defined to $0$.  The results generated by this
    step are $3$ RD curves (one per subband).
  \item Put all the RD points in a list, tracking\footnote{Remembering
  the combination subband index and quantization step used for each
  point.} the quantization pattern used for each RD point. Compute the
    list of points of the RD curve of the reconstructed image as the
    (in descending order) sorted list, by slope, of RD points.
  \item (Optional) If the image compressor used in the first step
    generates less data overhead when a color image is compressed
    considering the three channels at the same time, recompute the
    rate of each RD point of the definitive curve. For example, in the
    case of PNG~\cite{vruiz__PNG}, even when the intercomponent
    redundancy is not exploited, the number of headers is decreased if
    we perform only one compression per image.
  \end{enumerate}

  The orthonormality of the DCT also allows us to compute the
  definitive distortion directly in the DCT domain.

  Notice also that this algorithm has been designed considering the
  fact that, ideally (where the RD curve is continuous), the rate
  selected for each subband should be the
  same~\cite{vruiz__information_theory,vetterli2014foundations}, and
  therefore, should generate a RD curve such that the sum of the
  distances of the points of the RD curve to the point $(0,0)$ is
  minimized.
  
\end{enumerate}

As a final remark, take into consideration that the quantization steps
used in each subband should be selected considering aspects such as
the ganularity of the rate-control\footnote{How many points has the RD
curve.} and the features of the decoding process\footnote{For example, if we
want to provide progressive bit-plane decoding, the quantization steps
should be powers of $2$.}

\section{An even faster (but coarser) rate-control algorithm}

The running time of the previous algorithm depends on the number of
subbands, that in our case is only $3$. However, we can develop a
faster rate-control procedure (avoiding to compute the RD curves of
the subbands that is the heaviest part) if we suppose that the RD
slopes are basically determined by the distortion (i.e., without
considering the rate)\footnote{Notice that this basically means that
all the subband's curves have the
same \footnote{https://en.wikipedia.org/wiki/Domain_of_a_function}{domain}
and that the RD points occurs at the same rates}.

Obviously, in the practice it is very unlikely to happen, but if the
shapes of the curves are close enough, we can suppose that the slopes
are the same for the same points in the different curves. In the case
of the DCT that is orthonormal (the gains of the forward and the
backward filters is $1$), this is the same that define the
quantization pattern
\begin{equation}
  \mathbf{\Delta}_0 = \mathbf{\Delta}_1 = \mathbf{\Delta}_2.
\end{equation}

Notice that one way to check if this quantization algorithm is optimal
(for the obtained RD points) is to check if the ``slower'' algorithm
outputs such quantization patterns. If this is true, the ``faster''
rate-control algorithm is optimal, at least for the encoded image.

\end{comment}
