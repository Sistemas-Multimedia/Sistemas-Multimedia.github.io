% Emacs, this is -*-latex-*-

% https://www.youtube.com/watch?v=ZCDF9f1Wo0Q ( Self-Organizing Maps for Color Quantization (python) )
% https://stackoverflow.com/questions/57817645/how-do-i-re-train-an-existing-k-means-clustering-model
% https://www.deeplearningbook.org/
% https://www.researchgate.net/profile/Wenpeng-Ding/publication/242268750_Rate-distortion_optimized_color_quantization_for_compound_image_compression_-_art_no_65082Q/links/5448579f0cf2f14fb8142446/Rate-distortion-optimized-color-quantization-for-compound-image-compression-art-no-65082Q.pdf?origin=publication_detail (RATE-DISTORTION OPTIMIZED COLOR QUANTIZATION FOR COMPOUND IMAGE COMPRESSION)
% https://pyclustering.github.io/docs/0.8.2/html/da/d22/classpyclustering_1_1cluster_1_1kmeans_1_1kmeans.html#details
% https://ieeexplore.ieee.org/document/839021 (Centroid neural network for unsupervised competitive learning)
% https://github.com/tranleanh/centroid-neural-networks
% https://towardsai.net/p/l/centroid-neural-network-and-vector-quantization-for-image-compression
% https://pub.towardsai.net/centroid-neural-network-an-efficient-and-stable-clustering-algorithm-b2fa8cbb2a27
% https://towardsai.net/p/l/centroid-neural-network-for-clustering-with-numpy
% https://matthew-parker.rbind.io/post/2021-01-16-pytorch-keras-clustering/
% https://theswissbay.ch/pdf/Gentoomen%20Library/Information%20Theory/Compression/
% https://scikit-learn-extra.readthedocs.io/en/stable/modules/cluster.html
%

%\newcommand{\SM}{\href{http://cms.ual.es/UAL/estudios/masteres/plandeestudios/asignaturas/asignatura/MASTER7114?idAss=71142105&idTit=7114}{Sistemas Multimedia}}
\newcommand{\SM}{\href{https://sistemas-multimedia.github.io/}{Sistemas Multimedia}}

\newcommand{\theproject}{\href{https://github.com/Sistemas-Multimedia/MRVC}{MRVC}}

\newcommand{\SW}{\href{https://github.com/Sistemas-Multimedia/MRVC}{MRVC}}

\title{\SM{} - \href{https://github.com/Sistemas-Multimedia/Sistemas-Multimedia.github.io/tree/master/contents/color_transforms}{Color Transforms}}

\maketitle
\tableofcontents

\section{Spectral decorrelation}

Most image and video compressors exploit the statistical (and also
perceptual\footnote{This will be addresed latter in this course.})
correlation between the $\text{RGB}$ color
\emph{components}\footnote{A component of a pixel in the $\text{RGB}$
domain refer to one of the values $\text{R}$ (red), $\text{G}$ (green)
and $\text{B}$ (blue) coordinates in the $\text{RGB}$ color 3D space.}
of the pixels, using a color transform.

Color transforms are pixel-wise operators. As a resul, each pixel is
represented in a different domain where three new
\emph{coefficients}\footnote{Most part of the transforms, including
the color ones, analyze the signal information from a frequency
perspective, generating the so called coefficients whose index in the
transform domain is related to a different frequency of the signal.}
express the same\footnote{In general, the color transforms can be
considered lossless, although this is only true when fixed-point
arithmetic is used.} information but in a different color domain.

\section{Luma and chromas}

Most color transforms are designed to split the color information of a
pixel in \emph{luminance} (luma) and \emph{chrominance} (chroma). The
luma is basically the low frequency\footnote{It is worth to understand
  that the frequency concept in the color transform domain is not
  related to the frequency concept in the original pixel domain. For
  example, the $\text{R}$ component or a pixel represents the amount
  of red in the pixel, and in the visible spectrum we are refering to
  frequencies that are smaller than the frequency that the $\text{G}$
  and $\text{B}$ components represent. However, in a color transformed
  domain, the luma measures brightness level of the pixel, and we
  cannot found a subband of frequencies in the visible spectrum that
  can represent such information.} information of the pixel, and the
chroma (logically) the high frequency information.

For example, in
\href{https://en.wikipedia.org/wiki/JPEG#JPEG_codec_example}{JPEG} and
in
\href{https://en.wikipedia.org/wiki/Advanced_Video_Coding#Fidelity_range_extensions_and_professional_profiles}{H.264/AVC}
the color information of each pixel is transformed from the
$\text{RGB}$ color space to the $\text{YCrCb}$ color space, and in
\href{https://en.wikipedia.org/wiki/JPEG_XR#Description}{JPEG XR}, the
$\text{YCoCg}$ color space is used. In these luma-based color spaces,
$\text{Y}$ represents the luma (coefficient) of the pixel. The other
two form the chroma.

Notice that the chrominance of a pixel is determined by two chromas.

\subsection*{Channels and subbands (a word about notation)}

Apart from using the terms of component and coefficient, we will use
the word \emph{channel} to refeer to the same index component of all
the pixels of an image (or video), and \emph{subband} to denote the
same index coefficient generated after the transformation of all the
pixels of an image (or video). For example, the $\mathbf{R}$ channel of
a color image corresponds to the \emph{monochromatic}
image\footnote{That can be considered as a
  \emph{single-channel}/\emph{mono-component}/\emph{scalar} image.}
generated by the $\text{R}$ component of all the pixels of the image,
and for example, the $\mathbf{Y}$ subband of a transformed
($\text{RGB}$) image corresponds to the $\text{Y}$ coefficients of all
the pixels (also a ``\emph{monochromatic}'' image).

In image and video coding, most color transforms map 3 channels
($\text{RGB}$) into 3 subbands.

\section{Benefits of color transforms}

Color transforms applied to natural visual information generally have
two key advantages:
\begin{enumerate}
\item \textbf{Energy concentration:} In general, transforms ``move''
  the energy\footnote{In general, the information provided by the
    signals.} between subbands, accumulating most of the energy in a
  reduced number of them (aspect related to the so called coding gain
  of the transform~\cite{vruiz__transform_coding}). In our case, where
  the transformations are between color spaces, in the transform
  domain most of the energy is concentrated in the $\text{Y}$
  subband. As a consequence of this, usually, \textbf{the
    entropy~\cite{vruiz__information_theory} is decreased} and
  \textbf{the dynamic range of the signal is increased}. The first
  means that will compress more\footnote{For the same bit-rate.}, and
  the second that we will be able to use a higher range of
  quantization step
  sizes~\cite{vruiz__scalar_quantization,sayood2017introduction},
  increasing also the number of feasible points in the RD
  curve~\cite{vruiz__information_theory}.
\item \textbf{Luma/croma analysis:} Our visual system is more
  sensitive in terms of spatial resolution to the luma
  (``black-and-white'') information than to the chroma (``color'')
  information, which basically means that we can quantize more the
  chroma.\footnote{Notice again, that we will study this effect in a
  posterior session.}
\end{enumerate}

\section{Scalar quantization}

If the color transform is orthogonal or bi-orthogonal, i.e., the luma
and the cromas are independent, the quantization noise generated in
the subbands is additive~\cite{burger2016digital}. Therefore, from a
pure RD point of view, the quantization step sizes for each subband
should be selected using the same RD-slope in all the subbands (see
the notebook
\href{https://github.com/vicente-gonzalez-ruiz/color_transforms/blob/main/docs/RGB/RGB_SQ.ipynb}{Scalar
  Quantization of RGB images}. Notice that this implies to compute the
RD curves.

Considering a generic luma/croma transform $\text{YUV}$, if we expect that
\begin{equation}
  \lambda^{\text{Y}} \approx \lambda^{\text{U}} \approx \lambda^{\text{V}}
  \label{eq:optimal_lambda}
\end{equation}
for a given quantization step size $\Delta$, the
RDO~\cite{vruiz__information_theory} can be ignored.

In the notebook
\href{https://github.com/Sistemas-Multimedia/Sistemas-Multimedia.github.io/blob/master/contents/RGB_SQ/RGB_SQ.ipynb}{Scalar
  Quantization of RGB images} we can explore (at least, visually) the
grade of compilance of Eq.~\eqref{eq:optimal_lambda}.

\subsection*{Resources}
See the notebooks
\href{https://github.com/vicente-gonzalez-ruiz/color_transforms/blob/main/docs/3DCT/3DCT_over_RGB.ipynb}{Removing
  RGB redundancy with the DCT},
\href{https://github.com/vicente-gonzalez-ruiz/color_transforms/blob/main/docs/YCoCg/YCoCg_over_RGB.ipynb}{Removing
  RGB redundancy with the $\text{YCoCg}$ transform} and
\href{https://github.com/vicente-gonzalez-ruiz/color_transforms/blob/main/docs/YCrCb/YCrCb_over_RGB.ipynb}{Removing
  RGB redundancy with the $\text{YCrCb}$ transform}.

\section{Vector quantization of $\text{RGB}$ images}

SQ (Scalar
Quantization)~\cite{vruiz__scalar_quantization,sayood2017introduction}
would be an optimal solution only if the image colors are uniformly
distributed within
\href{https://en.wikipedia.org/wiki/RGB_color_model}{the RGB
  cube}. However, the typical color distribution in natural images is
anything but uniform, with some regions of the color space being
densely populated and many potentially used colors entirely
missing. In this case, depending on the quantization step
size~\cite{vruiz__signal_quantization}, SQ could be suboptimal because
the used colors may not be sampled with sufficient density while at the
same time the encoding system is considering colors that do not appear
in the image at all~\cite{burger2016digital}.

On the other hand, VQ (Vector
Quantization)~\cite{vruiz__vector_quantization,sayood2017introduction}
applied to the color domain does not treat the individual $\text{RGB}$
components separately as does scalar quantization, but each used color
vector ${\mathbf C}_i = (\text{R}_i, \text{G}_i, \text{B}_i )$ in the
image is treated as a minimum structure. VQ determines a code-book of
$K$ code-vectors (centroids) that minimizes the distortion between the
original image and the reconstructed one. Notice that the code-book
must be known by the decoder to find a reconstruction.

\subsection*{Resources}
See the notebooks
\href{https://github.com/vicente-gonzalez-ruiz/vector_quantization/blob/main/docs/RGB_VQ.ipynb}{Vector
  Quantization (in the color domain) of a RGB image} and
\href{https://github.com/vicente-gonzalez-ruiz/vector_quantization/blob/main/docs/spatial_color_VQ.ipynb}{Vector
  Quantization (in the 2D domain) of a color (RGB) image}.

\section{To-Do}
\begin{enumerate}
\item Modify VCF to allow the use of the (color) $\text{DCT}$ in the
  compression pipeline. Complexity 1.
\item Modify VCF to allow the use of the $\text{YCrCb}$ transform in
  the compression pipeline. Complexity 1.
\item Modify VCF to allow the use of VQ (Vector Quantization) (applied to
  the color domain) in the compression pipeline. Complexity 4.
\end{enumerate}

\section{References}

\renewcommand{\addcontentsline}[3]{}% Remove functionality of \addcontentsline
\bibliography{maths,data_compression,signal_processing,DWT,image_compression,image_processing,information_theory,quantization}
