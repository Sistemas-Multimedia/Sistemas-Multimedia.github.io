% Emacs, this is -*-latex-*-

%\newcommand{\SM}{\href{http://cms.ual.es/UAL/estudios/masteres/plandeestudios/asignaturas/asignatura/MASTER7114?idAss=71142105&idTit=7114}{Sistemas Multimedia}}
\newcommand{\SM}{\href{https://sistemas-multimedia.github.io/}{Sistemas Multimedia}}

\newcommand{\theproject}{\href{https://github.com/Sistemas-Multimedia/MRVC}{MRVC}}

\newcommand{\SW}{\href{https://github.com/Sistemas-Multimedia/MRVC}{MRVC}}

\title{\SM{} - Study Guide - Milestone 3 - Task 1: Scalar Quantization of Gray-scale Images}

\maketitle
\tableofcontents

\section{Description}
%{{{

Gray-Scale (GS) images have only one (gray or luminance)
channel\footnote{They are usually represented in color displays using
gray tones (colors where the same amount of red, green and blue
components have been mixed). However, it is quite difficult to find a
color display with more than 256 gray colors.}. In most of the cases,
the number of bits per pixel (the so called ``depth'' of the image) is
$8$, but it is not rare to use up to $16$ bits/pixel in specific
contexts such as in
\href{https://en.wikipedia.org/wiki/Medical_imaging}{Medical Imaging}
and \href{https://en.wikipedia.org/wiki/Remote_sensing}{Remote
  Sensing}~\cite{burger2016digital}.

In this milestone we are going to see how to
\href{https://vicente-gonzalez-ruiz.github.io/quantization/}{quantize}
a GS image, and then compression the image with PNG, to create a
\href{https://en.wikipedia.org/wiki/Lossy_compression}{lossy image
  compressor}.

\subsection{Scalar and vector quantization of GS images}
%{{{

The compression ratio of our lossy image compressor can be controled
by means of quantization. At this point we have basically two
alternatives:
\begin{enumerate}
\item \textbf{Scalar Quantization (SQ)}: where each pixel is quantized
  without considering the rest of pixels in the
  image~\cite{vruiz__scalar_quantization}.
\item \textbf{Vector Quantization (VQ)}: when the pixels are quantized
  block-by-block (2D vectors)~\cite{vruiz__vector_quantization}.
\end{enumerate}
Notice that VQ exploits the spatial correlation, but not SQ. For this
reason, we will use PNG after SQ, because PNG can remove the spatial
redundancy. In the case of VQ, only the statistical redundancy
remains, that can be exploited by any entropy codec (and obviously,
any image codec).

%}}}

\subsection{Bit-rate control in SQ + PNG}
%{{{

The bit-rate (the number of bits/pixel) obtained after using
quantization + PNG depends on:
\begin{enumerate}
\item The number of output indexes generated by the quantizer.
\item The capability of PNG for removing spatial/statistical
  redudancy, aspect that we cannot control with accuracy (only some
  levels of compression are usually available).
\item In the case of SQ, the quantization step size $\Delta$
  used. Notice that, usually, the higher the $\Delta$, the higher the
  compression ratio, the lower the number of bits/pixel, and the lower
  the quality of the reconstructed image.
\end{enumerate}
Variying $\Delta$ we should be able to generate a Rate/Distortion (RD)
curve, where the X-axis represents the bit-rate (in bit/pixel, for
example) and the Y-axis represents the distortion (the
\href{https://en.wikipedia.org/wiki/Root-mean-square_deviation}{Root
  Mean Square Error (RMSE)}, for example).

Normaly, RD curves are convex (see Fig.~\ref{fig:RD_slopes}), which
means that if $\lambda_i$ is the slope of the curve measured at the
$i$-th point of the curve (starting at the lowest bit-rate), it
usually hold that
\begin{equation}
  \lambda_i > \lambda_{i+1}.
  \label{eq:convexity}
\end{equation}
where $\lambda$ quantifies the trade-off between decreasing the
distortion\footnote{For this reason, the slopes are negative.} while
the bit-rate
increases~\cite{vetterli1995wavelets,sayood2017introduction}. However,
Eq.~\ref{eq:convexity} is not always true in the real world. This can
be seen in the notebooks
\href{https://github.com/Sistemas-Multimedia/Sistemas-Multimedia.github.io/blob/master/contents/scalar_quantization/midtread.ipynb}{midtread\_quantization},
\href{https://github.com/Sistemas-Multimedia/Sistemas-Multimedia.github.io/blob/master/contents/scalar_quantization/midrise.ipynb}{midrise\_quantization},
\href{https://github.com/Sistemas-Multimedia/Sistemas-Multimedia.github.io/blob/master/contents/scalar_quantization/deadzone.ipynb}{deadzone\_quantization}
and
\href{https://github.com/Sistemas-Multimedia/Sistemas-Multimedia.github.io/blob/master/contents/scalar_quantization/companded.ipynb}{companded\_quantization}.

As it can be also seen in the notebook
\href{https://github.com/Sistemas-Multimedia/Sistemas-Multimedia.github.io/blob/master/contents/scalar_quantization/compare_quantizers.ipynb}{quantizers\_comparison},
the performance of the quantizers is not the same: usually midrise and
midtread, performs better than deadzone at intermediate bit-rates, but
deadzone is the best a low bit-rates. Deadzone has also another
advantage over midread and midtread: when $\Delta$ is a power of 2
(which corresponds to a bit-plane encoding), the obtained RD point is
near optimal in the RD space.

%}}}

%}}}

\section{What do I have to do?}
%{{{

\begin{enumerate}
\item Create a notebook named ``bit-rate\_control.ipynb'' that, given
  a maximum bit-rate, finds the quantization step $\Delta^*$ that
  minimizes the distortion. Use the quantizer that your prefer.
\end{enumerate}

%}}}

\section{Timming}

Please, finish this milestone before the next class session.

\section{Deliverables}

None.

\section{Resources}

\renewcommand{\addcontentsline}[3]{} % Remove functionality of \addcontentsline
\bibliography{data-compression,signal-processing,DWT,image-processing}
