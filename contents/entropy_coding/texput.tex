% Emacs, this is -*-latex-*-

%\newcommand{\SM}{\href{http://cms.ual.es/UAL/estudios/masteres/plandeestudios/asignaturas/asignatura/MASTER7114?idAss=71142105&idTit=7114}{Sistemas Multimedia}}
\newcommand{\SM}{\href{https://sistemas-multimedia.github.io/}{Sistemas Multimedia}}

\newcommand{\theproject}{\href{https://github.com/Sistemas-Multimedia/MRVC}{MRVC}}

\newcommand{\SW}{\href{https://github.com/Sistemas-Multimedia/MRVC}{MRVC}}


\title{\SM{} - Entropy Coding}

\maketitle

\tableofcontents

\section{What is Entropy Coding?}

Entropy Coding (EC) encompasses a whole series of coding techniques
that exploit the statistical redundancy of data with the ultimate goal
of finding a more compact representation.

Entropy Coding is related to the definition of entropy in the context
of the Information Theory~\cite{vruiz__information_theory}. In this
area, the entropy quantifies the volume of information represented by
a data set, so that the higher the entropy, the better the efficiency
of such representation would be.

There are basically two types of entropy encoders depending on how the
sequence of symbols that make up the data to be encoded are processed:

\begin{enumerate}
\item Those that process the sequence symbol by symbol. Examples are
  Huffman Coding~\cite{vruiz__huffman_coding} and Arithmetic
  Coding~\cite{vruiz__arithmetic_coding}.
\item Those that process the sequence by blocks of symbols
  (strings). Examples are Run-Length Encoding (RLE)~\cite{vruiz__rle} and all the
  dictionary-based text compressors, such as LZW~\cite{vruiz__LZW}.
\end{enumerate}
  
Those of the first type generally achieve more compact
representations, but are more computationally expensive. The
algorithms of the second class are usually faster, but slighlt worse
in compression ratio.

\section{Portable Network Graphics (PNG)}

\href{https://en.wikipedia.org/wiki/Portable_Network_Graphics}{PNG}~\cite{vruiz__PNG}
(pronounced ``ping'') a dictionary-based
\href{https://en.wikipedia.org/wiki/Lossless_compression}{lossless
  image compression format} used for representing
\href{https://en.wikipedia.org/wiki/Digital_data}{digital}
\href{https://en.wikipedia.org/wiki/Digital_image}{images} and
\href{https://en.wikipedia.org/wiki/Video}{videos}~\cite{vruiz__image_video}.

PNG is the default EC in the
\href{https://github.com/Sistemas-Multimedia/VCF}{VCF project} because
is fast and the compression performance is good.

\section{Resources}

\renewcommand{\addcontentsline}[3]{}% Remove functionality of \addcontentsline
\bibliography{text-compression,image-formats,image-video-theory}
