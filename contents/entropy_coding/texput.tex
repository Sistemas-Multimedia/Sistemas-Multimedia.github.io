% Emacs, this is -*-latex-*-

%\newcommand{\SM}{\href{http://cms.ual.es/UAL/estudios/masteres/plandeestudios/asignaturas/asignatura/MASTER7114?idAss=71142105&idTit=7114}{Sistemas Multimedia}}
\newcommand{\SM}{\href{https://sistemas-multimedia.github.io/}{Sistemas Multimedia}}

\newcommand{\theproject}{\href{https://github.com/Sistemas-Multimedia/MRVC}{MRVC}}

\newcommand{\SW}{\href{https://github.com/Sistemas-Multimedia/MRVC}{MRVC}}


\title{\SM{} - \href{https://sistemas-multimedia.github.io/contents/entropy_coding}{Entropy Coding}}

\maketitle

\tableofcontents

\section{What is entropy coding?}

\href{https://en.wikipedia.org/wiki/Entropy_coding}{Entropy coding}
(EC)~\cite{vruiz__entropy_coding} encompasses a whole series of coding
techniques that exploit the
\href{https://en.wikipedia.org/wiki/Redundancy_(information_theory)}{statistical
  redundancy} of data with the ultimate goal of finding a more compact
representation. EC is related to the definition of
\href{https://en.wikipedia.org/wiki/Entropy_(information_theory)}{entropy}
in the context of the information
theory~\cite{vruiz__information_theory}. In this area, the entropy
quantifies the volume of
\href{https://en.wikipedia.org/wiki/Information}{information}
represented by a data set, so that the higher the entropy, the better
the efficiency of such representation
(\href{https://en.wikipedia.org/wiki/Data}{data}).

\section{A classification of entropy encoders}

There are basically two types of entropy encoders depending on how the
sequence of \href{https://en.wikipedia.org/wiki/Symbol}{symbols} that
make up the data to be encoded are processed:

\begin{enumerate}
\item Those that process the sequence symbol by symbol. Examples of
  this type of algorithms are Huffman
  coding~\cite{vruiz__huffman_coding} and arithmetic
  coding~\cite{vruiz__arithmetic_coding}. These encoders are also
  called ``probabilistic encoders'' because the number of bits of code
  assigned to a symbol $s$ depends on the probability of the symbol
  $p(s)$.
\item Those that process the sequence by blocks of symbols
  (strings). Examples are Run-Length Encoding (RLE)~\cite{vruiz__rle}
  and all the dictionary-based text compressors, such as
  LZW~\cite{vruiz__LZW}.
\end{enumerate}
  
Those of the first type generally achieve more compact
representations, but computationally are more expensive. The
algorithms of the second class are usually faster, but slightly worse
in compression ratio.

\section{\href{https://en.wikipedia.org/wiki/Huffman_coding}{Huffman coding}}

A Huffman code~\cite{vruiz__huffman_coding,ruiz2000compresion} is a
\href{https://en.wikipedia.org/wiki/Prefix_code}{prefix code} that
allows to ``navigate'' through the so called Huffman (inverted) Tree
from the trunk to one of its leaves, without
uncertainty. The Huffman Tree satisfies that
\begin{equation}
  l(c(s)) = \lceil I(s)\rceil,
  \label{eq:huffman_performance}
\end{equation}
where $l(c(s))$ is the length of the (Huffman) code assigned to the
symbol $s$ and $I(s)$ is the amount of information represented by $s$,
measured bits of information~\cite{vruiz__information_theory}.

Notice that the minimum number of bits that can be used for
representing a symbol (using Huffman) is 1, which can be a problem
when the length of the alphabet is small.\footnote{In this case,
Huffman is ineffective.} Another drawback of Huffman coding is that
the Huffman tree must be updated each time the probabilistic model is
updated, action very frequent in adaptive models.

\subsection*{Resources}
\begin{enumerate}
\item \href{https://www.inference.org.uk/mackay/python/compress/#Huff}{Compression algorithms in python}.
\item \href{https://favtutor.com/blogs/huffman-coding}{Huffman Coding Implementation in Python with Example}.
\item \href{https://www.programiz.com/dsa/huffman-coding}{Huffman Coding}.
\item \href{https://en.wikipedia.org/wiki/Prediction_by_partial_matching}{Prediction by partial matching}.
\item \href{https://en.wikipedia.org/wiki/Lossless_JPEG}{Lossless JPEG}.
\item \href{https://en.wikipedia.org/wiki/Context-adaptive_variable-length_coding}{Context-adaptive variable-length coding}.
\end{enumerate}

\section{\href{https://en.wikipedia.org/wiki/Arithmetic_coding}{Arithmetic
    coding}}

In an arithmetic codec~\cite{vruiz__arithmetic_coding,ruiz2000compresion}, the number of
bits of data that can be used for representing a symbol can match
exactly the number of bits of information, i.e,
\begin{equation}
  l(c(s)) = I(s).
\end{equation}

This also means that even if the size of the alphabet is small, the
coding performance of an arithmetic code is optimal, although this
optimality is only fully satisfied if the number of symbols to encode
is infinite. Notice also that, if the alphabet is large, the encoding
performance difference between a Huffman code (or any other prefix
code) and an arithmetic code, vanishes.

\subsection*{Resources}
\begin{enumerate}
\item
  \href{https://github.com/vicente-gonzalez-ruiz/arithmetic_coding/blob/master/src/arithmetic_coding/arithmetic_coding.ipynb}{Arithmetic
    coding (notebook)}.
\item \href{https://www.nayuki.io/page/reference-arithmetic-coding}{Reference arithmetic coding}/\href{https://github.com/nayuki/Reference-arithmetic-coding}{Reference arithmetic coding}.
\item \href{https://www.inference.org.uk/mackay/python/compress/#AC}{Compression algorithms in python}.
\item \href{https://en.wikipedia.org/wiki/Prediction_by_partial_matching}{Prediction by partial matching}.
\item \href{https://en.wikipedia.org/wiki/Lossless_JPEG}{Lossless JPEG}.
\item \href{https://en.wikipedia.org/wiki/Context-adaptive_binary_arithmetic_coding}{Context-adaptive binary arithmetic coding}.
\end{enumerate}

\section{\href{https://zlib.net/}{\texttt{zlib}}}

\texttt{zlib} is based on
\href{https://en.wikipedia.org/wiki/Deflate}{DEFLATE}, which in turn
is based on LZ77~\cite{vruiz__LZW,ruiz2000compresion} and Huffman
coding. Therefore, \texttt{zlib} exploits the repetition of symbols
and also, the the 0-order statistical redundancy\footnote{In other
words, that some symbols are more frequently than others.}. One of the
main advantages of \texttt{zlib} is that is quite fast.

\subsection*{Resources}
\begin{enumerate}
\item \href{https://docs.python.org/3/library/zlib.html}{\texttt{zlib}
    — Compression compatible with gzip}.
\item \href{https://en.wikipedia.org/wiki/Lossless_JPEG}{Lossless JPEG}.
\end{enumerate}

\section{Portable Network Graphics (PNG)}

\href{https://en.wikipedia.org/wiki/Portable_Network_Graphics}{PNG}~\cite{vruiz__PNG}
(pronounced ``ping'') a dictionary-based
\href{https://en.wikipedia.org/wiki/Lossless_compression}{lossless
  image compression format} used for representing
\href{https://en.wikipedia.org/wiki/Digital_data}{digital}
\href{https://en.wikipedia.org/wiki/Digital_image}{images} and
\href{https://en.wikipedia.org/wiki/Video}{videos}~\cite{vruiz__image_video}
in III... format~\cite{vruiz__video_compression}. The entropy encoder
of PNG is based on Huffman coding and LZSS, and a pixel predictor that
removes the spatial redundancy.

We must bear in mind that as such an image compressor, we can only
interact with PNG at the image level, that is, it only accepts images
(in shades of gray or in color, with the possibility of an
\href{https://en.wikipedia.org/wiki/Alpha_compositing}{alpha channel})
and returns images.

% PNG is the default EC in the \href{https://github.com/Sistemas-Multimedia/VCF}{VCF project} because: (1) is lossless, (2) is fast, and (3) the compression performance is reasonable.

% \section{Entropy Image Coding with \href{https://en.wikipedia.org/wiki/Better_Portable_Graphics}{BPG (Better Portable Graphics)}}

\section{To-Do}
Please, select one of the following tasks to develop:
\begin{enumerate}
\item Modify \href{https://github.com/Sistemas-Multimedia/VCF}{VCF} to
  allow the use of Huffman coding as a entropy codec in the
  compression pipeline. A context-based probabilistic model should be
  used to minimize the probabilities of the symbols. Complexity 3.
\item Modify VCF to allow the use of arithmetic coding as a entropy
  codec in the compression pipeline. Again, a context-based
  probabilistic model should be used and in this case, to speed up the
  decoding of the symbols and to increase the compression ratio, it is
  convenient that they follow a Laplace statistical distribution that
  can be easily obtained using a spatial predictor. Complexity 3.
\item Modify VCF to allow the use of \texttt{zlib} as a entropy codec
  in the compression pipeline. Again, use spatial predictor to
  increase the compression ratio. Complexity 2.
\end{enumerate}

\section{References}

\renewcommand{\addcontentsline}[3]{}% Remove functionality of \addcontentsline
\bibliography{text_compression,image_formats,image_video_theory,information_theory,video_compression}
