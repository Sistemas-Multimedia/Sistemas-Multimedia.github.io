% Emacs, this is -*-latex-*-

%\newcommand{\SM}{\href{http://cms.ual.es/UAL/estudios/masteres/plandeestudios/asignaturas/asignatura/MASTER7114?idAss=71142105&idTit=7114}{Sistemas Multimedia}}
\newcommand{\SM}{\href{https://sistemas-multimedia.github.io/}{Sistemas Multimedia}}

\newcommand{\theproject}{\href{https://github.com/Sistemas-Multimedia/MRVC}{MRVC}}

\newcommand{\SW}{\href{https://github.com/Sistemas-Multimedia/MRVC}{MRVC}}


\title{\SM{} - Perceptual Coding}

\maketitle

\tableofcontents

\section{What is perceptual coding?}

Deadzone quantization (example)

\section{Quantization in the $\texttt{RGB}$ domain}

The color-recognition accuracy of the HVS is
finite~\cite{vruiz__visual_redundancy}. For this reason, quantization
in the $\texttt{RGB}$ color space can be imperceptible.

any alternative different from
\begin{equation}
  \mathbf{\Delta}^{\text{R}}_i = \mathbf{\Delta}^{\text{G}}_i =
  \mathbf{\Delta}^{\text{B}}_i,
  \label{eq:simple_Q}
\end{equation}
where $i$ represents the $i$-th quantization step size, will produce
some alteration in the color

\section{Chroma redundancy}

Humans do not perceive detail in the chrominance as well as in they
does in the
luminance.~\cite{burger2016digital}. As in
$\text{YCrCb}$, the
\href{https://en.wikipedia.org/wiki/Sampling_(signal_processing)}{sampling
  rate} of the chroma is usually reduces to 1/4 without any noticeable
distortion. This feature is used in some of the last image and video
compression systems such as
\href{https://en.wikipedia.org/wiki/JPEG_XR#Description}{JPEG XR} and
\href{https://en.wikipedia.org/wiki/High_Efficiency_Video_Coding#Video_coding_layer}{HEVC}.

\section{Redundancy in the RGB domain}

The Human Visual System (HVS) is more sensitive to luminance than to
chrominance~\cite{vruiz__visual_redundancy}. This basically means that
we can an encoding process we modify the color information of an image
or a video without noticing a high distortion.

Since the \href{https://en.wikipedia.org/wiki/Visual_system}{HVS
  (Human Visual System)} is not able to perceive detail in the
chrominance as well as it does in the luminance, the amount of
information, and consequently
\href{https://en.wikipedia.org/wiki/Sampling_(signal_processing)}{sampling
  rate}, used in the chrominance can be generally reduced to 1/4 of
the used for the luminance without a noticeable distortion (see
Fig.~\ref{fig:san-diego_chroma_subsampled})~\cite{burger2016digital}. This
\href{https://en.wikipedia.org/wiki/Bandwidth_(computing)}{fact is
  used when compressing} digital still images and is one of the reason
why, for example, the $\text{YCrCb}$ transform is used in the
\href{https://en.wikipedia.org/wiki/JPEG}{JPEG} image compressor.

#  V E R   R E P O   V I S U A L   R E D U N D A N C Y

\begin{figure}
  \centering
  \png{san-diego_chroma_subsampled}{1000}
  \caption{Visual effect of chroma subsamplig in the YCrCb domain. See
    this
    \href{https://github.com/Sistemas-Multimedia/Sistemas-Multimedia.github.io/blob/master/milestones/06-YUV_compression/chroma_subsampling.ipynb}{notebook}.}
  \label{fig:san-diego_chroma_subsampled}
\end{figure}


\section{To-Do}
Please, select one of the following tasks to develop:
\begin{enumerate}
\item
\end{enumerate}

\section{References}

\renewcommand{\addcontentsline}[3]{}% Remove functionality of \addcontentsline
\bibliography{quantization}
