% Emacs, this is -*-latex-*-

% https://vcgit.hhi.fraunhofer.de/jvet
% https://www.etsist.upm.es/uploaded/392/02_Video_Compression_Basics_MPEG1_MPEG2_MPEG4.pdf

\newcommand{\SM}{\href{http://cms.ual.es/UAL/estudios/masteres/plandeestudios/asignaturas/asignatura/MASTER7114?idAss=71142105&idTit=7114}{Sistemas Multimedia}}

\newcommand{\theproject}{\href{}}{MCDWT}
\title{\SM{} - Temporal Transforms}

\maketitle
\tableofcontents

\section{Temporal redundancy}
%{{{

In general, neighbour \emph{frames} (images) in (video)
\emph{sequences} exhibit a high temporal correlation that can be
exploited to increase the compression ratios. The removal of such
redundancy can be performed by transforms applied to the temporal
domain.

A temporal transform inputs two o more frames of pixels or
coefficients, and outputs at least one \emph{residual} (frame) in
which the residual-pixels have a higher dynamic range but also a
smaller entropy. This potentially increases the compression ratios.

%}}}

\section{Motion Estimation and Compensation (MEC)}
%{{{

In most video coding standards,
\href{https://en.wikipedia.org/wiki/Motion_compensation}{Motion
  Compensation (MC)} is name of the technique used to generate the
residuals. MC exploits the temporal correlation to minimize the
entropy of the residuals.

However, in general MC requires to estimate the
motion~\cite{vruiz__ME} between the frames that are
compensated. Basically, MC consists in substracting to the video data
a prediction performed with the information that must be also
avaliable\footnote{In order to make a reversible process.} at the
decoder~\cite{vruiz__MCT}. The better the prediction, the lower the
entropy of the residuals. Notice that the number of residual-pixels is
equal to the number of pixels in the compensated frame.

%}}}

\section{MEC and RDO}
%{{{

ME can be designed to minimize the distortion $D$ of the residues or
to minimize the lagrangian
\begin{equation}
  J = R + \lambda D,
\end{equation}
which also takes into consideration the bit-rate $R$ generated by the
entropy encoding of the motion data and the residues.

Notice that, in general, $D$ will decrease if the ``motion part'' of
$R$ increases. However, if the motion information can be infered by
the decoder, $R$ will be only affected by the entropy encoding of the
residues and the main objective of ME is to minimize the distortion.

On the other hand, when the motion information is infered at the
decoder, this information will be less accurate that if we use all the
visual information available at the encoder.\footnote{The decoder, in
general, has a limited access to the original information because the
encoding system is lossy.}

%}}}

\section{Resources}

\section{To-Do}
%{{{

\begin{enumerate}
\item Modify VCF to encode/decode a sequence of images using a
  III... scheme. Complexity 1.
\item Modify VCF to encode/decode a sequence of images using a
  IPP... scheme, without motion compensation. Complexity 2.
\item Modify VCF to encode/decode a sequence of images using a
  IPP... scheme, with motion compensation. Complexity 4.
\end{enumerate}

%}}}
  
\section{References}
%{{{

\renewcommand{\addcontentsline}[3]{}% Remove functionality of \addcontentsline
\bibliography{image_pyramids,DWT,motion_estimation,HEVC}

%}}}
