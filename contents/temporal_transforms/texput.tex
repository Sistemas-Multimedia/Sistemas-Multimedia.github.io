% Emacs, this is -*-latex-*-

% https://vcgit.hhi.fraunhofer.de/jvet
% https://www.etsist.upm.es/uploaded/392/02_Video_Compression_Basics_MPEG1_MPEG2_MPEG4.pdf

%\newcommand{\SM}{\href{http://cms.ual.es/UAL/estudios/masteres/plandeestudios/asignaturas/asignatura/MASTER7114?idAss=71142105&idTit=7114}{Sistemas Multimedia}}
\newcommand{\SM}{\href{https://sistemas-multimedia.github.io/}{Sistemas Multimedia}}

\newcommand{\theproject}{\href{https://github.com/Sistemas-Multimedia/MRVC}{MRVC}}

\newcommand{\SW}{\href{https://github.com/Sistemas-Multimedia/MRVC}{MRVC}}

\title{\SM{} - \href{https://github.com/Sistemas-Multimedia/Sistemas-Multimedia.github.io/tree/master/contents/temporal_transforms}{Temporal Transforms}}

\maketitle
\tableofcontents

\section{Temporal transforms exploit temporal redundancy}

\section{Block-ing}



\subsection{Fixed}

Usually depending on the available computational resources (for example, the available memory).

\subsection{Adaptive}

Usually controlled RDO. Better RD curves.

\section{GOF-ing}

\subsection{Fixed}

Usually determined by the random-access mode.

\subsection{Adaptive}

Usually controlled by RDO.

\section{Intra coding (III...)}

All frames are encoded as independent images.  All blocks in the
frames are I-type, by definition. Intra-coded frames are also named
keyframes.

Advantages:
\begin{itemize}
\item [+] Minimal pipeline delay.
\item [+] No error propagation.
\item [+] Maximum temporal scalability.
\end{itemize}

Disadvantages:
\begin{enumerate}
\item [-] Low compression ratios.
\end{enumerate}

Mainly used in video edition.

\section{Low-delay coding (IPPP...)}

In a GOP, only the first frame is intra-coded and the rest,
predictive-codec using motion compensation. In a P-frame, blocks can
be of (depending on RDO):
\begin{enumerate}
\item I-type: the block is intra-coded.
\item P-type: the block is predicted using a block that belong to a previous frame.
\item S-type: the residue of the prediction is so small that it is considered as zero.
\end{enumerate}

Advantages:
\begin{itemize}
\item [+] Low pipeline delay (one frame-time).
\item [+] Medium-high compression ratios.
\end{itemize}

Disadvantages:
\begin{enumerate}
\item [-] Error propagation sequentially, until the next GOF.
\item [-] Low temporal scalability.
\end{enumerate}

Mainly used in surveillance.

\section{``Random''-access mode (IBPB...)}

In a GOP, some or all the residue frames can be P or B-type. In a
B-type frame, blocks can be (depending on RDO) of B-type, which means
that at leas two blocks, one of them in a past frame and other in a
future frame, are used as reference blocks during the motion
compensation.

Notice that. by definition, the low-delay coding mode is also a
random-access mode.

Advantages:
\begin{itemize}
\item [+] High compression ratios.
\end{itemize}

Disadvantages:
\begin{enumerate}
\item [-] High pipeline delay (until a GOF-time).
\item [-] Error propagation (depending on the block-type decissions),
  until the next GOF.
\item [-] Low temporal scalability.
\end{enumerate}

\section{To-Do}
\begin{enumerate}
\item Using RDO, determine the optimal block sizes in a DCT-based
  image compressor. Implement such codec in VCF. Complexity 4.
\end{enumerate}

\section{Temporal redundancy}
%{{{

In general, neighbour \emph{frames} (images) in (video)
\emph{sequences} exhibit a high temporal correlation that can be
exploited to increase the compression ratios. The removal of such
redundancy can be performed by transforms applied to the temporal
domain.

A temporal transform inputs two o more frames\footnote{With pixels or
coefficients, depending on the current domain of the frame.}, and
outputs at least one \emph{residual} (frame) in which the
residual-coefficients have a higher dynamic range but also a smaller
entropy. This potentially increases the compression ratios.

%}}}

\section{Motion Compensation (MC)}
%{{{

In most video coding standards,
\href{https://en.wikipedia.org/wiki/Motion_compensation}{Motion
  Compensation (MC)} is name of the technique used to generate the
residuals. MC exploits the temporal correlation to minimize the
entropy of the residuals.

However, in general MC requires to estimate the
motion~\cite{vruiz__ME} between the frames that are
compensated. Basically, MC consists in substracting to the video data
a prediction performed with the information that must be also
avaliable\footnote{In order to make a reversible process.} at the
decoder~\cite{vruiz__MCT}. The better the prediction, the lower the
entropy of the residuals. Notice that, after using MC, the number of
residual-pixels is equal to the number of pixels in the compensated
frame.

%}}}

\section{Motion Estimation (ME) and RDO}
%{{{

ME can be designed to minimize the distortion $D$ of the residues or
to minimize the lagrangian
\begin{equation}
  J = R + \lambda D,
\end{equation}
which also takes into consideration the bit-rate $R$ generated by the
entropy encoding of the motion data and the residues.

Notice that, in general, $D$ will decrease if the ``motion part'' of
$R$ increases. However, if the motion information can be infered by
the decoder, $R$ will be only affected by the entropy encoding of the
residues and the main objective of ME is to minimize the distortion.

On the other hand, when the motion information is infered at the
decoder, this information will be less accurate that if we use all the
visual information available at the encoder.\footnote{The decoder, in
general, has a limited access to the original information because the
encoding system is lossy.}

%}}}

\section{Resources}

\section{To-Do}
%{{{

\begin{enumerate}
\item Modify VCF to encode/decode a sequence of images using a
  III... scheme. Complexity 1.
\item Modify VCF to encode/decode a sequence of images using a
  IPP... scheme, without motion compensation. Complexity 2.
\item Modify VCF to encode/decode a sequence of images using a
  IPP... scheme, with motion compensation. Complexity 4.
\end{enumerate}

%}}}
  
\section{References}
%{{{

\renewcommand{\addcontentsline}[3]{}% Remove functionality of \addcontentsline
\bibliography{image_pyramids,DWT,motion_estimation,HEVC}

%}}}
