% Emacs, this is -*-latex-*-

\newcommand{\SM}{\href{http://cms.ual.es/UAL/estudios/masteres/plandeestudios/asignaturas/asignatura/MASTER7114?idAss=71142105&idTit=7114}{Sistemas Multimedia}}

\newcommand{\theproject}{\href{}}{MCDWT}
\title{\SM{} - Study Guide - Milestone 3 - Task 2: Vector Quantization of Gray-scale Images}

\maketitle
\tableofcontents

\section{Description}
%{{{

Vector Quantization (VQ)~\cite{vruiz__vector_quantization} applied to
the spatial domain of images can provide good RD curves. After
quantizing the image we will use PNG as an entropy codec, basically
removing the statistical redundancy, because the spatial redundancy
has been removed by VQ.

%}}}

\subsection{Bit-rate control in VQ + PNG}
%{{{

The number of bits/pixels depends on  the ratio between the number
  of different combinations of patters (of a given size) found in the
  image $M$ and the number of vectors $N$ used in the code-book.
\end{enumerate}
Like in SQ (where we change $\Delta$), variying $N/M$ we should be
able to generate a Rate/Distortion (RD) curve, where the X-axis
represents the bit-rate (in bit/pixel) and the Y-axis represents the
distortion (again, the
\href{https://en.wikipedia.org/wiki/Root-mean-square_deviation}{Root
  Mean Square Error (RMSE)} can be used).

%}}}

%}}}

\section{What do I have to do?}
%{{{

\begin{enumerate}
\item Create a notebook named ``bit-rate\_control.ipynb'' that, given
  a maximum bit-rate, finds the combination of $N$ and $M$ that
  minimizes the distortion.
\end{enumerate}

%}}}

\section{Timming}

Please, finish this milestone before the next class session.

\section{Deliverables}

None.

\section{Resources}

\renewcommand{\addcontentsline}[3]{} % Remove functionality of \addcontentsline
\bibliography{data-compression,signal-processing,DWT,image-processing}
