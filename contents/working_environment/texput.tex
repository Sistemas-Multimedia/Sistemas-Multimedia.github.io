% Emacs, this is -*-latex-*-

%\newcommand{\SM}{\href{http://cms.ual.es/UAL/estudios/masteres/plandeestudios/asignaturas/asignatura/MASTER7114?idAss=71142105&idTit=7114}{Sistemas Multimedia}}
\newcommand{\SM}{\href{https://sistemas-multimedia.github.io/}{Sistemas Multimedia}}

\newcommand{\theproject}{\href{https://github.com/Sistemas-Multimedia/MRVC}{MRVC}}

\newcommand{\SW}{\href{https://github.com/Sistemas-Multimedia/MRVC}{MRVC}}

\title{\SM{} - \href{https://sistemas-multimedia.github.io/working_environment}{Working Environment}}

\maketitle

\tableofcontents

\section{About the Operating System}

\subsection{Running/Developing VCF ``Natively''}
The VCF project~\cite{vruiz__VCF} has been designed to work in
\href{https://en.wikipedia.org/wiki/Linux}{Linux}. Said that, it is
also possible to run VCF in other operating systems such as OSX and
Windows because Python~\cite{vruiz__YAPT} and
Git~\cite{vruiz__GitHub}, the two main tools that we will use, are
also available. In the case of Linux, you can use your favorite
distribution.

The developing of VCF in a native environment is the preferred option
because is the most efficient one from a pure computational point of
view.

\subsection{Running/Developing CVF in a Linux Container}
The second most efficient option in terms of computational
requirements and speed is to run Linux in a Linux container, such
as the provided by \href{https://hub.docker.com/}{Docker}
(\href{https://hub.docker.com/_/ubuntu}{example for Ubuntu})
or \href{https://linuxcontainers.org/}{LXD}. Notice however, that you
should be able to run graphical applications (such
as \href{https://linuxcontainers.org/}{ImageMagick}
and \href{https://www.mozilla.org/firefox}{Firefox}) in your
container.

\subsection{Running/Developing CVF in a Linux Virtual Machine}
VCF works also in any Linux installed in a virtual machine (VM)
environment such as \href{https://www.virtualbox.org/}{Virtual Box},
\href{https://www.vmware.com/}{VMware} and \href{https://www.vmware.com/}{Parallels}.
The only drawback of this way of running CVF is that you will require
some extra memory and CPU power, compared to the containers.

Obviously, any remote host (virtualized or not) running Linux is
completely adequate.

\subsection{Running CVF in a ``Notebook''}
Finally, it is possible to run CVF services that able to run
``notebooks'', such as
\href{https://colab.research.google.com/}{Google Colaboratory} or
\href{https://mybinder.org/}{Binder}. Notice, however, that in this
case, you will probably need to configure your notebooks for
installing all the necessary stuff from the scratch if you use these
cloud services. On the contrary, if you opt for using
\href{https://jupyter.org/}{Jupyter}, this last action is not
necessary.

\section{About the Programming Language(s)}

VCF is written in \href{https://www.python.org/}{Python}. You will need
also to interact with
a \href{https://en.wikipedia.org/wiki/Command-line_interface}{commands
line terminal}.

\section{About the Project Hosting Platform}

VCF is hosted at \href{https://github.com}{GitHub}.

\section{References}

\renewcommand{\addcontentsline}[3]{}% Remove functionality of \addcontentsline
\bibliography{projects}
