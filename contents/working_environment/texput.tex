% Emacs, this is -*-latex-*-

%\newcommand{\SM}{\href{http://cms.ual.es/UAL/estudios/masteres/plandeestudios/asignaturas/asignatura/MASTER7114?idAss=71142105&idTit=7114}{Sistemas Multimedia}}
\newcommand{\SM}{\href{https://sistemas-multimedia.github.io/}{Sistemas Multimedia}}

\newcommand{\theproject}{\href{https://github.com/Sistemas-Multimedia/MRVC}{MRVC}}

\newcommand{\SW}{\href{https://github.com/Sistemas-Multimedia/MRVC}{MRVC}}

\title{\SM{} - \href{https://github.com/Sistemas-Multimedia/Sistemas-Multimedia.github.io/tree/master/contents/working_environment}{Working Environment}}

\maketitle

\tableofcontents

\section{Recommended Operating System}

\subsection{Running/Developing VCF ``Natively''}
The VCF project~\cite{vruiz__VCF} is written in
Python~\cite{python}. Therefore, you should be able to run the VCF
modules on any machine with a Python interpreter installed. Said that,
it is highly recommended to use Linux or a UNIX-based OS such as
OSX. However, with some configuration effort (for example, after
installing Python), VCF should be able to run also in Windows.

Running VCF natively is the preferred solution.

\subsection{Running/Developing CVF in a Linux Container}
The second most convinient option in terms of computational
requirements and speed is to run Linux in a Linux container, such
as the provided by \href{https://hub.docker.com/}{Docker}
(\href{https://hub.docker.com/_/ubuntu}{example for Ubuntu})
or \href{https://linuxcontainers.org/}{LXD}. Notice, however, that you
should be able to run graphical applications (such
as \href{https://linuxcontainers.org/}{ImageMagick}
and \href{https://www.mozilla.org/firefox}{Firefox}) in your
container.

Recall that in Windows there exists the possibility of using the
\href{https://learn.microsoft.com/en-us/windows/wsl/install}{Windows
  Subsystem for Linux (WSL)}.

\subsection{Running/Developing CVF in a Linux Virtual Machine}
VCF works also in any Linux installed in a virtual machine (VM)
environment such as \href{https://www.virtualbox.org/}{Virtual Box},
\href{https://www.vmware.com/}{VMware} and \href{https://www.vmware.com/}{Parallels}.
The only drawback of this option is that you will require
some extra memory and CPU power, compared to the previous solutions.


\subsection{Running CVF in a ``Notebook''}
VCF modules can be invoked from \href{https://jupyter.org/}{Jupyter}
notebooks and similar environments, such as
\href{https://colab.research.google.com/}{Google Colaboratory} and
\href{https://mybinder.org/}{Binder}. Notice, however, that in these
cases, you will probably need to configure your notebooks to install
all the necessary stuff from scratch.

\section{About the Programming Language(s)}

VCF is fully written in
\href{https://www.python.org/}{Python}~\cite{vruiz__YAPT}. You will
also need to interact with a
\href{https://en.wikipedia.org/wiki/Command-line_interface}{commands
  line terminal}. It is recommended to use
\href{https://docs.python.org/3/library/venv.html}{virtual
  environments}.

\section{About the Software Versions System}

VCF is hosted at \href{https://github.com}{GitHub} and Git~\cite{vruiz__GitHub}.

\section{References}

\renewcommand{\addcontentsline}[3]{}% Remove functionality of \addcontentsline
\bibliography{projects,python}
