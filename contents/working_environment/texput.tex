% Emacs, this is -*-latex-*-

%\newcommand{\SM}{\href{http://cms.ual.es/UAL/estudios/masteres/plandeestudios/asignaturas/asignatura/MASTER7114?idAss=71142105&idTit=7114}{Sistemas Multimedia}}
\newcommand{\SM}{\href{https://sistemas-multimedia.github.io/}{Sistemas Multimedia}}

\newcommand{\theproject}{\href{https://github.com/Sistemas-Multimedia/MRVC}{MRVC}}

\newcommand{\SW}{\href{https://github.com/Sistemas-Multimedia/MRVC}{MRVC}}

\title{Working Environment}

\maketitle

\tableofcontents

\section{Visual Coding Framework (VCF) Project}
\href{https://github.com/Sistemas-Multimedia/VCF}{VCF}~\cite{vruiz__VCF}
is a Python~\cite{python} application that implements a collection of
algorithms used in image and video compression. VCF is hosted at
GitHub and developed by the students of Sistemas Multimedia.

\section{Recommended Working Environments}

\subsection{Using your Operating System}
VCF modules should run on any machine with a Python interpreter
installed. Said that, it is recommended to use Linux or a
UNIX-based OS such as OSX. However, with some configuration effort
(for example, after installing Python), VCF should run also in
Windows.

Running VCF natively is the preferred solution. Recall that in Windows
there exists the possibility of using the
\href{https://learn.microsoft.com/en-us/windows/wsl/install}{Windows
  Subsystem for Linux (WSL)}.

\subsection{Using a Container}
The second most convinient option in terms of computational
requirements and speed is to run Linux in a container, such
as the provided by \href{https://hub.docker.com/}{Docker}
(\href{https://hub.docker.com/_/ubuntu}{example for Ubuntu})
or \href{https://linuxcontainers.org/}{LXD}. Notice, however, that you
will need to run graphical applications (such
as \href{https://linuxcontainers.org/}{ImageMagick}
and \href{https://www.mozilla.org/firefox}{Firefox}) in your
container.

\subsection{Using a Virtual Machine}
VCF works also in any Linux installed in a virtual machine (VM)
environment such as \href{https://www.virtualbox.org/}{Virtual Box},
\href{https://www.vmware.com/}{VMware} and \href{https://www.vmware.com/}{Parallels}.
The only drawback of this option is that you will require
some extra memory and CPU power, compared to the previous solutions.

In this category also fall cloud notebook services, such as
\href{https://colab.research.google.com/}{Google Colaboratory},
\href{https://mybinder.org/}{Binder}, and
\href{https://www.kaggle.com/}{Kaggle}, that allow to install Python
packages and run shell scripts.

\begin{comment}
\subsection{Running CVF in a ``Notebook''}
VCF modules can be invoked from \href{https://jupyter.org/}{Jupyter}
notebooks and similar environments, such as
\href{https://colab.research.google.com/}{Google Colaboratory} and
\href{https://mybinder.org/}{Binder}. Notice, however, that in these
cases, you will probably need to configure your notebooks to install
all the necessary stuff from scratch.

\section{About the Programming Language(s)}

VCF is fully written in
\href{https://www.python.org/}{Python}~\cite{vruiz__YAPT}. You will
also need to interact with a
\href{https://en.wikipedia.org/wiki/Command-line_interface}{commands
  line terminal}. It is recommended to use
\href{https://docs.python.org/3/library/venv.html}{virtual
  environments}.
\end{comment}

\section{Programming environment}

The most convenient way of developping a Python application is using a
\href{https://docs.python.org/3/library/venv.html}{Python environment}:
\begin{verbatim}
python3 -m venv SM
source SM/bin/activate # In Linux and OSX
# SM\Scripts\activate.bat # In Windows using cmd.exe
# SM\\Scripts\Activate.ps1 # In Windows using PowerShell
\end{verbatim}

\section{About the Software Versions System}

VCF is hosted by \href{https://github.com}{GitHub} and
Git~\cite{vruiz__GitHub} in the associated software version system.

\section{Basic usage of VCF}

See the VFC's \href{https://github.com/Sistemas-Multimedia/VCF/blob/main/README.md}{README.md} file.

\section{References}

\renewcommand{\addcontentsline}[3]{}% Remove functionality of \addcontentsline
\bibliography{projects,python}
