% Emacs, this is -*-latex-*-

%\newcommand{\SM}{\href{http://cms.ual.es/UAL/estudios/masteres/plandeestudios/asignaturas/asignatura/MASTER7114?idAss=71142105&idTit=7114}{Sistemas Multimedia}}
\newcommand{\SM}{\href{https://sistemas-multimedia.github.io/}{Sistemas Multimedia}}

\newcommand{\theproject}{\href{https://github.com/Sistemas-Multimedia/MRVC}{MRVC}}

\newcommand{\SW}{\href{https://github.com/Sistemas-Multimedia/MRVC}{MRVC}}

\title{\SM{} - Study Guide - Milestone 5 - Task 1 - Removing Spatial Redundancy with the 2D-Discrete Cosine Transform}

\maketitle

\tableofcontents

\section{Description}

This task shows the effect in the RD curve of using the
\href{https://en.wikipedia.org/wiki/Discrete_cosine_transform}{DCT}~\cite{vruiz__DCT}
when it is applied by 2D blocks over the space domain of an image. The
DCT is the key of the spatial decorrelation (information
concentration) used in some of the most important image and video
compressors, such as \href{https://en.wikipedia.org/wiki/JPEG}{JPEG}
and \href{https://en.wikipedia.org/wiki/Advanced_Video_Coding}{H.264},
\href{https://en.wikipedia.org/wiki/Advanced_Video_Coding}{HEVC} and
\href{https://en.wikipedia.org/wiki/Versatile_Video_Coding}{VVC}.

\subsection{Spatial decorrelation with the DCT}

The visual information of an image can be concentrated in a small
number of DCT coefficients using the following algorithrm:
\begin{enumerate}
\item Divide the image into blocks. Usually, the blocks are not
  overlapping, but this is not mandatory because overlapped blocks
  could minimize the visually-perceptible blocking artifacts generated
  at low bit-rates. Notice that the block sizes can be adapted to the
  local characteristics of the image with the aim of minimizing the RD
  curve (for example, in those regions where the RD performance of the
  DCT is small, the block size should be also small).
\item Transform each block using the 2D-DCT to each block-component,
  generating a subband-component per block-component.
\item Quantize and compress each $i$-th block with a quantization step
  size ${\mathbf\Delta}_i$ such that the overall RD is minimized. To
  achieve this, the selected RD slope used in each subband-component
  should be the same~\cite{vetterli2014foundations}. Notice that,
  because the DCT is orthonormal, the distortion (generated by the
  quantization) is not affected by the DCT and therefore, the
  distortion in the reconstructed image can be computed directly in
  the DCT-domain.
\item Compress the quantization indexes of all the
  subband-components. For the sake of simplicity, we can consider the
  indexes as ``pixels'' and use a lossless image compressor, such as
  PNG.
\end{enumerate}

\section{What do I have to do?}
Run the \href{https://github.com/Sistemas-s-Multimedia.github.io/blob/master/contents/2D_DCT/RD_performance.ipynb}{notebook}.
%With the aim of checking that distortion can be measured in both, the
%transform and the image (RGB) domain, extend the
%previous \href{https://github.com/Sistemas-s-Multimedia.github.io/blob/master/milestones/07-%DCT/block_DCT_compression.ipynb}{notebook}
%to compute the distortion in the RGB domain. Check that the result is
%the same.

\section{Timming}

Please, finish this milestone before the next class session.

\section{Deliverables}

None.

\section{Resources}

\renewcommand{\addcontentsline}[3]{}% Remove functionality of \addcontentsline
\bibliography{maths,data-compression,signal-processing,DWT,image-compression,image-processing}
