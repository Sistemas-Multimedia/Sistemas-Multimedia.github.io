% Emacs, this is -*-latex-*-

\newcommand{\SM}{\href{http://cms.ual.es/UAL/estudios/masteres/plandeestudios/asignaturas/asignatura/MASTER7114?idAss=71142105&idTit=7114}{Sistemas Multimedia}}

\newcommand{\theproject}{\href{}}{MCDWT}
\title{\SM{} - Study Guide - Milestone 4 - Task 1 - The Color-DCT Applied to the $\text{RGB}$ Domain}

\maketitle

\tableofcontents

\section{Description}

This task explains how the RD tradeoff (expressed usually as a curve
that represents the
\href{https://en.wikipedia.org/wiki/Mean_squared_error}{MSE} vs the
bit-rate) can be improved (closer to the point $(0,0)$) when the color
redundancy is exploited to concentrate information in one of the color
channels, in this case, the luminance channel.

\subsection{Redundancy in the $\text{RGB}$ (Red, Green, and Blue) color space}
%{{{ 

In original (untransformed) images, the color of a pixel depends on
the \href{https://en.wikipedia.org/wiki/Visible_spectrum}{frequency of
  the light that the pixel captures}. Such information can be
represented in a number of different encoding systems known as
\href{https://en.wikipedia.org/wiki/Color_space}{color spaces}. Among
all those systems, the $\text{RGB}$ color space is the most used
because it can be obtained directly from the light signal using color
filters.\footnote{Specifically, a red filter, a green filter and a
blue filter.}

The $\text{RGB}$ color model is easy to implement (physically), but in
general is quite
\href{https://en.wikipedia.org/wiki/Data_redundancy}{redundant}.  In
the case of a color image expressed in the $\text{RGB}$ domain (a
$\text{RGB}$ image), the three components of each pixel can be
\href{https://en.wikipedia.org/wiki/Correlation_and_dependence}{correlated}
in the sense that if, for example, the R component is high, the other
components probably will be also high. This means that if we use an
encoding system that takes into consideration this redundancy, we can
express the same (or almos the same amount of) information using a
small number of bits (reducing the length of the code-stream).

\subsection{Color-DCT decorrelation}

One way of achieving this goal (the decorrelation in the color domain)
is to use
\href{https://vicente-gonzalez-ruiz.github.io/transform_coding/}{Transform
  Coding}~\cite{vruiz__transform_coding} applied between the color
components to ``concentrate'' information\footnote{That is usually
estimated through the measurement of the
\href{https://en.wikipedia.org/wiki/Variance}{variance} or the
\href{https://en.wikipedia.org/wiki/Entropy}{entropy}.} of the image
in a small set of coefficients, that after quantization can be
compressed more efficiently with
\href{https://en.wikipedia.org/wiki/Entropy_coding}{Entropy
  Coding}. For this task we will use the
Color-DCT~\cite{vruiz__color_DCT}.

%\footnote{For example,
%if the energy of a color subband is low, quantization could completely
%makes zero such subband, but the reconstruction of the image would be
%reasonable. The most part of entropy codecs reach higher compression
%ratios with sequences of zeros.}

%This
%\href{https://github.com/Sistemas-Multimedia/Sistemas-Multimedia.github.io/blob/master/contents/color_DCT/coding_gain.ipynb}{notebook}
%shows the near-lossless coding gain of an image after using the
%color-DCT.

%This
%\href{https://github.com/Sistemas-Multimedia/Sistemas-Multimedia.github.io/blob/master/milestones/06-YUV_compression/color_redundancy.ipynb}{notebook}
%quantifies the redundancy related to the color domain.

%}}}

\subsection{RD gain provided by the Color-DCT}

All transforms change the representation of the transformed signal,
and decorrelating transforms try to accumulate as much as possible
energy (information) in one of the channels\footnote{Or subbands,
... the name depends on the context.}.

The improvement resulting of the use of the Color-DCT can be
quantified if we compare the RD curves considering or not the
transform.\footnote{This kind of comparison is commonly called
``\emph{Ablation Study}''.}

One way of measuring the amount
of energy compaction is to quantize the coefficients (the output of
the forward transform) and to entropy encode the result. This RD
(Rate/Distortion) tradeoff has been computed in this
\href{https://github.com/Sistemas-Multimedia/Sistemas-Multimedia.github.io/blob/master/contents/color_DCT/RD_performance.ipynb}{notebook}.

\section{What do I have to do?}

\begin{enumerate}
\item Please, run the
  \href{https://github.com/Sistemas-Multimedia/Sistemas-Multimedia.github.io/blob/master/contents/color_DCT/RD_performance.ipynb}{notebook}
  to learn some insights about the compression of images using the Color-DCT.
\item Try to compress different images. Is the Color-DCT always
  effective (are RD curves better than compressing the image in
  the RGB domain)?
\end{enumerate}

\section{Timming}

Please, finish this milestone before the next class session.

\section{Deliverables}

None.

\section{Resources}

\renewcommand{\addcontentsline}[3]{}% Remove functionality of \addcontentsline
\bibliography{maths,data-compression,signal-processing,DWT,image-compression,image-processing}
