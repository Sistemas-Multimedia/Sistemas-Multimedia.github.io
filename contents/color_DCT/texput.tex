% Emacs, this is -*-latex-*-

\newcommand{\SM}{\href{http://cms.ual.es/UAL/estudios/masteres/plandeestudios/asignaturas/asignatura/MASTER7114?idAss=71142105&idTit=7114}{Sistemas Multimedia}}

\newcommand{\theproject}{\href{}}{MCDWT}
\title{\SM{} - Study Guide - Milestone 4 - Task 1 - The Color-DCT Applied to the $\text{RGB}$ Domain}

\maketitle

\tableofcontents

\section{Description}

\subsection{The $\text{RGB}$ color space}
%{{{ 

The color of a pixel depends on the
\href{https://en.wikipedia.org/wiki/Visible_spectrum}{frequency of the
  light that the pixel captures}. Such information can be represented
in a number of different encoding systems known as
\href{https://en.wikipedia.org/wiki/Color_space}{color spaces}. Among
all those systems, the RGB color space is the most used because it can
be obtained directly from the light signal using color
filters.\footnote{Specifically, a red filter, a green filter and a
blue filter.}

%}}}

\subsection{$\text{RGB}$ color redundancy}
%{{{

The RGB color model is easy to implement, but in general is
quite
\href{https://en.wikipedia.org/wiki/Data_redundancy}{redundant}. This
means that usually some part of the data in an image can be removed
without loss of information. In the case of a color image in the RGB
domain, the three components of each pixel (each one measuring the
energy in a different band of the
\href{https://en.wikipedia.org/wiki/Visible_spectrum}{visible
  spectrum}) can be highly
\href{https://en.wikipedia.org/wiki/Correlation_and_dependence}{correlated}. Therefore,
\href{https://vicente-gonzalez-ruiz.github.io/transform_coding/}{transform
  coding} applied between the color components can concentrate the
information (energy) of the image in a small set of coefficients, that
after quantization and/or entropy coding, can be compressed more
efficiently.\footnote{For example, if the energy of a color subband is
low, quantization could completely makes zero such subband, but the
reconstruction of the image would be reasonable. The most part of
entropy codecs reach higher compression ratios with sequences of
zeros.}

%This
%\href{https://github.com/Sistemas-Multimedia/Sistemas-Multimedia.github.io/blob/master/contents/color_DCT/coding_gain.ipynb}{notebook}
%shows the near-lossless coding gain of an image after using the
%color-DCT.

%This
%\href{https://github.com/Sistemas-Multimedia/Sistemas-Multimedia.github.io/blob/master/milestones/06-YUV_compression/color_redundancy.ipynb}{notebook}
%quantifies the redundancy related to the color domain.

%}}}

\subsection{Rate/Distortion gain provided by the Color-DCT}

All transforms change the representation of the transformed signal,
and decorrelating transforms try to accumulate as much as possible
energy (information) in one of the channels\footnote{Or subbands,
... the name depends on the context.}. One way of measuring the amount
of energy compaction is to quantize the coefficients (the output of
the forward transform) and to entropy encode the result. This RD
(Rate/Distortion) tradeoff has been computed in this
\href{https://github.com/Sistemas-Multimedia/Sistemas-Multimedia.github.io/blob/master/contents/color_DCT/RD_performance.ipynb}{notebook}.

\section{What do I have to do?}

\begin{enumerate}
\item Please, run the
  \href{https://github.com/Sistemas-Multimedia/Sistemas-Multimedia.github.io/blob/master/contents/color_DCT/RD_performance.ipynb}{notebook}
  to learn some insights about the compression of images using the Color-DCT.
\item Try to compress different images. Is the Color-DCT always
  effective (are RD curves better than compressing the image in
  the RGB domain)?
\end{enumerate}

\section{Timming}

Please, finish this milestone before the next class session.

\section{Deliverables}

None.

\section{Resources}

\renewcommand{\addcontentsline}[3]{}% Remove functionality of \addcontentsline
\bibliography{maths,data-compression,signal-processing,DWT,image-compression,image-processing}
