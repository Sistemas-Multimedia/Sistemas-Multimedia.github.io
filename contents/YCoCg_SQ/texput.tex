% Emacs, this is -*-latex-*-

%\newcommand{\SM}{\href{http://cms.ual.es/UAL/estudios/masteres/plandeestudios/asignaturas/asignatura/MASTER7114?idAss=71142105&idTit=7114}{Sistemas Multimedia}}
\newcommand{\SM}{\href{https://sistemas-multimedia.github.io/}{Sistemas Multimedia}}

\newcommand{\theproject}{\href{https://github.com/Sistemas-Multimedia/MRVC}{MRVC}}

\newcommand{\SW}{\href{https://github.com/Sistemas-Multimedia/MRVC}{MRVC}}

\title{\SM{} - Removing $\text{RGB}$ Redundancy with the $\text{YCoCg}$ Color Transform}

\maketitle
\tableofcontents

\section{Objective}
This task is almost identical to
\href{https://sistemas-multimedia.github.io/contents/YCoCg_SQ/}{Removing
  $\text{RGB}$ Redundancy with the DCT}, except that we use the
$\text{YCoCg}$ transform~\cite{vruiz__YCoCg}.

\section{Color decorrelation with the $\text{YCoCg}$ transform}
The $\text{YCoCg}$ is a near-orthogonal~\cite{vruiz__YCoCg}
transform~\cite{vruiz__transform_coding} that has been specifically
designed to accumulate energy in the $\text{Y}$ luma
subband~\cite{vruiz__image_IO}. The subbands $\text{Co}$ and
$\text{Cg}$ represents the croma (orange and green, respectively). The
size of the subbands is the same than the number of pixels in the
original image.

\section{Effects of the transform}
Basically the same than in the case of the DCT:
\begin{enumerate}
\item Smaller entropy (better RD curves).
\item High dynamic range (more points in the RD curves).
\end{enumerate}

\section{Quantization in the $\text{YCoCg}$ domain}
Unlike the DCT, the gains of the synthesis filters of the inverse
$\text{YCoCg}$ transform are not the same. As it can be seen in
\href{https://github.com/Sistemas-Multimedia/Sistemas-Multimedia.github.io/blob/master/contents/YCoCg_SQ/YCoCG_SQ.ipynb}{YCoCG_SQ.ipynb},
the $\text{Y}$ and the $\text{Cg}$ subbands are $3/2$ times more
energetic than the $\text{Co}$ subband. This basically means that the
quantization error~\cite{vruiz__signal_quantization} generated in the
$\text{Y}$ and $\text{Cg}$ subbands is more relevant that the
quantization error generated in the $\text{Co}$ subband (exactly,
$1.5$ times more important). Therefore, the 

Again, like in the color-DCT, ignoring the possible effects of the
entropy encoding stage (that could compress more some color subbands),
the previous gains suggest to use
\begin{equation}
  \frac{3}{2}\Delta_{\text{Y}} = \Delta_{\text{Co}} = \frac{3}{2}\Delta_{\text{Cg}}.
\end{equation}
See this \href{https://github.com/Sistemas-Multimedia/Sistemas-Multimedia.github.io/blob/master/milestones/06-YUV_compression/YCrCb_matrix.ipynb}{notebook}.

If the RD slope of each point depends also on the performance of DEFLATE (something that is normal), we can find the optimal RD curve with the algorithm:
\begin{enumerate}
\item Varilling the $\Delta$, estimate (remember that YCoCg is only  near-orthonal), the RD curve of each YCoCg subband can de found (without considering the rest of subbands).
\item Sort the RD points by their slope.
\item Apply progressively the combinations of quantization
  steps. Measure the distortion in the RGB domain (in the transform
  domain could be only estimated).
\end{enumerate}

%}}}

\section{What do I have to do?}

\begin{enumerate}
\item Please, run the previous
  \href{https://github.com/Sistemas-Multimedia/Sistemas-Multimedia.github.io/blob/master/study_guide/06-color_transform/performance.ipynb}{notebook}
  to learn some insights about the problem of the optimal
  quantization in the color domain.
\item Include in the previous
  \href{https://github.com/Sistemas-Multimedia/Sistemas-Multimedia.github.io/blob/master/study_guide/06-color_transform/performance.ipynb}{notebook}
  an implementation of the
  \href{https://en.wikipedia.org/wiki/JPEG_2000#Color_components_transformation}{RCT
    (Reversible Color Transform)} and compare it's RD performance with
  the other transforms.
\item Implement the transform described in Eq.~\ref{eq:optimal}, and
  compare it with the other transforms.
\end{enumerate}

\subsection{Rate-control}
%{{{
When the transform is orthogonal, the quantization step of a subband should be inversely proportional to the subband gain.  
%}}}

\section{Timming}

Please, finish this milestone before the next class session.

\section{Deliverables}

None.

\section{Resources}

\renewcommand{\addcontentsline}[3]{}% Remove functionality of \addcontentsline
\bibliography{maths,data-compression,signal-processing,DWT,image-compression,image-processing}
