% Emacs, this is -*-latex-*-

\newcommand{\SM}{\href{http://cms.ual.es/UAL/estudios/masteres/plandeestudios/asignaturas/asignatura/MASTER7114?idAss=71142105&idTit=7114}{Sistemas Multimedia}}

\newcommand{\theproject}{\href{}}{MCDWT} \title{\SM{} - Study Guide - Image
  Compression using the $\text{YCoCg}$ Color Transform}

\maketitle
\tableofcontents

\section{Objective}
Let's use the
\href{https://en.wikipedia.org/wiki/YCbCr}{$\text{YCoCg}$} color (from
the \text{RGB} domain) transform to compress $\text{RGB}$ images.

\section{The $\text{YCoCg}$ color domain}
$\text{YCoCg}$ is a \emph{luma}-based
(luminance-based\footnote{Luminance can be considered as the intensity
part of a viual stimuli.} color domain. This domain rely on the idea
of separating the luminance coefficients (Y) from two \emph{chroma}
coefficients (orange and green in the case of $\text{YCoCg}$).

The $\text{YCoCg}$ color model is a succesor of th $\text{YCrCb}$ one,
where the transform filters has been redefined to be more efficient
(considering more and new images). Moreover, it can be also considered
a the digital version of the
\href{https://en.wikipedia.org/wiki/YUV}{$\text{YUV}$ (analog) color
  model}.

\section{Chroma redundancy}

Humans do not perceive detail in the chrominance as well as in they
does in the
luminance.~\cite{vruiz__visual_redundancy,burger2016digital}. As in
$\text{YCrCb}$, the
\href{https://en.wikipedia.org/wiki/Sampling_(signal_processing)}{sampling
  rate} of the chroma is usually reduces to 1/4 without any noticeable
distortion. This feature is used in some of the last image and video
compression systems such as
\href{https://en.wikipedia.org/wiki/JPEG_XR#Description}{JPEG XR} and
\href{https://en.wikipedia.org/wiki/High_Efficiency_Video_Coding#Video_coding_layer}{HEVC}.

\section{Quantization in the YCoCg domain}

%{{{ 

Again, like in the color-DCT, ignoring the possible effects of the
entropy encoding stage (that could compress more some color subbands),
the previous gains suggest to use
\begin{equation}
  \frac{3}{2}\Delta_{\text{Y}} = \Delta_{\text{Co}} = \frac{3}{2}\Delta_{\text{Cg}}.
\end{equation}
See this \href{https://github.com/Sistemas-Multimedia/Sistemas-Multimedia.github.io/blob/master/milestones/06-YUV_compression/YCrCb_matrix.ipynb}{notebook}.

If the RD slope of each point depends also on the performance of DEFLATE (something that is normal), we can find the optimal RD curve with the algorithm:
\begin{enumerate}
\item Varilling the $\Delta$, estimate (remember that YCoCg is only  near-orthonal), the RD curve of each YCoCg subband can de found (without considering the rest of subbands).
\item Sort the RD points by their slope.
\item Apply progressively the combinations of quantization
  steps. Measure the distortion in the RGB domain (in the transform
  domain could be only estimated).
\end{enumerate}

%}}}

\section{What do I have to do?}

\begin{enumerate}
\item Please, run the previous
  \href{https://github.com/Sistemas-Multimedia/Sistemas-Multimedia.github.io/blob/master/study_guide/06-color_transform/performance.ipynb}{notebook}
  to learn some insights about the problem of the optimal
  quantization in the color domain.
\item Include in the previous
  \href{https://github.com/Sistemas-Multimedia/Sistemas-Multimedia.github.io/blob/master/study_guide/06-color_transform/performance.ipynb}{notebook}
  an implementation of the
  \href{https://en.wikipedia.org/wiki/JPEG_2000#Color_components_transformation}{RCT
    (Reversible Color Transform)} and compare it's RD performance with
  the other transforms.
\item Implement the transform described in Eq.~\ref{eq:optimal}, and
  compare it with the other transforms.
\end{enumerate}

\subsection{Rate-control}
%{{{
When the transform is orthogonal, the quantization step of a subband should be inversely proportional to the subband gain.  
%}}}

\section{Timming}

Please, finish this milestone before the next class session.

\section{Deliverables}

None.

\section{Resources}

\renewcommand{\addcontentsline}[3]{}% Remove functionality of \addcontentsline
\bibliography{maths,data-compression,signal-processing,DWT,image-compression,image-processing}
