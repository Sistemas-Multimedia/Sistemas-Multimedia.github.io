%\newcommand{\SM}{\href{http://cms.ual.es/UAL/estudios/masteres/plandeestudios/asignaturas/asignatura/MASTER7114?idAss=71142105&idTit=7114}{Sistemas Multimedia}}
\newcommand{\SM}{\href{https://sistemas-multimedia.github.io/}{Sistemas Multimedia}}

\newcommand{\theproject}{\href{https://github.com/Sistemas-Multimedia/MRVC}{MRVC}}

\newcommand{\SW}{\href{https://github.com/Sistemas-Multimedia/MRVC}{MRVC}}

\title{\SM{} - Study Guide - Motion Compensated CL-LPT}

\maketitle

Technically, the LPT is a frame expansion, that generates an expanded
octave-band decomposition. The ratio of redundancy in the
decomposition $R\rightarrow 2$ with the number of DWT levels.

\section{Description}

The 2D-DWT can a used on sequences of images, by simply iterating as
it is described in the Algorithm~\ref{alg:MDWT}. $I$ is the number of
images in the sequence $V$.

\begin{pseudocode}{\text{MDWT}}{V}
  \label{alg:MDWT}
  \FOR i \GETS 0 \TO I-1 \DO
  V_i = \text{2D-DWT}(V_i)
\end{pseudocode}

\section{What you have to do?}
  
Please, complete this
\href{https://github.com/Sistemas-Multimedia/Sistemas-Multimedia.github.io/blob/master/study_guide/MDWT/MDWT.ipynb}{notebook}.

\section{Timming}

Please, finish this notebook before the next class session.

\section{Deliverables}

None.

\section{Resources}

\bibliography{python}
