% Emacs, this is -*-latex-*-

%\newcommand{\SM}{\href{http://cms.ual.es/UAL/estudios/masteres/plandeestudios/asignaturas/asignatura/MASTER7114?idAss=71142105&idTit=7114}{Sistemas Multimedia}}
\newcommand{\SM}{\href{https://sistemas-multimedia.github.io/}{Sistemas Multimedia}}

\newcommand{\theproject}{\href{https://github.com/Sistemas-Multimedia/MRVC}{MRVC}}

\newcommand{\SW}{\href{https://github.com/Sistemas-Multimedia/MRVC}{MRVC}}

\title{\SM{} - Study Guide - Milestone 4 - Task 1 - Vector Quantization in the $\text{RGB}$ Color Domain}

\maketitle

\tableofcontents

\section{Objective}
%{{{

In this task we are going to implement a lossy image compressor based
on VQ~\cite{vruiz__vector_quantization} and PNG~\cite{vruiz__PNG}.

%}}}

\section{Scalar versus vector quantization of RGB images}
%{{{

SQ (Scalar
Quantization)~\cite{vruiz__scalar_quantization,sayood2017introduction}
would be an optimal solution only if the image colors are uniformly
distributed within
\href{https://en.wikipedia.org/wiki/RGB_color_model}{the RGB
  cube}. However, the typical color distribution in natural images is
anything but uniform, with some regions of the color space being
densely populated and many potentially used colors entirely
missing. In this case, depending on the quantization step
size~\cite{vruiz__signal_quantization}, SQ is not optimal because the
used colors may not be sampled with sufficient density while at the same
time the encoding system is considering colors that do not appear in
the image at all~\cite{burger2016digital}.

On the other hand, VQ (Vector
Quantization)~\cite{vruiz__vector_quantization,sayood2017introduction}
applied to the color domain does not treat the individual RGB
components (also refered by
\href{https://en.wikipedia.org/wiki/Color_image}{channel}s) separately
as does scalar quantization, but each used color vector ${\mathbf C}_i
= ({\mathbf R}_i, {\mathbf G}_i, {\mathbf B}_i )$ in the image is
treated as a minimum structure. Starting from a set of original color
tuples ${\mathbf C} = \{{\mathbf C}_1, {\mathbf C}_2, \ldots ,{\mathbf
  C}_m\}$, where $m$ is the number of different colors found in the
image, the task of a vector quantizer in this context is to:
\begin{enumerate}
\item Find a set of $n$ representative color vectors (the so called
  \emph{code-book}) ${\mathbf C}' = \{{\mathbf C}'_1, {\mathbf C}'_2
  ,\ldots , {\mathbf C}'_n \}$, where $n < m$.
\item Replace each original color ${\mathbf C}_i$ by one of the new
  color vectors ${\mathbf C}'_j\in {\mathbf C}'$, where the resulting
  deviation from the original image shall be minimal.
\end{enumerate}

%}}}

\section{What do I have to do?}
%{{{

\begin{enumerate}
\item Please, using this
  \href{https://github.com/Sistemas-Multimedia/Sistemas-Multimedia.github.io/blob/master/milestones/05-RGB_compression/RGB_compression.ipynb}{notebook}
  try to find a quantization steps combination where
  Eq.~\ref{eq:simple_Q} is not optimal (or at least there is a
  different configuration of QSs better that this equation).
\item Do you think that our lifes would be easier, to compress a RGB
  image, if we had an gray-image (lossy) compressor that allows to
  select the quantization step by its slope?
\end{enumerate}
%\begin{enumerate}
%\item Please, modify this
%  \href{https://github.com/Sistemas-Multimedia/Sistemas-Multimedia.github.io/blob/master/milestones/05-quantization/performance.ipynb}{notebook}
%  in order to use the
%  \href{https://docs.opencv.org/master/d4/da8/group__imgcodecs.html}{TIFF
%    and JPEG 2000 image formats} instead of PNG. Compare the RD
%  curves.
%\item In the previous
%  \href{https://github.com/Sistemas-Multimedia/Sistemas-Multimedia.github.io/blob/master/milestones/05-quantization/performance.ipynb}{notebook}
%  the three color channels, R, G, and B has been quantized using the
%  same QS ($\Delta_{\text{R}} = \Delta_{\text{G}} =
%  \Delta_{\text{B}}$). Do you think that this strategy minimizes the
%  quantization error?
%\item Compare the estimation provided by the entropy with the
%  DEFLATE's bit-rates.
%\end{enumerate}

%}}}

\section{Timming}

Please, finish this milestone before the next class session.

\section{Deliverables}

None.

\section{Resources}

\renewcommand{\addcontentsline}[3]{} % Remove functionality of \addcontentsline
\bibliography{data-compression,signal-processing,DWT,image-processing}
